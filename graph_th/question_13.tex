\section{Теорема Томассена о почти регулярном факторе почти регулярного графа.}
\begin{theorem}[Томассен, 1981]
    Пусть $G$ --- граф, степени всех вершин которого равны или $k$ или $k+1$, а $r \le k$. Тогда существует остовный подграф $F$ графа $G$, степени всех вершин которого равны либо $r$, либо $r+1$.
\end{theorem}
\begin{proof}
    Индукция по $r$. 
	\begin{description}
		\item[База:] $r = k$, очевидно, подойдет $H = G$.
		\item[Переход:] от $k$ к $k-1$. Пусть граф имеет остовный подграф $F$, степени вершин которого равны $r$ или $r+1$.

			Будем удалять из графа $F$ по очереди ребра, соединяющие вершины степени $r+1$. В какой-то момент мы получим граф  $F'$, степени вершин которого равны $r$ или $r+1$, при этом любые две вершины степени $r+1$ несмежны.
			Пусть $V_{r+1}$ --- множество всех вершин степени  $r+1$ в $F'$. Если $V_{r+1} = \varnothing$, то $F'$ уже подходит.

			Пусть $V' = V(G) \setminus V_{r+1}$, $B$ --- двудольный граф с долями $V_{r+1}$ и $V'$, ребра которого --- $E_{F'}(V_{r+1}, V')$.

			Для каждой вершины $x \in  V_{r+1}$ мы имеем $d_B(x)= r+1$, для каждой $y \in V'$ имеем $d_B(y) \le y$.

			По следствию из теоремы Холла в графе $B$ есть паросочетание $M$, покрывающее все вершины из $V_{r+1}$.

			Тогда удалим его и все вершины степени $r+1$ потеряют по одному ребру, а степени $r$ не более одного. Итого получится граф $H = F' -M$, где степени равны $r$ или $r-1$.
	\end{description}
\end{proof}
