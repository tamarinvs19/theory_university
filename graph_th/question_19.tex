\section{Следствие о веере путей из теоремы Менгера. Теорема Дирака о цикле, содержащем заданные k вершин.}
\begin{definition}
    Пусть $X, Y \subset V(G)$, $R \subset V(G) \cup E(G)$.
	\begin{enumerate}
		\item Через $G - R$ обозначим граф, полученный из $G$ в результате удаления всех вершин и ребер из $R$, а также всех ребер инцидентных вершинам из $R$.
		\item Назовем множество $R$ \selectedFont{разделяющим}, если граф $G -R$ несвязен. Обозначим за $\RR(G)$ множество всех разделяющих множеств. 
	\end{enumerate}
\end{definition}
\begin{definition}
    Граф $G$ является \selectedFont{$k$-связным}, если  $v(G) \ge k+1$ и минимальное вершинное разделяющее множество в графе $G$ содержит хотя бы $k$ вершин.
\end{definition}
\begin{definition}
    \begin{enumerate}
		\item Пусть $x, y \in V(G)$ --- несмежные вершины. Обозначим за $\kappa_G(x, y)$ размер минимального множества $R \subset V(G)$ такого, что $R$ разделяет $x$ и $y$. Если $x$ и $y$ смежны, положим $\kappa_G(x, y) = + \infty$. Назовем $\kappa_G(x, y) $ \selectedFont{связностью} вершин $x$ и $y$.

			Пусть $X, Y \subset V(G)$. Обозначим через $\kappa_G(X, Y)$ размер минимального множества $R \subset V(G)$ такого, что $R$ разделяет $X$ и $Y$. Если такого множества нет, положим $\kappa_G(X, Y) = + \infty$.
    \end{enumerate}
\end{definition}
В $k$-связном графе $G$ для любых двух множество вершин $X, Y \subset V(G)$ выполнено $\kappa_G(X, Y) \ge k$.

\begin{theorem}[Менгер, 1927]
    Пусть $X, Y \subset V(G)$, $\kappa_G(X, Y) \ge k$, $\lvert X \rvert \ge k$, $\lvert Y \rvert \ge k$. Тогда в графе $G$ существуют $k$ непересекающихся $XY$-путей.
\end{theorem}

\begin{corollary}\label{cor:connectivity_2}
    Пусть $x \in  V(G)$, $Y \subset V(G)$, $x \notin Y$, $k = \min(\lvert Y \rvert, \kappa_G(x, Y))$. Тогда существуют $k$ путей от $x$ до различных вершин множества $Y$, не имеющих  общих внутренних вершин.
\end{corollary}
\begin{proof}
    Пусть $X = N_{G}(x)$. Так как $\kappa_G(x, Y) \ge k$, $\lvert N_{G}(x) \ge k \rvert$.

	Так как $x\notin Y$, любое множество вершин $R$, отделяющее $X$ от $Y$ отделяет вершину $x$ от множества $Y$. Следовательно, $\lvert R \rvert \ge k$.

	Так как $\lvert Y \rvert \ge k$ по теореме Менгера существует  $k$ непересекающихся путей от $x$ до различных вершин множества $Y$.
\end{proof}

\begin{theorem}[Уитни, 1932]
	Пусть $G$ --- $k$-связный граф. Тогда для любых двух вершин $x , y \in V(G)$ существует $k$ независимых $xy$-путей.
\end{theorem}
\begin{proof}
    Индукция по $k$. 
	\begin{description}
		\item[База:] $k=1$, очевидно
		\item[Переход:] пусть мы доказали для меньших $k$. Если вершины $x$ и $y$ несмежны, то утверждение следует из следствия \ref{cor:connectivity_2}.

			Разберем случай смежных $x$ и $y$.

			Если $G - xy$  --- это $(k-1)$-связный граф, то про индукционному предположению существует $k-1$ независимый $xy$-путь в графе $G-xy$ и еще один путь --- ребро $xy$.

			Теперь предположим, что в $G-xy$  существует разделяющее множество $T$, $\lvert T \rvert \le k-2$. Так как $T$ не разделяет $G$, в графе $G - (T \cup \{xy\})$ ровно две компоненты связности: $U_x \ni x$ и $U_y \ni y$.

			Пусть $T_x = T \cup \{x\}$. Если $U_x \neq \{x\}$, то $T_x$ отделяет $U_x \setminus \{x\}$ от $U_y$ в $G$, но это невозможно, так как $\lvert T_x \rvert \le  k-1$.

			Тогда $U_x = \{x\}$ и, аналогично, $U_y = \{y\}$. Получается, что в графе $G$ не более $k$ вершин: $T$, $x$ и $y$. Но такой граф не может быть $k$-связным.
	\end{description}
\end{proof}

\begin{theorem}[Дирак]
	Пусть $k \ge 2$. В $k$-связном графе для любых $k$ вершин существует простой цикл, содержащий все эти вершины.
\end{theorem}
\begin{proof}
    Индукция по $k$.
	\begin{description}
		\item[База:] $k=2$, следует из теоремы Уитни.
		\item[Переход:] $k-1 \to k$. Рассмотрим $k$-связный граф $G$ и его вершины $v_1, \ldots , v_{k}$. 
			Так как $G$ и  $k-1$-связный тоже, по индукционному предположению  есть простой цикл $Z$, содержащий вершины $v_1, \ldots , v_{k-1}$.
			
			Разберем два случая:
			\begin{enumerate}
				\item Пусть $v(Z) <k$. Тогда $V(Z) = \{v_1, \ldots , v_{k-1}\}$, по следствию \ref{cor:connectivity_2} существуют непересекающиеся пути от $v_k$ до всех вершин $Z$.

					Теперь можно вставить в $Z$ еще одну вершину $v_k$ между двумя соседними.
				\item $v(Z) > k$. По следствию \ref{cor:connectivity_2} существует  $k$ непересекающихся путей от $v_k$ до $Z$. 

					Обозначим концы этих путей $x_1, \ldots , x_k \in  V(Z)$. Они делят круг на $k$-дуг и внутренность еще одной из этих дуг. Поэтому хотя бы одна не содержит ни одной из вершин $v_1, \ldots , v_{k-1}$. Ее мы можем заменить на путь от начала до $v_k$  и от $v_k$ до конца, тем самым получив искомый цикл.
			\end{enumerate}
	\end{description}
\end{proof}
