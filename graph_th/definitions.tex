\begin{definition}
    Цикл $C$ графа $G$ \selectedFont{неразделяющий}, если граф $G - V(C)$ связен. 
\end{definition}
\begin{definition}
    Цикл $C$ \selectedFont{индуцированный}, если не имеет хорд.
\end{definition}
\begin{definition}
    Пусть $G$ и $G'$ --- два плоских графа, а биекция $\varphi \colon V(G) \to V(G')$ удовлетворяет следующим условиям:
	\begin{enumerate}
		\item $xy \in E(G) \Longleftrightarrow \varphi(x) \varphi(y) \in E(G')$ ;
		\item $U \subset V(G)$ является множестов граничных вершин некоторой грани графа $ G$, согда $\varphi(U) = \{\varphi(x) \colon x \in  U\}$ является множеством граничных вершин некторой грани графа $G'$.
	\end{enumerate}
	Тогда $\varphi$ --- \selectedFont{изоморфизм плоских графов} $G$ и $G'$, а плоские графе $G$ и $G'$ \selectedFont{изоморфны}.
\end{definition}

\begin{definition}
    Граф $H'$ --- \selectedFont{подразбиение} графа $H$, если $H'$ может быть получен из $H$ заменой некоторый ребре на простые пути, все добавленные вершины различны и имеют степень 2.

	Вершины $H'$, соответствующие вершинам $H$, называются \selectedFont{главными}.
\end{definition}
