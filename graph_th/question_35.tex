\section{Списочное хроматическое число $k$-редуцируемого графа.}

\begin{definition}[Списочные раскраски]
	Каждой вершине $v \in V(G)$ сопоставляется \selectedFont{ список} $L(v)$, после чего раскраска считается правильной, если цвет каждой вершины входит в ее список.

	Минимальное такое  $k \in \N$, что для любых списков из $k$ цветов существует правильная раскраска вершин графа $G$, обозначается через $\ch(G)$  и называется \selectedFont{ списочное хроматическое число}.

	Аналогично определяются \selectedFont{списочная раскраска ребер} и \selectedFont{списочный хроматический индекс}.
\end{definition}
\begin{note}
	Известны графы, где $\ch(G) > \chi(G)$, но не известны такие, где $\ch'(G) > \chi'(G)$.
\end{note}
 \begin{definition}
    Пусть $k \in \N$. Граф называется \selectedFont{$k$-редуцируемым}, если его вершины можно занумеровать $v_1, \ldots , v_n $ так, что каждая вершины смежна менее чем с $k$ вершинами с б\'oльшим номером.
\end{definition}
\begin{lemma}\label{lm:coloring_3}
	Пусть $G$ --- $k$-редуцируемый граф. Тогда $\chi(G) \le \ch(G) \le k$.
\end{lemma}
\begin{proof}
    Пусть $v_1, \ldots , v_n $ --- нумерация вершин графа из определения, причем каждой вершине $v_i$ соответствует список $L(v_i)$ длины $l(v_1) \ge k$.

	Покрасим вершины в порядке, обратном нумерации. При покраске вершины $v_i$ количество запретов на цвет не превосходит количество соседей среди вершин с б\'oльшим номером, а таких не более $k-1$. Значит, мы можем покрасить вершину $v_i$ в цвет из ее списка.
\end{proof}
\begin{lemma}\label{lm:coloring_4}
    Граф $G$ является редуцируемым, согда для любого его подграфа $H$ выполняется $\delta(H) \le k-1$.
\end{lemma}
\begin{proof}
    \begin{description}
		\item[$ \implies$ ] Пронумеруем вершины графа $G$ как в определении.
			Пусть какой-то подграф  $ H$ имеет $\delta(H) \ge k$.

			Рассмотрим вершину с наименьшим номером $v_i \in V(H)$. Она смежна не менее чем с $d_H(v_i) \ge \delta(H) \ge k$ вершинами с б\'oльшими номерами. Противоречие. 
		\item [$ \impliedby$ ] Пусть $v_1$ --- вершина графа  $G$ наименьшей степени.
			Она смежна не более чем с $d_G(v_1)=\delta(G) \le k-1$ вершиной.

			Пусть вершины $v_1, \ldots , v_{i-1} $ уже построены. 

			Рассмотрим граф $G_i = G - \{v_1, \ldots , v_{i-1}\}$. В нем должна быть вершина степени не более $\delta(G_i) \le  k-1$, которую мы и возьмем в качестве $v_i$.
    \end{description}
\end{proof}
