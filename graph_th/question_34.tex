\section{Теорема Брукса}
\begin{theorem}[Брукс, 1941]
	Пусть $d \ge 3$, $G$ --- связный граф, отличный от $K_{d+1}$, $\Delta(G) \le d$. Тогда $\chi(G) \le d$.
\end{theorem}
\begin{proof}
	Достаточно рассмотреть случай регулярного графа степени $d$, иначе можно воспользоваться леммой \ref{lm:coloring_1}.
	Рассмотрим два случая.
	\begin{itemize}
		\item Пусть в графе $G$ есть точка сочленения $a$. Тогда $G = G_1 \cup G_2$, где $V(G_1) \cap V(G_2) = \{a\}$, а сами $G_1$ и $G_2$ связны.

			Так как $a$ смежна хотя бы с одной вершиной и из $G_1$ и из $G_2$, то $d_{G_1}(a) < d$ b $d_{G_2}(a) <d$. 
			По лемме \ref{lm:coloring_1} можем покрасить $G_1$ и $G_2$ в $d$ цветов. 

			Согласуем раскраски, чтобы цвет вершины $a$ был одинаковый, и получим правильную раскраску всего $G$.
		\item Теперь пусть $G$ двусвязен.
			По лемме \ref{lm:coloring_2} существуют такие $a, b, c \in V(G)$, что $ab, bc \in E(G)$, $ac \notin E(G)$ и граф $G-a-c = G'$ связен.

			Рассмотрим такой $G'$ и его остовное дерево $T$.

			Подвесим дерево за $b$. Пронумеруем уровни так, чтобы номер совпадал с расстоянием от корня. 

			Пусть $\rho(a) = \rho(c) = 1$. Будем красить остальные вершины дерева в порядке убывания номеров их уровней, начиная с листьев.

			Пусть $x \neq b$ --- очередная вершина, причем на момент ее рассмотрения мы покрасили все вершины больших уровней, но не красили вершины меньших. Тогда ее предок еще не имеет цвета, поэтому соседи покрашены максимум в $d-1$ цвет. Следовательно, хотя бы один свободный останется.

			Посмотрим на момент, когда осталась только вершина $b$. У нее два соседа $a$ и $c$ имеют один цвет, поэтому опять есть свободный цвет.

			Так мы покрасили все вершины графа $G$ в $d$ цветов.
\end{itemize}
\end{proof}
