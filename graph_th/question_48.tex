\section{Хроматический многочлен и компоненты связности. Кратность корня 0 хроматического многочлена графа.}
\begin{lemma}\label{lm:coloring_13}
    Пусть  $G_1, \ldots , G_{n} $ --- все компоненты графа $G$. Тогда $\chi_{G}(k) = \prod_{i=1}^{n}\chi_{G_i}(k)$.
\end{lemma}
\begin{proof}
    Очевидно
\end{proof}

\begin{theorem}
    Для любого графа $G$ число $0$ является корнем $\chi_G(k)$ кратности, равной количеству компонент связности.
\end{theorem}
\begin{proof}
    $0$ --- корень любого хроматического многочлена, так как раскрасок в $0$ цветов быть не может.

	Докажем, что для связного графа $G$ кратность корня $0$ у $\chi_G(k)$ равна $1$. Далее по лемме \ref{lm:coloring_13} получим утверждение теоремы.

	Пусть $v(G) = n$. Индукцией по количеству вершин докажем для связного графа $G$, что коэффициент при $k$ многочлена $\chi_G(k)$ не равен $0$ и имеет такой же знак как $(-1)^{n}$.
	\begin{description}
		\item[База:] 	$n=0$, очевидно.
		\item[Переход:] пусть $G$ --- связный граф c $v(G) = n  \ge 2$, для меньшего $n$ утверждение доказано, $T$ --- остовное дерево графа $G$. $\chi_T(k) = k(k-1)^{n-1}$.

			Существует последовательность графов $ G_0 = T, \ldots , G_n = G$, в которой $G_{i+1} = G_i + e$, где $e \notin E(G_i)$.

			Пусть $a_i$ --- коэффициент при $k$ многочлена $\chi_{G_i}(k)$. Докажем по индукции, что $a_i \neq 0$ и имеет такой же знак, что и $(-1)^{n-1}$. 
			\begin{description}
				\item[Бaза:] $i=0$, очевидно, по формуле для дерева.
				\item [Переход:] пусть $a_i \neq 0$ и имеет знак $(-1)^{n-1}$. По лемме \ref{lm:coloring_122} $\chi_{G_{i+1}}(k) = \chi_{G_i}(k) - \chi_{G_i \cdot e}(k)$.
					
					Граф $G_i \cdot e_i$ связен. По индукционному предположению у многочлена $\chi_{G_i \cdot e_i}(k)$ знак коэффициента $b$ при $k$ такой же, как $(-1)^{n-2}$, то есть отличается от знака $a_{i}$.

					Поэтому $a_{i+1} = a_i - b$ имеет такой же знак, что и $a_i$ и отличный от $0$.
			\end{description}
	\end{description}
\end{proof}

