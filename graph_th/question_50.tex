\section{Теорема Жордана для ломаной.}
\begin{theorem}[Жордан, 1887]
    Замкнутая несамопересекающаяся ломаная $P$ делит точки плоскости, не лежащие на $P$, на две такие части, что выполнены следующие условия:
	\begin{enumerate}[label=(\arabic*)]
		\item любые две точки из одной части можно соединить ломаной, не пересекающей $P$ ;
		\item любая ломаная, соединяющая две точки из разных частей пересекает $P$.
	\end{enumerate}
\end{theorem}
\begin{proof}
    Пусть $ P_1 \ldots P_m$ --- вершины $P$ в порядке обхода по часовой стрелке. Обозначим через $M$ множество всех точек плоскости, не лежащих на $P$.

	Зафиксируем на прямой вектор $l$, не параллельный ни одной из сторон $P$. Из каждой точки $A \in M$ выпустим луч $l(A)$ в направлении $l$.

	Если $l(A)$ содержит вершину $P_i$ многоугольника $P$, то стороны $P_{i-1}P_i$ и $P_iP_{i+1}$ лежат в одной полуплоскости относительно $l(A)$, будем говорить, что многоугольник $P$ \selectedFont{касается } $l(A) $ в вершине $P_i$.

	Посчитаем число $p(A)$ точек пересечения $l(A) $ c $P$, не являющихся касаниями. Оно точно конечное.

	Обозначим за $M_0$ ту часть, которая состоит из всех точек $A \in M$, для которых $p(A)$ четно, и за $M_1$ --- нечетно.

	\begin{statement}\label{st:planar_1}
	    $M_0$ и $M_1$ непусты.
	\end{statement}
	\begin{proof}
	    Рассмотрим прямую $l_0$, параллельную вектору $l$, проходящую через внутреннюю точку ломаной $P$.

		Найдем последнее пересечение во внутренней точке прямой $l_0$ и $P$ в направлении вектора $l$ --- обозначим за ее $X$.

		Рассмотрим содержащий $X$ малый отрезок $[Y, Z]$ на $l_0$, не пересекающий $P$ в отличных от $X$ точках. Пусть $Y$ лежит перед $X$ при движении в направлении $l$. Тогда $p(Y) = 1$, а $p(Z) = 0$.
	\end{proof}
	\begin{statement}\label{st:planar_2}
		Пусть $A, B \in M$ и отрезок $[A, B]$ не пересекает $P$. Тогда $p(A) \equiv p(B) \pmod 2$.  В частности, выполнено второе условие теоремы.
	\end{statement}
	\begin{proof}
	    Если $AB \parallel l$, то утверждение очевидно.

		Если нет, отметим на отрезке $AB$ все такие точки $A_1, \ldots , A_k$ в направлении от $A$ к $B$, что $l(A_i)$ касается $P$ (если такие есть). И обозначим $A_0 = A$, $A_{k+1}  = B$.

		Тогда для каждого $i \in [0..k]$ все точки отрезка $[A_i, A_{i+1}]$ имеют одинаковое значение функции $p$, при переходе на соседний отрезок значение может изменится на четное число (см. рис. \ref{fig:jordan-theorem}).

		В любом случае, на всем отрезке $[A, B]$ четность одинаковая.
	\end{proof}
	\begin{figure}[ht]
		\centering
		\begin{subfigure}{0.48\textwidth}
			\centering
			\incfig{jordan-theorem}
			\caption{}
			\label{fig:jordan-theorem}
		\end{subfigure}
		\hfill
		\begin{subfigure}{0.48\textwidth}
			\centering
			\incfig{jordan-theorem-2}
			\caption{}
			\label{fig:jordan-theorem-2}
		\end{subfigure}
			\caption{}
	\end{figure}
	\textbf{Докажем первое утверждение теоремы}

	Пусть $A, B \in M_i$. Если отрезок $[A, B]$ не пересекает $P$, то все уже доказано. Тогда найдем ближайшие к $A$ и к $B$ точки пересечения $A_1$ и $B_1$ соответственно.

	Отметим на отрезке $[A, A_1]$ точку $A'$ очень близко к $A_1$, на отрезке $[B_1, B]$ --- точку $B'$ очень близко к $B$, обозначим <<очень близко>> за $\delta$. Тогда $p(A) = p(A')$ и $p(B) = p(B')$. См. рис. \ref{fig:jordan-theorem-2}.

	Проведем вдоль каждой стороны многоугольника две параллельных прямых на расстоянии $\delta$ с обоих сторон от $P$. Получим два новых многоугольника $P'$ и $P''$. Подбираем $\delta$ так, чтобы эти многоугольники не пересекали сторон исходного.

	НУО $A'$ лежит на $P'$. Если $B'$ тоже лежит на $P'$, то мы можем дополнить ее до точек $A$ и $B$, тем самым получив ломаную от $A$ до $B$, не пересекающую $P$.

	Пусть $B'$ лежит на $P''$. Тогда обозначим за $B^*$ точку пересечения $P'$ c $AB$ около $B$ на расстоянии $\delta$.

	Тогда $p(B^*) - p(B') = \pm 1$. Но по утверждению \ref{st:planar_2} должно выполнятся сравнение 
	\[
	p(B^*) \equiv p(A') \equiv p(A) \equiv p(B) \equiv p(B') \pmod 2
	.\] 
	Противоречие. 
\end{proof}
