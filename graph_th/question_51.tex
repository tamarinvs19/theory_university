\section{Грань плоского графа и ее граница. Свойства.}
\begin{definition}
    Граф называется \selectedFont{планарным}, если его можно изобразить на плоскости так, чтобы его ребра не пересекались во внутренних точках. 
\end{definition}
\begin{definition}
    \selectedFont{Плоский граф} --- конкретное изображение планарного графа бед пересечений и самопересечений ребер.
\end{definition}
\begin{definition}[Грань]
	Пусть $M$ --- множество всех точек плоскости, не входящих в изображение $G$. Запись $A \sim B $ означает, что точки $A, B \in M$  можно соединить ломаной, не пересекающей изображение графа $G$. $\sim$ --- отношение эквивалентности.

	Назовем классы эквивалентности по $\sim$ \selectedFont{гранями}. Обозначим множество всех граней $F(G)$ и  $f(G) = \lvert F(G) \rvert$.
\end{definition}

\begin{definition}
    Рассмотрим ребро $e$ плоского графа $G$. Если по разные стороны $e$ расположены разные грани, то это ребро \selectedFont{граничное}, если одна и та же, что \selectedFont{внутреннее}.
	Обозначим через $E_d$ множество всех граничных и внутренних ребер.
\end{definition}
\begin{definition}
    \selectedFont{Граничные вершины} грани $d$ --- вершины, до которых можно дойти по ломаной от внутренних точек этой грани, не пересекая изображение графа $G$. Обозначим их множество через $V_d$.
\end{definition}
\begin{definition}
    \selectedFont{Граница} грани $d$ --- подграф $B(d)$ графа $G$ с множеством вершин $V_d$ и множеством ребер $E_d$.
\end{definition}
\begin{definition}
    \selectedFont{Размер границы} грани $d$ --- количество граничный ребер этой грани плюс удвоенное количество внутренних. Обозначение: $b(d)$.

	$\sum_{d \in F(G)}^{} b(d) = 2 e(G)$.
\end{definition}
\begin{lemma}\label{lm:planar_1}
    \begin{enumerate}
		\item Любые две точки на границе грани $d$ можно соединить ломаной, проходящей в $d$.
		\item Если две точки $A$ и $B$ на изображении графа $G$ можно соединить ломаной $L$, не пересекающей изображения $G$, то $A$ и $B$ лежат на границе некоторой грани.
    \end{enumerate}
\end{lemma}
\begin{proof}
    \begin{enumerate}
		\item Пусть $A$ --- внутренняя точка грани $d$. От нее можно провести ломаные, не пересекающие изображение $G$, до любых двух граничных. Все точки на этих ломаных лежат в $d$.
		\item $A$ и $B$ точно лежат на границе грани $d$, содержащей все внутренние точки $L$.
    \end{enumerate}
\end{proof}
