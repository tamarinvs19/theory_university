\section{Теорема Петерсена о паросочетании в кубическом графе.}
\begin{definition}
    \selectedFont{Кубический граф} --- граф, все вершины которого имеют степень $3$.
\end{definition}
\begin{definition}
    \selectedFont{Мост графа} --- ребро, не входящее ни в один цикл.
\end{definition}
\begin{theorem}[Петерсон, 1891]
	Пусть $G$ --- связный кубический граф, в котором не более двух мостов. Тогда в графе есть совершенное паросочетание.
\end{theorem}
\begin{proof}
    Пусть совершенного паросочетания нет. Тогда по Теореме Тaтта существует такое множество $S \subset V(G)$, что $o(G-S) > \lvert S \rvert$.

	Так как в кубическом графе четное число вершин, $S \neq \varnothing$ и $o(G-S) \equiv \lvert S \rvert \pmod 2$.

	Пусть $U_1, \ldots , U_n$ --- все нечетные компоненты связности графа $G - S$. $n \ge \lvert S \rvert + 2$.

	Пусть $m_i = e_G(U_i, S)$. Это число нечетное, так как:
	\[
	m_i = \sum_{v \in U_i}^{} d_G(v) - 2 e(G(U_i)) = 3 \lvert U_i \rvert - 2e(G(U_i))
	.\] 

	В $G$ не больше двух мостов, поэтому не более, чем два числа из $m_i$ равны $1$, а остальные не меньше $3$, так как нечетные.
	\[
	3 \lvert S \rvert = \sum_{v \in S}^{} d_G(v) \ge  \sum_{i=1}^{n} m_i \ge 3n-4 \ge 3(\lvert S \rvert+2) -4 > 3 \lvert S \rvert
	.\] 
	Противоречие. 
\end{proof}
