\section{Лемма о галочке}
\begin{definition}
	\selectedFont{Раскраска вершин} графа $G$ в  $k$ цветов --- функция $\rho\colon V(G) \to M$, где $\lvert M \rvert = k$. Раскраска $\rho$ называется \selectedFont{правильной}, если $\rho(v) \neq \rho(u)$ для любой пары смежных вершин   $u$ и $v$ .

	Через $\chi(G)$ обозначим \selectedFont{хроматическое число} графа $G$ --- наименьшее натуральное число, для которого существует правильная раскраска вершин графа $G$ в такое количество цветов.

	\selectedFont{Раскраска ребер} графа $G$ в $k$ цветов --- функция $\rho\colon E(G) \to M$, где $\lvert M \rvert = k$. Раскраска называется \selectedFont{правильной}, если $\rho(v) \neq \rho(u)$ для любой пары смежных ребер $u$ и $v$.

	Через $\chi'(G)$ обозначим \selectedFont{хроматический индекс} графа $G$ --- наименьшее натуральное число, для которого существует правильная раскраска ребер графа $G$ в такое количество цветов.
\end{definition}

\begin{lemma}\label{lm:coloring_1}
	Пусть $G$ --- связный граф, $\Delta(G) \le d$, причем хотя бы одна из верши графа имеет степень менее $d$. Тогда $\chi(G) \le d$.
\end{lemma}
\begin{proof}
    Индукция по количеству вершин.
	\begin{description}
		\item [База:] Если в графе не более $d$ вершин, его точно можно покрасить в $d$ цветов.
		\item [Переход:] Пусть мы уже доказали утверждение для любого меньшего связного графа с меньшим числом вершин.

			Пусть $u \in V(G)$ --- вершина степени менее $d$. Рассмотрим граф $G - u$. Пусть $G_1, \ldots , G_k  $ --- компоненты $G-u$.

			В каждом из графов $G_i$ есть вершина $u_i$, смежная с $u$ в графе $G$. Тогда у $u_i$ в $G_i$ степень не более $d-1$, и $\Delta(G_i) \le d$.

			По индукционному предположению, каждый $G_i$ можно покрасить в $d$ цветов. Далее докрашиваем $u$, мы можем это сделать, так как у нее только $d-1$ ребро.
	\end{description}
\end{proof}

\begin{lemma}\label{lm:coloring_2}
    Если $G$ --- двусвязный неполный граф с $\delta(G) \ge 3$, существуют такие вершины $a, b, c \in V(G)$, что $ab, bc \in E(G)$, $ac \notin E(G)$ и граф $G-a-c$ связен.
\end{lemma}

\begin{proof}
	\begin{itemize}
		\item Пусть $G$ трехсвязен.

			Так как $G$ неполный, существуют такие вершины $a, b, c \in V(G)$, что $ab, bc \in E(G)$ и $ac \notin E(G)$. Граф $G-a-c$ точно связен. 
		\item Путь $G$ не трехсвязен, тогда существует вершина $b \in V(G)$, что граф $G' = G - b$ не двусвязен.

			Граф $G'$ имеет хотя бы два крайних блока. Так как исходный $G$ двусвязен, вершина  $b$ должна быть смежна хотя бы с одной внутренней вершиной каждого крайнего блока $G'$. Пусть $a$ и $c$ --- смежные с $b$ внутренние вершины двух разных крайних блоков $B_a$ и $B_c$ графа $G'$.
		\begin{figure}[ht]
			\centering
			\incfig{checkmark-lemma}
			\caption{}
			\label{fig:checkmark-lemma}
		\end{figure}
		
		Тогда графы $B_a - a$ и $B_c - c$ связны, поэтому и $G'- a -c $ связен.

		Так как $d_G(b) \ge 3$, вершина $b$ смежна с $G'-a-c$, а значит, и граф $G -a-c$ связен.
	\end{itemize}
\end{proof}
