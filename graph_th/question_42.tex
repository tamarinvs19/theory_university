\section{Гипотеза Хайоша, случай $k = 4$.}
\begin{definition}
    Пусть $H$ --- произвольный граф. Назовем граф $H'$ \selectedFont{подразбиением} графа $H$, если $H'$ может быть получен из $H$ заменой нескольких ребер на простые непересекающиеся пути.
\end{definition}
\begin{theorem}[Дирак, 1953]
    Если $\chi(G) = 4$, то граф содержит в качестве подграфа подразбиение $K_4$.
\end{theorem}
\begin{proof}
    Достаточно доказать для $4$-критических графов. Рассмотрим такой граф $G$. Будем доказывать по индукции по количеству вершин в графе.

	\begin{description}
		\item[База:] $G$ --- трехсвязный граф. $\delta(G) \ge 3$, тогда в графе $G$ существует простой цикл $Z = a_1a_2\ldots a_n$, длины хотя бы $4$.

			Так как $G - \{a_1, a_3\}$ связен, существует простой путь $P$ от $a_2$ до $a_m$, не проходящий по другим вершинам цикла.

			Эти две вершины делят цикл на две непустые дуги: $B = \{a_3, \ldots , a_{m-1}\}$ и $B' = \{a_{m+1}, \ldots , a_1\}$.

			Так как $G - \{a_2, a_m\}$ связен, существует $BB'$-путь $Q$, не проходящий через $a_2$ и $a_m$. Пусть $a_x \in B$ и $a_y \in B'$ --- концы $Q$.

			Рассмотрим два случая
			\begin{itemize}
				\item $V(P) \cap V(Q) = \varnothing$.

					Рассмотрим подграф $H$ графа $G$ --- объединение цикла $Z$ и путей $P$ и $Q$. Этот граф --- подразбиение $K_4$.  См. рис. \ref{fig:dirac-lemma-1}.
				\item $V(P) \cap V(Q) \neq \varnothing$.

					Пусть $u$ --- первая точка пересечения c $Q$ на пути $P$ (от $a_2$ ). Тогда подграф $H$, равный объединению цикла $Z$, пути $Q$ и участка $a_2Pu$ --- подразбиение $K_4$. См. рис. \ref{fig:dirac-lemma-2}.
			\end{itemize}
		\begin{figure}[ht]
			\centering
			\begin{subfigure}{0.3\textwidth}
				\incfig{dirac-lemma}
				\caption{}
				\label{fig:dirac-lemma-1}
			\end{subfigure}
			\hfill
			\begin{subfigure}{0.3\textwidth}
				\incfig{dirac-lemma-2}
				\caption{}
				\label{fig:dirac-lemma-2}
			\end{subfigure}
			\hfill
			\begin{subfigure}{0.3\textwidth}
				\incfig{dirac-lemma-3}
				\caption{}
				\label{fig:dirac-lemma-3}
			\end{subfigure}
		\end{figure}
	\item[Переход:] граф $G$ не трехсвязен.

		Пусть $S $ --- минимальное разделяющее множество графа $G$. Тогда $\lvert S \rvert \le 2$. Пусть $S = \{a, b\}$.

		По лемме \ref{lm:coloring_9} вершины $a$ и $b$ несмежны,  $\Part(S) = \{F_1, F_2\}$, $G_i = G(F_i)$, причем части можно занумеровать так, что граф  $G_1^* = G_1 + ab$ --- $4$-критический.

		По индукционному предположению, в графе $G_1^*$ есть подграф $H$, являющийся подразбиением $K_4$. Если $H$ --- уже подграф $G$, то переход доказан.

		Предположим, что $H$ не подграф $G$, тогда $ ab \in H$ . 

		Граф $G_2$ связен, поэтому существует $ab$-путь $P$ c $\Int(P) \cap V(H) = \varnothing$. Тогда граф $H' = H - ab \cup P$ -- подграф $G$ и подразбиение $K_4$. См. рис. \ref{fig:dirac-lemma-3}.
	\end{description}
\end{proof}
