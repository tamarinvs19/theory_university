\section{Теорема Хватала о гамильтоновых последовательностях.}
\begin{definition}
	\begin{enumerate}
		\item Пусть $a_1 \le a_2 \le  \ldots a_n$ и $b_1 \le b_2 \le \ldots b_n$ --- две упорядоченные последовательности. Последовательность $\{a_i\}_{i \in [1..n]}$ \selectedFont{мажорирует} последовательность $\{b_i\}_{i \in [1..n]}$, если $ \forall i \in  [1..n]\colon a_i \ge b_i$.
		\item Пусть $G$ --- граф на $n$ вершинах. \selectedFont{Степенная последовательность} графа $G$ --- упорядоченная последовательность степеней его вершин $d_1 \le \ldots d_n$. 
		\item Будем говорить, что граф $G$ \selectedFont{мажорирует} граф $H$, если $v(G) = v(H)$ и степенная последовательность графа $G$ мажорирует степенную последовательность графа $H$.
		\item Последовательность  $a_1 \le a_2 \le \ldots a_n$ называется \selectedFont{гамильтоновой}, если $a_n \le n-1$ и любой граф на $n$ вершинах, степенная последовательность которого мажорирует $a_1, \ldots a_n$, имеет гамильтонов цикл.
	\end{enumerate}
\end{definition}

\begin{theorem}[Критерий Хватала, 1972]
    Пусть $0 \le a_1 \le a_2 \le \ldots \le a_n \le n-1$, $n \ge 3$. Следующие два утверждения равносильны:
	\begin{enumerate}
		\item Последовательность $a_1, \ldots a_n $ гамильтонова.
		\item Для каждого $s < \frac{n}{2}$ из $a_s \le s$ следует $a_{n-s} \ge  n-s$.
	\end{enumerate}
\end{theorem}
\begin{proof}
	\begin{description}
		\item[$2 \implies 1$] Предположим, что наша последовательность негамитльтонова. Рассмотрим негамильтонов граф $G$ на $n$ вершинах с максимальным числом ребер, степенная последовательность $\{d_i\}_{i \in [1..n]}$ которого мажорирует $\{a_i\}_{i \in [1..n]}$. 

			По лемме \ref{lm:circle_2} граф $G$ совпадает со своим замыканием, так как граф максимальный, и сумма степеней любых двух несмежных вершин менее $n$.

			Рассмотрим две несмежные вершины $x, y \in V(G)$ c максимальной суммой $d_G(x) + d_G(y)$, такие есть, иначе граф полный, и точно гамильтонов. Не умаляя общности $d_G(x) \le d_G(y)$.

			Так как $d_G(x) + d_G(y) < n$, имеем $d_G(x) = s < \frac{n}{2}$, поэтому $d_G(y) \le n-1 -s$.

			Пусть $W_x$ --- множество всех вершин графа $G$, отличных от $x$ и не смежных с $x$, $W_y$ --- аналогично для $y$.
			\[
			\lvert W_x \rvert = n-1-d_G(x) = n-1-s; \qquad \lvert W_y \rvert = n-1-d_G(y) \ge d_G(x) = s
			.\] 

			Степени всех вершин множества $W_y$ не превосходят $s $, так как $s$ дает максимальную сумму с $d_G(y)$. Поэтому  $a_s \le d_s \le s$. В множестве $W_x \cup \{x\}$ будет $n-s$ вершин, причем их степени не превосходят $d_G(y) \le n-1-s$, поэтому $a_{n-s} \le d_{n-s} \le n-s-1$.

			Но это противоречит условию. Следовательно, последовательность $\{a_i\}$ гамильтонова.
		\item[$1 \implies 2$] Докажем, что последовательность  $\{a_i\}$ не может быть гамильтоновой, если не выполнено второе условие.
			
			Пусть $h < \frac{n}{2}, a_h \le h$ и $a_{n-h} \le n - h -1$. Построим негамильтонов граф $G_{n, h}$, степенная последовательность которого мажорирует $\{a_i\}$.

			Пусть $A = \{v_1, \ldots , v_h\}$, $B = \{v_{n-h+1}, \ldots , v_n\}$, $D = \{v_{h+1}, \ldots, v_{n-h} \}$. Граф $G_{n, h}$ будет объединением $K_{h, h}$ с долями $A$ и $B$ и $K_{n-h}$ на вершинах $B \cup D$ :
		\begin{figure}[ht]
			\centering
			\incfig{existing-gamiltonov-cycle}
			\caption{Граф $G_{n,h}$}
			\label{fig:existing-gamiltonov-cycle}
		\end{figure}
	\end{description}

	Здесь все степени в $A$ равны $h$, в $B$ --- $n-1$, в $D$ --- $n-h-1$. Степенная последовательность выглядит следующим образом:
	 \[
		 \underbrace{h, \ldots , h}_{h}, \underbrace{n-h-1, \ldots , n-h-1}_{n-2h}, \underbrace{n-1, \ldots , n-1}_{h}
	.\] 

	Эта последовательность мажорирует $a_1, \ldots , a_n$.
	
	Всего компонент связности $c(G_{n,h} - B) = h+1 = \lvert B \rvert$, это $D$ и отдельные вершины в $A$.

	Так как $c(G_{n,h}-B) >h = \lvert B \rvert$, можем применить лемму \ref{lm:circle_4} и получить, что $G_{n, h}$ не является гамильтоновым.

\end{proof}
