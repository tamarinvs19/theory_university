\chapter{Пути и циклы} 

\lecture{1}{15 feb}
Все материалы можно найти на сайте \url{https://logic.pdmi.ras.ru/~dvk/MKN/graph_th}.
\begin{note}
    В этом разделе возможны кратные ребра.
\end{note}

\section{Эйлеров путь и цикл}

\begin{definition}[]

	\selectedFont{Эйлеров путь} в графе $G$ --- путь, проходящий по каждому ребру ровно один раз. 
	
	\selectedFont{Эйлеров цикл} в графе $G$ --- цикл, проходящий по каждому ребру ровно один раз. 

	 Граф $G$ --- \selectedFont{эйлеров}, если в нем есть эйлеров цикл.
\end{definition}

\begin{theorem}
	Связный граф $G$ --- эйлеров, согда степени всех вершин $G$ четны.
\end{theorem}
\begin{corollary}[]
	Связный граф $G$ имеет эйлеров путь, согда в нем либо нет вершин с нечетной степенью, либо их ровно две.
\end{corollary}

\section{Гамильтонов путь и цикл}

\begin{definition}[]
	\selectedFont{Гамильтонов путь} --- простой путь, проходящий по каждой вершине графа.

	\selectedFont{Гамильтонов цикл} --- простой цикл, проходящий по каждой вершине графа.

	\selectedFont{Гамильтонов граф}  --- граф, в котором есть гамильтонов цикл.
\end{definition}

\begin{lemma}\label{lm:circle_1}
    Пусть $n > 2$, $a_1 \ldots a_n$ --- максимальный путь (по ребрам) в графе $G$, причем $d_{G}(a_1) + d_{G}(a_{n}) \ge n$. Тогда в графе есть цикл длины $n$.
\end{lemma}
$N_G(v)$ --- все вершины достижимые из вершины $v$ в графе $G$.

$d_G(v)$ --- степень вершины  $v$ в графе $G$.

\begin{proof}
	Разберем несколько случаев:
	\begin{itemize}
		\item Если $a_1$ и $a_{n}$ смежны, то $a_1a_2\ldots a_{n}$ --- искомый цикл.
		\item Иначе $N_{G}(a_1), N_{G}(a_{n}) \subset \{a_2, \ldots a_{n-1}\}$, так как удлинить путь нельзя.

			\begin{minipage}{0.6\textwidth}
				Если есть вершина $a_{k}$ смежная с $a_{n}$ и вершина $a_{k+1}$ смежная с $a_1$, то в графе есть цикл из $n$ вершин
					\[
					a_1a_2\ldots a_{k}a_{n}a_{n-1}\ldots a_{k+1}
					.\] 
			\end{minipage}
			\hfill
			\begin{minipage}{0.25\textwidth}
				\centering
				\incfig{lm-circle}
				\label{fig:lm-circle}
			\end{minipage}
			Пусть $N_G(a_{n}) = \{a_{i_1}, \ldots , a_{i_l}\}$.

			Если хотя бы одна из вершин $a_{i_{1}+1}, \ldots , a_{i_{l}+1}$ лежит в $N_{G}(a_1)$, то, согласно утверждению выше, в графе есть цикл длины $n$. 

			Иначе $d_G(a_1) \le n-1 - d_G(a_n)$, а это противоречит условию.
	\end{itemize}
\end{proof}

\begin{theorem}[Критерий Оре, 1960]
	\begin{enumerate}
		\item Если для любых двух несмежных вершин $u, v \in V(G)$ выполняется 
			\[
			d_G(u) + d_{G}(v) \ge v(G) - 1
			,\] 
			то в графе $G$ есть гамильтонов путь.
		\item Если  $v(G) > 2$ и для любых двух несмежных вершин  $u, v \in V(G)$ выполняется
			\[
			d_{G}(u) + d_{G}(v) \ge v(G)
			,\] 
			то в графе $G$ есть гамильтонов цикл.
	\end{enumerate}
\end{theorem}
\begin{proof}
\begin{enumerate}
	\item Докажем первое утверждение
	\begin{itemize}
	\item Для двух вершин все очевидно. Далее предположим, что  $v(G) > 2$.
	\item Рассмотрим две вершины $a$ и $b$ и  предположим, что они несмежные. По условию $d_G(a)+ d_{G}(b) \ge v(G) -1 $, поэтому $N_G(a) \cap N_G(b) \neq \varnothing$, следовательно, $a$ и $b$ связаны. Тогда граф $G$ связен.
	\item Теперь найдем наибольший простой путь $a_1\ldots a_{n}$ в графе $G$. Так как вершин больше двух, и граф связен, $n \ge 3$. Предположим, что это не гамильтонов путь, то есть $n \le v(G) - 1$.
	\item Если $a_1\ldots a_{n}$ не цикл, то по лемме \ref{lm:circle_1} существует цикл $Z$ из $n$ вершин, так как 
		\[
		d_G(a_1) + d_G(a_{n}) \ge v(G) - 1 \ge n
		.\] 
	\item Так как граф связен, существует не вошедшая в этот цикл вершина, смежная с хотя бы одной из вершин цикла. Тогда из нее и цикла можно получить путь длиной $n+1$, противоречие.
	\end{itemize}
\item По первому пункту уже есть гамильтонов путь, обозначим его за $a_1\ldots a_{n}$, где $n = v(G)$.

	Если $a_1$ и $a_n$ смежны, то мы нашли гамильтонов цикл. Иначе
	\[
	d_G(a_1) + d_G(a_{n}) \ge v(G) = n
	.\] 
	А тогда по лемме \ref{lm:circle_1} в графе есть гамильтонов цикл.
\end{enumerate}    
\end{proof}
\begin{corollary}[Критерий Дирака, 1952]
	\begin{enumerate}
		\item Если $\delta(G) \ge \frac{v(G)-1}{2}$, то в графе $G$ есть гамильтонов путь.
		\item  Если $\delta(G) \ge \frac{v(G)}{2}$, то в графе $G$ есть гамильтонов цикл.
	\end{enumerate}
\end{corollary}

\begin{lemma}
    Пусть вершины $a$ и $b$ не смежны и $d_G(a) + d_G(b) \ge v(G)$. Тогда граф $G$ гамильтонов, согда граф $G + ab$ тоже гамильтонов.
\end{lemma}
% TODO: proof

\begin{definition}[]
	Рассмотрим произвольный граф $G$. Пока существуют две вершины $a, b \in V(G)$, для которых $d_G(a) + d_G(b) \ge v(G)$, добавим в граф соответствующее ребро $ab$. Полученный граф называется \selectedFont{замыканием} графа $G$, обозначается $C(G)$.
\end{definition}

\begin{lemma}[Хватал, 1974]
    Граф $G$ гамильтонов, согда его замыкание $C(G)$ --- гамильтонов граф.
\end{lemma}

\begin{lemma}[о единственности замыкания]
    Замыкание графа $G$ определено однозначно, то есть не зависит от порядка добавления ребер.
\end{lemma}
% TODO: proof

\begin{lemma}
    Пусть граф $G$ гамильтонов. Тогда для любого множества $S \subset V(G)$
	выполняется неравенство $c(G - S) \le \lvert S \rvert$.
\end{lemma}
% TODO: proof

\begin{theorem}[Хватал, Эрдёшь, 1972]
    Пусть $v(G) \ge 3$ и $\kappa(G) \ge \alpha(G)$, тогда $G$ гамильтонов.
\end{theorem}

\begin{definition}[]
	Пусть $a_1 \le a_2 \le \ldots \le a_{n} $ и $b_1 \le b_2 \le  \ldots \le a_{n}$. Последовательность $\{a_i\}_{i \in [1..n]}$ \selectedFont{мажорирует} последовательность $\{b_i\}_{i \in [1..n]}$, если $a_i \ge b_i$ для всех $i \in [1..n]$ .
\end{definition}
\begin{definition}[]
	Пусть $G$ --- граф на $n$ вершинах. \selectedFont{Степенная последовательность} графа $G$ --- упорядоченная последовательность степеней его вершин $d_1 \le d_2 \le \ldots \le d_{n}$.
\end{definition}
\begin{definition}[]
	Граф $G$ \selectedFont{мажорирует} граф $H$, если $v(G) = v(H)$ и степенная последовательность графа $G$ мажорирует степенную последовательность графа $H$.
\end{definition}
\begin{definition}[]
	Последовательность $a_1 \le a_2 \le \ldots \le a_{n}$ называется \selectedFont{гамильтоновой}, если $a_{n} \le n-1$ и любой граф на $n$ вершинах, степенная последовательность которого мажорирует $a_1, \ldots , a_{n}$ имеет гамильтонов цикл.
\end{definition}

\begin{theorem}[Хватал, 1972]
	Пусть $0 \le a_1 \le a_2 \le \ldots \le a_{n} \le n-1$, $n \ge 3$. Следующие два утверждения равносильны:
	\begin{enumerate}
		\item Последовательность $\{a_i\}$ гамильтонова.
		\item Для каждого $s < \frac{n}{2}$ из $a_s \le s$ следует, что $a_{n-s} \ge n-s$.
	\end{enumerate}
\end{theorem}
\section{Гамильтонов цикл в кубе графа}
\begin{definition}[]
	Для графа $G$ и натурального $d$ обозначим за $G^{d}$ граф на вершинах из $V(G)$, в котором вершины  $x$ и $y$ смежны, согда $ \dist_{G}(x, y) \le d$.
\end{definition}
\begin{theorem}[Хартланд, Капур, 1969]
	Для любого связного графа $G$ с $v(G) \ge 3$ и ребра $e \in E(G)$ в графе $G^{3}$ существует гамильтонов цикл, содержащий ребро $e$.
\end{theorem}

\begin{definition}[]
	\selectedFont{Обхват} графа $G$ ($g(G)$ ) --- длина наименьшего цикла в графе $G$.
\end{definition}

\begin{theorem}[Татт]
	Пусть $k, g, n \in \N$, причем $k, g \ge 3$, $kn \equiv 0 \pmod 2$ и \[
	n > \frac{k(k-1)^{g-1} -2}{k-2}
	.\] 
	Тогда существует регулярный граф $G$ степени $k$ с $g(G) = g$ и $v(G) = n$.
\end{theorem}

\chapter{Паросочетания}
\section{Определения}
\begin{definition}[]
	Множество  вершин $U \subset V(G)$ называется \selectedFont{независимым}, если никакие две его вершины не смежны. Обозначим через $\alpha(G)$ количество вершин в максимальном независимом множестве графа $G$. 
\end{definition}
\begin{definition}[]
	Множество ребер $M \subset E(G)$ называется \selectedFont{паросочетанием}, если никакие два его ребра не имеют общей вершины. Обозначим через $\alpha'(G)$ количество  ребер в максимальном паросочетании графа $G$.
\end{definition}

\begin{definition}[]
	Будем говорить, что множество вершин $W \subset V(G)$ \selectedFont{покрывает} ребро $e \in  E(G)$, если существует вершина $w \in W$, инцидентная $e$. Будем говорить, что множество ребер $F \subset E(G)$ \selectedFont{покрывает} вершину $v \in V(G)$, если существует ребро $f \in  F$, инцидентное $v$.
\end{definition}

\begin{definition}[]
	Паросочетание $M$ графа $G$ называется \selectedFont{совершенным}, если оно покрывает все вершины графа.
\end{definition}

\begin{definition}[]
	Множество вершин $W \subset V(G)$ называется \selectedFont{вершинным покрытием}, если оно покрывает все ребра графа. Обозначим через $\beta(G)$ количество вершин в минимальном вершинном покрытии графа $G$.
\end{definition}

\begin{definition}[]
	Множество ребер $F \subset E(G)$ называется \selectedFont{реберным покрытием}, если оно покрывает все вершины графа. Обозначим через $\beta'(G)$ количество ребер в минимальном реберном покрытии графа $G$.
\end{definition}

\begin{lemma}
    \begin{enumerate}
    	\item  $U \subset V(G)$ --- независимое множество, согда $V(G) \setminus U$ --- вершинное покрытие.
		\item $\alpha(G) + \beta(G) = v(G)$.
    \end{enumerate}
\end{lemma}

\begin{theorem}[Галлаи, 1959]
	Пусть $G$ --- граф с $\delta(G)>0$. Тогда $\alpha'(G) + \beta'(G) = v(G)$.
\end{theorem}

\section{Чередующиеся и дополняющие пути}
\begin{definition}[]
	Пусть $M$ --- паросочетание в графе $G$.
	\begin{enumerate}
		\item Назовем путь \selectedFont{$M$-чередующимся}, если  в нем чередуются ребра из $M$ и ребра не из $M$.
		\item Назовем $M$-чередующийся путь \selectedFont{$M$-дополняющим}, если его начало и конец не покрыты паросочетанием $M$.
	\end{enumerate}
\end{definition}

\begin{theorem}[Берж, 1957]
	Паросочетание $M$ в графе $G$ максимально, согда нет $M$-дополняющих путей.
\end{theorem}

\section{Паросочетания в двудольном графе}
Пусть $G = (V_1, V_2, E)$ --- двудольный граф с долями $V_1$ и $V_2$.
\begin{theorem}[Холл, 1935]
	В двудольном графе $G$ есть  паросочетание, покрывающее все вершины доли $V_1$, согда для любого множества $U \subset V_1$ выполняется $ \lvert U \rvert \le \lvert N_{G}(U) \rvert$.
\end{theorem}

\begin{corollary}[]
	В двудольном графе $G = (V_1, V_2, E)$ все вершины из $V_1$ имеют степени не меньше $k$, а все вершины $V_2$ имеют степени не больше $k$. Тогда есть паросочетание, покрывающее $V_1$.
\end{corollary}

\begin{corollary}[Кенинг, 1916]
	Пусть $G = (V_1, V_2, E)$ --- регулярный двудольный граф степени $k$. Тогда $G$ --- объединение $k$ своих совершенных паросочетаний.
\end{corollary}

\begin{theorem}[Кенинг, 1931]
	Пусть $G$ --- двудольный граф. Тогда $\alpha'(G) = \beta(G)$.
\end{theorem}

\begin{corollary}[]
	Пусть $G$ --- двудольный граф с $\delta(G) >0$. Тогда $\alpha(G) = \beta'(G)$.
\end{corollary}

\section{Паросочетания с предпочтениями}
\begin{definition}[]
	Пусть для каждой вершины $v \in V(G)$ задано линейное отношение (нестрогого) порядка $\le_{v}$ на множестве всех инцидентных  $v$ ребер из $E(G)$. Тогда $\le = \{\le_v\}_{v \in  V(G)}$ --- \selectedFont{множество предпочтений}.
\end{definition}
\begin{definition}[]
	Паросочетание $M$ называется \selectedFont{стабильным} для множества предпочтений $\le$, если для любого ребра $d \not \in M$ существует такое ребро $e \in M$, что $e$ и $f$ имеют общий конец и $f \le _{v} e$.
\end{definition}

\begin{theorem}[Гейл, Шепли, 1962]
	Пусть $G$ --- двудольный граф. Тогда для любого множества предпочтений в графе $G$ существует  стабильное паросочетание.
\end{theorem}

\section{Паросочетания в произвольном графе}
\begin{definition}[]
	Для произвольного графа $G$ обозначим через $o(G)$ количество нечетных компонент связности графа $G$.
\end{definition}
\begin{theorem}[Татт, 1947]
	В графе $G$ существует совершенное паросочетание, согда для любого $S \subset V(G)$ выполняется условия $o(G - S) \le \lvert S \rvert$
\end{theorem}
\section{Совершенное паросочетание в кубическом графе}
\begin{definition}[]
	Граф, все вершины которого имеют степень $3$, называется \selectedFont{кубическим}.
\end{definition}
\begin{definition}[]
	\selectedFont{Мост} графа --- ребро, не входящее ни в один цикл.
\end{definition}

\begin{theorem}[Петерсон, 1891]
	Пусть $G$ --- связный кубический граф, в котором не более двух мостов. Тогда в графе $G$ есть совершенное паросочетание.
\end{theorem}

\begin{theorem}[Плешник, 1972]
	Пусть $G$ --- регулярный граф степени $k$ с четным числом вершин, причем $\lambda(G) \ge k-1$, а граф $G'$ получен из $G$ удалением не более, чем $k-1$ ребер. Тогда в графе $G'$ есть совершенное паросочетание.
\end{theorem}

\begin{corollary}[]
	Пусть $G$ --- регулярный граф степени $k$ с четным числом вершин, причем $\lambda(G) \ge k-1$. Тогда для любого ребра $e \in E(G)$ существует совершенное паросочетание графа $G$, содержащее $e$.
\end{corollary}

\section{Факторы регулярного графа}
\begin{definition}[]
	\selectedFont{$k$-фактор} графа $G$ --- остовный регулярный подграф степени $k$ графа $G$.
\end{definition}

\begin{theorem}[Петерсен, 1891]
	У регулярного графа степени $2k$ есть $2$-фактор.
\end{theorem}
\begin{corollary}[]
	Следующие утверждения:
	\begin{enumerate}
		\item Регулярный граф степени $2k$ есть объединение $k$ своих $2$-факторов.
		\item Для любого $r \le k$ регулярный граф степени $2k$ имеет $2r$-фактор.
	\end{enumerate}
\end{corollary}

\begin{theorem}[Томасен, 1981]
	Пусть $G$ --- граф, степени всех вершин которого равны $k$ или $k+1$, а $r \ge k$. Тогда существует остовный подграф $H$ графа $G$, степени всех вершин которого равны либо $r$, либо $r+1$.
\end{theorem}

\begin{corollary}[Lovasz, 1970]
	Пусть $s, t \in \N$. Тогда любой граф максимальной степени $s + t - 1$ представляется в виде объединения графа максимальной степени не более $t$.
\end{corollary}
