\section{Оптимальные раскраски ребер и их свойства. Xроматический и покрывающий индексы двудольного графа.}

\begin{definition}
     \selectedFont{Хроматический индекс} графа $G$ --- наименьшее натурально число $\chi'(G)$, для которого существует правильная раскраска ребер графа $G$ в такое количество цветов.

	 Назовем раскраску \selectedFont{покрывающей}, если ребра каждого цвета образуют покрытие, то есть покрывают все вершины.

	 \selectedFont{Покрывающий индекс} графа $G$ --- наибольшее натуральное число $\kappa'(G)$, для которого существует покрывающая раскраска ребер графа $G$. 
\end{definition}

\begin{definition}
    Пусть $\rho$ --- раскраска ребер графа $G$ в $k$ цветов. 

	Будем говорить, что в раскраске $\rho$ цвет $i$ \selectedFont{представлен} в вершине $v$, если существует инцидентное $v$ ребро $e$ такое, что $\rho(e) = i$. Обозначим за $\rho(v)$ количество цветов, представленных в вершине $v$.

	Введем обозначение $\rho(G) = \sum_{v \in V(G)}^{} \rho(v)$. Назовем раскраску $\rho$ \selectedFont{$k$-оптимальной}, если для любой другой раскраски $\rho'$ ребер графа $G$ в $k$ цветов $\rho(G) \ge \rho'(G)$.
\end{definition}
Пусть $\rho$ --- правильная раскраска ребер графа $G$ в не более чем  $k$ цветов. Тогда для каждой вершины $v \in V(G)$ имеем $\rho(v) = d_{G}(v) \ge \rho'(v)$  для любой  другой раскраски $\rho'$. Поэтому правильная раскраска всегда $k$-оптимальна.

\begin{lemma}\label{lm:coloring_10}
	Пусть $G$ --- связный граф, отличный от простого цикла нечетной длины. Тогда существует такая раскраска ребер $G$ в два цвета, что в каждой вершине степени не менее двух представлены оба цвета.
\end{lemma}
\begin{proof}
	\begin{itemize}
		\item Если все вершины имеют степень $2$, то это четный цикл, утверждение верно.
		\item Если  графе есть вершины нечетной степени, то добавим новую вершину $w$ и соединим со всеми вершинами нечетной степени. Получим граф со всеми четными степенями $\tilde{G}$.
		\item Если в графе $G$ есть вершины нечетной степени, то положим $a = w$.
		\item Если  все степени четные, то $\tilde{G} = G$, а в качестве  $a$ возьмем вершину степени хотя бы $4$, такая есть, так как $G$ не является четным циклом.
	\end{itemize}
	В графе $\tilde{G}$ есть эйлеров цикл. Покрасим ребра в порядке обхода по нему, начиная с $a$ и чередуя цвета.

	Пусть $x \neq a$. Если $d_G(x) \ge 2$, мы прошли через $x$ не меньше раза, поэтому у $x$ есть два ребра $G$ разных цветов. 

	Если $a = w$, то ничего проверять не нужно. Тогда остался случай, когда $a$ --- вершина степени хотя бы $4$. Тогда есть в $G$ есть два ребра инцидентных $a$ разных цветов.
\end{proof}

Любая раскраска $\rho$ ребер графа в цвета $[1 .. k]$ --- разбиение множества $E(G)$ в объединение непересекающихся множеств $E_1, \ldots , E_k$, где $\rho$ принимает значение $i$ на ребрах $E_i$.
\begin{lemma}\label{lm:coloring_11}
    Пусть $\rho$ --- $k$-оптимальная раскраска ребер графа $G$. Предположим, что вершина  $w$ и цвета $i$ и $j$ таковы, что в вершине $w$ хотя бы два раза представлен цвет $i$ и не представлен цвет  $j$. Пусть $H = G(E_{i} \cap E_j)$, а $H_w$ --- компонента графа $H$, содержащая  вершину $w$. Тогда $H_w$ --- простой цикл нечетной длины.
\end{lemma}
\begin{proof}
    Пусть $H_w$ не является простым циклом нечетной длины.

	Построим новую раскраску $\rho'$, отличающуюся только раскраской ребер $H_w$: раскрасим из в цвета $i$ и $j$ так, чтобы в каждой вершине $x$ степени $d_{H_w}(x) \ge 2$ были представлены оба цвета, это возможно по лемме \ref{lm:coloring_10}.

	Тогда $\rho'(w) \ge \rho(w) + 1$, а для любой другой вершины $x$, очевидно, что $\rho'(x) \ge \rho(x)$. Тогда $\rho'(G) > \rho(G)$, поэтому $\rho$ не может быть $k$-оптимальной. Противоречие. 
\end{proof}
\begin{note}
    Очевидно, что $\chi'(G) \ge \Delta(G)$ : все ребра, инцидентные одной вершине наибольшей степени должны быть разноцветными.
\end{note}
\begin{theorem}[Кениг, 1916]
    Пусть $G$ --- двудольный граф (возможно, с кратными ребрами). Тогда $\chi'(G) = \Delta(G)$.
\end{theorem}
\begin{proof}
    Пусть $\Delta = \Delta(G)$. Рассмотрим $\Delta$-оптимальную раскраску $\rho$ ребер графа $G$.

	Предположим, что $\rho$ --- неправильная. Тогда существует вершина $v$ и цвет $i$ такие, что $i$ дважды представлен в вершине $v$. 

	Так как $d_G(v) \le \Delta$, существует цвет $j$, который не представлен в вершине $v$. По лемме \ref{lm:coloring_11} в $G$ есть нечетный цикл, противоречие.
\end{proof}

\begin{theorem}[Гупта, 1966]
    Если граф $G$ двудольный, то $\kappa'(G) = \delta(G)$.
\end{theorem}
\begin{proof}
    Рассмотрим $\delta(G)$-оптимальную раскраску $\rho$ ребер графа $G$.

	Предположим, что $\rho$ не является покрывающей. Тогда существует вершина $v$ и цвет $i$ такие, что $i$ не представлен в вершине $v$. 

	Так как $d_G(v) \ge  \delta$, существует цвет $j$, который представлен в вершине $v$ дважды. По лемме \ref{lm:coloring_11} в $G$ есть нечетный цикл, противоречие.
\end{proof}
