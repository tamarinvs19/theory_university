\section{Теорема Петерсена о выделении 2-фактора в 2k-регулярном графе и ее следствие о регулярных факторах.}
\begin{definition}
    \selectedFont{$k$-фактором} графа $G$ называется его остовный регулярный подграф степени $k$.
\end{definition}

\begin{theorem}[Петерсон, 1891]
    У регулярного графа степени $2k$ есть 2-фактор.
\end{theorem}

\begin{proof}
    Так как все степени четные, есть эйлеров цикл. Обойдем его в некотором направлении и ориентируем каждое ребро в направлении обхода. Теперь в каждую вершину $\overline{G}$ входит и выходит ровно по $k$ стрелок.

	Построим граф $G^*$ следующим образом: разделим каждую вершину $v \in V(G)$ на две вершины $v_1$ и $v_2$, если ребро $xy \in E(G)$ было ориентировано от $x$ к $y$, то проведем в графе $G^*$ ребро $x_1y_2$.

	Таким образом, существует биекция $\varphi\colon E(G) \to E(G^*)$, заданная правилом $\varphi(xy) = x_1y_2$.

	$G^*$ --- регулярный двудольный граф степени $k$ с долями $\{v_1\}_{v \in V(G)}$ и $\{v_2\}_{v \in  V(G)}$.

	По следствию \ref{cor:matching_2} в графе $G^*$ есть совершенное паросочетание $M^*$.

	Пусть $M = \varphi^{-1}(M^*)$. Для любой вершины $x \in V(G)$ каждая из вершин $x_1, x_2 \in  V(G^*)$ инцидентна ровно одному ребру из $M^*$.

	Поэтому $x$ инцидентна ровно двум ребрам из $M$, то есть $M$ --- 2-фактор графа $G$.
\end{proof}

\begin{corollary}
    \begin{enumerate}
		\item Регулярный граф степени $2k$ есть объединение $k$ своих $2$-факторов.
		\item Для любого $r \le k$ регулярный граф степени $2k$ имеет $2r$-фактор.
    \end{enumerate}
\end{corollary}
