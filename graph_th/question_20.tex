\section{Разделяющие множества в k-связном графе, части разбиения. Внутренность и граница части разбиения.}
Пусть $\S \subset \RR(G)$.
\begin{definition}
    Множество $A \subset V(G)$ --- \selectedFont{часть $\S$-разбиения}, если никакие две вершины из $A$ нельзя разделить никаким множеством из $\S$, но любая другая вершина графа $G$ отделена от множества $A$ хотя бы одним из множеств набора $\S$.

	Множество всех частей разбиения графа $G$ набором разделяющих множеств  $\S$ мы будем обозначать через $\Part(\S)$. Если граф не очевиден  $\Part(G; \S)$.

	Вершину части $A \in \Part(\S)$ назовем \selectedFont{внутренней}, если она не входит ни в одно из множеств набора $\S$. Множество таких вершин --- $\Int(A)$ --- \selectedFont{внутренность} части $A$.

	Вершины, входящие в какие-либо множества из $\S$, будем называть \selectedFont{ граничными}, а все их множество \selectedFont{границей} и обозначать через $\Bound(A)$.
\end{definition}

Внутренняя вершина части $A \in \Part(\S)$ может быть концом ребра , входящего в множество $S \in \S$.

Пусть $A, B \in \Part(\S)$, $A \neq B$, $A \cap B \neq \varnothing$. Тогда существует такое $S \in \S$, что $A \cap B \subset S$.

Разделяющее множество $S \subset V(G)$ в  $k$-связном графе $G$ должно содержать на менее $k$ вершин. Мы обозначим через $\RR_{k}(G)$ множество всех $k$-вершинных разделяющих множеств графа $G$.

Пусть $S \in \RR_{k}(G)$, $A \in \Part(\S)$. Тогда $\Int(A) \neq \varnothing$, $G(\Int(A))$ связен ---  это компонента графа $G - S$. Для любой вершины $x \in S$ существует вершина $y \in \Int(A)$, смежная с $x $ (иначе $S \setminus \{x\}$ отделяет $\Int(A)$ от $G - A$ ).

Однако, если $\S \subset\RR_K(G)$, $B \in \Part(\S)$, то возможно, что $\Int(B) = \varnothing$. Кроме того, при $\Int(B) \neq \varnothing$ индуцированный подграф $G(\Int(B))$ не обязательно связен.

\begin{lemma}\label{lm:connectivity_5}
    Пусть $\S \subset \RR_k(G)$, $A \in \Part(\S)$. Тогда верно:
	\begin{enumerate}
	    \item  Вершина $x \in \Int(A)$ не смежна ни с одной другой из вершин множества $V(G)\setminus A$.
		\item Если $\Int(A) \neq \varnothing$, то $\Bound(A)$ отделяет $\Int(A)$ от $V(G) \setminus A$.
	\end{enumerate}
\end{lemma}
\begin{proof}
    \begin{enumerate}
		\item Пусть вершина $x \in  \Int(A)$ смежна с вершиной $y \in V(G) \setminus A$. Существует множество $S \in \S$, отделяющее $ y$ от $\Int(A)$ в $G$. Тогда $x, y \notin S$, причем они смежны. Противоречие. 
		\item Следует из прошлого пункта.
    \end{enumerate}
\end{proof}

\begin{theorem}
    Пусть $G$ --- $k$-связный граф, $\S, \T \subset\RR_k(G)$.
	\begin{enumerate}
		\item Пусть $A \in  \Part(\S)$. Тогда $\Bound(A)$ --- множество всех вершин части $A$, смежных хотя бы с одной из $V(G) \setminus A$.
		\item Пусть $A \in \Part(\S)$ и $A \in \Part(\T)$. Тогда граница $A$ как части $\Part(\S)$ совпадает с границей $A$ как части $\Part(\T)$.
	\end{enumerate}
\end{theorem}
\begin{proof}
    \begin{enumerate}
		\item Пусть  $x \in \Bound(A)$. Существует такое множество $S \in \S$, что $x \in S$.

			Множество вершин $S$ не разделяет $A$, следовательно $A$ может пересекать внутренность не более чем одной части $\Part(S)$. Тогда существует такая часть $B \in \Part(S)$, что $\Int(B) \cap A = \varnothing$. Тогда существует вершина $y \in \Int(B)$, смежная с $x$.  

			По следствию \ref{cor:connectivity_1} ни одна из вершин множества $\Int(A)$ не может быть смежна с вершиной из $V(G) \setminus A$.
		\item  В первом пункте мы построили $\Bound(A)$ вне зависимости от $\S$ или $\T$, поэтому совпадать с границей обоих будет совпадать.
    \end{enumerate}
\end{proof}
