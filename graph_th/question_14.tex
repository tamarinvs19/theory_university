\section{Теорема Ловаса о разбиении графа.}
\begin{corollary}[Ловас, 1970]
    Пусть $s, t \in \N$. Тогда любой граф максимальной степени $s+t-1$ представляется в виде объединения  графа максимальной степени не более $s$ и графа максимальной степени не более $t$.
\end{corollary}
\begin{proof}
    Пусть $G$ --- граф с $\Delta(G) = s+t-1$. Добавим в граф вершины и ребра, чтобы он стал регулярным степени $k=s+t-1$.

	По теореме Томассена граф  $H$ имеет остовный подграф $H_1$, степени вершин которого равны $t$ или $t-1$.

	Тогда оставшиеся ребра графа $H$ образуют подграф $H_2$, степени вершин которого равны $s-1$ или $s$.

	Теперь удалим из подграфов $H_1$ и $H_2$ добавленные вершины и ребра, получим подграфы $G_1$ и $G_2$ графа $G$ с $\Delta(G_1) \le t$ и $\Delta(G_2) \le s$. При этом $G = G_1 \cup G_2$.
\end{proof}
