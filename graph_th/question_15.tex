\section{Блоки и точки сочленения. Лемма о пересечении блоков.}
Здесь граф $G$ связен.
\begin{definition}
	Вершина $a \in V(G)$ называется \selectedFont{точкой сочленения}, если граф $G-a$ несвязен.
\end{definition}
\begin{definition}
    \selectedFont{Блок} --- максимальный по включению подграф графа $G$.
\end{definition}
\begin{definition}
    Блоки и точки сочленения несвязного графа --- блоки и точки сочленения его компонент.
\end{definition}

\begin{lemma}\label{lm:connectivity_1}
    Пусть $B_1$ и $B_2$ --- два разных блока графа $G$, причем $V(B_1) \cap v(B_2) \neq \varnothing$. Тогда $V(B_1) \cap V(B_2)$ состоит из одной точки сочленения $a$ графа $G$, причем $a$ --- единственная  точка сочленения, отделяющая $B_1$ от $B_2$.
\end{lemma}
\begin{proof}
    \begin{description}
		\item[Единственность] Пусть $\lvert V(B_1) \cap V(B_2) \rvert \ge 2$. Тогда для любой вершины $x \in  V(B_1 \cup B_2)$ граф $B_1 \cup B_2 - x$ связен, так как $B_1 - x$ связен, $B_2-x$ связен, плюс остается хотя бы одна общая вершина. Следовательно, $B_1 \cup B_2$ содержится в блоке $B$ графа $G$, но тогда $B_1$ и $B_2$ не максимальные по включению.

			Пусть $V(B_1) \cap V(B_2) = \{a\}$. Так как $a $ --- общая вершина блоков $B_1$ и $B_2$, отделять $B_{1}$ от $B_2$ в графе $G$ может только $a$.
		\item[Точка сочленения] Если $a$ не отделяет $B_1$ от $B_2$, в графе $G$ должен быть $V(B_1)V(B_2)$-путь $P$.
			
			Пусть $H = B_1 \cup B_2 \cup P$. Граф $H-x$ связен для любой вершины $x \in V(H)$. Поэтому $H$ содержится в одном блоке графа $G$. Но блок $B_1$  --- его собственный подграф. Противоречие. 

		В итоге $a$ --- единственная вершина, отделяющая $B_1$ от $B_2$, следовательно, граф $G - a$ несвязен, поэтому $a$ --- точка сочленения.
    \end{description}
\end{proof}
