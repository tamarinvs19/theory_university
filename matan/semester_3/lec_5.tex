\lecture{5}{30 Sept}{\dag}
\begin{thm}[О сосхранении измеримости при гладком отображении]\label{th:save_measuring}
	Пусть $ G \subset \R^{m}$ и $ G$ --- открытое,  $ C^{1}(G) \ni \Phi\colon G \to \R^{m} $ --- гладкая функция на $ G$.
	Тогда
	\begin{enumerate}[noitemsep,label=(\arabic*),noitemsep]
		\item  если  $ e \subset G$ и $ \lambda (e) = 0$, то $ \Phi(e) \in \Omega _{m}$ и $ \lambda (\Phi(e)) = 0$ ;
		\item если $ E \subset G$ и $ E \in \Omega _{m}$, то $ \Phi(E) \in \Omega _{m}$,
	\end{enumerate}
	где $ \Omega _{m}$ --- семейство измеримых по Лебегу множеств.
\end{thm}
\begin{proof}
	~\begin{description}
		\item \boxed{ 1 \Longrightarrow 2}
			Представим $ E = e \cup \bigcap_{n=1}^{\infty} K_n$, где $ K_n$ --- компактны и $ \lambda (e) = 0$.
			Так как $ \Phi$ гладкая, она переводит компакт в компакт.
			\[
				\Phi(E)  =\Phi(e) \cap \bigcap_{n=1}^{\infty} \Phi(K_n)
			.\]
			$ \bigsqcup\limits_{n=1}^{\infty} \Phi(K_n)$ --- объединение замкнутых, следовательно, само является замкнутым, а поэтому измеримо по Лебегу.
			По первому пункту $ \Phi(e) $ измеримо. А тогда и $ \Phi(E)$ измеримо.
		\item \boxed{ 1} Пусть $ \lambda (e) = 0$.
			\begin{enumerate}
				\item Рассмотрим случай, когда $ e$ входит в $ G$ внутри некоторой ячейки.
					$ e \subset P\subset \overline{P} \subset G$, где $ P$ --- ячейка, поэтому $ \overline{P}$ --- компакт.

					По теореме о конечном приращении $ \Phi\!\!\bigm|_{\overline{P}}$ --- липшицево, то есть\footnote{Здесь под нормой подразумевается евклидова норма, то есть просто модуль. В качестве константы $ c$ достаточно взять максимум дифференциала на $ \overline{P}$.}
					\[
						\exists c\colon \forall x, y \in \overline{P} \quad\| \Phi(x) - \Phi(y) \| \le  c \cdot \| x - y \|
					.\]
					Воспользуемся регулярностью меры Лебега: зафиксируем $ \varepsilon >0$. Тогда существует открытое $ g$, что
					\[
						e \subset g \subset G \text{ и } \lambda (g) < \varepsilon
					.\]
					Теперь представим $ g$ в виде объединения дизъюнктных ячеек: $ g = \bigsqcup\limits_{j=1}^{\infty} Q_j$.
					Обозначим ребро ячейки $ Q_j$ за $ h_j$ и перепишем условие  $ \lambda (g) < \varepsilon $, используя счетную аддитивность:
					\[
						\lambda (g) = \sum_{j=1}^{\infty} h_j^{m} < \varepsilon
					.\]
					Заметим, что $ \diam Q_j = h_j \sqrt{ m} $.
					Так как $ \Phi $ липшицево, $ \diam \Phi (Q_j) \le c \cdot h_j \sqrt{ m} $.
					Тогда $ \Phi (Q_j)$ можно погрузить в ячейку:
					\[
						\begin{aligned}
						&\Phi (Q_j) \subset Q_j, \quad h_j' \le 2ch_j \sqrt{ m} \\
						& \lambda (Q_j') \le \underbrace{(2c \sqrt{ m} )^{m} }_{\text{не зависит от номера ячейки}}(h_j)^{m}
						\end{aligned}
					\]
					Мы знаем, что
					\[
						\Phi (e) \subset \Phi (g) \subset \bigsqcup_{j=1}^{\infty} \Phi (Q_j) \subset \bigsqcup_{j=1}^{\infty} Q_j'
					.\]
					При этом
					\[
						\sum_{j=1}^{\infty} \lambda (Q_j') \le (2c \sqrt{ m} )^{m} \cdot \sum_{j=1}^{\infty} (h_j)^{m} < (2c \sqrt{ m} )^{m} \cdot \varepsilon
					.\]
					Следовательно, $ \Phi (e)$ можно покрыть множеством сколь угодно малой меры. Тогда $ \Phi (e)$ измеримо и $ \lambda (\Phi (e)) = 0$.
					\begin{figure}[ht]
						\centering
						\incfig{safe-measure}
						\caption{Отображение $ Q_j$}
						\label{fig:safe-measure}
					\end{figure}
				\item Если $ e$ не помещается в диадическую ячейку: $ e \subset G$. Так как $ G$ открыто, каждая точка входит с некоторой окрестностью $ B$, для можно найти некоторою  ячейку $ P_j$, что точка принадлежит $ P_j$, а $ \overline{P_j}\subset B \subset G$.

					Поэтому
					\[
						G = \bigcup_{j=1}^{\infty} P_j, \quad P_j \text{ --- ячейки}, ~ \overline{P_j} \subset G
					.\]
					Тогда можем рассмотреть $ e_j = e \cap P_j \subset P_j \subset G$. В этом случае $ \lambda (e \cap P_j) \le  \lambda (e) = 0$ и $ e \cap P_j \subset P_j \subset  \overline{P_j} \subset G$.

					Можно применить первый пункт и получить, что
					\[
						\Phi (e \cap P_j) \text{ измеримо и }  \lambda (\Phi (e \cap P_j)) = 0
					.\]
					Просуммируем:
					$
					e = \bigsqcup_{j=1}^{\infty} (e \cap P_j)
					$ и
					\[
						\Phi (e) = \bigcup_{j=1}^{\infty} \Phi (e \cap P_j) \Longrightarrow \lambda (\Phi (e)) \le  \sum_{j=1}^{\infty} \lambda (\Phi (e \cap P_j)) = 0
					.\]
			\end{enumerate}
	\end{description}
\end{proof}
\begin{rem}
	На самом деле гладкость не нужна, а достаточно локальной липшицевости.
\end{rem}
\begin{rem}
	Подпространство меньшей размерности, чем объемлющая, имеет меру нуль.
\end{rem}
\begin{cor}
	Возьмем открытое $ G \subset \R^{m} $ и функцию $ f\colon G \to \R$, $ f \in C^{1}(G)$. \[
		\Gamma _{f} = \{(x, f(x) \mid x \in G\} \subset \R^{m+1}
	.\]
	Тогда $ \lambda _{m+1}(\Gamma _f) = 0$.
\end{cor}
\begin{proof}
	Рассмотрим множество $ G\times \{0\} $. $ \lambda _{m+1}(G \times \{0\}) = 0$, так как для любого покрытия параллелепипедами можно сколь угодно уменьшить объем (уменьшаем последнюю сторону, а это можно делать, так как покрыть нужно точку~$ 0$). Теперь возьмем $ \Phi (x, 0) = (x, f(x))$.
	По доказанной теореме $ \lambda _{m+1}(\Gamma _{f}) = 0$.
\end{proof}
\begin{probl}
	Измеримость не сохраняется при непрерывном отображении. Привести пример. \textit{Намек на решение: канторова лестница.}
\end{probl}

\section{Инвариантность меры Лебега при движении}
\begin{thm}[Инвариантность при сдвиге]
	Рассмотрим $ v \in  \R^{m} $ и $ E \in \mathfrak{A}_{m}$. Тогда
	$ v + E = \{v + x \mid x \in E\}$ тоже измеримо и $ \lambda (v + E) = \lambda (E)$.
\end{thm}
\begin{proof}
	Так как сдвиг --- липшицево отображение с коэффициентом $ 1$, по прошлой теореме \ref{th:save_measuring},  $ v + E$ измеримо.

	Определим $ \mu(E) = \lambda (v + E)$. $ \mu$ --- мера на $ \A$, так как множества измеримы по $ \lambda $, согда они измеримы по $ \mu$.

	Тогда для ячейки $ P \in \P_m$ верно $ \mu(P) = \lambda (v + P) = \lambda (P)$.

	Так как $ \lambda $ --- стандартное продолжение объема на ячейках, а $ \mu$ --- продолжение Каратеодори, причем они совпадают на полукольце, то по единственности стандартного продолжения они совпадают на $ \A_{m}$. Следовательно, $ \lambda (E) = \lambda (v + E)$.
\end{proof}

\begin{thm}
	Пусть $ \mu$ --- инвариантная относительно сдвига (то есть $ \forall E \in \A_{m} ~ \forall v \colon \mu(v + E) = \mu(E)$) мера на  $ \A_{m}$.

	Дополнительно потребуем, что $ \mu$ конечна на всех ограниченных измеряемых множествах (достаточно потребовать для ячеек).

	Тогда существует такое $ k \in [0, +\infty)$, что $ \mu = k \lambda $, то есть 
	\[
		\forall E \in \A_m \colon \mu(E) = k \lambda (E)
	.\] 
\end{thm}
\begin{proof}
	Рассмотрим $ Q = [0, 1]^{m}$. Пусть $ k = \mu(Q)$ и $ \widetilde{ \mu} = \frac{\mu}{k}$.
	Тогда $ \widetilde{ \mu} (Q) = 1 = \lambda (Q)$. 

	Так как $ \widetilde{ \mu} $ инвариантно относительно сдвигов, $ \widetilde{ \mu}  = \lambda $ на диадических ячейках $ \P_{m}^{d}$ (для $ \P_{m}^{1}$ уже знаем, для большего $ d$ можем раздробить ячейку на меньшие и сдвинуть туда меньшую).

	Следовательно, $ \widetilde{ \mu}  = \lambda $ на $ \A_m$.
\end{proof}

\begin{thm}[Об инвариантности относительно вращения]
	Пусть $ U\colon \R^{m} \to \R^{m} $ --- ортогональное преобразование. 
	Тогда, если $ E$ измеримо, то $ U(E)$ тоже измеримо, причем $ \lambda (U(E)) = \lambda (E)$.
\end{thm}
\begin{proof}
	Так как $ U$ липшицево, $ U(E)$ измеримо. Пусть $ \mu(E) = \lambda (U(E))$ --- тоже мера на $ \Am$ (все аксиомы просто наследуются). Притом $ \mu$ точно конечна на всех ограниченных множествах.

	Проверим инвариантность относительно сдвига
	\[
		v \in \R^{m} \colon \mu(v + E) = \lambda (U(v + E)) = \lambda (Uv + U(E)) = \lambda (U(E)) = \mu(E)
	.\] 
	Тогда существует такое $ k$, что $ \mu = k\cdot \lambda $.

	Но на единичном шаре $ B$ 
	\[
		\mu(B) = \lambda (U(B)) = \lambda (B) \Longrightarrow k = 1
	.\] 
	Получаем, что $ \lambda $ инвариантно относительно поворота.
\end{proof}
\begin{cor}
    Мера Лебега инвариантна относительно движений.
\end{cor}
\begin{prac}
	~\begin{enumerate}[noitemsep]
        \item Как меняется мера при проекции?
		\item Как меняется мера при других преобразованиях, например, гомотетии?
    \end{enumerate} 
\end{prac}

\section{Изменение меры Лебега при линейном отображении}
\begin{lm}
	Пусть  $ L\colon \R^{m} \to \R^{m} $. Тогда существуют ортонормированные базисы $ \{g_j\}_{j=1}^{m}$ и $ \{e_j\}_{j=1}^{m}$ и $ s_j > 0$ такие, что
	 \[
	Lx = \sum_{j=1}^{m} s_j \langle x, g_j \rangle e_j \quad \forall x \in \R^{m} 
	,\] 
	при этом  $ \lvert \det L \rvert = \prod_{j=1}^{m} s_j$.
	Это называется \textsf{полярным разложением оператора}.
\end{lm}
\begin{proof}
	Рассмотрим $ L^{*}$ --- сопряженный оператор и $ A = L L^{*}$ --- самосопряженный оператор. Так как $ L$ определено на $ \R^{m} $, матрица $ A$ будет вещественной, а поэтому еще и симметричной. У симметричной матрицы можем взять ортонормированный базис из собственных векторов $ \{g_j\}_{j=1}^{m}$. 

	Пусть $ g_j$ --- собственный вектор собственного числа $  \lambda _j$.  
	\[
		\lambda _j \underbrace{\langle g_j, g_j \rangle}_{=1} = \langle Ag_j , g_j \rangle = \langle LL^{*}g_j, g_j \rangle = \langle L^{*}g_j, L^{*}g_j \rangle \ge 0 \Longrightarrow \lambda _j \ge 0
	.\] 
	Пусть $ s_j = \sqrt{ \lambda _j} $. Тогда $ x \in \R^{m}  $ представим в виде $ \sum_{j=1}^{m} \langle x, g_j \rangle g_j$. Тогда
	\[
		Lx = \sum_{j=1}^{m} \langle x, g_j \rangle \underbrace{Lg_j}_{s_j e_j}
	.\] 
	Поэтому $ \{e_j\} = \{\frac{Lg_j}{s_j}\}$ --- базис. Докажем, что он ортонормированный.
	\[
	\begin{aligned}
		s_ks_n \langle e_k , e_n \rangle &= \langle Le_k, Le_n \rangle = \langle L^{*} L g_k, g_n \rangle = \langle \lambda _kg_k, g_n \rangle s_{k}^2\delta _{k, n} \\
		&&\delta _{k, n} = 
		\begin{cases}
			0 & k\ne n\\
			1 & k = n
		\end{cases}
	\end{aligned}
	\]
	Докажем утверждение про определитель:
	\[
		\det A = \prod_{j=1}^{m} \lambda_{j} = \left( \prod_{j=1}^{m}s_j \right) ^2
	.\] 
	С другой стороны,
	\[
		\det  A = \det L \cdot \det L^{*}  =(\det L)^2
	.\] 
\end{proof}
