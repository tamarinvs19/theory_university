\lecture{5}{30 Sept}{\dag}
\begin{thm}[О сосхранении измеримости при гладком отображении]
    Пусть $ G \subset \R^{m}$ и $ G$ --- открытое,  $ C^{1}(G) \ni \Phi\colon G \to \R^{m} $ --- гладкая функция на $ G$.
    Тогда
     \begin{enumerate}[noitemsep,label=(\arabic*),noitemsep]
	 \item  если  $ e \subset G$ и $ \lambda (e) = 0$, то $ \Phi(e) \in \Omega _{m}$ и $ \lambda (\Phi(e)) = 0$ ;
	 \item если $ E \subset G$ и $ E \in \Omega _{m}$, то $ \Phi(E) \in \Omega _{m}$,
    \end{enumerate} 
    где $ \Omega _{m}$ --- семейство измеримых по Лебегу множеств.
\end{thm}
\begin{proof}
    $ $
    \begin{description}
        \item \boxed{ 1 \Longrightarrow 2} 
			Представим $ E = e \cup \bigcap_{n=1}^{\infty} K_n$, где $ K_n$ --- компактны и $ \lambda (e) = 0$.
			Так как $ \Phi$ гладкая, она переводит компакт в компакт.
			\[
				\Phi(E)  =\Phi(e) \cap \bigcap_{n=1}^{\infty} \Phi(K_n)
			.\] 
			$ \bigsqcup\limits_{n=1}^{\infty} \Phi(K_n)$ --- объединение замкнутых, следовательно, само является замкнутым, а поэтому измеримо по Лебегу.
			По первому пункту $ \Phi(e) $ измеримо. А тогда и $ \Phi(E)$ измеримо.
		\item \boxed{ 1} Пусть $ \lambda (e) = 0$.
			\begin{enumerate}
			    \item Рассмотрим случай, когда $ e$ входит в $ G$ внутри некоторой ячейки.
					$ e \subset P\subset \overline{P} \subset G$, где $ P$ --- ячейка, поэтому $ \overline{P}$ --- компакт.

					По теореме о конечном приращении $ \Phi\!\!\bigm|_{\overline{P}}$ --- липшицево, то есть\footnote{Здесь под нормой подразумевается евклидова норма, то есть просто модуль. В качестве константы $ c$ достаточно взять максимум дифференциала на $ \overline{P}$.}
					\[
						\exists c\colon \forall x, y \in \overline{P} \quad\| \Phi(x) - \Phi(y) \| \le  c \cdot \| x - y \| 
					.\] 
					Воспользуемся регулярностью меры Лебега: зафиксируем $ \varepsilon >0$. Тогда существует открытое $ g$, что
					\[
						e \subset g \subset G \text{ и } \lambda (g) < \varepsilon 
					.\] 
					Теперь представим $ g$ в виде объединения дизъюнктных ячеек: $ g = \bigsqcup\limits_{j=1}^{\infty} Q_j$.
					Обозначим ребро ячейки $ Q_j$ за $ h_j$ и перепишем условие  $ \lambda (g) < \varepsilon $, используя счетную аддитивность:
					\[
						\lambda (g) = \sum_{j=1}^{\infty} h_j^{m} < \varepsilon 
					.\] 
					Заметим, что $ \diam Q_j = h_j \sqrt{ m} $.
					Так как $ \Phi $ липшицево, $ \diam \Phi (Q_j) \le c \cdot h_j \sqrt{ m} $.
					Тогда $ \Phi (Q_j)$ можно погрузить в ячейку:
					\[
					\begin{aligned}
						&\Phi (Q_j) \subset Q_j, \quad h_j' \le 2ch_j \sqrt{ m} \\
						& \lambda (Q_j') \le \underbrace{(2c \sqrt{ m} )^{m} }_{\text{не зависит от номера ячейки}}(h_j)^{m}
					\end{aligned}
					\]
					Мы знаем, что
					\[
						\Phi (e) \subset \Phi (g) \subset \bigsqcup_{j=1}^{\infty} \Phi (Q_j) \subset \bigsqcup_{j=1}^{\infty} Q_j'
					.\] 
					При этом
					\[
						\sum_{j=1}^{\infty} \lambda (Q_j') \le (2c \sqrt{ m} )^{m} \cdot \sum_{j=1}^{\infty} (h_j)^{m} < (2c \sqrt{ m} )^{m} \cdot \varepsilon 
					.\] 
					Следовательно, $ \Phi (e)$ можно покрыть множеством сколь угодно малой меры. Тогда $ \Phi (e)$ измеримо и $ \lambda (\Phi (e)) = 0$.
					\begin{figure}[ht]
						\centering
						\incfig{safe-measure}
						\caption{Отображение $ Q_j$}
						\label{fig:safe-measure}
					\end{figure}
					\item Если $ e$ не помещается в диадическую ячейку: $ e \subset G$. Так как $ G$ открыто, каждая точка входит с некоторой окрестностью $ B$, для можно найти некоторою  ячейку $ P_j$, что точка принадлежит $ P_j$, а $ \overline{P_j}\subset B \subset G$.

						Поэтому 
						\[
						G = \bigcup_{j=1}^{\infty} P_j, \quad P_j \text{ --- ячейки}, ~ \overline{P_j} \subset G
						.\] 
						Тогда можем рассмотреть $ e_j = e \cap P_j \subset P_j \subset G$. В этом случае $ \lambda (e \cap P_j) \le  \lambda (e) = 0$ и $ e \cap P_j \subset P_j \subset  \overline{P_j} \subset G$. 

						Можно применить первый пункт и получить, что
						\[
							\Phi (e \cap P_j) \text{ измеримо и }  \lambda (\Phi (e \cap P_j)) = 0
						.\] 
						Просуммируем:
						$
							e = \bigsqcup_{j=1}^{\infty} (e \cap P_j) 
							$ и 
							\[
								\Phi (e) = \bigcup_{j=1}^{\infty} \Phi (e \cap P_j) \Longrightarrow \lambda (\Phi (e)) \le  \sum_{j=1}^{\infty} \lambda (\Phi (e \cap P_j)) = 0
							.\] 
			\end{enumerate} 
    \end{description} 
\end{proof}

