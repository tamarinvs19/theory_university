\documentclass[11pt,dvipsnames]{report}
\usepackage[english, russian]{babel}
\usepackage{xltxtra}
\usepackage{polyglossia}

\usepackage{mathpazo}

\defaultfontfeatures{Ligatures=TeX,Mapping=tex-text}

\setmainfont{STIX2Text-Regular.otf}[
ExternalLocation={/home/vyacheslav/builds/STIXv2.0.2/OTF/},
BoldFont=STIX2Text-Bold.otf,
ItalicFont=STIX2Text-Italic.otf,
BoldItalicFont=STIX2Text-BoldItalic.otf
]
\setmathrm{STIX2Math.otf}[
ExternalLocation={/home/vyacheslav/builds/STIXv2.0.2/OTF/}
]


\usepackage{makeidx}
\usepackage{amssymb, amsthm}
\usepackage{amsmath}
\usepackage{mathtools}
\usepackage{needspace}
\usepackage{enumitem}
\usepackage{cancel}
\usepackage{fdsymbol}
\usepackage{fontawesome}


% разметка страницы и колонтитул
\usepackage[left=2cm,right=2cm,top=1cm,bottom=1.1cm,bindingoffset=0cm]{geometry}
\usepackage{fancybox,fancyhdr}
\fancyhf{}
\fancyhead[R]{\thepage}
\fancyhead[L]{\rightmark}
\fancyfoot{}
\fancyhfoffset{0pt}
\addtolength{\headheight}{13pt}
\pagestyle{fancy}

% Отступы
\setlength{\parindent}{3ex}
\setlength{\parskip}{3pt}

\usepackage{graphicx}
\usepackage{hyperref}

\usepackage{import}
\usepackage{xifthen}
\usepackage{pdfpages}

\newcommand{\incfig}[1]{%
    \def\svgwidth{\columnwidth}
    \import{./figures/}{#1.pdf_tex}
}


\usepackage{xifthen}
\makeatother
\def\@lecture{}%
\newcommand{\lecture}[3]{
    \ifthenelse{\isempty{#3}}{%
        \def\@lecture{Лекция #1}%
    }{%
        \def\@lecture{Лекция #1: #3}%
    }%
    \subsection*{\@lecture}
    \marginpar{\small\textsf{\mbox{#2}}}
}
\makeatletter


\usepackage{xcolor}
\definecolor{Aquamarine}{cmyk}{50, 0, 17, 100}
\definecolor{ForestGreen}{cmyk}{76, 0, 76, 45}
\definecolor{Pink}{cmyk}{0, 100, 0, 0}
\definecolor{Cyan}{cmyk}{56, 0, 0, 100}
\definecolor{Gray}{gray}{0.3}


\usepackage{mdframed}
\mdfsetup{skipabove=3pt,skipbelow=3pt}
\mdfdefinestyle{defstyle}{%
    linecolor=red,
	linewidth=3pt,rightline=false,topline=false,bottomline=false,%
    frametitlerule=false,%
    frametitlebackgroundcolor=red!0,%
    innertopmargin=4pt,innerbottommargin=4pt,innerleftmargin=7pt
    frametitlebelowskip=1pt,
    frametitleaboveskip=3pt,
}
\mdfdefinestyle{thmstyle}{%
    linecolor=cyan!100,
	linewidth=2pt,topline=false,bottomline=false,%
    frametitlerule=false,%
    frametitlebackgroundcolor=cyan!20,%
    innertopmargin=4pt,innerbottommargin=4pt,
    frametitlebelowskip=1pt,
    frametitleaboveskip=3pt,
}
\theoremstyle{definition}
\mdtheorem[style=defstyle]{defn}{Определение}

\newmdtheoremenv[nobreak=true,backgroundcolor=Aquamarine!10,linewidth=0pt,innertopmargin=0pt,innerbottommargin=7pt]{cor}{Следствие}
\newmdtheoremenv[nobreak=true,backgroundcolor=CarnationPink!20,linewidth=0pt,innertopmargin=0pt,innerbottommargin=7pt]{desc}{Описание}
\newmdtheoremenv[nobreak=true,backgroundcolor=Gray!10,linewidth=0pt,innertopmargin=0pt,innerbottommargin=7pt,font={\small}]{ex}{Пример}
\newmdtheoremenv[nobreak=false,backgroundcolor=Cyan!10,linewidth=0pt,innertopmargin=0pt,innerbottommargin=7pt]{thm}{Теорема}
\newmdtheoremenv[nobreak=true,backgroundcolor=Pink!10,linewidth=0pt,innertopmargin=0pt,innerbottommargin=7pt]{lm}{Лемма}

\newtheorem*{st}{Утверждение}
\newtheorem*{prop}{Свойства}

\theoremstyle{plain}
\newtheorem*{name}{Обозначение}

\theoremstyle{remark}
\newtheorem*{rem}{Ремарка}
\newtheorem*{com}{Комментарий}
\newtheorem*{note}{Замечание}
\newtheorem*{prac}{Упражнение}
\newtheorem*{probl}{Задача}


\renewcommand{\proofname}{Доказательство}
\renewenvironment{proof}
{ \hspace{\stretch{1}}\\ \faSquareO\quad \small  }
{ \hspace{\stretch{1}}  \faSquare \normalsize }


\numberwithin{ex}{section}
\numberwithin{thm}{section}
\numberwithin{equation}{section}



\newcommand{\K}{\mathcal{K}}
\newcommand{\Z}{\mathbb{Z}}
\newcommand{\N}{\mathbb{N}}
\newcommand{\Real}{\mathbb{R}}
\newcommand{\Q}{\mathbb{Q}}
\newcommand{\Cm}{\mathbb{C}}
\newcommand{\Pm}{\mathbb{P}}
\newcommand{\ord}{\operatorname{ord}}
\newcommand{\lcm}{\operatorname{lcm}}
\newcommand{\sign}{\operatorname{sign}}
\newcommand{\E}{\mathbb{E}}

\renewcommand{\o}{o}
\renewcommand{\O}{\mathcal{O}}
\renewcommand{\le}{\leqslant}
\renewcommand{\ge}{\geqslant}

\def\mybf#1{\textbf{#1}}
\def\selectedFont#1{\textbf{#1}}
\def\ComplexityFont#1{\textmd{\textbf{\textsf{#1}}}}
\def\LanguageFont#1{{\textbf{\texttt{#1}}}}


\newcommand{\Cclass}{\mathcal{C}}
\newcommand{\Dclass}{\mathcal{D}}


\renewcommand{\P}{\ComplexityFont{P}}
\newcommand{\DTIME}{\ComplexityFont{DTime}}
\newcommand{\DTime}{\ComplexityFont{DTime}}
\newcommand{\DSpace}{\ComplexityFont{DSpace}}
\newcommand{\PSPACE}{\ComplexityFont{PSPACE}}
\newcommand{\NTIME}{\ComplexityFont{NTime}}
\newcommand{\NSpace}{\ComplexityFont{NSpace}}
\newcommand{\coNSpace}{\ComplexityFont{coNSpace}}
\newcommand{\NPSPACE}{\ComplexityFont{NPSPACE}}
\newcommand{\poly}{\ComplexityFont{poly}}
\newcommand{\RP}{\ComplexityFont{RP}}
\newcommand{\coRP}{\ComplexityFont{co-RP}}
\newcommand{\ZPP}{\ComplexityFont{ZPP}}
\newcommand{\BPP}{\ComplexityFont{BPP}}
\newcommand{\BQP}{\ComplexityFont{BQP}}
\newcommand{\coBPP}{\ComplexityFont{co-BPP}}
\newcommand{\NP}{\ComplexityFont{NP}}
\newcommand{\NL}{\ComplexityFont{NL}}
\newcommand{\coNL}{\ComplexityFont{co-NL}}
\renewcommand{\L}{\ComplexityFont{L}}
\newcommand{\NPcomp}{\ComplexityFont{NP-complete}}
\newcommand{\tP}{\widetilde{\P}}
\newcommand{\tNP}{\widetilde{\NP}}
\newcommand{\tBH}{\widetilde{\BH}}
\newcommand{\Class}{{\ComplexityFont{C}}}
\newcommand{\coC}{\ComplexityFont{co-}\mathcal{C}}
\newcommand{\coNP}{\ComplexityFont{co-NP}}
\newcommand{\PH}{\ComplexityFont{PH}}
\newcommand{\EXP}{\ComplexityFont{EXP}}
\newcommand{\Size}{\ComplexityFont{Size}}
\newcommand{\Ppoly}{\ComplexityFont{P}/\ComplexityFont{poly}}
\newcommand{\NC}{\ComplexityFont{NC}}


\newcommand{\FACTOR}{\LanguageFont{FACTOR}}
\newcommand{\kQBF}{{\LanguageFont{QBF{\tiny k}}}}
\newcommand{\QBFk}{{\LanguageFont{QBF{\tiny k}}}}
\newcommand{\QBF}{{\LanguageFont{QBF}}}
\newcommand{\STCON}{\LanguageFont{STCON}}
\newcommand{\USTCON}{\LanguageFont{USTCON}}
\newcommand{\CircuitSat}{{\LanguageFont{CIRCUIT\_SAT}}}
\newcommand{\tCircuitSat}{\widetilde{{\LanguageFont{CIRCUIT\_SAT}}}}
\newcommand{\SAT}{\LanguageFont{SAT}}
\newcommand{\tSAT}{\widetilde{{\LanguageFont{SAT}}}}
\newcommand{\UNSAT}{{\LanguageFont{UNSAT}}}
\newcommand{\tThreeSAT}{\widetilde{{\LanguageFont{3\text{-}SAT}}}}
\newcommand{\ThreeSAT}{{\LanguageFont{3\text{-}SAT}}}
\newcommand{\BH}{\LanguageFont{BH}}
\newcommand{\CircuitEval}{{\LanguageFont{CIRCUIT\_EVAL}}}


\newcommand{\const}{\textmd{const}}
\newcommand{\logspace}{\textmd{logspace}}
\newcommand{\PATH}{\textmd{PATH}}


\newcommand{\readonly}{\textsf{read-only}}
\newcommand{\writeonly}{\textsf{write-only}}


\usepackage{ upgreek }
\newcommand{\PI}{\Uppi}
\newcommand{\SIGMA}{\Upsigma}
\newcommand{\DELTA}{\Updelta}

\begin{document}

\begin{ex}
    Рассмотрим функции $ f_n(x) = x^{n}$ на отрезке $ (0, 1)$. Так как $ \forall x \in  (0, 1)\colon  x^{n} \mathrel{\rightarrow}_{n \to  \infty} 0$, $ f_n \to  f \equiv 0$. Но $ f_n \not\rightrightarrows 0$, потому что, например, для  $ \varepsilon = \frac{1}{2}$ каким бы ни было $ N$ для всех $ n > N$ можно взять такое $ x$ рядом с единицей, что  $ \lvert x^{n}-0 \rvert > \frac{1}{2}$.
\end{ex}

\begin{st}
    $ f_n \rightrightarrows f$ на $ E$ равносильно тому, что
     \[
	 \sup_{x \in E} \lvert f_n(x)-f(x) \rvert \stackrel{n \to  \infty} \longrightarrow 0
    .\] 
\end{st}
\begin{rem}
    Если мы смотрим на множество непрерывных функций на компакте $ C(K)$, где норма 
    \[
	\| f \| _{C(K)} = \max_{x \in  K} \lvert f(x) \rvert 
    ,\] 
    то из поточечной сходимости следует равномерная:
    \[
    f_{n} \to  f \Longrightarrow \| f_n - f \| \to  0 \Longleftrightarrow f_n \rightrightarrows f \text{ на }  K
    .\] 
    Аналогично будет с множеством ограниченных функций на $ E$ ($ l^{\infty}(E)$) с нормой 
    \[
	\| f \| _{\infty}  = \sup_{x \in E}\lvert f(x) \rvert 
    .\] 
\end{rem}

\begin{defn}[Равномерная ограниченность]
    Последовательность функций $ f_n \colon E \to \R (\Cm)$ называется {\sf равномерно ограниченной на $ E$}, если существует такое $ M$, что
    \[
	\forall x \in E ~ \forall n \in \N \colon \lvert f_n(x) \rvert \le M
    .\] 
\end{defn}

\begin{ex}
    Пусть $ f_n \in C(K)$. Тогда равномерная ограниченность $ \{f_n\}$ равносильна ограниченности по норме, то есть все функции содержатся в некотором шаре с центром в нуле.
\end{ex}

\begin{prop}
    \begin{enumerate}
	\item Из равномерной сходимости следует поточечная
	\item Если для всех $ x \in E$ выполнено 
	    $$ \lvert f_n(x) - f(x) \rvert \le a_n,$$
	    где $ \{a_n\}$ --- последовательность, стремящаяся к нулю при $ n \to  \infty$, то $ f_n$ равномерно сходится к $ f$ на $ E$. 
	\item Если существует $ \varepsilon_0$ и $ x_n \in E$ для всех $ n$ такие, что
	     \[
		 \lvert f_n(x_n) - f(x_n) \rvert \ge \varepsilon_0
	    ,\] 
	    то $ f_n$ не сходится равномерно к $ f$ на $ E$.
	\item Пусть $ \{f_n\} \rightrightarrows f$ на $ E$ и $ \{g_n\}$ равномерно ограничена на $E $. Тогда $ f_ng_n \rightrightarrows 0$.
	    \begin{proof}
	        \[
		    \sup_{x \in E}\lvert f_n(x)g_n(x) \rvert \le M_{g_n} \cdot \underbrace{\sup_{x \in  E}\lvert f_n(x) \rvert }_{ \to 0} \stackrel{n \to \infty} \longrightarrow 0
	        .\] 
	    \end{proof}
	\item {\bf Критерий Коши}. Пусть $ f_n \colon E \to  \R(\Cm)$. $ f_n$ равномерно сходится на $ E$, согда\footnote{С этого момента буду писать <<согда>> вместо <<тогда и только тогда, когда>>, чтобы упростить формулировки} для любого положительного $ \varepsilon $ существует $ N$, что
	    \[
		\forall n, m > N ~ \forall x \in  E \colon \lvert f_n(x) - f_m(x) \rvert < \varepsilon 
	    .\] 
	    \begin{proof}
	        $ $
	        \begin{description}
	            \item \boxed{ 1 \Longrightarrow 2} Запишем определение равномерной сходимости на $ E$ для $ \frac{\varepsilon}{2}$:
			\[
			    \forall \varepsilon >0 ~ \exists N \colon  \forall n > N ~ \forall x \in E \quad  \lvert f_n(x) - f(n) \rvert < \frac{\varepsilon}{2}
			.\] 
			Тогда для любых $ n, m > N$
			\begin{align*}
			    \lvert f_m(x) - f(x)_n \rvert & \le \\
							  & \le \lvert f_m(x) - f(x) \rvert + \lvert f_n(x) - f(x) \rvert \le  \\
							  & \le \frac{\varepsilon}{2} + \frac{\varepsilon}{2} = \varepsilon 
			\end{align*}
	            \item \boxed{ 2 \Longrightarrow 1} 
			Из условия Коши получаем, что для всех $ x \in E$ последовательность $ {f_n(x)}$ фундаметальна. Следовательно, существует предел  $ f(x) \coloneqq \lim_{n \to \infty} f_n(x)$.

			Устремим $ m \to  \infty$. Тогда \[
			    \lvert f_n(x) - f(x) \rvert \le  \varepsilon 
			.\] 
			По определению равномерной сходимости получаем, что $ f_n \rightrightarrows f$ на $ E$.
	        \end{description} 
	    \end{proof}
	\item \label{prop_5} Пусть $ E$ --- метрическое пространство. Рассмотрим последовательность непрерывных в точке $ x \in E$ функций $ f_n \colon E \to \R(\Cm) $. Если $ f_n \rightrightarrows f$ на $ E$, то  $ f$ тоже непрерывна в точке  $ a$.
	     \begin{proof}
	        Проверим, что 
		\[
		    \lim_{x \to  a} f(x) = f(a)
		.\] 
		А именно, для любого $ \varepsilon  > 0$ существует $ \delta > 0$ такое, что
		\[
		    \forall x \in E \quad \rho(x, a) < \delta \Longrightarrow \lvert f(x) - f(a) \rvert  < \varepsilon 
		.\] 

		Используем равномерную сходимость: для любого  $ \varepsilon > 0$ существует $ N$ такое, что
		\begin{equation}\label{eq:koshi_1}
		    \forall n > N ~ \forall x \in E \quad \lvert f_n(x) - f(x) \rvert  < \frac{\varepsilon}{3}
		.\end{equation}
		Так как $ f_n$ непрерывна в точке $ a$, можем записать определение для  $ \frac{\varepsilon}{3}$ и заодно взять $ n > N$:
		\[
		    \exists \delta >0 \colon \forall x \in  E \quad \rho(x, a) < \delta \Longrightarrow \lvert f_n(x) - f_n(a) \rvert \le \frac{\varepsilon}{3}
		.\] 
		Используем два полученых неравенства:
		\begin{align*}
		    \lvert f(x) - f(a) \rvert \le  & \\
					       \le  & \lvert f(x) - f_n(x) \rvert + \\
					       + & \lvert f_n(x) - f_n(a) \rvert + \\
					       + & \lvert f_n(a) - f_n(a) \rvert < \\
					       < & \frac{\varepsilon}{3} * 3 = \varepsilon 
		\end{align*}
	    \end{proof}

	\item {\bf Теорема Стокса-Зайделя}. Пусть $ f_n \in C(E)$. Если $ f_n \rightrightarrows f$, то $ f$ непрерывна на $ E$. 
	    \begin{proof}
		Следствие из \ref{prop_5}[прошлого свойства].
	    \end{proof}
    \end{enumerate}
\end{prop}
\end{document}
