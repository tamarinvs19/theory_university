% \documentclass[11pt,dvipsnames]{report}
% \usepackage[english, russian]{babel}
\usepackage{xltxtra}
\usepackage{polyglossia}

\usepackage{mathpazo}

\defaultfontfeatures{Ligatures=TeX,Mapping=tex-text}

\setmainfont{STIX2Text-Regular.otf}[
ExternalLocation={/home/vyacheslav/builds/STIXv2.0.2/OTF/},
BoldFont=STIX2Text-Bold.otf,
ItalicFont=STIX2Text-Italic.otf,
BoldItalicFont=STIX2Text-BoldItalic.otf
]
\setmathrm{STIX2Math.otf}[
ExternalLocation={/home/vyacheslav/builds/STIXv2.0.2/OTF/}
]


\usepackage{makeidx}
\usepackage{amssymb, amsthm}
\usepackage{amsmath}
\usepackage{mathtools}
\usepackage{needspace}
\usepackage{enumitem}
\usepackage{cancel}
\usepackage{fdsymbol}
\usepackage{fontawesome}


% разметка страницы и колонтитул
\usepackage[left=2cm,right=2cm,top=1cm,bottom=1.1cm,bindingoffset=0cm]{geometry}
\usepackage{fancybox,fancyhdr}
\fancyhf{}
\fancyhead[R]{\thepage}
\fancyhead[L]{\rightmark}
\fancyfoot{}
\fancyhfoffset{0pt}
\addtolength{\headheight}{13pt}
\pagestyle{fancy}

% Отступы
\setlength{\parindent}{3ex}
\setlength{\parskip}{3pt}

\usepackage{graphicx}
\usepackage{hyperref}

\usepackage{import}
\usepackage{xifthen}
\usepackage{pdfpages}

\newcommand{\incfig}[1]{%
    \def\svgwidth{\columnwidth}
    \import{./figures/}{#1.pdf_tex}
}


\usepackage{xifthen}
\makeatother
\def\@lecture{}%
\newcommand{\lecture}[3]{
    \ifthenelse{\isempty{#3}}{%
        \def\@lecture{Лекция #1}%
    }{%
        \def\@lecture{Лекция #1: #3}%
    }%
    \subsection*{\@lecture}
    \marginpar{\small\textsf{\mbox{#2}}}
}
\makeatletter


\usepackage{xcolor}
\definecolor{Aquamarine}{cmyk}{50, 0, 17, 100}
\definecolor{ForestGreen}{cmyk}{76, 0, 76, 45}
\definecolor{Pink}{cmyk}{0, 100, 0, 0}
\definecolor{Cyan}{cmyk}{56, 0, 0, 100}
\definecolor{Gray}{gray}{0.3}


\usepackage{mdframed}
\mdfsetup{skipabove=3pt,skipbelow=3pt}
\mdfdefinestyle{defstyle}{%
    linecolor=red,
	linewidth=3pt,rightline=false,topline=false,bottomline=false,%
    frametitlerule=false,%
    frametitlebackgroundcolor=red!0,%
    innertopmargin=4pt,innerbottommargin=4pt,innerleftmargin=7pt
    frametitlebelowskip=1pt,
    frametitleaboveskip=3pt,
}
\mdfdefinestyle{thmstyle}{%
    linecolor=cyan!100,
	linewidth=2pt,topline=false,bottomline=false,%
    frametitlerule=false,%
    frametitlebackgroundcolor=cyan!20,%
    innertopmargin=4pt,innerbottommargin=4pt,
    frametitlebelowskip=1pt,
    frametitleaboveskip=3pt,
}
\theoremstyle{definition}
\mdtheorem[style=defstyle]{defn}{Определение}

\newmdtheoremenv[nobreak=true,backgroundcolor=Aquamarine!10,linewidth=0pt,innertopmargin=0pt,innerbottommargin=7pt]{cor}{Следствие}
\newmdtheoremenv[nobreak=true,backgroundcolor=CarnationPink!20,linewidth=0pt,innertopmargin=0pt,innerbottommargin=7pt]{desc}{Описание}
\newmdtheoremenv[nobreak=true,backgroundcolor=Gray!10,linewidth=0pt,innertopmargin=0pt,innerbottommargin=7pt,font={\small}]{ex}{Пример}
\newmdtheoremenv[nobreak=false,backgroundcolor=Cyan!10,linewidth=0pt,innertopmargin=0pt,innerbottommargin=7pt]{thm}{Теорема}
\newmdtheoremenv[nobreak=true,backgroundcolor=Pink!10,linewidth=0pt,innertopmargin=0pt,innerbottommargin=7pt]{lm}{Лемма}

\newtheorem*{st}{Утверждение}
\newtheorem*{prop}{Свойства}

\theoremstyle{plain}
\newtheorem*{name}{Обозначение}

\theoremstyle{remark}
\newtheorem*{rem}{Ремарка}
\newtheorem*{com}{Комментарий}
\newtheorem*{note}{Замечание}
\newtheorem*{prac}{Упражнение}
\newtheorem*{probl}{Задача}


\renewcommand{\proofname}{Доказательство}
\renewenvironment{proof}
{ \hspace{\stretch{1}}\\ \faSquareO\quad \small  }
{ \hspace{\stretch{1}}  \faSquare \normalsize }


\numberwithin{ex}{section}
\numberwithin{thm}{section}
\numberwithin{equation}{section}



\newcommand{\K}{\mathcal{K}}
\newcommand{\Z}{\mathbb{Z}}
\newcommand{\N}{\mathbb{N}}
\newcommand{\Real}{\mathbb{R}}
\newcommand{\Q}{\mathbb{Q}}
\newcommand{\Cm}{\mathbb{C}}
\newcommand{\Pm}{\mathbb{P}}
\newcommand{\ord}{\operatorname{ord}}
\newcommand{\lcm}{\operatorname{lcm}}
\newcommand{\sign}{\operatorname{sign}}
\newcommand{\E}{\mathbb{E}}

\renewcommand{\o}{o}
\renewcommand{\O}{\mathcal{O}}
\renewcommand{\le}{\leqslant}
\renewcommand{\ge}{\geqslant}

\def\mybf#1{\textbf{#1}}
\def\selectedFont#1{\textbf{#1}}
\def\ComplexityFont#1{\textmd{\textbf{\textsf{#1}}}}
\def\LanguageFont#1{{\textbf{\texttt{#1}}}}


\newcommand{\Cclass}{\mathcal{C}}
\newcommand{\Dclass}{\mathcal{D}}


\renewcommand{\P}{\ComplexityFont{P}}
\newcommand{\DTIME}{\ComplexityFont{DTime}}
\newcommand{\DTime}{\ComplexityFont{DTime}}
\newcommand{\DSpace}{\ComplexityFont{DSpace}}
\newcommand{\PSPACE}{\ComplexityFont{PSPACE}}
\newcommand{\NTIME}{\ComplexityFont{NTime}}
\newcommand{\NSpace}{\ComplexityFont{NSpace}}
\newcommand{\coNSpace}{\ComplexityFont{coNSpace}}
\newcommand{\NPSPACE}{\ComplexityFont{NPSPACE}}
\newcommand{\poly}{\ComplexityFont{poly}}
\newcommand{\RP}{\ComplexityFont{RP}}
\newcommand{\coRP}{\ComplexityFont{co-RP}}
\newcommand{\ZPP}{\ComplexityFont{ZPP}}
\newcommand{\BPP}{\ComplexityFont{BPP}}
\newcommand{\BQP}{\ComplexityFont{BQP}}
\newcommand{\coBPP}{\ComplexityFont{co-BPP}}
\newcommand{\NP}{\ComplexityFont{NP}}
\newcommand{\NL}{\ComplexityFont{NL}}
\newcommand{\coNL}{\ComplexityFont{co-NL}}
\renewcommand{\L}{\ComplexityFont{L}}
\newcommand{\NPcomp}{\ComplexityFont{NP-complete}}
\newcommand{\tP}{\widetilde{\P}}
\newcommand{\tNP}{\widetilde{\NP}}
\newcommand{\tBH}{\widetilde{\BH}}
\newcommand{\Class}{{\ComplexityFont{C}}}
\newcommand{\coC}{\ComplexityFont{co-}\mathcal{C}}
\newcommand{\coNP}{\ComplexityFont{co-NP}}
\newcommand{\PH}{\ComplexityFont{PH}}
\newcommand{\EXP}{\ComplexityFont{EXP}}
\newcommand{\Size}{\ComplexityFont{Size}}
\newcommand{\Ppoly}{\ComplexityFont{P}/\ComplexityFont{poly}}
\newcommand{\NC}{\ComplexityFont{NC}}


\newcommand{\FACTOR}{\LanguageFont{FACTOR}}
\newcommand{\kQBF}{{\LanguageFont{QBF{\tiny k}}}}
\newcommand{\QBFk}{{\LanguageFont{QBF{\tiny k}}}}
\newcommand{\QBF}{{\LanguageFont{QBF}}}
\newcommand{\STCON}{\LanguageFont{STCON}}
\newcommand{\USTCON}{\LanguageFont{USTCON}}
\newcommand{\CircuitSat}{{\LanguageFont{CIRCUIT\_SAT}}}
\newcommand{\tCircuitSat}{\widetilde{{\LanguageFont{CIRCUIT\_SAT}}}}
\newcommand{\SAT}{\LanguageFont{SAT}}
\newcommand{\tSAT}{\widetilde{{\LanguageFont{SAT}}}}
\newcommand{\UNSAT}{{\LanguageFont{UNSAT}}}
\newcommand{\tThreeSAT}{\widetilde{{\LanguageFont{3\text{-}SAT}}}}
\newcommand{\ThreeSAT}{{\LanguageFont{3\text{-}SAT}}}
\newcommand{\BH}{\LanguageFont{BH}}
\newcommand{\CircuitEval}{{\LanguageFont{CIRCUIT\_EVAL}}}


\newcommand{\const}{\textmd{const}}
\newcommand{\logspace}{\textmd{logspace}}
\newcommand{\PATH}{\textmd{PATH}}


\newcommand{\readonly}{\textsf{read-only}}
\newcommand{\writeonly}{\textsf{write-only}}


\usepackage{ upgreek }
\newcommand{\PI}{\Uppi}
\newcommand{\SIGMA}{\Upsigma}
\newcommand{\DELTA}{\Updelta}

% \begin{document}

\lecture{11}{23 Apr}{\dag}
\begin{defn}
    Рассмотрим $ \sum_{k=}^{\infty} a_k$. $ \sum_{k=1}^{A_k} $ --- Группировка ряда $ \sum_{k=1}^{\infty} a_k$, если $ A_1= a_1 + \ldots a_{n_1}$, $ A_2 = a_{n_1+1} + \ldots  + a_{n_2}$, то есть $ n_j$ --- возрастающая последовательность натуральных чисел, $ n_0= 0$. $ A_j = \sum_{k=n_{j-1}+1}^{n_j} a_k$.
\end{defn}
\begin{thm}[о группировке] 

    $ $
    \begin{enumerate}[noitemsep]
	\item Если ряд сходится, его группировка тоже сходится, причем $ \sum_{k=1}^{\infty} a_k = \sum_{k=1}^{\infty} A_k$.
	    \item Пусть $ a_n \to  0$ и в каждом  $ A_k$ не более $ L$ слагаемых. Тогда, если $ \sum_{k=1}^{\infty} A_k$ сходится, то $ \sum_{k=1}^{\infty} a_k$ сходится.

		\begin{proof}
		    Рассмотрим $ S_n = \sum_{j=1}^{n} a_j$, $ n_j < n \le n_{j+1}$. Посмотрим на $ S_{n_j}$ и $ S_{n_{j+1}}$.
		    \[
		    \exists \varepsilon 
		    .\] 
		TODO: дописать доказательство
		\end{proof}
	    \item Пусть ряд числовой. Для любого $ A_k$ в сумме участвуют только слагаемые одного знака.
		\begin{proof}
		     Если $ n_i < n < n_j$, то  $ S_{n}$ лежит между $ S_{n_j}$ и $ S_{n_i}$. Можно добиться, чтобы расстояния были меньше $ \varepsilon $, тогда и $ S_n$ будет отличаться на малую величину.
		\end{proof}
    \end{enumerate}
\end{thm}
\section{Положительные ряды}
\begin{defn}[положительный ряд]
    Числовой ряд называется {\sf положительным}, если все его члены неотрицательны.  
\end{defn}
\begin{prop}

    $ $
    \begin{description}
	\item[\boxed{\text{1}}] Ряд сходится тогда и только тогда, когда  $ \{S_n\}$ ограничена (сверху).
	\item[\boxed{\text{Признак сравнения}}] $ 0 \le a_n \le b_n$, то
	    \begin{enumerate}
		\item $ \sum_{=1}^{\infty} b_n$ сходится, тогда $ \sum_{n=1}^{\infty} a_n $ сходится
		\item $ \sum_{n=1}^{\infty} a_n$ расходится, тогда $ \sum_{n=1}^{\infty} b_n$ Тоже расходится.
	    \end{enumerate}
	\item[\boxed{\text{2'}}] $ 0 \le a_n, b_n$, $ a_n = \O(b_n)$ и  $ \sum_{j=1}^{\infty} b_j$ сходится, тогда $ \sum_{n=1}^{\infty} a_n$ сходится. 
	\item[\boxed{\text{2''}}] $ 0 \le a_n , b_n$, если  $ a_n ~ b_n$, то  $ \sum_{n=1}^{\infty} a_n$ сходится тогда и только тогда, когда  $ \sum_{n=1}^{\infty} b_n$ сходится.
	\item[\boxed{\text{Признак Коши}}] Пусть $ a_n \ge 0$ и $ q = \overline{\lim}_{n \to \infty} \sqrt[n]{ a_n} $
	    \begin{enumerate}
		\item $ q < 1$, то  $ \sum_{n=1}^{\infty} a_n $ сходится
		    \item $ q > 1$, то $ \sum_{n=1}^{\infty}a_n $ расходится
	    \end{enumerate}
	    \begin{proof}

		$ $
	        \begin{enumerate}
		    \item Выберем $ 0<\tilde q < 1 $, с некоторого места мы не выходим сильно правее  $ q$, поэтому  $ \exists  N ~ \forall  n > N \colon \sqrt[n]{ a_n} < \tilde q  $, тогда $ a_n < (\tilde q)^{n}$.
		    \item $ \forall   N   \exists  n > N \colon  a_n > 1 \Longrightarrow a_n  \not\to 0$, следовательно, ряд расходится.
	        \end{enumerate}
	    \end{proof}
	\item[\boxed{\text{Признак Даламбера}}] $ a_n > 0$ и  $ \exists \lim_{n to + \infty} \frac{a_{n+1}}{a_n} = q$. Тогда 
	    \begin{enumerate}
		\item $ q> 1$, то ряд расходится
		\item  $ q < 1$, то ряд сходится
	    \end{enumerate}
	    \begin{proof}
	        
		$ $
		\begin{enumerate}
		    \item $ a_{n+1} > a_n$, пожтому ряд точно не сходится.
		    \item Возьмем $ q < \tilde q < 1$, тогда  $ \exists  N ~ \forall  n > N \colon \frac{a_{n+1}}{a_n} < \tilde q$.
			Запишем 
			 \[
			     a_{n+1} = \frac{a_{n+1}}{a_n} \cdot \frac{a_n}{a_{n-1}} \cdot  \ldots \frac{a_{N+1}}{a_N} \cdot a_N < (q)^{n - N + 1} \cdot a _{N^2} = C(\tilde q)^{n+1}
			.\] 
		\end{enumerate}
	    \end{proof}
	\item[\boxed{\text{Интегральный признак}}] Пусть $f \ge 0 $, монотонно убывает $ f::[1, + \infty) \to  \R$. Тогда 
	    \[
		\sum_{n = 1}^{\infty} f(n) \text{ сходится } \Longleftrightarrow \int_{1}^{n} f(x) dx \text{ сходится}
	    .\] 
	    \begin{proof}
	        Просто смотрим по определению интеграла.
	    \end{proof}
    \end{description}
\end{prop}
\section{Числовые ряды с произвольными членами}
\begin{defn}
    $ x_k \in  X$ --- нормированное пространство. $ \sum_{k=1}^{\infty} x_k$ {\sf абсолютно сходится}, если сходится $ \sum_{k=1}^{\infty} \| x_k \| $.
\end{defn}
\begin{prop}

    $ $
    \begin{description}
	\item[\boxed{\text{1}}] $ \sum x_k, \sum y_k $ абсолютно сходятся,  $ \alpha , \beta $ --- скаляры. Тогда ряд $ \sum_{}^{} ( \alpha x_k + \beta y_k)$ абсолютно сходится, так как
	    \[
	    \|  \alpha x_k + \beta  y_k \|  \le \| \alpha  \| \cdot \|  x_k \|  + \| \beta  \| \cdot  \|  y_k \| 
	    .\] 
	\item[\boxed{\text{2}}] Если $ \sum_{k=1}^{\infty} x_k$ сходится, $ \sum_{k=1}^{\infty} \| x_k \| $ сходится, то $ \| \sum_{k=1}^{\infty}  x_k\| \le \sum_{k=1}^{\infty} \| x_k \| $, так как
	    \[
		\| S \| \stackrel{n \to  \infty}{\longleftarrow} \| S_n \| \le \sum_{k=1}^{n} \|  x_k \|  \stackrel{n \to  \infty}{\longrightarrow} \sum_{k=1}^{\infty} \| x_k \| 
	    .\] 
	\item[\boxed{\text{3}}] $ X$ --- полное нормированное пространство. $ \sum_{k=1}^{\infty} \| x_k \|   $ сходится, тогда $ \sum_{k=1}^{\infty} x_k$ сходится.
	    \begin{proof}
		    $ \forall  \varepsilon >0 ~ \exists  N \colon  \forall n > N , p \in  \N ~ \sum_{k = n+1}^{n+p} \| x_k \|  < \varepsilon $, следовательно, $ \| \sum_{k = n+1}^{n+p} x_k \| < \varepsilon $. Получили, что $ \sum_{k=1}^{\infty} x_k$ сходится.
	    \end{proof}
	\item[\boxed{\text{4}}] В полном нормированном пространстве $ \sum_{k=1}^{\infty} x_k$ сходится абсолютно, $ \sum_{k=1}^{\infty} y_k$ сходится условно, тогда $ \sum_{k=1}^{\infty} (x_k+ y_k)$ сходится условно.
	\item[\boxed{\text{5}}] $ X $ --- полное, $ \overline{\lim}_{n \to  \infty} \sqrt[n]{ \|  x_n \| } $ , $ \lim_{n \to \infty} \frac{\| x_{n+1} \| }{\| x_n \| }$

    \end{description}
\end{prop}
\begin{defn}
    Если ряд сходится, но не сходится абсолютно, он называется {\sf условно сходящимся.}  
\end{defn}
\begin{lm}[преобразование Абеля]
    Пусть $ \{a_n\}, \{b_n\}$ --- последовательности. Пусть $A_n = \sum_{k=1}^{n} a_n , ~ A_0 = 0$.
    Рассмотрим
    \begin{align*}
	\sum_{k=1}^{n}  a_k b_k &= \sum_{k=1}^{n} (Ak - A_{k+1}) b_k = \sum_{k=1}^{n} A_kb_k - \sum_{k=1}^{n} A_{k-1}b_k = \\
				&= \sum_{k=1}^{n} A_k b_k - \sum_{k=1}^{n-1} A_k b_k+1 = A_n b_n  \sum_{k=1}^{n-1} A_k(b_k-b_{k+1})
    \end{align*}
    Получили дискретный аналог интегрирования по частям.
\end{lm}
\begin{thm}[Признаки Дирихле и Абеля]
    $ \{a_n\}, \{b_n\}$ --- числовые последовательности.  $ b_n$ --- монотонная последовательность, $ b_n \in \R, a_n in \Cm$

\begin{description}
    \item[Признак Дирихле] $ \{A_n\} $ --- ограниченная последовательность, $ b_n \to  0$.
    \item[Признак Абеля] $ \sum_{k=1}^{n} a_k $ сходится, $ b_n $ ограничено
\end{description}
тогда $ \sum_{n=1}^{\infty} a_nb_n$ сходится.
\end{thm}
\begin{proof}
     \[
	 \sum_{k=1}^{n} a_k b_k = A_n b_n + \sum_{k=1}^{n-1} A_k (b_k - b_{k+1})
     .\] 
     Первое слагаемое сходится при условии обоих признаков. 

     Для признака Абеля сразу все хорошо: второе слагаемое сходится.

     Для признака Дирихле проверим $ \sum_{k=1}^{\infty}  \left| A_k(b_k - b_{k+1}) \right| \le X \sum_{k=1}^{\infty} | b_k - b_{k+1} | $
     В атом случае сходится даже без модуля $ \sum_{k=1}^{\infty} b_k - b_{k+1} $, так как $ \sum_{k=1}^{n}  b_{n+1} - b_1$.
\end{proof}
\begin{thm}[Признак Лейбница] $ b_n$ убывает к нулю, тогда ряд $ \sum_{n=1}^{\infty} (-1)^{n} b_n$ сходится.
\end{thm}
\begin{proof}
    $ a_n = (-1)^{n}$, $ A_n \in  \{1, 0\}$ --- ограничено. По признаку Дирихле ряд произведения сходится.
\end{proof}
\begin{note}
    $ S_n = \sum_{k=1}^{n} (-1)^{k}b_k$, $ S$ --- сумма ряда. Тогда $ \lvert S- S_n \rvert  \le b_{n+1}$. 
\end{note}
\begin{ex}[Ряд Лейбница]
    \[
	\sum_{k=1}^{\infty} (-1)^{k+1}\frac{1}{2k-1} \text{ сходится условно }
    .\] 
\end{ex}
\begin{ex}
    \[
	\sum_{k=1}^{\infty} (-1)^{k+1}\frac{1}{k} \text{ тоже сходится условно}
    .\] 
\end{ex}
\begin{ex}
    \[
	\sum_{k=1}^{\infty} \frac{\sin k}{k}, ~ \sum_{k=1}^{\infty} \frac{\cos k}{k} \text{ сходятся}
    .\] 
    \[
	A_n = \sum_{k=1}^{n}  \sin k= \sum_{k=1}^{n}  \im (x\cos k  i \sin k) = \im \sum_{k=1}^{n}  e^{ik}
    .\] 
    \[
	\sum_{k=1}^{n}  e^{ik} =  e^{i} \frac{e^{n_i} - 1}{e^{i} - 1}=
e^{i }\frac{e^{\frac{ni}{2}} \left(e^{\frac{ni}{2}}-e^{-\frac{ni}{2}}\right)\cdot \frac{1}{2i}}{e^{\frac{i}{2}}\left(e^{\frac{i}{2}}-e^{-\frac{i}{2}}\right)\cdot \frac{1}{2i}} = e^{\frac{n+1}{2}i} \frac{\sin \frac{n}{2}}{\sin \frac{1}{2}} 
    .\] 
    Теперь берем мнимую часть 
    \[
    A_n  = \frac{\sin \frac{n+1}{2}\sin \frac{n}{2}}{\sin\frac{1}{2}} \le \frac{1}{\sin \frac{1}{2}}
    .\] 
    Для косинуса аналогично.
\end{ex}
\begin{thm}[О перестановке членов абсолютно сходящегося ряда]
    $ \sum_{k=1}^{\infty}a_k $ --- абсолютно сходящийся ряд.
    $ \varphi \colon \N \to \N$ --- биекция, тогда $ \sum_{k=1}^{\infty} a_{\phi(k)}$ сходится к той же сумме.
\end{thm}
\begin{proof}

    $ $
    \begin{enumerate}
	\item $ a_k > 0$,  $ S_n = \sum_{k=1}^{n} a_k$, $ T_n = \sum_{k=1}^{n} a_{\varphi(k)}$
	    \[
	    \forall  n ~ \exists  n_1, n_2 \colon S_n \le T_{n_1} \le S_{n_2} \Longrightarrow T_n \to  S = \lim_{n \to \infty} S_n
	    .\] 
	\item $ a_k \in  \R$. Запишем $ a_k = (a_k)_+ - (a_k)_-, ~ \lvert a_k \rvert  = (a_k)_+ (a_k)_-$.
	    Тогда 
	    \[
		\sum{\lvert a_k \rvert }  \text{ сходится } \Longrightarrow   \sum_{k=1}^{\infty} (a_k)_+, ~\sum_{k=1}^{\infty} (a_k)_- \text{ сходятся}.
	    .\] 
	    Применим прошлый пункт: $ \sum{(a_k)_\pm}  = \sum(a_{\varphi(k)})_\pm  $.
	    \[
		\sum_{k=1}^{\infty} a_k = \sum_{k=1}^{\infty}(a_k)_+ - \sum_{k=1}^{\infty} (a_k)_- = \sum_{k=1}^{\infty} (a_{ \varphi (k)})_+ - \sum_{k=1}^{\infty} (a_{ \varphi (k)})_- = \sum_{k=1}^{\infty}  a_{ \varphi (t)}  
	    .\] 
	\item $ a_k \in \Cm$, $ a_ k= b_k + i c_k$. Применяем второй пункт.
    \end{enumerate}
\end{proof}
\begin{thm}[Теорема Римана]
    $ a_k \in  \R$. $ \sum_{k=1}^{\infty} a_k$ сходится условно. Тогда $$ \forall S \in  \overline{\R} ~ \exists \varphi  \colon \N \to \N \colon \sum_{k=1}^{\infty} a_{\varphi(k)} = S
    $$
\end{thm}
\begin{thm}[Коши об умножении рядов]
    $ \sum_{k=1}^{\infty} a_k, ~ \sum_{k=1}^{\infty}b_k $ --- абсолютно сходящиеся численные ряды. Тогда $ \sum_{k, n =1}^{\infty}a_k b_n $ сходится при любых порядках слагаемых, при этом $ \sum_{k,n=1}^{\infty} a_kb_n = \sum_{k=1}^{\infty} a_k \cdot \sum_{n=1}^{\infty}b_n $
\end{thm}
\begin{proof}
    Пусть $ \sum_{k=1}^{n} a_k = A_k, \sum_{k=1}^{n} \lvert a_k \rvert = \overline{A_n}, \sum_{k=1}^{\infty}  =A, \sum_{k=1}^{\infty} \lvert a_k \rvert = \overline{A} $, аналогично для $ b$.
    
    Зафиксируем на множестве пар некоторый порядок.

    $ S_m$ --- частичная сумма $ \sum \lvert a_k \rvert \lvert b_n \rvert $, $ N $ --- максимальный из встречающихся индексов.
    \[
	S_m \le \sum_{k=1}^{N}  \lvert a_k \rvert \sum_{k=1}^{N}  \lvert b_k \rvert \le \overline{A} \overline{B} \Longrightarrow \text{ hряд } \sum \lvert a_k \rvert \lvert b_n \rvert  \text{сходится}
    .\] 
    Теперь просуммируем по квадратам
    \[
	n^2 \le m <(n+1)^2
    .\] 
    \[
    S \leftarrow S_{n^2} = A_n\cdot  B_n \to  A\cdot B
    .\] 
    \[
	\lvert S_{n^2} - S_m \rvert \le \lvert a_{n+1} \rvert \cdot \overline{B} + \lvert b_{n+1} \rvert \cdot \overline{A} \stackrel{n \to  \infty}{\longrightarrow} 0
    .\] 
\end{proof}
\begin{defn}[Произведение рядов по Коши]
    $ \sum_{n=1}^{\infty} a_n$, $ \sum_{n=1}^{\infty} b_n$ --- ряды. $ c_n = a_1b_n + a_2b_{n-1} + \ldots a_nb_1$. Тогда ряд $ \sum_{n=1}^{\infty} c_n$ называется {\sf  произведением рядов}.  
\end{defn}
\begin{thm}[Мергенс]
    $ \sum_{n=1}^{\infty} a_n$ сходится абсолютно, $ \sum_{k=1}^{\infty} b_n$ сходится. Тогда $ \sum_{n=1}^{\infty} c_n $ сходится и равно $ \sum_{n=1}^{\infty} a_n \sum_{n=1}^{\infty} b_n$.
\end{thm}
\begin{thm}[Абель]
    $ \sum_{n=1}^{\infty} a_n, \sum_{n=1}^{\infty} b_n, \sum_{n=1}^{\infty} c_n$ сходится, тогда $ \sum_{n=1}^{\infty} c_n = \sum_{n=1}^{\infty} a_n \sum_{n=1}^{\infty} b_n$
\end{thm}
\begin{ex}
    $ a_n = b_n = (-1)^{n}\frac{1}{\sqrt{ n} } \Longrightarrow  \lvert a_n \rvert \ge 1$
\end{ex}
\section{Бесконечные произведения}
\begin{defn}
    Частичные произведения $ \prod_{k=1}^{n}p_k =P_n$. Частичные произведения сходятся к $ P$если  $ \exists  \lim_{n \to \infty} P_n = P$ и $ P\ne 0, P \ne \infty$. Если $ P=0$, говорят, что расходится к 0, если к  $ \pm \infty$, говорят, что расходится к $ \pm \infty$.
\end{defn}
\begin{ex}
    \[
	\prod_{n=2}^{\infty}\left(1- \frac{1}{n^2}\right)
    .\] 
    \[
	P_n = \frac{1}{2}\cdot \frac{n+1}{n} \to  \frac{1}{2}
    .\] 
\end{ex}
\begin{ex}
    \[
	\prod_{n=1}^{\infty} \left( 1 - \frac{1}{4n^2} \right)  = \frac{2}{\pi} \text{(формула Ваниса)}
    .\] 
\end{ex}
\begin{prop}
    Будем считать, что $ p_n \ne 0$.
    \begin{description}
	\item[\boxed{1}]  $ \prod_{n=1}^{\infty} p_n$ сходится, тогда $ p_n \to  1$ 
	\item[\boxed{2}]  Первые несколько слагаемых ряда можно отбросить, на сходимость это не повлияет
	\item[\boxed{3}] Всегда можно считать, что $ p_n > 0$
	\item[\boxed{4}] $ \prod_{n=1}^{\infty}p_n, p_n >0$. 
	    \[
	    \prod_{n=1}^{\infty} p_n \text{ сходится} \Longleftrightarrow \prod_{n=1}^{\infty} \ln p_n \text{ сходится}
	    .\] 
	    $ \ln P_n = S_n$
    \end{description}
\end{prop}
\begin{ex}
    Пусть $ p_n$ --- $ n$-ое простое число. 
    \[
	\prod _{n=1}^{\infty} \frac{p_n}{p_n-1} \text{ расходится}
    .\] 
    \[
	\prod_{n=1}^{\infty} \frac{p_n}{p_n -1}= \prod_{n=1}^{\infty} \frac{1}{1- \frac{1}{p_n}} = \prod_{n=1}^{\infty}\sum_{k=0}^{\infty} \frac{1}{p_n^{k}} \stackrel{?}{=}
    .\] 
    Оценим
    \[
	P_n =  \prod_{k=1}^{n} \frac{p_k}{p_k -1} = \prod _{k=1}^{n} \frac{1}{1-\frac{1}{p_k}} \ge \prod _{k=1}^{n} \sum_{m=0}^{n} \frac{1}{p_k^{m}} = \sum_{0 \le \alpha_j \le  n }^{}  \frac{1}{p_1^{ \alpha _j} \cdot  \ldots  \cdot  p_n^{ \alpha _n}} \ge 1 + \frac{1}{2} + \frac{1}{3} + \ldots +\frac{1}{n} = \ln n + C
    .\] 
    \[
	\sum_{n=1}^{\infty} \ln\left( \frac{p_n}{p_n-1} \right) , ~ \ln\left( \frac{p_n}{p_n -1} \right) = - \ln\left( 1- \frac{1}{p_n} \right) \sim \frac{1}{p_n}
    .\] 
    Тогда ряд $ \sum_{n=1}^{\infty} \frac{1}{p_n}$ расходится.

   Следовательно,
   \[
   \stackrel{?}{=} \sum_{}^{} \frac{1}{p_1}^{ \alpha _1} \cdot  \ldots  p_s ^{ \alpha _s} = \sum_{n=1}^{\infty} \frac{1}{n} \to  +\infty
   .\] 
\end{ex}
% \end{document}
