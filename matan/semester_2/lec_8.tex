% \documentclass[11pt,dvipsnames]{report}
% \usepackage[english, russian]{babel}
\usepackage{xltxtra}
\usepackage{polyglossia}

\usepackage{mathpazo}

\defaultfontfeatures{Ligatures=TeX,Mapping=tex-text}

\setmainfont{STIX2Text-Regular.otf}[
ExternalLocation={/home/vyacheslav/builds/STIXv2.0.2/OTF/},
BoldFont=STIX2Text-Bold.otf,
ItalicFont=STIX2Text-Italic.otf,
BoldItalicFont=STIX2Text-BoldItalic.otf
]
\setmathrm{STIX2Math.otf}[
ExternalLocation={/home/vyacheslav/builds/STIXv2.0.2/OTF/}
]


\usepackage{makeidx}
\usepackage{amssymb, amsthm}
\usepackage{amsmath}
\usepackage{mathtools}
\usepackage{needspace}
\usepackage{enumitem}
\usepackage{cancel}
\usepackage{fdsymbol}
\usepackage{fontawesome}


% разметка страницы и колонтитул
\usepackage[left=2cm,right=2cm,top=1cm,bottom=1.1cm,bindingoffset=0cm]{geometry}
\usepackage{fancybox,fancyhdr}
\fancyhf{}
\fancyhead[R]{\thepage}
\fancyhead[L]{\rightmark}
\fancyfoot{}
\fancyhfoffset{0pt}
\addtolength{\headheight}{13pt}
\pagestyle{fancy}

% Отступы
\setlength{\parindent}{3ex}
\setlength{\parskip}{3pt}

\usepackage{graphicx}
\usepackage{hyperref}

\usepackage{import}
\usepackage{xifthen}
\usepackage{pdfpages}

\newcommand{\incfig}[1]{%
    \def\svgwidth{\columnwidth}
    \import{./figures/}{#1.pdf_tex}
}


\usepackage{xifthen}
\makeatother
\def\@lecture{}%
\newcommand{\lecture}[3]{
    \ifthenelse{\isempty{#3}}{%
        \def\@lecture{Лекция #1}%
    }{%
        \def\@lecture{Лекция #1: #3}%
    }%
    \subsection*{\@lecture}
    \marginpar{\small\textsf{\mbox{#2}}}
}
\makeatletter


\usepackage{xcolor}
\definecolor{Aquamarine}{cmyk}{50, 0, 17, 100}
\definecolor{ForestGreen}{cmyk}{76, 0, 76, 45}
\definecolor{Pink}{cmyk}{0, 100, 0, 0}
\definecolor{Cyan}{cmyk}{56, 0, 0, 100}
\definecolor{Gray}{gray}{0.3}


\usepackage{mdframed}
\mdfsetup{skipabove=3pt,skipbelow=3pt}
\mdfdefinestyle{defstyle}{%
    linecolor=red,
	linewidth=3pt,rightline=false,topline=false,bottomline=false,%
    frametitlerule=false,%
    frametitlebackgroundcolor=red!0,%
    innertopmargin=4pt,innerbottommargin=4pt,innerleftmargin=7pt
    frametitlebelowskip=1pt,
    frametitleaboveskip=3pt,
}
\mdfdefinestyle{thmstyle}{%
    linecolor=cyan!100,
	linewidth=2pt,topline=false,bottomline=false,%
    frametitlerule=false,%
    frametitlebackgroundcolor=cyan!20,%
    innertopmargin=4pt,innerbottommargin=4pt,
    frametitlebelowskip=1pt,
    frametitleaboveskip=3pt,
}
\theoremstyle{definition}
\mdtheorem[style=defstyle]{defn}{Определение}

\newmdtheoremenv[nobreak=true,backgroundcolor=Aquamarine!10,linewidth=0pt,innertopmargin=0pt,innerbottommargin=7pt]{cor}{Следствие}
\newmdtheoremenv[nobreak=true,backgroundcolor=CarnationPink!20,linewidth=0pt,innertopmargin=0pt,innerbottommargin=7pt]{desc}{Описание}
\newmdtheoremenv[nobreak=true,backgroundcolor=Gray!10,linewidth=0pt,innertopmargin=0pt,innerbottommargin=7pt,font={\small}]{ex}{Пример}
\newmdtheoremenv[nobreak=false,backgroundcolor=Cyan!10,linewidth=0pt,innertopmargin=0pt,innerbottommargin=7pt]{thm}{Теорема}
\newmdtheoremenv[nobreak=true,backgroundcolor=Pink!10,linewidth=0pt,innertopmargin=0pt,innerbottommargin=7pt]{lm}{Лемма}

\newtheorem*{st}{Утверждение}
\newtheorem*{prop}{Свойства}

\theoremstyle{plain}
\newtheorem*{name}{Обозначение}

\theoremstyle{remark}
\newtheorem*{rem}{Ремарка}
\newtheorem*{com}{Комментарий}
\newtheorem*{note}{Замечание}
\newtheorem*{prac}{Упражнение}
\newtheorem*{probl}{Задача}


\renewcommand{\proofname}{Доказательство}
\renewenvironment{proof}
{ \hspace{\stretch{1}}\\ \faSquareO\quad \small  }
{ \hspace{\stretch{1}}  \faSquare \normalsize }


\numberwithin{ex}{section}
\numberwithin{thm}{section}
\numberwithin{equation}{section}



\newcommand{\K}{\mathcal{K}}
\newcommand{\Z}{\mathbb{Z}}
\newcommand{\N}{\mathbb{N}}
\newcommand{\Real}{\mathbb{R}}
\newcommand{\Q}{\mathbb{Q}}
\newcommand{\Cm}{\mathbb{C}}
\newcommand{\Pm}{\mathbb{P}}
\newcommand{\ord}{\operatorname{ord}}
\newcommand{\lcm}{\operatorname{lcm}}
\newcommand{\sign}{\operatorname{sign}}
\newcommand{\E}{\mathbb{E}}

\renewcommand{\o}{o}
\renewcommand{\O}{\mathcal{O}}
\renewcommand{\le}{\leqslant}
\renewcommand{\ge}{\geqslant}

\def\mybf#1{\textbf{#1}}
\def\selectedFont#1{\textbf{#1}}
\def\ComplexityFont#1{\textmd{\textbf{\textsf{#1}}}}
\def\LanguageFont#1{{\textbf{\texttt{#1}}}}


\newcommand{\Cclass}{\mathcal{C}}
\newcommand{\Dclass}{\mathcal{D}}


\renewcommand{\P}{\ComplexityFont{P}}
\newcommand{\DTIME}{\ComplexityFont{DTime}}
\newcommand{\DTime}{\ComplexityFont{DTime}}
\newcommand{\DSpace}{\ComplexityFont{DSpace}}
\newcommand{\PSPACE}{\ComplexityFont{PSPACE}}
\newcommand{\NTIME}{\ComplexityFont{NTime}}
\newcommand{\NSpace}{\ComplexityFont{NSpace}}
\newcommand{\coNSpace}{\ComplexityFont{coNSpace}}
\newcommand{\NPSPACE}{\ComplexityFont{NPSPACE}}
\newcommand{\poly}{\ComplexityFont{poly}}
\newcommand{\RP}{\ComplexityFont{RP}}
\newcommand{\coRP}{\ComplexityFont{co-RP}}
\newcommand{\ZPP}{\ComplexityFont{ZPP}}
\newcommand{\BPP}{\ComplexityFont{BPP}}
\newcommand{\BQP}{\ComplexityFont{BQP}}
\newcommand{\coBPP}{\ComplexityFont{co-BPP}}
\newcommand{\NP}{\ComplexityFont{NP}}
\newcommand{\NL}{\ComplexityFont{NL}}
\newcommand{\coNL}{\ComplexityFont{co-NL}}
\renewcommand{\L}{\ComplexityFont{L}}
\newcommand{\NPcomp}{\ComplexityFont{NP-complete}}
\newcommand{\tP}{\widetilde{\P}}
\newcommand{\tNP}{\widetilde{\NP}}
\newcommand{\tBH}{\widetilde{\BH}}
\newcommand{\Class}{{\ComplexityFont{C}}}
\newcommand{\coC}{\ComplexityFont{co-}\mathcal{C}}
\newcommand{\coNP}{\ComplexityFont{co-NP}}
\newcommand{\PH}{\ComplexityFont{PH}}
\newcommand{\EXP}{\ComplexityFont{EXP}}
\newcommand{\Size}{\ComplexityFont{Size}}
\newcommand{\Ppoly}{\ComplexityFont{P}/\ComplexityFont{poly}}
\newcommand{\NC}{\ComplexityFont{NC}}


\newcommand{\FACTOR}{\LanguageFont{FACTOR}}
\newcommand{\kQBF}{{\LanguageFont{QBF{\tiny k}}}}
\newcommand{\QBFk}{{\LanguageFont{QBF{\tiny k}}}}
\newcommand{\QBF}{{\LanguageFont{QBF}}}
\newcommand{\STCON}{\LanguageFont{STCON}}
\newcommand{\USTCON}{\LanguageFont{USTCON}}
\newcommand{\CircuitSat}{{\LanguageFont{CIRCUIT\_SAT}}}
\newcommand{\tCircuitSat}{\widetilde{{\LanguageFont{CIRCUIT\_SAT}}}}
\newcommand{\SAT}{\LanguageFont{SAT}}
\newcommand{\tSAT}{\widetilde{{\LanguageFont{SAT}}}}
\newcommand{\UNSAT}{{\LanguageFont{UNSAT}}}
\newcommand{\tThreeSAT}{\widetilde{{\LanguageFont{3\text{-}SAT}}}}
\newcommand{\ThreeSAT}{{\LanguageFont{3\text{-}SAT}}}
\newcommand{\BH}{\LanguageFont{BH}}
\newcommand{\CircuitEval}{{\LanguageFont{CIRCUIT\_EVAL}}}


\newcommand{\const}{\textmd{const}}
\newcommand{\logspace}{\textmd{logspace}}
\newcommand{\PATH}{\textmd{PATH}}


\newcommand{\readonly}{\textsf{read-only}}
\newcommand{\writeonly}{\textsf{write-only}}


\usepackage{ upgreek }
\newcommand{\PI}{\Uppi}
\newcommand{\SIGMA}{\Upsigma}
\newcommand{\DELTA}{\Updelta}

% \begin{document}

\lecture{8}{3 Apr}{\dag}
\section{Странные примеры экстремумов}
\subsection{Задача Гюйгенса}
\begin{desc}
    Есть два шара с массами $ M$ и $ m \in (0, M)$. Шар с массой $ M$ летит со скоростью  $ V$ на покоящийся нар массой  $ m$. Какая скорость будет у малого шара после столкновения? И как ее вообще найти?
\end{desc}
\begin{center}
\begin{figure}[ht]
    \centering
    \incfig{balls}
    \caption{Столкновение шаров}
    \label{fig:balls}
\end{figure}
\end{center}
После столкновения посчитаем импульс и энергию.
По закону сохранения импульса и закону сохранения энергии
\begin{align*}
    m_1v_1 + m_2v_2 &= m_1 \widetilde{ v_1} + m_2 \widetilde{ v_2} \\
m_1v_1^2 + m_2v_2^2 &= m_1 \widetilde{ v_1}^2 + m_2 \widetilde{ v_2}^2 
\end{align*}
\begin{align*}
    m_1(v_1-\widetilde{ v_1} ) &= m_2(\widetilde{ v_2} -v_2)\\
    m_1(v_1^2-\widetilde{ v_1}^2 ) &= m_2(\widetilde{ v_2}^2 -v_2^2)\\
\end{align*}
Поделим одно на другое и получим, что $ v_1+\widetilde{ v_1} = v_2+\widetilde{ v_2}  $.
Дальше можно подставить в первое уравнение и получить
\[
    m_1v_1+ m_2v_2 =m_1 \widetilde{ v_1} +m_2(v_1+ \widetilde{ v_1} -v_2)
.\] 
Тогда
\[
    \widetilde{ v_1}= \frac{(m_1-m_2)v_1 +2m_2v_2}{m_1+m_2}
.\] 
\[
    \widetilde{ v_2}= \frac{(m_2-m_1)v_2 +2m_1v_1}{m_1+m_2}
.\] 
Если $ v_2=0$, 
\[
	\widetilde{ v_2} = \frac{2m_1v_1}{m_1+m_2}  \in (v_1, 2v_1)
.\] 
\begin{defn}[Задача Гюйгенса]
    С какими массами $ m_1, \ldots m_n$ разместить по пути покоящиеся шары, чтобы передался максимальный импульс?
\end{defn}
\begin{figure}[ht]
    \centering
    \incfig{huygens-task}
    \caption{Задача Гюйгенса}
    \label{fig:huygens-task}
\end{figure}
\[
	\widetilde{v} = v \cdot  \frac{2M}{M+m_1} \cdot  \frac{2m_1}{m_1+m_2} \cdot \ldots \cdot \frac{2m_n}{m_n +m} = f(m_1, \ldots m_n) \cdot v \cdot 2^{n+1}
.\] 

Нужно найти максимум этой функции. Он существует, так как в бесконечности  одной и переменных значение стремиться к 0.
Обозначим $ m_0 = M, ~ m_{n+1} = m$.

Посчитаем частные производные и приравняем к $ 0$ 
\[
    \partial _j f(\ldots ) = 0 \Longleftrightarrow  m_j^2=m_{j-1}m_{j+1}
.\] 
Тогда 
\[
q =     \frac{M}{m_1}= \frac{m_1}{m_2} = \ldots = \frac{m_n}{m}
.\] 
 \[
     q^{n+1} = \frac{M}{m}, \quad q = \sqrt[n+1]{\frac{M}{m}}
 .\] 
 А скорость
 \[
     \widetilde{ v} = 2^{n+1} \left( \frac{q}{q+1} \right) ^{n+1} v
 .\] 
 При $ n = 0$, получается  $ \widetilde{v} = 2\cdot \frac{\frac{M}{m}}{\frac{M}{n} + 1}  < 2$ 
\begin{prac}
    \[
		\lim_{n \to \infty} \left( \frac{2 \sqrt[n]{\frac{M}{m} }}{\sqrt[n]{\frac{M}{m}+1 } }  \right) ^{n+1} = ?
    .\] 
\end{prac}
\subsection{Кратчайшее расстояние до линейного подпространства}
\begin{thm}
	Пусть $ H$ --- пространство со скалярным произведением,  $ L \subset H$ --- линейное подмножество (подпространство), $ x_0 \in H$. Пусть $ z_0$ --- наилучшее приближение к $ x_0$ в $ L$, то есть
	 \[
	\| x_0 - z_0 \| = \min_{z \in  L} \| x_0 - z \| 
	,\] 
	тогда $ x_0-z_0 \perp L$, то есть $ \forall z \in L\colon \langle x_0 - z_0, z \rangle = 0$.
\end{thm}
\begin{proof}
	Введем функцию $ f\colon L \to  R$, $ f(z) = \| x_0 - z \|^2 $. В точке $ z_0$ эта функция имеет минимум.
	Хотим минимизировать $ f$.
	 \[
		 f(z) = \langle x_0 -z, x_0 -z \rangle = \langle z, z \rangle - \langle x_0, z \rangle - \langle z, x_0 \rangle + \langle x_0, x_0 \rangle
	.\] 
	Продифференцируем:
	\[
	\begin{aligned}
		df(z_0)h &= \langle z_0, h \rangle + \langle h, z_0 \rangle - \langle x_0, h \rangle - \langle h, x_0 \rangle \\
				 &= \langle z_0-x_0, h \rangle+ \langle  h, z_0 - x_0 \rangle = \\
				 &=2 \re \langle h, z_0-x_0 \rangle
	\end{aligned}
	.\] 
	Так как $ \forall h \in L\colon df(z_0)h = 0$, в веществественном случае получаем перпендикулярность.
	
	Если поле комплексное, то для всех $ \theta $ 
	\[
		2 \re \langle he^{i \theta }, z_0- x_0 \rangle = 0
	.\] 
	Выберем $ \theta $ так, что $ \langle he^{i \theta }, z_0 -x_0 \rangle \in  \R$,  поэтому можно вынести скаляр $ e^{i \theta }$ и получить $ \langle h, z_0-x_0 \rangle = 0$.
\end{proof}
\begin{defn}[Аффинное подпространство]
   Пусть $ L \subset X$, $ x_0, l_0 \in H$, $ L_0 \subset H$ --- линейное подпространство. Подпространство $ L = \{l_0 + z \mid z \in L_0\}$ называется  {\sf аффинным}.  
\end{defn}
Рассмотрим функцию $ f\colon L \to  \R$, $ f(z) = \| z-x_0 \| ^2$. Нужно найти ее минимум. Пусть $ z_0$ --- точка минимума.

\[
	df(z_0) \colon L_0 \to  \R
.\] 
\[
	f(z_0 + h) = f(z_0) + df(z_0) h + \o(h)
.\] 
Будем брать $ h\colon z_0 + h \in L \Longleftrightarrow h \in  L_0$ --- область допустимых приращений.

\subsection{Задача о брахистороне}
\paragraph{Постановка задачи}
Пусть есть координатная плоскость с осями $ x$ и  $ y$. Мы находимся в точке  $ (0, 0)$ и хотим попасть в точку  $ (x', y')$, выбрав оптимальную траекторию.
\begin{figure}[ht]
    \centering
    \incfig{brachistochrone-problem}
    \caption{Задача о брахистороне}
    \label{fig:brachistochrone-problem}
\end{figure}
Хотим минимизировать время, затраченное на спуск, по всем функциям $ f$.
Обозначим множество функций
\[
	L = \{f \in C^{1}[0, x_{*} \mid f(0) = 0, ~f(x_*) = y_*\}
.\] 

Посчитаем мгновенную скорость: 
\[
	\frac{mv(x)^2}{2} = mgx \Longrightarrow v(x) = \sqrt{ 2gx} 
.\] 

Чтобы найти время, нужно разбить путь на малые отрезки, на них разделить расстояние на скорость и просуммировать. То есть проинтегрировать функцию. Воспользуемся утверждением о том, что при достаточно малом кусочке длина дуги будет равна $ \sqrt{ 1+ f'(x)^2} $ :
\[
	T(f) = \int_{0}^{x_*} \frac{\sqrt{ 1+f'(x)^2}}{\sqrt{ 2gx} }dx  
.\] 
Заведем функционал $ J\colon L \to  \R$:
 \[
	 J[f] = \int_{0}^{x_*} \frac{\sqrt{ 1+f'(x)^2}}{\sqrt{ x} }dx  
.\] 

\subparagraph{Общий вид}
В более общем виде функционал $ J[f]$, где $ F\colon \R^3 \to  \R$, принимает такой вид:
\[
	J[f] = \int_{a}^{b} F(x, f(x), f'(x)) dx  
.\] 
В нашем случае $ F(u_1, u_2, u_3) = \frac{\sqrt{ 1 + u_3^2}}{\sqrt{ u_1} }  $.

\begin{prac}
Если $ F \in C^{1}(\R^3)$, то $ J$  дифференцируема.
\end{prac}
\begin{proof}
	Пусть $ F \in C^{1}(\R^3), ~ F\colon \R^3 \to  \R$, $ J\colon L \to \R$
	\[
		L = \{f \in C^{1}[a, b]\mid f(a)  = A,~ f(b) = B\}
	.\] 
Определим $ L_0 $ --- пространство  допустимых приращений к функции:
\[
	L_0 = \{f \in  C^{1}[a, b] \mid f(a) = f(b) = 0\}
.\] 

Тогда $ dJ(f)\colon L_0 \to  \R$ --- линейное непрерывное отображение.

Пусть $ J = J_2 \circ J_1 $, где 
\[
\begin{aligned}
	J_2 &\colon C[a, b] \to  \R, &&J_2(f) = \int_{a}^{b} f(x)dx \\
	J_1 &\colon C^{1}[a, b] \to C[a, b], && J_1(f)(x) = F(x, f(x), f'(x)
\end{aligned}
\]
Тогда $ dJ(f) = J_2 \circ d J_1(f)$. Чтобы доказать это, докажем, что $ dJ_2(J_1(f)) = J_2$. Пусть $ q = J_1(f)\colon \R \to  \R$ и $ h \to  0$ --- приращение. Тогда
\[
	J_2(q + h) - J_2(q) = \int_{a}^{b} q(x)dx = J_2(q) 
.\] 
Теперь нужно проверить, что $ J_1$ дифференцируемо, так как с $ J_2$ уже все в порядке.
\[
	d_hJ(f) = J_2 \circ d_hJ_1(f)
.\] 
\[
	h \in L_0 = \lim_{t \to  \infty} \frac{J_1(f+th)-J_1(f)}{t}
.\] 
Таким образом, для всех $ x$  нужно посчитать 
\begin{equation}\label{eq:limf}
	\lim_{t \to  \infty} \frac{F(x, f(x) + th(x), f'(x) + th'(x) ) - F(x, f(x) , f'(x))}{t}
\end{equation}
Пусть 
\[
	\varphi (t) = F(x, f(x) +th(x), f'(x)+th'(x))
.\] 
Тогда \ref{eq:limf} равна $ \varphi '(0)$.
При этом 
\[
	\varphi '(t) = \partial _2 F(x, f(x) + th(x), f'(x)+th'(x))h(x) + \partial _3 F(x, f(x)+th(x), f'(x)+th'(x))h'(x)
,\] 
из чего следует, что
\ref{eq:limf} равно
\[
	\partial _2 F(x, f(x), f'(x))h(x) + \partial _3 F(x, f(x), f'(x))h'(x)
.\] 
Проинтегрируем
\[
	\partial _hJ(f) = \int_{a}^{b} \partial _2 F(x, f(x), f'(x))h(x) + \partial _3 F(x, f(x), f'(x))h'(x) dx  
.\] 
Необходимое условие экстремума состоит в том, что $ \forall h \in L_0\colon \partial _h J(f) = 0$.

Заметим, что
\[
\begin{aligned}
	\partial _h J(f) &= \int_{a}^{b} \left( \partial _2F(x, f(x), f'(x)) - \frac{d}{dx} \Bigl( \partial _3 F(\ldots )\Bigr) \right)h(x) dx + \partial _3 F(\ldots ) h(x) \Bigm|_{a}^{b} = \\
					 &=\int_{a}^{b} g(x)h(x) dx = 0
\end{aligned}
\]
Так как это равенство верно для всех $ h$ из $ L_0$, $ g(x) = 0$: пусть  $ g(x') \ne 0$. Тогда по непрерывности $ g(x) \ne 0$ в некоторой окрестности $ x'$. Тогда существует $ h$ такое, что  $ h(x) \ne  0$ только в этой окрестности $ x'$, поэтому
 \[
	 \int_{a}^{b} g(x)h(x)dx \ne 0 
.\] 
\begin{figure}[ht]
    \centering
    \incfig{proff-g-is-zero}
    \label{fig:proff-g-is-zero}
\end{figure}

Следовательно, $ f$ --- экстремум. Тогда  \[
	\partial _2 F(x, f(x), f'(x)) - \frac{d}{dx}\partial _3 f(x, f(x), f'(x)) = 0, ~ f(a) = A , ~ f(b) = B
.\] 
Полученное дифференциальное уравнение от $ f$ называется  {\sf уравнением Эйлера-Лагранжа}.  
\paragraph{Применим для решения первоначальной задачи}
\[
	F(u_1, u_2, u_3) = F(x, f(x), f'(x)) = \sqrt{ \frac{1+f'(x)^2}{x}} 
.\] 
Тогда $ \partial _2 F(\ldots )$  в уравнении просто равно 0, а
\[
	\partial _3 F(\ldots ) = \frac{f'(x)}{\sqrt{ x} \cdot \sqrt{ 1 + f'(x)^2}  }
.\] 
Поэтому 
\[
	\left( \frac{f'(x)}{\sqrt{ x} \cdot \sqrt{ 1 + f'(x)^2} } \right)' = 0, \quad f(x) = y', \quad f(0) = 0
.\] 
Следовательно, 
\[
	\frac{f'(x)}{\sqrt{ x} \cdot \sqrt{ 1+f'(x)^2} } = c
.\] 
Возведем в квадрат и получим, что
\[
	\frac{x}{c^2} = \frac{1}{f'(x)^2} +1 \Longrightarrow \frac{const -x}{x} = \frac{1}{f'(x)^2} \Longrightarrow f'(x) - \sqrt{ \frac{x}{const-x}} 
.\] 
Таким образом,
\[
	f(x) = \int_{0}^{x} \sqrt{ \frac{s}{const - s}}dx  
,\] 
при этом $ const$ можно подобрать так, что $ f(x^{*}) = y^{*}$. Это {\sf циклоида}.  
\begin{figure}[ht]
    \centering
    \incfig{cycloid}
    \caption{Циклоида}
    \label{fig:cycloid}
\end{figure}
\end{proof}
% \end{document}
