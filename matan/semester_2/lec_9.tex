% \documentclass[11pt,dvipsnames]{report}
% \usepackage[english, russian]{babel}
\usepackage{xltxtra}
\usepackage{polyglossia}

\usepackage{mathpazo}

\defaultfontfeatures{Ligatures=TeX,Mapping=tex-text}

\setmainfont{STIX2Text-Regular.otf}[
ExternalLocation={/home/vyacheslav/builds/STIXv2.0.2/OTF/},
BoldFont=STIX2Text-Bold.otf,
ItalicFont=STIX2Text-Italic.otf,
BoldItalicFont=STIX2Text-BoldItalic.otf
]
\setmathrm{STIX2Math.otf}[
ExternalLocation={/home/vyacheslav/builds/STIXv2.0.2/OTF/}
]


\usepackage{makeidx}
\usepackage{amssymb, amsthm}
\usepackage{amsmath}
\usepackage{mathtools}
\usepackage{needspace}
\usepackage{enumitem}
\usepackage{cancel}
\usepackage{fdsymbol}
\usepackage{fontawesome}


% разметка страницы и колонтитул
\usepackage[left=2cm,right=2cm,top=1cm,bottom=1.1cm,bindingoffset=0cm]{geometry}
\usepackage{fancybox,fancyhdr}
\fancyhf{}
\fancyhead[R]{\thepage}
\fancyhead[L]{\rightmark}
\fancyfoot{}
\fancyhfoffset{0pt}
\addtolength{\headheight}{13pt}
\pagestyle{fancy}

% Отступы
\setlength{\parindent}{3ex}
\setlength{\parskip}{3pt}

\usepackage{graphicx}
\usepackage{hyperref}

\usepackage{import}
\usepackage{xifthen}
\usepackage{pdfpages}

\newcommand{\incfig}[1]{%
    \def\svgwidth{\columnwidth}
    \import{./figures/}{#1.pdf_tex}
}


\usepackage{xifthen}
\makeatother
\def\@lecture{}%
\newcommand{\lecture}[3]{
    \ifthenelse{\isempty{#3}}{%
        \def\@lecture{Лекция #1}%
    }{%
        \def\@lecture{Лекция #1: #3}%
    }%
    \subsection*{\@lecture}
    \marginpar{\small\textsf{\mbox{#2}}}
}
\makeatletter


\usepackage{xcolor}
\definecolor{Aquamarine}{cmyk}{50, 0, 17, 100}
\definecolor{ForestGreen}{cmyk}{76, 0, 76, 45}
\definecolor{Pink}{cmyk}{0, 100, 0, 0}
\definecolor{Cyan}{cmyk}{56, 0, 0, 100}
\definecolor{Gray}{gray}{0.3}


\usepackage{mdframed}
\mdfsetup{skipabove=3pt,skipbelow=3pt}
\mdfdefinestyle{defstyle}{%
    linecolor=red,
	linewidth=3pt,rightline=false,topline=false,bottomline=false,%
    frametitlerule=false,%
    frametitlebackgroundcolor=red!0,%
    innertopmargin=4pt,innerbottommargin=4pt,innerleftmargin=7pt
    frametitlebelowskip=1pt,
    frametitleaboveskip=3pt,
}
\mdfdefinestyle{thmstyle}{%
    linecolor=cyan!100,
	linewidth=2pt,topline=false,bottomline=false,%
    frametitlerule=false,%
    frametitlebackgroundcolor=cyan!20,%
    innertopmargin=4pt,innerbottommargin=4pt,
    frametitlebelowskip=1pt,
    frametitleaboveskip=3pt,
}
\theoremstyle{definition}
\mdtheorem[style=defstyle]{defn}{Определение}

\newmdtheoremenv[nobreak=true,backgroundcolor=Aquamarine!10,linewidth=0pt,innertopmargin=0pt,innerbottommargin=7pt]{cor}{Следствие}
\newmdtheoremenv[nobreak=true,backgroundcolor=CarnationPink!20,linewidth=0pt,innertopmargin=0pt,innerbottommargin=7pt]{desc}{Описание}
\newmdtheoremenv[nobreak=true,backgroundcolor=Gray!10,linewidth=0pt,innertopmargin=0pt,innerbottommargin=7pt,font={\small}]{ex}{Пример}
\newmdtheoremenv[nobreak=false,backgroundcolor=Cyan!10,linewidth=0pt,innertopmargin=0pt,innerbottommargin=7pt]{thm}{Теорема}
\newmdtheoremenv[nobreak=true,backgroundcolor=Pink!10,linewidth=0pt,innertopmargin=0pt,innerbottommargin=7pt]{lm}{Лемма}

\newtheorem*{st}{Утверждение}
\newtheorem*{prop}{Свойства}

\theoremstyle{plain}
\newtheorem*{name}{Обозначение}

\theoremstyle{remark}
\newtheorem*{rem}{Ремарка}
\newtheorem*{com}{Комментарий}
\newtheorem*{note}{Замечание}
\newtheorem*{prac}{Упражнение}
\newtheorem*{probl}{Задача}


\renewcommand{\proofname}{Доказательство}
\renewenvironment{proof}
{ \hspace{\stretch{1}}\\ \faSquareO\quad \small  }
{ \hspace{\stretch{1}}  \faSquare \normalsize }


\numberwithin{ex}{section}
\numberwithin{thm}{section}
\numberwithin{equation}{section}



\newcommand{\K}{\mathcal{K}}
\newcommand{\Z}{\mathbb{Z}}
\newcommand{\N}{\mathbb{N}}
\newcommand{\Real}{\mathbb{R}}
\newcommand{\Q}{\mathbb{Q}}
\newcommand{\Cm}{\mathbb{C}}
\newcommand{\Pm}{\mathbb{P}}
\newcommand{\ord}{\operatorname{ord}}
\newcommand{\lcm}{\operatorname{lcm}}
\newcommand{\sign}{\operatorname{sign}}
\newcommand{\E}{\mathbb{E}}

\renewcommand{\o}{o}
\renewcommand{\O}{\mathcal{O}}
\renewcommand{\le}{\leqslant}
\renewcommand{\ge}{\geqslant}

\def\mybf#1{\textbf{#1}}
\def\selectedFont#1{\textbf{#1}}
\def\ComplexityFont#1{\textmd{\textbf{\textsf{#1}}}}
\def\LanguageFont#1{{\textbf{\texttt{#1}}}}


\newcommand{\Cclass}{\mathcal{C}}
\newcommand{\Dclass}{\mathcal{D}}


\renewcommand{\P}{\ComplexityFont{P}}
\newcommand{\DTIME}{\ComplexityFont{DTime}}
\newcommand{\DTime}{\ComplexityFont{DTime}}
\newcommand{\DSpace}{\ComplexityFont{DSpace}}
\newcommand{\PSPACE}{\ComplexityFont{PSPACE}}
\newcommand{\NTIME}{\ComplexityFont{NTime}}
\newcommand{\NSpace}{\ComplexityFont{NSpace}}
\newcommand{\coNSpace}{\ComplexityFont{coNSpace}}
\newcommand{\NPSPACE}{\ComplexityFont{NPSPACE}}
\newcommand{\poly}{\ComplexityFont{poly}}
\newcommand{\RP}{\ComplexityFont{RP}}
\newcommand{\coRP}{\ComplexityFont{co-RP}}
\newcommand{\ZPP}{\ComplexityFont{ZPP}}
\newcommand{\BPP}{\ComplexityFont{BPP}}
\newcommand{\BQP}{\ComplexityFont{BQP}}
\newcommand{\coBPP}{\ComplexityFont{co-BPP}}
\newcommand{\NP}{\ComplexityFont{NP}}
\newcommand{\NL}{\ComplexityFont{NL}}
\newcommand{\coNL}{\ComplexityFont{co-NL}}
\renewcommand{\L}{\ComplexityFont{L}}
\newcommand{\NPcomp}{\ComplexityFont{NP-complete}}
\newcommand{\tP}{\widetilde{\P}}
\newcommand{\tNP}{\widetilde{\NP}}
\newcommand{\tBH}{\widetilde{\BH}}
\newcommand{\Class}{{\ComplexityFont{C}}}
\newcommand{\coC}{\ComplexityFont{co-}\mathcal{C}}
\newcommand{\coNP}{\ComplexityFont{co-NP}}
\newcommand{\PH}{\ComplexityFont{PH}}
\newcommand{\EXP}{\ComplexityFont{EXP}}
\newcommand{\Size}{\ComplexityFont{Size}}
\newcommand{\Ppoly}{\ComplexityFont{P}/\ComplexityFont{poly}}
\newcommand{\NC}{\ComplexityFont{NC}}


\newcommand{\FACTOR}{\LanguageFont{FACTOR}}
\newcommand{\kQBF}{{\LanguageFont{QBF{\tiny k}}}}
\newcommand{\QBFk}{{\LanguageFont{QBF{\tiny k}}}}
\newcommand{\QBF}{{\LanguageFont{QBF}}}
\newcommand{\STCON}{\LanguageFont{STCON}}
\newcommand{\USTCON}{\LanguageFont{USTCON}}
\newcommand{\CircuitSat}{{\LanguageFont{CIRCUIT\_SAT}}}
\newcommand{\tCircuitSat}{\widetilde{{\LanguageFont{CIRCUIT\_SAT}}}}
\newcommand{\SAT}{\LanguageFont{SAT}}
\newcommand{\tSAT}{\widetilde{{\LanguageFont{SAT}}}}
\newcommand{\UNSAT}{{\LanguageFont{UNSAT}}}
\newcommand{\tThreeSAT}{\widetilde{{\LanguageFont{3\text{-}SAT}}}}
\newcommand{\ThreeSAT}{{\LanguageFont{3\text{-}SAT}}}
\newcommand{\BH}{\LanguageFont{BH}}
\newcommand{\CircuitEval}{{\LanguageFont{CIRCUIT\_EVAL}}}


\newcommand{\const}{\textmd{const}}
\newcommand{\logspace}{\textmd{logspace}}
\newcommand{\PATH}{\textmd{PATH}}


\newcommand{\readonly}{\textsf{read-only}}
\newcommand{\writeonly}{\textsf{write-only}}


\usepackage{ upgreek }
\newcommand{\PI}{\Uppi}
\newcommand{\SIGMA}{\Upsigma}
\newcommand{\DELTA}{\Updelta}

% \begin{document}
%
\lecture{9}{10 Apr}{\dag}
\section{Поверхности и криволинейные координаты}
\begin{description}
    \item[Поверхность-график] Пусть $ f \colon U \subset \R^2 \to  \R$ --- непрерывная функция на открытом множестве, график функции, поверхность ---
	\[
	    S = \Gamma _f = \{(x, y, z) \mid z = f(x, y), ~ (x, y) \in U \}
	.\]
    \item  [Параметризация] Отображение $ F\colon U \to  S$, такое что $ F(x, y) = (x, y, f(x, y))$ --- непрерывное, биектинвное отображениoе
    \item [Пути на $ S$] Если  $ \gamma \colon [a, b] \to  U$ --- путь в $ U$, то  $ F \circ \gamma $ -- путь в $ S$, и наоборот.
    \item [Криволинейные координаты на $ S$]  $ (x, y) $ выполняют роль координат на  $ S$. Образы координатных линий --- координатные кривые на  $ S$.
\end{description}
\subsection{Касательная плоскость к графику функции}
\begin{itemize}
    \item Пусть $ f$  дифференцируемо в точке $ (x_0, y_0) \in U$. Тогда
	\[
	    f(x, y) = f(x_0, y_0) + A(x-x_0)+B(y-y_0)+\o(\ldots ), \qquad (x, y) \to  (x_0, y_0)
	.\]
	\[
	    df(x_0, y_0) = \left( \partial _xf(x_0, y_0), \partial _yf(x_0, y_0) \right)
	.\]
    \item Множество точек  $ (x, y, z) \in \R^3$, удовлетворяющий уравнению
	\[
	    z = f(x_0, y_0) + A(x-x_0) + B(y-y_0)
	.\]
	называется {\sf касательной плоскостью} к $ S$ в точке  $ (x_0, y_0, f(x_0, y_0)$.
    \item  Эта плоскость единственна и
	\[
	    A= \partial _xf(x_0, y_0) , \qquad B= \partial _yf(x_0, y_0)
	.\]
    \item Нормаль к плоскости
	\[
	    n = \left( \partial _xf(x_0, y_0), \partial _yf(x_0, y_0) \right) = \left( \nabla f(x_0, y_0), -1 \right)
	.\]
\end{itemize}
\subsection{Касательный вектор}
\begin{itemize}
    \item  Если гладкий путь в $ \Gamma \colon [a,b] \to  \R^3, ~\Gamma (t) = \left( x(t), y(t), z(t) \right)  $, то касательный вектор к нему это $ \left( x'(t), y'(t),z'(t) \right) $.  Если путь лежит на поверхности $ S$, то есть $ \Gamma = F \circ \gamma $, то
	\[
	    \Gamma '(t) = \left( x'(t), y'(t), \partial _xf(x(t), y(t) + \partial _yf(x(t), y(t))y'(t) \right)
	.\]
    \item Касательный вектор к пути на поверхности перпендикулярен нормали и лежит в касательной плоскости.

	Уравнение нормали
	\[
	    n=\left( \partial _xf(x_0, y_0), \partial _y(x_0, y_0), -1 \right)
	.\]
    \item Верно и обратное:  любой вектор из касательной плоскости является касательным к некоторому пути на поверхности.
	\[
	    (u, v, w) \bot n \qquad x(t) = x_0+ ut, ~y(t) =y_0+vt \text{ --- путь в } U
	.\]
	\[
	    \Gamma (t) = (x(t), y(t), f(x(t), y(t)))
	.\]
	Продифференцировав это, мы получим равенство выше.
\end{itemize}
\subsection{Чуть более общая ситуация}
\begin{itemize}
    \item Если $ f\colon U \subset \R^{n} \to  \R^{m} , ~ f= (f_1, \ldots f_m)$, то получим график отображения
	\[
	    S = \Gamma _f = \{(x, y) \in \R^{n} \times \R^{m} \mid x \in U, y \in f(x)\}
	\]
	--- $ n$-мерная поверхность в  $ \R^{n+m}$.
    \item $ F\colon U \to  S$, $ F(x) - (x, f(x)$ --- параметризация поверхности.
    \item Касательное пространство $ n$-мерно и состоит из касательных векторов.
    \item Пространство нормалей  $ m$-мерное.
\end{itemize}
\section{Теорема о неявном отображении (функции)}
\subsection{Мотивация}
\begin{itemize}
    \item
	Рассмотрим множество $ \{x^2+y^2-1= 0\}$ --- окружность на плоскости. Это не график функции $ y = f(x)$, но почти для всех точек можем взять окрестность, которая будет графиком.
    \item  Можно честно решить относительно $ y$ уравнение  $ y = \pm \sqrt{1-x^2 } $

	TODO: Дописать $\circlearrowleft$
\end{itemize}

\subsection{Подстановка}

\begin{itemize}
    \item Пусть задана система уравнений
	\[
	    \begin{cases}
		f_1(x_1, \ldots x_n, y_1, \ldots y_m) = 0 \\
		\vdots
		\\
		f_m(x_1, \ldots x_n, y_1, \ldots y_m) = 0
	    \end{cases}
	.\]
    \item  Хотим разрешить относительно $ y = (y_1, \ldots y_n)$
	\[
	    \begin{cases}
		y_1= g_1(x_1, \ldots x_{n})\\
		\vdots\\
		y_m = g_m(x_1, \ldots x_n)
	    \end{cases}
	.\]
    \item Тем самым, получить задание $ m$-мерной поверхности в   $ \R^{m+n}$.
\end{itemize}
\section{Теорема о неявном отображении}
\begin{thm}[О неявном отображении]
    $ $
    \begin{itemize}[noitemsep]
	\item Пусть $ X, Y, Z$ --- нормированные пространства,   $ Y$--- полное, $ (x_0, y_0) \in W \subset X \times Y$.
	\item Отображение непрерывно $ F \colon W \to  Z$ в точке $ (x_0, y_0)$, $ F(x_0, y_0) = 0$
	    \item В $ W$ существует частный дифференциал  $ F$ по  $ y$ ( $ \exists  \partial _y F \colon W \to  L(Y, Z)$) и непрерывно в точке $ (x_0, y_0)$.
	    \item  Оператор обратим $ (\partial _yF(x_0, y_0) )^{-1} \in  L(Z, Y)$
    \end{itemize}
    Тогда существуют $ U \subset X$ --- окрестность точки $ x_0$, $ V \subset Y$ --- окрестность точки $ y_0$, $ f\colon U \to  V$ такие, что $ U \times V \subset W$ и
    \[
	\{F(x, y) = 0\} \cap (U \times V) = \Gamma _f = \{(x, f(x)) \mid x \in U\}
    .\] 
\end{thm}
\begin{proof}
    Пусть $ (x_0, y_0) = (0, 0)$
    \begin{enumerate}[noitemsep]
	\item Пусть $ g_x (y) = y - \left( \partial _yF(0, 0) \right)^{-1}F(x, y) , \quad g_X\colon Y \to  Y $.
	    \[
		F(x, y) = 0 \Longleftrightarrow y \text{ --- неподвижная точка  } g_X
	    .\] 
	    Докажем это. Нужно выделить подмножество $ Y$, где отображение действует.
	     \[
		 d g_x(y) = I_Y - \left( \partial _yF(0, 0) \right)^{-1}\partial _y F(x, y)
	    .\] 
	    Если $ (x, y)$ стремиться к  $ (0, 0)$, то последнее слагаемое будет стремиться к тождественному отображению  $ I_Y$, то есть правая часть равенства стремиться к 0. 

	    \[
		\exists  \Delta > 0\colon \| x \|  < \Delta , \| y \| < \Delta  \Longrightarrow \| d g_{x}(y) \|  <\frac{1}{2}
	    .\] 
	    Возьмем $ \Delta > \varepsilon > 0$. $ g_0(0) = 0$
	     \[
		 \exists \delta >0 ~ \forall x , \|  x  \| < \delta \colon \| g_x(0) \| \le \varepsilon /2  
	    .\] 

	\item {\bf Ключевой момент:}  
	    так как производные меньше $ \frac{1}{2}$, и $ \|g_x(0)\| \le \varepsilon /2$
	    \[
		g_x\left(\{\| y \| \le \varepsilon  \}\right) \subset \{\| y \| \le \varepsilon  \}
	    .\] 
	    Применим теорему о сжимающем отображении
	    \[
		y \colon \|  y \| \le \varepsilon, \quad	\| g_x(y) - g_y(x) \|  \le \sup_{0 < \Theta < 1} \|  dg_x(\Theta) \| \cdot \| y \| \le  \frac{\varepsilon}{2}
	    .\] 
	    Так как $ Y$ полное, шар $ M$, где действует $ g$, является метрическим,отображение $ g_x$ сжимающее. Следовательно, существует единственная неподвижная точка
	     \[
		 \exists ! ~y\colon \| y \| \le \varepsilon ,   g_x(y) = y
	     .\] 
	     Рассмотрим $ U = B_{\delta }(0)$. Оно подойдет.
    \end{enumerate}
\end{proof}
\begin{note}
    Отображение $ f$ непрерывно в точке $ x_0$.
\end{note}
\begin{note}
    Если случай конечномерный, то достаточно требовать только обратимость (без непрерывности)
\end{note}
\begin{note}
    $
    \begin{pmatrix}
        \partial f_k \\ \partial y_j
    \end{pmatrix}
    $ --- обратимая матрица,  то есть ее определитель не 0.
\end{note}
\begin{thm}
    Если в условиях прошлой теоремы отображения  $ F, ~\partial _yF$ непрерывны не только в точке $ (x_0, y_0)$, но в целой окрестности, то $ f$  непрерывно в окрестности точки  $ x_0$
\end{thm}
\begin{proof}
    Хотим проверить, что $ \exists ~ (d_y D(x, y))^{-1} \in L(Z, Y)$, при $ (x, y)$ близких к $ (x0, y0)$. Уже знаем, что $ \exists ~\left( \partial _yF(x_0, y_0) \right)^{-1} \in L(Z, Y)$.
    \begin{lm}[об обратимости оператора близкого к тождественному]\label{lm_ob:2}
	$ Y$ --- полное,  $ B \in L(Y, Y)$, $ \| B \|  \le  1 $. Тогда $ \exists ( I - B)^{-1} \in L(Y, Y)$.
    \end{lm}
    \begin{proof}
	Докажем, что
        \[
	    \forall v \in ~ \exists ! ~ u \in Y \colon (I-B) u = v  
        .\] 
	Последнее утверждение равносильно тому, что
	\[
	    u = c + Bu \quad g_v(u) = v + Bu
	.\] 
	Это сжимающее отображение так как
	\[
	    \|  g_v(u_1) - g_v(u_2) \| = \| Bu_1 - Bu_2 \| \le \| B \| \cdot \| u_1-u_2 \| 
	.\] 
	Тогда по теореме сжимающем отображении существует неподвижная точка.

	\[
	v_n \to  v_0j \Longrightarrow u_n \to  u, ~ u_n = v_n + B u_n \wedge  u_0  = v_0 + Bu_0
	.\] 
	$ u_n  - u_0 = v_n - v_0 + B\left( u_n - u_0 \right) $
	\[
	\| u_n - u_0 \| \le \| v_n - v_0 \| + \| B \| \cdot \|  u_n-u_0 \|    
	.\] 
	\[
	\|  u_n - u_0 \| \le \frac{1}{1-\|  B \| }  \|  v_n - v_0 \| \to  0
	.\] 
    \end{proof}
    \begin{lm}[об обратимости обератора близкого к обратимому]\label{lm_obr:1}
	$ Y $ --- полное пространство. $ A, A_0 \in  L(Y, Z)$, $ \exists  A_0^{-1} \in L(Z, Y)$. Если $ \| A-A_0 \| < \frac{1}{\| A_0^{-1} \| }$, то $ A^{-1} \in L(Z, Y)$ 
    \end{lm}
    \begin{proof}
	Применяем лемму \ref{lm_ob:2}
	\[
	\underbrace{A}_{L(Y, Z)} = A_0 + A - A_0 = \underbrace{A_0}_{\text{обратимо и }L(Y, Z)} \underbrace{(I_Y+A_0^{-1}(A-A_0))}_{\text{обратимо и }L(Y, Y)} , \quad \| B \| \le \| A-A_0 \| \cdot \|  A_0^{-1} \| <1 
	.\] 
    \end{proof}
    По лемме \ref{lm_obr:1} получаем утверждение теоремы.
\end{proof}
\begin{thm}
    Если в условиях теоремы $ 1$ отображения  $ F$ дифференцируемо в точке  $ (x_0, y_0)$, то и $ f$ дифференцируемо в точке  $ x_0$ и 
    \[
	df(x_0) = -(\partial )yF(x_0, y_0))^{-1} \partial _x F(x_0, y_0)
    .\] 
\end{thm}
\begin{proof}
    Пусть $ (x_0, y_0) = (0, 0)$.
    \[
	F(x, y) = F(0, 0) - \partial _xF(0, 0)x + \partial _yF(0, 0) y +\underbrace{\o(\| x \|+ \|  y \| ) }_{\alpha (x, y)}
    .\] 
    Знаем, что $ F(x, y) = 0 \Longleftrightarrow y = f(x)$.
    \[
	0 = \partial _xF(0, 0) x + \partial _y F(0, 0)f(x) + \alpha (x, y)
    .\] 
    \[
	f(x) = -\left( \partial _y F(0, 0) \right)^{-1}\partial _x F(0, 0) x - (\partial _y F(0, 0))^{-1} \underbrace{\alpha (x, f(x))}_{?= \o(\| x \| )}
    .\] 
    Если $ x \to  0$, $ f(x) \to  0$.
    \[
	\exists \delta > 0\colon \|  x  \| < \delta \Longrightarrow \frac{ \|  \alpha (x, f(x))  \|}{\| x \| + \|  f(x) \| }  \le \frac{1}{\|  d_yF(0, 0)^{-1} \| \cdot \frac{1}{2} }
    .\] 
    \[
	\|  \partial _yF(0, 0)^{-1} \alpha (x, f(x)) \| \le \frac{1}{2}\left( \|  x \| + \|  f(x) \|  \right)  
    .\] 
    \[
	\| f(x) \| \le C \| x \| + \frac{1}{2} \left( \| x \| + \| f(x) \|   \right)  
    .\] 
    \[
	\frac{1}{2}\|  f(x) \| \le  C \|  x \| + \frac{1}{2}  \| x \| \Longrightarrow  \| f(x)  \|\le \widetilde{ c} \| x \|  \Longrightarrow  \o(\| x \|+\| f(x) \|) = \o(\| x \| )  
    .\] 
\end{proof}
\begin{note}
    Можно попросить большую дифференцируемость $ F$ и получить большую дифференцируемость  $ f$.  Аналогично можно попросить дифференцируемость в окрестности и получить дифференцируемостьв окрестности.
\end{note}
\begin{thm}[об обратном отображении]
    Пусть  $ F\colon W \subset Y \to  X$, $ Y$--- полно, $ F(y_0) = x_0$, $ F$ дифференцируемо в  $ W$, $ dF$ непрерывна в точке   $ y_0$, и существует  $ \left( dF(y_0) \right)^{-1} \in L(X, Y) $. Тогда существуют окрестности $ U \subset W$ точки $ x_0 $ и $ V$ точка  $ y_0$ такие, что $ F \colon V \to  U$ --- биекция, то есть существует $ F^{-1} \colon U \to  V$, $ F ^{-1}$ --- дифференцируемо  в точке $ x_0$ и 
    \[
	dF^{-1}(x_0) = \left( dF(y_0) \right)^{-1}
    .\] 
\end{thm}
\begin{proof}
    Рассмотрим $ G(x, y) = X - F(y), \quad F\colon X \times Y \to  X$.
    Заметим, что $ G(x, y) = 0 \Longleftrightarrow x = F(y)$. 
    $ G(x_0, y_0) = 0$.

    \[
	\partial _yG(x_0, y_0) = -dF(y_0) \text{ --- обратимо}
    .\] 
    \[
	\exists (\partial _yG(x_0, y_0))^{-1} \in L(Y, X)
    .\] 
    По теореме о неявной функции получаем, что существует 
    \[
	f\colon U \to  V \qquad G(x, f(x) ) = 0 \Longleftrightarrow x - F(f(x))= 0
    .\] 
    И  $ f = F^{-1}$ на  $ U$.
    \[
	dF^{-1}(y_0) = df(y_0) = \ldots = dF(y_0))^{-1}
    .\] 
\end{proof}
\begin{note}
    Можно попросить большую дифференцируемость $ F$ и получить большую дифференцируемость  $ f$.
\end{note}
% \end{document}
