\documentclass[12pt]{report}
\usepackage [utf8] {inputenc}
\usepackage [T2A] {fontenc}
\usepackage {amsfonts}
% \usepackage{eufrak}
\usepackage{amssymb, amsthm}
\usepackage{amsmath}
\usepackage{mathtools}
\usepackage{needspace}
\usepackage{etoolbox}
\usepackage{lipsum}
\usepackage{comment}
\usepackage{cmap}
\usepackage[pdftex]{graphicx}
\usepackage{hyperref}
\usepackage{epstopdf}

\usepackage{import}
\usepackage{xifthen}
\usepackage{pdfpages}
\usepackage{transparent}

\newcommand{\incfig}[1]{%
    \def\svgwidth{\columnwidth}
    \import{./figures/}{#1.pdf_tex}
}


\pagestyle{plain}

\usepackage{fullpage}
% \usepackage[left=15mm,top=15mm,left=15mm,bottom=30mm,nohead,nofoot]{geometry}

\begin{document}
\renewcommand{\proofname}{Доказательство}

\theoremstyle{plain}
\newtheorem{thm}{Theorem}[section]
\newtheorem*{aks}{Аксиома}[section]
\newtheorem*{lm}{Lemma}
\newtheorem*{st}{Statement}
\newtheorem*{prop}{Properties}

\theoremstyle{definition}
\newtheorem{defn}{Def}
\newtheorem*{ex}{Example}
\newtheorem*{exs}{Examples}
\newtheorem*{cor}{Corollary}
\newtheorem*{name}{Name}

\theoremstyle{remark}
\newtheorem*{rem}{Remain}
\newtheorem*{note}{Note}
\newtheorem*{probl}{Exercise}

\newcommand{\Z}{\mathbb{Z}}
\newcommand{\N}{\mathbb{N}}
\newcommand{\R}{\mathbb{R}}
\newcommand{\Q}{\mathbb{Q}}
\newcommand{\K}{\mathbb{K}}
\newcommand{\Cm}{\mathbb{C}}
\newcommand{\Pm}{\mathbb{P}}
% \newcommand{\Zero}{\mathbb{O}}
\newcommand{\ilim}{\int\limits}
\newcommand{\slim}{\sum\limits}

\title{Конспект по матанализу в формате вопросов коллоквиума \\ (лекции Кислякова Сергея Витальевича)}                      
\maketitle
\clearpage
\tableofcontents
\clearpage
\chapter{Введение}
\section{Простейшие свойства вещественных чисел}
\begin{enumerate}
    \item Алгебраические операции
	\begin{enumerate}
	    \item сложение $a, b \in \R$ : сумма $a+b$ определяется единственным образом
		\begin{enumerate}
		    \item  $a+b = b+a$ (коммутативность)
		    \item  $(a+b)+c = a+(b+c)$ (ассоциативность)
		    \item  $\exists 0: a +0 = a, \forall a \in \R$ (нейтральный по сложению)
		    \item  $\forall a \in \R \exists a': a +a' = a' + a = 0 $ (обратный по сложению)
		\end{enumerate}
	    \item умножение $x,y \in \R$ : произведение $x\cdot y$ определяется единственным образом
		\begin{enumerate}
		    \item  $x y = y x$ (коммутативность)
		    \item  $(xy)z = x(yz)$ (ассоциативность)
		    \item  $\exists 1: x \cdot 1 = x, \forall x \in \R$ (нейтральный по умножению)
		    \item  $x(a+b) =xa + xb$ (дистрибутивность)
		    \item  $\forall x\ne 0 \in \R \exists y  \stackrel{def} = x^{-1}: xy = 1$ (обратный по умножению)
		\end{enumerate}
	\end{enumerate}
    \item Порядок на $\R$
	\begin{defn}
	    Упорядоченная пара $(u, v) = \{\{u\}, \{u, v\}\}$ .
	\end{defn}
	\begin{defn}
	    Декартово произведение $X \times Y = \{(x, y) \mid \forall x \in X, y \in Y\}$.
	\end{defn}
	\begin{defn}
	    Отношение между элементами множеств $X, Y$ - $A \subset X \times Y$
	\end{defn}
	Отношения порядка: $a < b, a>b, a=b$
	 \begin{enumerate}
	    \item $\forall a, b \in \R: \left [ 
		    \begin{matrix}
		        a = b \\ a > b \\ a< b
		    \end{matrix}
		\right. $ (антисимметричность)
	    \item $a<b \wedge b < c \Rightarrow a < c$ (транзитивность)
	    \item $a<b \wedge c \in \R \Rightarrow a + c < b + c$ 
	    \item $a<b \wedge c > 0 \Rightarrow ac < bc$ 
	    \item $u < v \wedge x < y \Rightarrow u+x  < v + y$ 
	\end{enumerate}
\end{enumerate}
\section{Множества на $\R$ }
\begin{defn}[Отрезки, интервалы, сегменты] 
    $a, b \in \R, a \le b$
    $$
    [a, b] = \{a \in \R \mid a \le x \le b\} \mbox{(замкнутый отрезок)}
    $$
    $$
    (a, b] = \{a \in \R \mid a < x \le b\} \mbox{(открытый слева отрезок)}
    $$
    $$
    [a, b) = \{a \in \R \mid a \le x < b\}  \mbox{(открытый справа отрезок)}
    $$
    $$(a, b) = \{a \in \R ~|~ a < x < b\} \mbox{(открытый отрезок)}$$
\end{defn}

\begin{defn}[Лучи] $a \in \R$
$$[a, +\infty) = \{x \in \R \mid x \ge a\}$$
$$(a, +\infty) = \{x \in \R \mid x > a\}  $$
$$(-\infty, a] = \{x \in \R \mid x \le a\}$$
$$ (-\infty, a) =\{x \in \R \mid x < a\}$$
\end{defn}

\begin{defn}$ $

Множество $A \subseteq \R$ ограничено сверху, если $\exists ~x \in \R: a \le x ~\forall a \in A$. Любое такое $x$ - верхняя граница      $A$.

Множество $A \subseteq \R$ ограничено снизу, если $\exists ~y \in \R: a \ge y ~\forall a \in A$. Любое такое $y$ - нижняя граница $     A$.

//$\pm\infty$ - не нижняя/верхняя граница.

Ограниченное множество - ограниченное сверху и снизу. 
\end{defn}

\begin{aks}[Архимед]
    Множество натуральных чисел не ограниченно сверху.
\end{aks}
\begin{lm}
    $x > 0 \Rightarrow \exists~n \in \N: \frac{1}{n} < x$
\end{lm}
\begin{proof}
    Предположим противное. $\forall n \in \N: x \le \frac{1}{n}$. Тогда $\forall n: n < x^{-1}$, а это противоречит аксиоме Архимеда.
\end{proof}
\end{document}
