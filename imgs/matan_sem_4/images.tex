\documentclass[12pt,a4paper]{article}

\usepackage [utf8] {inputenc}
\usepackage [T2A] {fontenc}
\usepackage {amsfonts}
\usepackage{amssymb, amsthm}
\usepackage{amsmath}
\usepackage{mathtools}
\usepackage{needspace}
\usepackage{cmap}
\usepackage[pdftex]{graphicx}
\usepackage{fullpage}
\usepackage{textcomp}

% figure support
\usepackage{import}
\usepackage{xifthen}
\pdfminorversion=7
\usepackage{pdfpages}
\usepackage{transparent}
\newcommand{\incfig}[1]{%
    \def\svgwidth{\columnwidth}
    \import{./figures/}{#1.PDF_Tex}
}

\def\ml{\mathcal}

\begin{document}
\section{Лекция 1}
\begin{figure}[ht]
    \centering
    \incfig{example-orientation}
    \caption{Однорукий червяк на Земле и на ленте Мёбиуса}
    \label{fig:example-orientation}
\end{figure}

\begin{figure}[ht]
    \centering
    \incfig{smooth-manifold}
    \label{fig:smooth-manifold}
\end{figure}

\begin{figure}[ht]
    \centering
    \incfig{matched-maps}
    \label{fig:matched-maps}
\end{figure}

\begin{figure}[ht]
    \centering
    \incfig{orientation}
    \caption{Ориентация через нормали}
    \label{fig:orientation}
\end{figure}

\begin{figure}[ht]
    \centering
    \incfig{proof-second-theorem}
    \label{fig:proof-second-theorem}
\end{figure}
	
\end{document}
