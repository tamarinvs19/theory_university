\lecture{5}{29 April}{\dag}
\chapter{Коммуникационная сложность}
Пусть у нас есть два игрока: Алиса и Боб. Они могут отправлять друг другу сообщения и хотят посчитать функцию (или отношение) $ f \colon X \times Y \to Z$.
Будем говорить, что они \textit{решают коммуникационную задачу} для функции $ f$, если:
\begin{enumerate}
	\item множества $ X, Y, Z$ и функция $ f$ известны обоим игрокам,
	\item	Алиса знает некоторое $x \in  X$,
	\item	Боб знает некоторое $ y \in Y$,
	\item	Алиса и Боб стремятся вычислить $ f(x, y)$.
\end{enumerate}

\section{Детерминированная модель}
\begin{defn}[Коммуникационный протокол]
	\selectedFont{Коммуникационный протокол} для функции $ f\colon X\times Y \to  Z$ --- корневое двоичное дерево, которое описывает совместное вычисление Алисой и Бобом функции $ f$. В этом дереве:
	   \begin{itemize}
		   \item каждая внутренняя вершина $ v$ помечена меткой $ a$ или $ b$, означающей очередь хода Алисы или Боба соответственно;
		   \item для каждой вершины $ v$, помеченной $ a$, определена функция  $ g_v\colon X \to \{0, 1\}$, которая задает бит, который Алиса отправит  Бобу, если вычисление будет находится в $ v$; аналогично для каждой вершины $ v$ с пометкой $ b$ определена функция  $ h_v\colon Y \to \{0, 1\}$ для сообщений Боба;
		\item каждая внутренняя вершина имеет двух потомков, ребро к первому помечено нулем, ко второму --- единицей;
		\item каждый лист помечен значением из множества $ Z$.
	   \end{itemize}
	   Каждая пара входов $ (x, y)$ определяют путь от корня до листа в этом дереве. 
	   Будем  говорить, что коммуникационные протокол \selectedFont{вычисляет} функцию $ f$, если для всех пар  $ (x, y) \in  X \times Y$ этот путь заканчивается в листе с пометкой $ f(x, y)$.

	   \selectedFont{Коммуникационной сложностью} функции $ f$ называется наименьшая глубина протокола, вычисляющего  $ f$. Обозначается  $ D(f)$ или $ D^{cc}(f)$.

	   Каждой функции можем сопоставить матрицу  $ X \times Y$, в каждой клетке $ (x_i, y_j)$ которой стоит значение  $ f(x_i, y_j)$.
\end{defn}

\subsection{Нижние оценки для детерминированного случая}
% картинка
Пусть наша функция $ f\colon X \times Y \to  Z$. 
Запишем  для нее коммуникационную матрицу $ M$ размера $ \lvert X \rvert \times \lvert Y \rvert $, где $ M_{x, y} = f(x, y)$.

Рассмотрим такое подмножество $ R_v$ множества $ X \times Y$, что $(x, y) \in R_v \Longleftrightarrow \text{ протокол приводит в }  v$.

\begin{lm}
	$ R_v = X_v \times Y_v$ --- комбинаторный прямоугольник.
\end{lm}
Рассмотрим два доказательства:
\begin{proof}
	Пусть $ (x, y)$ и  $ (x', y')$ принадлежат  $ R_v$. Тогда  $ (x, y')$ и  $ (x', y)$ тоже принадлежат  $ R_v$, так как $ a(x) = a(x')$ и  $ b(y) = b(y')$.

	А из этого следует, что это комбинаторный прямоугольник.
\end{proof}
\begin{proof}
	Посмотрим на множество $ X\times Y$ как на таблицу.

    Пусть Алиса перешла по какому-то ребру. Вся таблица разделилась на две части по горизонтали: какие-то строки соответствуют $ x \in X$, для которых Алиса отсылает  $ 0$, а какие-то для $ 1$. 

	Если потом ход делает Боб, то он делит все текущие прямоугольники тоже на две части, но по вертикали. 

% \begin{figure}[ht]
%     \centering
%     \incfig{table-proof}
%     \label{fig:table-proof}
% \end{figure}
И так далее.

В прямоугольнике для листа у всех элементов одинаковый ответ. То есть исходную матрицу можно разбить на комбинаторные прямоугольники, причем они естественно не пересекаются.
\end{proof}

\vspace{1em}
Рассмотрим величины $ \chi_{0}(f)$ и $ \chi_1(f)$, первая равна минимальному числу непересекающихся прямоугольников, которыми можно покрыть все нули в таблице, в вторая --- все единицы.

Тогда листьев в двоичном дереве протокола будет хотя бы $ \chi_0(f) + \chi_1(f)$. Следовательно, $ D(f) \ge \log( \chi_0(f) + \chi_1(f))$.

Но эта оценка не всегда точна. Можно рассмотреть следующее разбиение на картинке \ref{fig:bad-rect}. 

Здесь $ \chi_0(f)  + \chi_1(f) = 4 + 1 = 5$.
\begin{figure}[h]
    \centering
    \incfig{bad-rect}
    \caption{Пример неточной оценки}
    \label{fig:bad-rect}
\end{figure}
Заметим, что для такого разбиения не существует дерева протокола. Посмотрим на первое действие игроков. Прямоугольник должен разделиться на две части, но любой разрез вдоль сторон разрежет один из внутренних прямоугольников.

Поэтому любой протокол не будет соответствовать этому разбиению. А тогда листьев будет больше пяти.

\vspace{1.5em}
\begin{thm}[G,PW 16]
Пусть $ \chi$ --- минимальное число одноцветных прямоугольников в разбиении.
	Существует $ f$ для которой
	 \[
		 D(f) \ge \log^{2 - \varepsilon } \chi (M_f)
	.\] 
	Без доказательства.
\end{thm}

\section{Вероятностная модель}
Теперь Алиса и Боб могут подбрасывать монетки. Либо эти монетки (оракулы) \textit{публичны} (оба видят значения), либо \textit{приватны} (тогда никто не видит, кроме пользователя).

Так как Алиса или Боб в случае публичного оракула, могут закрыть глаза на сообщения другого, публичный протокол не меньше приватного.

Скажем, что \selectedFont{протокол отработал корректно}, если 
\[
	\forall x, y \colon Pr_{r}[\pi(x, y) = f(x, y)] \ge \frac{2}{3}, \quad \pi(x, y) \text{ --- результат работы}
.\] 
\begin{defn}[]
	Будем говорить, что вероятностный протокол \selectedFont{$ \varepsilon$ -вычисляет $ f$}, если для любой пары $ x, y$ с вероятностью  $ r$ или  $ s$ не менее  $ 1 - \varepsilon $ результат протокола равен $ f(x, y)$ (с точки зрения обоих игроков).

	Через  $ R_{ \varepsilon }(f) $ обозначается минимальная высота вероятностного протокола $ \varepsilon $-вычисляющего $ f$. 

	Через $ R_{ \varepsilon }^{pub}$ --- минимальная высота в случае, если $ r = s$.
\end{defn}
\begin{lm}
	$ R^{pub}_{ \varepsilon } (\EQ) = \O( \log \varepsilon ^{-1})$
\end{lm}

\section{Методы оценки коммуникационной сложности}
\subsection{Метод ранга}
Разбирался на практике:
\[
	\operatorname{rk}_{\R} (M_f) \le \# \text{ одноцветных прямоугольников в разбиении}
.\] 
Здесь $ f$ --- функция.

Для функции $ \EQ = \colon \{0, 1\}^{n} \times \{0, 1\}^{n} \to \{0, 1\}$, будет диагональная матрица. Поэтому одноцветных прямоугольников будет не меньше $ 2^{n}$, а тогда коммуникационная сложность хотя бы $ n$.

\subsection{Fooling Set}
Рассмотрим коммуникационную матрицу. Пусть мы хотим выбрать некоторое множество 
\[
	S = \{(x_1, y_1), (x_2, y_2) , \ldots \}
,\] 
такое что каждая пара точек не лежит в одном прямоугольнике.

% картинка

Если две клетки в одном прямоугольнике, оставшиеся вершины тоже лежат в нем.

Тогда нужно для всех $ i, j, i \ne j$ либо $ (x_i, y_j)$, либо  $ (x_j, y_i)$ покрашена в другой цвет.

Для $ \EQ$ легко получить ту же оценку. Плюс, как как нужен хотя бы один лист для нуля, $ n$ не хватит, следовательно,  $ D(\EQ) = n+1$.

\begin{thm}
	Если существует Fooling set размера $ s$, то $ \operatorname{rk}_R s \ge s$.
\end{thm}

 

\section{Связь со схемами. Теорема Шеннона}
\begin{defn}[]
	\selectedFont{Формульная сложность} $ L(f)$ формулы  $ f$ --- минимальное возможное число листьев дерева, вычисляющего  $ f$.
\end{defn}

\begin{thm}[Шеннон]
	Существует $ f \colon \{0, 1\}^{n} \to  \{0, 1\}$, такая что 
	$$ L(f) \ge \Omega\left( \tfrac{2^{n}}{n} \right) .$$
\end{thm}
\begin{proof}
	% картинка
    Всего функций такого вида $ 2^{2^{n}}$, так как можно задать таблицей истинности.

	Посчитаем число схем. Это ациклический граф и то, что записано в его узлах.

	Пусть каждая вершина ($ S$ штук) выбирает себе двух предков. Так же в каждую вершину нужно что-то записать и на ребре можно ставить отрицание: хватит $ 3$ бит. Еще есть входные данные ($ n$ штук).

	Итого: $2^{ S \cdot 2 (\log S + \log n) + 3S}$.

	Схем должно быть не меньше  количества функций
	\[
		2^{S \cdot  2( \log S + \log n) + 2 S} \ge  2^{2^n}
	.\] 
	Отсюда получаем нужное неравенство.
\end{proof}

\textbf{Открытый вопрос:} Можно ли предъявить $ f \in \NP$, что $ L(f) \ge  10n$

\section{Теорема Карчмера-Вигдерсона}

Пусть нам дана $ f\colon \{0, 1\}^{n } \to  \{0, 1\}$.

Алиса получает число $ x \in f^{-1}(1)$, а Боб $ y \in f^{-1}(0)$. Их цель найти любой бит, в котором $x$ и $y$ отличаются.

\begin{thm}[Karchmer-Wigderson, 1990]
	$ L(f) $ --- размер минимальной формулы для  $ f$, согда $ L(f)$ --- размер минимального протокола для  $ KW_f$
\end{thm}
\begin{proof}
    $ $
    \begin{description}
		\item [\boxed{ 1 \Longrightarrow 2}] Нарисуем дерево вверх корнем. Также спустим все отрицания к листьям. Пусть в узле считается функция $ f = g \vee h$, где  $ g$ и  $h$ --- соседи  $ f$.

			Тогда $ f(x) = 1$,  $ f(y) = 0$. Тогда $g(y) = h(y) = 1$, а для Алисы хотя бы одно из двух значений единица.Тогда Алиса посылает информацию, где $ 1$, то есть куда нам нужно спуститься (потому что в этом отрезке битов точно есть отличие). Далее игра продолжается по тем же правила рекурсивно. Стоит заметить, что если в вершине конъюнкция, то Боб пошлет информацию, где происходит обнуление. Так за глубину формулы мы нашли решение для $KW_f$ той же глубины. Т.е. оптимальный размер протокола меньше или равен $L(f)$.
		\item [\boxed{ 2 \Longrightarrow 1}] 
			Пусть у нас есть некоторый протокол для игры. Это некоторое дерево. Обозначим за $ R_v \coloneqq X_v \times Y_v$ --- прямоугольник входов для вершины $ v$, из которых мы получаем $ v$.

			Мы хотим построить $ f\colon \{0, 1\}^{n} \to \{0, 1\}$.

			Будем подниматься снизу и строить формулу по протоколу. Обозначим за $ f_v$ построенную формулу в вершине  $ v$. Хотим получить следующие свойства для формулы: $ f_v(X_v) = 1$ и  $ f_v(Y_v) = 0$.

			Если они выполняются, то $ \forall x \in X_v \colon f(x) = 1$ и $ \forall y \in Y_v \colon f(y) = 0$.

			В корне $ f = f_r$.

			\begin{itemize}
				\item Если мы в листе $ l$. Здесь написан некоторый ответ. То есть $ \forall x \in X_l ~\forall y \in Y_l \colon x \ne  y$.

					Тогда либо $ \forall i \colon x_i = 0 \wedge  y_i  = 1$, либо наоборот. % Пояснение?

					В качестве $ f_{l}(z) $ можем в первом случае взять $ \neg z_i$, во втором $z_i$.
				\item Теперь мы находимся в вершине $ v$ с потомками  $ a $ и $ b$.

					Если ходит Алиса, то прямоугольник  $ R_v$ разрезается на   $ R_a$ и  $ R_b$ горизонтально (если  Боб, то наоборот вертикально).

					У нас уже есть две функции  $ f_a$ и  $ f_b$, построенные по предположению индукции.  $ \forall y \in Y_v\colon  f_a(y) = f_b(y) = 0$, так как $ Y_v = Y_a = Y_b$.

					А так как $ X_v \subseteq X_a \cup X_b$, $ \forall x \in X_v$ либо $ f_a(x) = 1$, либо  $ f_b(x) = 1$.

					Поэтому нам подходит  $ f_v \coloneqq f_a \vee f_b$.

					Если же ходит Боб нужно будет сделать конъюнкцию.
			\end{itemize}
    \end{description} 
\end{proof}
