\chapter{Информация по Хартли}
\section{Базовые свойства}
\lecture{1}{1 April}{\dag}
Пусть у нас есть конечное множество объектов $ A$. Выдернем какой-то элемент.

Мы хотим придумать описание этого элемента, которое будет отличать его от всех остальных. Хотим ввести некоторую \textit{меру информации}, то есть  сколько информации мы узнаем, когда вытягиваем некоторый элемент из $ A$.

Самый простой вариант --- число битов требуемое для записи объекта.

Свойства, которые мы хотим получить от меры информации $ \chi(A)$:
\begin{enumerate}
    \item $ \chi$ дает нам оценку на длину описаний
	\item $ \chi(A \cup B) \le \chi(A) + \chi(B)$
	\item 
Если $ A \coloneqq B \times C$, то чтобы описать элемент из $ A$, достаточно описать элемент из $ B$ и из $ C$:
\[
	\chi(A)  \le \chi(B) + \chi(C)
.\] 
\end{enumerate} 

\begin{defn}[Информация по Хартли]\index{информация по Хартли}
	Пусть $ A$ --- некоторое конечное множество. За \selectedFont{информацию в множестве }  $ A$ будем принимать следующую величину:\footnote{Здесь и далее под $ \log x$ подразумевается двоичный логарифм.}
	$ \chi(A) \coloneqq  \log \lvert A \rvert $
\end{defn}
\begin{note}
Очевидно, второе свойство выполнено для такого определения (проблемы есть, только когда одно из множеств одноэлементно и не содержится во втором). В третьем даже равенство. 

Описание --- например, битовая строка. Если логарифм нецелый, округляем вверх.
\end{note}
Пусть $ A  \subseteq X \times Y$. Обозначим \selectedFont{проекции} $ A_X$ и $ A_{Y}$.
\begin{figure}[ht]
    \centering
    \incfig{axy-img}
    \label{fig:axy-img}
\end{figure}
\begin{st}
	Пусть $ A \subseteq X\times Y$ и конечно. Тогда для $ \chi(A)$ выполнены следующие свойства:
\begin{enumerate}
	\item $ \chi(A) \ge 0$
	\item $ \chi(A_{X}) \le \chi(A)$ и $ \chi(A_{Y}) \le \chi(A)$
	\item $ \chi(A) \le \chi(A_{X}) + \chi(A_{Y})$
\end{enumerate} 
\end{st}
\begin{proof}
    Очевидно, следует из определения.
\end{proof}

\begin{figure}[h]
\begin{minipage}{0.5\textwidth}
    \centering
    \incfig{diag-img}
	\caption{Диагональное множество}
    \label{fig:diag-img}
\end{minipage}
\begin{minipage}{0.5\textwidth}
    \centering
    \incfig{corner-img}
	\caption{Неотличимые множества}
    \label{fig:corner-img}
\end{minipage}
\end{figure}
\begin{ex}
	Рассмотрим такой пример (см. рисунок \ref{fig:diag-img}): множество из точек на диагонали.
Здесь, зная первую координату, можно сразу понять вторую.
\end{ex}
Попробуем усилить третье свойство:
\begin{enumerate}
	\item[3'.] $ \chi(A) \le \chi(A_{X}) + \chi_{Y \mid X} (A)$, где $ \chi _{Y \mid X}(A)$ --- описание $ Y$ при условии $ X$.
\end{enumerate} 
Как будем определять $ \chi_{Y \mid X} (A)$?
Можно взять $ \max_{x \in X} \log(\lvert A_x \rvert)$.
Теперь для диагонального множества $ \chi_{Y \mid X}$ просто обнуляется и неравенство переходит в равенство.

\begin{ex}
	Но если взять одно множество из двух отрезков, а второе из всего прямоугольника, возникнут проблемы (см. рисунок \ref{fig:corner-img}). 

Во-первых, на первой картинке, передав $ x$-координату столбца, придется передавать  и $ y$ тоже. Во-вторых, мы не сможем отличить эти множества по одной координате.
\end{ex}

\begin{lm}
	Пусть $ A \subseteq A_1 \times A_2 \times A_3$ конечно.
	Тогда\footnote{Здесь $ A_{ij} = \{(x, y) \mid x \in A_i \land y \in  A_j \land \exists z\colon  (x, y, z) \in A \}$}
	\[
		2 \chi(A) \le \chi(A_{12}) + \chi(A_{13}) + \chi(A_{23})
	.\] 
\end{lm}
\begin{proof*}
	Перепишем неравенство из условия:
	\begin{align*}
		2 \log \lvert A \rvert &\le \log \lvert A_{12} \rvert  + \log  \lvert A_{13} \rvert  + \log \lvert 23 \rvert  \\
		\log \lvert A \rvert^2 & \le \log \bigl(\lvert A_{12} \rvert \cdot \lvert A_{13} \rvert \cdot \lvert A_{23} \rvert\bigr)   \\
		\lvert A \rvert ^2 &\le \lvert A_{12} \rvert \cdot \lvert A_{13} \rvert \cdot \lvert A_{23} \rvert \tag{$ \log$  --- возрастающая функция}  
	\end{align*}
	Пусть $ A = \{(x, y, z) \mid x \in A_1 , ~ y \in A_2 , ~ z \in A_3\}$

	Сгруппируем элементы $ A$ следующим образом: $ A = \bigsqcup B_{ij}$, где 
	$$ B_{ij} = \{(x_i, y_j, z_k) \mid 1 \le k \le K_{ij} , ~ x_i \in A_1 , ~ y_j \in  A_2 , ~ z_k \in A_3\}.$$

	То есть фиксируем $ x_i, y_j$, а $ z_k$ меняется в допустимых пределах. 
	При этом $ \sum_{i, j} K_{ij} = \lvert A \rvert $, так как это сумма размеров $ B_{ij}$, которая равна размеру $ A$. 
	
	Теперь рассмотрим множество 
	\[
		A_{13, 23} = \{(x_i, z_s, y_j, z_l) \mid (x_i, z_s) \in A_{13}, ~ (y_j, z_l)  \in A_{23}\}
	.\] 

	Оценим размер множества $ A_{13,23}$ сверху и снизу.
	Очевидно $ \lvert A_{13, 23} \rvert \le \lvert A_{13} \rvert \cdot \lvert A_{23} \rvert $.

	Теперь опишем элементы, которые заведомо есть в $ A_{13, 23}$, чтобы оценить его размер снизу.
	Для этого зафиксируем $ x_i, y_j$, тогда все $ z_k$ из $ B_{ij}$ точно сюда подходят в качестве $ z_s$ или $ z_l$.
	То есть таких четверок не меньше чем $ \lvert B_{ij}\rvert^2 = K_{ij}^2$. 
	А по всем $ i, j$ получается, что $ \lvert A_{13,23}  \rvert \ge \sum K_{ij}^2$.

	Таким образом,
	\[
		\lvert A_{13} \rvert 	\cdot  \lvert A_{23} \rvert  \ge \lvert A_{13,23} \rvert \ge \sum K_{ij}^2 
	.\] 

	Домножим на $ \lvert A_{12} \rvert $, которое не меньше количества $ K_{ij}$ (обозначим его за $ n$), так как каждый набор $ B_{ij}$ соответствует одному элементу $ A_{12}$:
	\begin{align*}
		\lvert A_{12} \rvert  \cdot \left( \lvert A_{13} \rvert \cdot \lvert A_{23} \rvert   \right) &\ge  n \cdot \sum K_{ij}^2 \ge \tag{неравенство о средних} \\
																									 & \ge \left( \sum K_{ij} \right)^2 = \lvert A \rvert ^2 
	\end{align*}
	А это то, что мы и хотели доказать.
\end{proof*}

\section{Угадывание числа}
Обозначим $ [n] \coloneqq \{1, \ldots , n\}$.
\paragraph{Симметричный вариант}
Пусть есть два игрока, первый загадывает число от 1 до $ n$, а второй должен его угадать. Сколько вопросов необходимо задать, чтобы угадать число?

Есть два варианта игры:
\begin{itemize}
	\item \textit{Адаптивная}, когда второй игрок получает ответ на вопрос сразу
	\item \textit{Неадаптивная}, когда он пишет сначала все вопросы, а потом получает на них все ответы.
\end{itemize}
Очевидно, что нам потребуется не менее логарифма запросов: нарисуем дерево, где вершины -- запросы, по двум ребрам из которых  можно перейти в зависимости от ответа. Листья должны содержать $ [n]$, поэтому глубина дерева не менее логарифма.
\begin{figure}[ht]
    \centering
    \incfig{graph-img}
    \label{fig:graph-img}
	\caption{Граф вопросов}
\end{figure}

Теперь подумаем с точки зрения теории информации.
Пусть $ h$ --- число запросов, $ Q_i$ --- ответ на $ i$-ый запрос (один бит) по некоторому протоколу, 
$$ B \coloneqq Q_1 \times \ldots \times Q_h,$$ 

Мы хотим минимизировать $ h$.

Рассмотрим $ [n]\times B$ --- все возможные пары по всем возможным значениям искомого числа и $ B$. 
Нас интересует множество  $ A \subseteq [n]\times B$, соответствующее некоторым корректным запросам:
\[
	A = \{(m, b) \mid b = (q_1, \ldots , q_h), ~ m \text{ --- согласовано с ответом}\}
.\] 

\begin{enumerate}
	\item $ \chi _{[n] \mid B}(A) = 0$. Ответы на запросы должны однозначно определять число $ m$. Это свойство говорит о корректности протокола, то есть нам ничего не нужно, чтобы, зная ответы, получить $ m$.
	\item $ \chi(A) \ge \log n$, так как уже первая проекция $A$ имеет $n$ элементов. 

		С другой стороны, $ \chi(A) \le \chi_{B}(A) + \chi_{[n] \mid B}(A) = \chi_{B}(A) \le \chi(B) = \sum_{i=1}^{h} \chi(Q_i)$. 

		А так как $ \chi(Q_i) = 1$, $  h \ge \log n $.
\end{enumerate} 

\paragraph{Асимметричный вариант}
Пусть теперь за ответ <<да>> мы платим 1, а за <<нет>> 2, вариант игры адаптивный. И мы хотим минимизировать не число запросов, а стоимость в худшем случае.

Будем делить запросом множество <<пополам>> с точки зрения стоимости.

Пусть $ Q_i $ --- ответ на вопрос <<верно ли, что загаданное число $ m$ лежит в множестве $ T_i$?

Пусть $ A_i$ --- множество элементов, в котором может лежать $ m$ после первых  $ i - 1$ вопросов. В начале это все $ [n]$, в конце -- одно число.
\[
	A_i = \{a \in [n] \mid a \text{ согласовано с } Q_1, \ldots Q_{i-1}\}
.\] 

\textit{Стратегия минимальной цены бита информации}: берем такое $ T_i \subseteq A_i$, что
 \[
	 2\bigl(\chi(A_i) - \chi(\underbrace{T_i}_{A_{i+1}})\bigr) = \chi(A_i) - \chi(\underbrace{A_i \setminus T_i}_{A_{i+1}})
.\] 

Где разность $ \chi(A_i) - \chi(A_{i + 1})$ обозначает количество информации, которое мы получаем задав $ i$-ый вопрос. То есть слева получили ответ <<нет>> ($ m \in T_i$), а справа <<да>> ($ m \in A_i \setminus T_i$).

Распишем по определению:
\begin{align*}
	2\bigl(\log \lvert A_i \rvert  - \log \lvert T_i \rvert \bigr) &= \log \lvert A_i \rvert - \log \lvert A_i \setminus T_i \rvert \\
	\log \lvert A_i \rvert &= 2 \log \lvert T_i \rvert - \log \lvert A_i \setminus T_i \rvert \\
	\lvert A_i \rvert  &= \frac{\lvert T_i \rvert^2 }{\lvert A_i \setminus T_i \rvert }
\end{align*}
Обозначим $ \lvert A_i \rvert = k$, $ \lvert T_i \rvert  = t$ и решим уравнение:
\begin{align*}
	k = \frac{t^2}{k-t} & \Longleftrightarrow t^2 = k(k-t) = k^2 -kt \Longleftrightarrow \\
	t^2 + kt - k^2 = 0 &\Longleftrightarrow t = \frac{-k \pm \sqrt{ k^2 + 4k^2} }{2} = k \cdot \frac{-1 \pm \sqrt{ 5}}{2}  = k \cdot \varphi 
\end{align*}

То есть нужно выбирать $ T_i$, чтобы $\varphi \cdot \lvert T_i \rvert = \lvert A_i \rvert$. 

Тогда <<средняя цена>> бита будет равна $ 2(\chi(A_i) - \chi(T_i)) = 2 \log \frac{1}{\varphi}$

\textbf{Докажем оптимальность.} Пусть второй игрок меняет число, чтобы мы заплатили как можно больше, причем он знает нашу стратегию.

Если в нашем неравенстве знак $ \ge $, он будет направлять на по <<нет>>, а при  $ \le$ <<да>>, за счет чего каждый бит он будет отдавать по цене большей, чем, если бы мы действовали в точности по стратегии. 

Следовательно, любая другая стратегия будет требовать большего вклада.


\begin{prac}[Задача про взвешивания монеток]
	Есть $ n$ монеток и рычажные весы. Хотим найти фальшивую (она одна). 
\begin{enumerate}
    \item Пусть $ n = 30$ и весы показывают, что больше, что меньше. Теперь запрос приносит $  \log 3$ информации, так как три ответа.
		\[
			\log 30 \le  \sum_{i=1}^{h} \chi(a_i) \le h \log 3 
		.\] 
	\item $ n = 15$, но мы не знаем относительный вес фальшивой монеты. В прошлом неравенстве можно заменить  $ 30$ на $ 29$. Если в какой-то момент у нас было неравенство, можем в конце узнать не только номер, но и относительный вес, поэтому у нас $ 29$ исходов.
	\item Вопрос: можно ли при $ n=14$?  Нет. 
\end{enumerate} 
\end{prac}
