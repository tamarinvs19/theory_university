\section{Алгоритмы}
\begin{itemize}[noitemsep]
	\item функторы
	\item обзор алгоритмов с примерами (swap, iter\_swap, sort, find, copy, unique, remove\_if, lower\_bound)
	\item erase-remove idiom
	\item реализация алгоритмов через итераторы
\end{itemize}
\subsection{Функторы}
Функторы --- классы, объекты котороых похожи на функцию и перегружен оператор (). Используется, когда нужно описать алгоритму, что нужно сделать, при этом можтет иметь поля.
\begin{minted}[fontsize=\footnotesize,numbersep=3pt,framesep=1mm,linenos,frame=single,label=]{cpp}
struct sum_sq {
    int operator() (int a, int b) const {
	    return a * a + b * b;
	}
};
sum_sq f;
int res = f(3, 4);
\end{minted}
\begin{minted}[fontsize=\footnotesize,numbersep=3pt,framesep=1mm,linenos,frame=single,label=сравниваем с данным числом]{cpp}
struct cmp {
    int value;
	cmp(int v): value (v) {}

	bool operator () (int a) const {
	    return a < value;
	}
};

cmp f(2);
bool b = f(13);
\end{minted}
Такой функтор (с bool) называют {\sf предикатом}.
\begin{minted}[fontsize=\footnotesize,numbersep=3pt,framesep=1mm,linenos,frame=single,label=Подсчет суммы]{cpp}
struct accum {
    int acc;
	accum() : acc(0) {}

	void operator () (int a) {
	    acc += a;
	}
};

accum f;
f(13); f(16);
\end{minted}
На функциях можно сделать аналогично черес статические переменные.

\subsection{Алгоритмы}
\subsubsection{Микро-алгоритмы}
\begin{itemize}[noitemsep]
	\item \begin{verbatim}
swap(T &a, T&b)
	\end{verbatim}
\item 
	\begin{verbatim}
	iter_swap(It p, It q)
	\end{verbatim} 
	меняет местами значения, на которые указывают итераторы
\item 
	\begin{verbatim}
	max(const T &a, const T &b)
	\end{verbatim}
\item 
	\begin{verbatim}
	min(const T &a, const T &b)
	\end{verbatim}
\item 
	\begin{verbatim}
	size_t count(It p, It q, const T& x)
	\end{verbatim}
\item 
	\begin{verbatim}
	size_t count_if(It p, It q, Pr pred)
	\end{verbatim}
\end{itemize}
\begin{minted}[fontsize=\footnotesize,numbersep=3pt,framesep=1mm,linenos,frame=single,label=пример использования предикаторов]{cpp}
struct Person {
    int age;
	string name;
	string city;
	Person (...) {...}
};
struct by_city {
    bool operator () (const Person &p1, const Person &p2) const {
		return p1.city < p2.city;
	}
};

Person p1(30, "V", "Msc"); Person p2(15, "K", "Spb");
by_city f;
std::max<Person, by_city> (p1, p2, f);
std::max(p1, p2, by_city()); // анонимная переменная
\end{minted}
\subsubsection{Алгоритмы типа find}
\begin{itemize}[noitemsep]
\item 
	\begin{verbatim}
	find(It p, It q, const T &x)
	\end{verbatim} 
	возвращает итератор на первое вхождение элемента x в последовательнось между p и q
\item 
	\begin{verbatim}
	find_if(It p, It q, Pr pred)
	\end{verbatim} 
	возвращает итератор на первое вхождение элемента, для которого pred вернул true
\item 
	\begin{verbatim}
	max_element(It p, It q)
	\end{verbatim} 
\item 
	\begin{verbatim}
	min_element(It p, It q)
	\end{verbatim} 
\item 
	\begin{verbatim}
	equal(It p, It q, Itr i)
	\end{verbatim} 
	Сравнивает две последовательности, первая p-q, вторя должна быть той же длины. Если короче, то undefined behaviour.
\item 
	\begin{verbatim}
	pair<It, Itr>mismach(It p, It q, Itr i)
	\end{verbatim} 
	Возвращает пару итераторов, указывающую на первое несовпадение.
\item 
	\begin{verbatim}
	F for_each(It p, It q, F func)
	\end{verbatim} 
	Применяет func для всех элементов, возвращает функтор.
\end{itemize}
\begin{minted}[fontsize=\footnotesize,numbersep=3pt,framesep=1mm,linenos,frame=single,label=]{cpp}
std::vector<int> v = {1, 2, 5, 3, 4, 5, 5, 2};
std::vector<int>::iterator two = find(v.begin(), v.end(), 2);
\end{minted}
\subsubsection{Модифицирующие алгоритмы}
\begin{itemize}[noitemsep]
\item 
	\begin{verbatim}
	fill(It p, It q, const T &x)
	\end{verbatim} 
\item 
	\begin{verbatim}
	generate(It p, It q, F gen)
	\end{verbatim} 
	Заполняет последовательность значениями, сгенерированными функтором gen
\item 
	\begin{verbatim}
	copy(It p, It q, Itr out)
	\end{verbatim} 
\item 
	\begin{verbatim}
	reverse(It p, It q)
	\end{verbatim} 
\item 
	\begin{verbatim}
	sort(It p, It q)
	\end{verbatim} 
\item 
	\begin{verbatim}
	transform(It p, It q, Itr out, F func)
	\end{verbatim} 
	К каждому элементу применяем функтор func и записывают в последовательность, начинающуюся с итератора out
\end{itemize}
\begin{minted}[fontsize=\footnotesize,numbersep=3pt,framesep=1mm,linenos,frame=single,label=]{cpp}
std::vector<int> v = {5, 2, 8, 9, 4, 1};
std::sort(v.begin(), v.end());

std::vector<int> v1 = {5, 3, 3, 4, 1, 5, 5, 4};
std::vector<int>::iterator last = std::unique(v1.begin(), v1.end());
v1.erase(last, v1.end());
\end{minted}

\subsubsection{erase-remove idiom}
Сначала поставим все нужные элементы в начало, а в конце оставим мусор, вернем итератор на начало мусора. Далее можно его удалить через erase. 
\begin{minted}[fontsize=\footnotesize,numbersep=3pt,framesep=1mm,linenos,frame=single,label=]{cpp}
std::vector<int> v = { 0, 1, 2, 3, 4, 5, 6, 7, 8, 9 };
v.erase(std::remove(std::begin(v), std::end(v), 5), std::end(v)); 
\end{minted}

\subsubsection{Реализация алгоритмов через итераторы}
Это удобно тем, что после одной реализации можно использовать для различных типов и структур данных, имеющих итераторы.
\begin{minted}[fontsize=\footnotesize,numbersep=3pt,framesep=1mm,linenos,frame=single,label=]{cpp}
template<class InputIt, class OutputIt>
OutputIt copy(InputIt first, InputIt last, OutputIt d_first) {
    while (first != last) 
	    *d_first++ = *first++;
    return d_first;
}

vector<int> v;
list<int> l;
copy(v.begin(), v.end(), l.begin());

template<class InputIt, class OutputIt, class UnaryPredicate>
OutputIt copy_if(InputIt first, InputIt last, OutputIt d_first, UnaryPredicate pred) {
    while (first != last) {
	    if (pred(*first))
			*d_first++ = *first;
		first++;
	}
    return d_first;
}
copy_if(v.begin(), v.end(), l.begin(), divide_by(8));
\end{minted}
