\section{Ассоциативные контейнеры}
\begin{itemize}[noitemsep]
	\item set, multiset, map, multimap
	\item unordered\_set, unordered\_map
	\item внутреннее устройство и основные операции
	\item итераторы и их инвалидация
\end{itemize}
Особенности:
\begin{enumerate}[noitemsep]
	\item Переупорядочивают элементы для быстрого поискка $ \mathcal(O)(\log n)$ 
	\item Возможные реализации: дерево поиска $ \mathcal{O}(\log n)$ 
	\item Требуют отношение порядка \texttt{operator<()}
	\item Нет произвольного доступа по индексу
\end{enumerate} 
Общие методы:
\begin{enumerate}[noitemsep]
    \item Конструктор по умолчанию, конструктор копирования, оператор присваивания, деструктор
	\item Операторы сравнения: \texttt{==, !=,   >, >=, <, <=}
	\item \texttt{size(), empty()}
	\item \texttt{swap(obj2)}
	\item \texttt{clear()}
	\item \texttt{begin(), end()}
	\item \texttt{erase} по ключу
	\item \texttt{insert}/\texttt{insert} c подсказкой (итератор)
	\item \texttt{count} число элементов с заданным ключом
	\item \texttt{find} поиск точного совпадения (возвращает  итератор)
	\item \texttt{lower\_bound, upper\_bound, equal\_range} 
\end{enumerate} 
\subsection{set, multiset}
В set все элементы уникальны, в multiset допустимы дубликаты. Реализованы с помощью BST с ключами в вершине. 
\begin{minted}[fontsize=\footnotesize,numbersep=3pt,framesep=1mm,linenos,frame=single,label=]{cpp}
#include <set>

std::set<int> s;
s.insert(10); s.insert(20); s.insert(10); // s.size() = 2

std::multiset<int> ms;
ms.insert(10); ms.insert(20); ms.insert(20); // ms.size() = 3

std::cout << s.count(20) << " " << ms.count(20) << std::endl;
\end{minted}
Порядок не сохраняется. 
\begin{minted}[fontsize=\footnotesize,numbersep=3pt,framesep=1mm,linenos,frame=single,label=]{cpp}
std::set<int> s;
s.insert(10);
s.insert(6);
s.insert(20);
s.insert(5);
s.insert(1);

for (std::set<int>::iterator it = s.begin(); it != s.end(); ++i) 
	std::cout << *it << " ";
} // выведет 1 5 6 10 20
\end{minted}
Присваивание запрещено.
\begin{minted}[fontsize=\footnotesize,numbersep=3pt,framesep=1mm,linenos,frame=single,label=]{cpp}
std::set<int> s;
s.insert(10);
std::set<int>::iterator it = s.find(10);
*it = 5; // запрещено
\end{minted}
\subsection{map, multimap}
В map элементы уникальны, в multimap --- нет. Реализация с помощью BST, в вершине $(key, value)$
Особый метод: \texttt{operator[]}.
\paragraph{Вспомогательный класс pair}
\begin{minted}[fontsize=\footnotesize,numbersep=3pt,framesep=1mm,linenos,frame=single,label=]{cpp}
template<class F, class S>
struct pair {
	F first;
	S second;
	... // конструкторы
};

template<class F, class S>
pair<F, S> make_pair(F const& f, S const& s);

template<class Key, class T, ...> class map {
	...
	typedef pair<const Key, T> value_type;
};
\end{minted}
\begin{minted}[fontsize=\footnotesize,numbersep=3pt,framesep=1mm,linenos,frame=single,label=]{cpp}
#include <map>

std::map<string, int> phonebook;
phonebook.insert(std::pair<string, int> ("Mary", 2128506));
phonebook.insert(std::make_pair("Alex", 9286385); // вывод типов параметров
phonebook.insert(std::make_pair("Bob", 2128506);

std::map<std::string, int>::iterator it = phonebook.find("John");
if (it != phonebook.end()) 
    std::cout << "Jonh`s p/n is " << it->second << std::endl;

for (it = phonebook.begin(); it != phonebook.end(); ++it)
    std::cout << it->first << ": " << it->second << "\n";
\end{minted}

\paragraph{\texttt{operator[]}}
Работает только с неконстантными map.
Требует наличие конструктора по умолчанию у T.
\begin{minted}[fontsize=\footnotesize,numbersep=3pt,framesep=1mm,linenos,frame=single,label=]{cpp}
T & operator[](Key const& k) {
    iterator i = find(k);
	if (i == end())
	    i = insert(k, T());
    return i->second;
}
\end{minted}
Требует $ \mathcal{O}(\log n)$ времени.
\begin{minted}[fontsize=\footnotesize,numbersep=3pt,framesep=1mm,linenos,frame=single,label=Пример]{cpp}
m["vasya"] = 10;
m["petya"] = 20;
int p = m["kola"]; // если такого ключа нет, будет создан новый элемент со значением 0
\end{minted}
\subsection{unordered\_map, unordered\_set}
\begin{minted}[fontsize=\footnotesize,numbersep=3pt,framesep=1mm,linenos,frame=single,label=]{cpp}
template<class Key, class Hash = std::hash<Key>,
		 class KeyEqual = std::equal_to<Key>,
		 class Allocator = std::allocator<Key>>
	class unordered_set;
\end{minted}
Реализовано как хэштаблица из каманов и списков. Операции find знимают $ \mathcal(O)(1)$, но, если все оказались в одном кармане, то  $ \mathcal{O}(n)$, insert аналогично, если вызывает рехэширование).

Для своего типа нужно реализовать hash:
\begin{minted}[fontsize=\footnotesize,numbersep=3pt,framesep=1mm,linenos,frame=single,label=]{cpp}
class Point {
private:
    int x, y;
private:
    bool operator == (const Point& rhs) const {
		return x == rhs.x && y == rhs.y;
	}
};

namespace std {
    template<>
	struct hash<Point> {
	    size_t operator() (Point const &p) const {
		    return (std::hash<int>()(p.getX()) * 53 + std::hash<int>()(p.getY())) * 239;
		}
	};
}

std::unordered_set<Point> points;
\end{minted}

\subsection{Инвалидация}
\begin{minted}[fontsize=\footnotesize,numbersep=3pt,framesep=1mm,linenos,frame=single,label=]{cpp}
std::map<string, int> m;
std::map<string, int>::iterator it = m.begin();
for (; it != m.end;)
    if (it->second == 0)
	    m.erase(++it);
	else
	    ++it;
\end{minted}
В С++11
\begin{minted}[fontsize=\footnotesize,numbersep=3pt,framesep=1mm,linenos,frame=single,label=]{cpp}
for (; it != m.end(); )
    if (it->second == 0) 
	    it = m.erase(it);
    else
	    ++it;
\end{minted}
\subsection{Собственный компаратор}
\begin{minted}[fontsize=\footnotesize,numbersep=3pt,framesep=1mm,linenos,frame=single,label=]{cpp}
struct Person {
    string name;
	string surname;
};
bool operator < (Person const& a, Person const& b) {
    return a.name < b.name || (a.name == b.name && a.surname < b.surname);
}
std::set<Person> s1; // уникальные по name и surname

struct PersonComp {
	bool operator () (Person const& a, Person const& b) const {
	    return a.surname < b.surname;
	}
};
std::set<Person, PersonComp> s2; // только по surname
\end{minted}
