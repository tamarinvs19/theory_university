\section{move семантика}
\begin{itemize}[noitemsep]
	\item rvalue и lvalue
	\item rvalue references
	\item move constructor, move assignment
	\item std::move
\end{itemize}
\subsection{lvalue и rvalue}
\begin{description}
	\item[lvalue]
		Может быть и в левой, и в правой части присваивания (переменная).
		\begin{itemize}[noitemsep]
			\item продолжает существовать за пределами выражения, где было использовано
			\item имеет имя
			\item можно взять адрес
		\end{itemize}
	\item[rvalue]
		Выражение, которое может быть только в правой части присваивания.
		\begin{itemize}[noitemsep]
			\item не существует за пределами выражения, где было использовано
			\item временное значение (temporary value)
	    \end{itemize}
\end{description}
\subsubsection{Примеры}
\begin{minted}[fontsize=\footnotesize,numbersep=3pt,framesep=1mm,linenos,frame=single,label=]{cpp}
int a = 42;
int b = 43;
int c = a * b; // a * b - rvalue
\end{minted}
\begin{minted}[fontsize=\footnotesize,numbersep=3pt,framesep=1mm,linenos,frame=single,label=]{cpp}
vector<Person> people;
people.push_back(Person("Name", 36); // Person("Name", 36) - rvalue
\end{minted}
\begin{minted}[fontsize=\footnotesize,numbersep=3pt,framesep=1mm,linenos,frame=single,label=]{cpp}
int square(int x) { return x*x; }
int sq square(10); // square(10) - rvalue
\end{minted}
\subsection{move}
Используется, если внутри класса храним какие-то ресурсы: указатель на динамически выделенную память, файлы, \ldots 
Оптимизируем работу с временными выражениями rvalue.
Пусть есть класс, обладающий описанными свойствами.

\begin{minipage}{0.35\textwidth}
\begin{minted}[fontsize=\footnotesize,numbersep=3pt,framesep=1mm,linenos,frame=single,label=]{cpp}
class X {
    int *array;
	X(size_t size);
	~X();
	X(const X&);
	X& operator = (const X&);
};

X foo();
X x;
x = foo();
\end{minted}
\end{minipage}
\hfill
\begin{minipage}{0.55\textwidth}
	Что происходит во время присваивания?
\begin{enumerate}[noitemsep]
	\item копируем ресурс из temporary (rvalue)
	\item освобождаем ресурс в x (delete, fclose, \ldots )
	\item присваиваем в x скопированный ресурс
	\item освобождаем ресурс в temporary
\end{enumerate} 
\end{minipage}
Можем обработать эффективно.
\begin{minted}[fontsize=\footnotesize,numbersep=3pt,framesep=1mm,linenos,frame=single,label=]{cpp}
class X {
    int *array = NULL; 
	X(size_t size);
	~X();
	X(const X& x); // X& - lvalue reference
	X& operator = (const X& x);

	X(X&& x) { std::swap(this->array, x.array); } // X&& - rvalue reference, это move constructor
	X& operator = (X&& x) { this->array = std::move(x.array) } // это move assignment
};

X x1(100);
X x2(x1); // вызываем X(const X& x), так как x1 - lvalue
X x3(X(100)); // вызываем X(X&& x), так как X(100) - rvalue
\end{minted}
% line 78 ? сойдет ??

\subsection{std::move}
Говорим, что мы хотим для этой переменной использовать move конструктор:
\begin{minted}[fontsize=\footnotesize,numbersep=3pt,framesep=1mm,linenos,frame=single,label=]{cpp}
template<class T>
void swap (T& a, T& b) {
	T tmp(std::move(a));
	a = std::move(b);
	b = std::move(tmp);
}
\end{minted}
