\usepackage[english, russian]{babel}
\usepackage{xltxtra}
\usepackage{polyglossia}

\usepackage{mathpazo}

\defaultfontfeatures{Ligatures=TeX,Mapping=tex-text}

\setmainfont{STIX2Text-Regular.otf}[
ExternalLocation={/home/vyacheslav/builds/STIXv2.0.2/OTF/},
BoldFont=STIX2Text-Bold.otf,
ItalicFont=STIX2Text-Italic.otf,
BoldItalicFont=STIX2Text-BoldItalic.otf
]
\setmathrm{STIX2Math.otf}[
ExternalLocation={/home/vyacheslav/builds/STIXv2.0.2/OTF/}
]


\usepackage{makeidx}
\usepackage{amssymb, amsthm}
\usepackage{amsmath}
\usepackage{mathtools}
\usepackage{needspace}
\usepackage{enumitem}
\usepackage{cancel}
\usepackage{fdsymbol}
\usepackage{fontawesome}


% разметка страницы и колонтитул
\usepackage[left=2cm,right=2cm,top=1cm,bottom=1.1cm,bindingoffset=0cm]{geometry}
\usepackage{fancybox,fancyhdr}
\fancyhf{}
\fancyhead[R]{\thepage}
\fancyhead[L]{\rightmark}
\fancyfoot{}
\fancyhfoffset{0pt}
\addtolength{\headheight}{13pt}
\pagestyle{fancy}

% Отступы
\setlength{\parindent}{3ex}
\setlength{\parskip}{3pt}

\usepackage{graphicx}
\usepackage{hyperref}

\usepackage{import}
\usepackage{xifthen}
\usepackage{pdfpages}

\newcommand{\incfig}[1]{%
    \def\svgwidth{\columnwidth}
    \import{./figures/}{#1.pdf_tex}
}


\usepackage{xifthen}
\makeatother
\def\@lecture{}%
\newcommand{\lecture}[3]{
    \ifthenelse{\isempty{#3}}{%
        \def\@lecture{Лекция #1}%
    }{%
        \def\@lecture{Лекция #1: #3}%
    }%
    \subsection*{\@lecture}
    \marginpar{\small\textsf{\mbox{#2}}}
}
\makeatletter


\usepackage{xcolor}
\definecolor{Aquamarine}{cmyk}{50, 0, 17, 100}
\definecolor{ForestGreen}{cmyk}{76, 0, 76, 45}
\definecolor{Pink}{cmyk}{0, 100, 0, 0}
\definecolor{Cyan}{cmyk}{56, 0, 0, 100}
\definecolor{Gray}{gray}{0.3}


\usepackage{mdframed}
\mdfsetup{skipabove=3pt,skipbelow=3pt}
\mdfdefinestyle{defstyle}{%
    linecolor=red,
	linewidth=3pt,rightline=false,topline=false,bottomline=false,%
    frametitlerule=false,%
    frametitlebackgroundcolor=red!0,%
    innertopmargin=4pt,innerbottommargin=4pt,innerleftmargin=7pt
    frametitlebelowskip=1pt,
    frametitleaboveskip=3pt,
}
\mdfdefinestyle{thmstyle}{%
    linecolor=cyan!100,
	linewidth=2pt,topline=false,bottomline=false,%
    frametitlerule=false,%
    frametitlebackgroundcolor=cyan!20,%
    innertopmargin=4pt,innerbottommargin=4pt,
    frametitlebelowskip=1pt,
    frametitleaboveskip=3pt,
}
\theoremstyle{definition}
\mdtheorem[style=defstyle]{defn}{Определение}

\newmdtheoremenv[nobreak=true,backgroundcolor=Aquamarine!10,linewidth=0pt,innertopmargin=0pt,innerbottommargin=7pt]{cor}{Следствие}
\newmdtheoremenv[nobreak=true,backgroundcolor=CarnationPink!20,linewidth=0pt,innertopmargin=0pt,innerbottommargin=7pt]{desc}{Описание}
\newmdtheoremenv[nobreak=true,backgroundcolor=Gray!10,linewidth=0pt,innertopmargin=0pt,innerbottommargin=7pt,font={\small}]{ex}{Пример}
\newmdtheoremenv[nobreak=false,backgroundcolor=Cyan!10,linewidth=0pt,innertopmargin=0pt,innerbottommargin=7pt]{thm}{Теорема}
\newmdtheoremenv[nobreak=true,backgroundcolor=Pink!10,linewidth=0pt,innertopmargin=0pt,innerbottommargin=7pt]{lm}{Лемма}

\newtheorem*{st}{Утверждение}
\newtheorem*{prop}{Свойства}

\theoremstyle{plain}
\newtheorem*{name}{Обозначение}

\theoremstyle{remark}
\newtheorem*{rem}{Ремарка}
\newtheorem*{com}{Комментарий}
\newtheorem*{note}{Замечание}
\newtheorem*{prac}{Упражнение}
\newtheorem*{probl}{Задача}


\renewcommand{\proofname}{Доказательство}
\renewenvironment{proof}
{ \hspace{\stretch{1}}\\ \faSquareO\quad \small  }
{ \hspace{\stretch{1}}  \faSquare \normalsize }


\numberwithin{ex}{section}
\numberwithin{thm}{section}
\numberwithin{equation}{section}



\newcommand{\K}{\mathcal{K}}
\newcommand{\Z}{\mathbb{Z}}
\newcommand{\N}{\mathbb{N}}
\newcommand{\Real}{\mathbb{R}}
\newcommand{\Q}{\mathbb{Q}}
\newcommand{\Cm}{\mathbb{C}}
\newcommand{\Pm}{\mathbb{P}}
\newcommand{\ord}{\operatorname{ord}}
\newcommand{\lcm}{\operatorname{lcm}}
\newcommand{\sign}{\operatorname{sign}}
\newcommand{\E}{\mathbb{E}}

\renewcommand{\o}{o}
\renewcommand{\O}{\mathcal{O}}
\renewcommand{\le}{\leqslant}
\renewcommand{\ge}{\geqslant}

\def\mybf#1{\textbf{#1}}
\def\selectedFont#1{\textbf{#1}}
\def\ComplexityFont#1{\textmd{\textbf{\textsf{#1}}}}
\def\LanguageFont#1{{\textbf{\texttt{#1}}}}


\newcommand{\Cclass}{\mathcal{C}}
\newcommand{\Dclass}{\mathcal{D}}


\renewcommand{\P}{\ComplexityFont{P}}
\newcommand{\DTIME}{\ComplexityFont{DTime}}
\newcommand{\DTime}{\ComplexityFont{DTime}}
\newcommand{\DSpace}{\ComplexityFont{DSpace}}
\newcommand{\PSPACE}{\ComplexityFont{PSPACE}}
\newcommand{\NTIME}{\ComplexityFont{NTime}}
\newcommand{\NSpace}{\ComplexityFont{NSpace}}
\newcommand{\coNSpace}{\ComplexityFont{coNSpace}}
\newcommand{\NPSPACE}{\ComplexityFont{NPSPACE}}
\newcommand{\poly}{\ComplexityFont{poly}}
\newcommand{\RP}{\ComplexityFont{RP}}
\newcommand{\coRP}{\ComplexityFont{co-RP}}
\newcommand{\ZPP}{\ComplexityFont{ZPP}}
\newcommand{\BPP}{\ComplexityFont{BPP}}
\newcommand{\BQP}{\ComplexityFont{BQP}}
\newcommand{\coBPP}{\ComplexityFont{co-BPP}}
\newcommand{\NP}{\ComplexityFont{NP}}
\newcommand{\NL}{\ComplexityFont{NL}}
\newcommand{\coNL}{\ComplexityFont{co-NL}}
\renewcommand{\L}{\ComplexityFont{L}}
\newcommand{\NPcomp}{\ComplexityFont{NP-complete}}
\newcommand{\tP}{\widetilde{\P}}
\newcommand{\tNP}{\widetilde{\NP}}
\newcommand{\tBH}{\widetilde{\BH}}
\newcommand{\Class}{{\ComplexityFont{C}}}
\newcommand{\coC}{\ComplexityFont{co-}\mathcal{C}}
\newcommand{\coNP}{\ComplexityFont{co-NP}}
\newcommand{\PH}{\ComplexityFont{PH}}
\newcommand{\EXP}{\ComplexityFont{EXP}}
\newcommand{\Size}{\ComplexityFont{Size}}
\newcommand{\Ppoly}{\ComplexityFont{P}/\ComplexityFont{poly}}
\newcommand{\NC}{\ComplexityFont{NC}}


\newcommand{\FACTOR}{\LanguageFont{FACTOR}}
\newcommand{\kQBF}{{\LanguageFont{QBF{\tiny k}}}}
\newcommand{\QBFk}{{\LanguageFont{QBF{\tiny k}}}}
\newcommand{\QBF}{{\LanguageFont{QBF}}}
\newcommand{\STCON}{\LanguageFont{STCON}}
\newcommand{\USTCON}{\LanguageFont{USTCON}}
\newcommand{\CircuitSat}{{\LanguageFont{CIRCUIT\_SAT}}}
\newcommand{\tCircuitSat}{\widetilde{{\LanguageFont{CIRCUIT\_SAT}}}}
\newcommand{\SAT}{\LanguageFont{SAT}}
\newcommand{\tSAT}{\widetilde{{\LanguageFont{SAT}}}}
\newcommand{\UNSAT}{{\LanguageFont{UNSAT}}}
\newcommand{\tThreeSAT}{\widetilde{{\LanguageFont{3\text{-}SAT}}}}
\newcommand{\ThreeSAT}{{\LanguageFont{3\text{-}SAT}}}
\newcommand{\BH}{\LanguageFont{BH}}
\newcommand{\CircuitEval}{{\LanguageFont{CIRCUIT\_EVAL}}}


\newcommand{\const}{\textmd{const}}
\newcommand{\logspace}{\textmd{logspace}}
\newcommand{\PATH}{\textmd{PATH}}


\newcommand{\readonly}{\textsf{read-only}}
\newcommand{\writeonly}{\textsf{write-only}}


\usepackage{ upgreek }
\newcommand{\PI}{\Uppi}
\newcommand{\SIGMA}{\Upsigma}
\newcommand{\DELTA}{\Updelta}


\begin{document}
\lstset{ %
language=C++,                 % выбор языка для подсветки (здесь это С)
basicstyle=\small\sffamily, % размер и начертание шрифта для подсветки кода
numbers=left,               % где поставить нумерацию строк (слева\справа)
numberstyle=\tiny,           % размер шрифта для номеров строк
stepnumber=1,                   % размер шага между двумя номерами строк
numbersep=7pt,                % как далеко отстоят номера строк от подсвечиваемого кода
backgroundcolor=\color{white}, % цвет фона подсветки - используем \usepackage{color}
showspaces=false,            % показывать или нет пробелы специальными отступами
showstringspaces=false,      % показывать или нет пробелы в строках
showtabs=false,             % показывать или нет табуляцию в строках
frame=single,              % рисовать рамку вокруг кода
tabsize=4,                 % размер табуляции по умолчанию равен 2 пробелам
captionpos=t,              % позиция заголовка вверху [t] или внизу [b]
breaklines=true,           % автоматически переносить строки (да\нет)
breakatwhitespace=false, % переносить строки только если есть пробел
escapeinside={\%*}{*)}   % если нужно добавить комментарии в коде
}

\documentclass[11pt]{book}
\usepackage[english, russian]{babel}
\usepackage{xltxtra}
\usepackage{polyglossia}

\usepackage{mathpazo}

\defaultfontfeatures{Ligatures=TeX,Mapping=tex-text}

\setmainfont{STIX2Text-Regular.otf}[
ExternalLocation={/home/vyacheslav/builds/STIXv2.0.2/OTF/},
BoldFont=STIX2Text-Bold.otf,
ItalicFont=STIX2Text-Italic.otf,
BoldItalicFont=STIX2Text-BoldItalic.otf
]
\setmathrm{STIX2Math.otf}[
ExternalLocation={/home/vyacheslav/builds/STIXv2.0.2/OTF/}
]


\usepackage{makeidx}
\usepackage{amssymb, amsthm}
\usepackage{amsmath}
\usepackage{mathtools}
\usepackage{needspace}
\usepackage{enumitem}
\usepackage{cancel}
\usepackage{fdsymbol}
\usepackage{fontawesome}


% разметка страницы и колонтитул
\usepackage[left=2cm,right=2cm,top=1cm,bottom=1.1cm,bindingoffset=0cm]{geometry}
\usepackage{fancybox,fancyhdr}
\fancyhf{}
\fancyhead[R]{\thepage}
\fancyhead[L]{\rightmark}
\fancyfoot{}
\fancyhfoffset{0pt}
\addtolength{\headheight}{13pt}
\pagestyle{fancy}

% Отступы
\setlength{\parindent}{3ex}
\setlength{\parskip}{3pt}

\usepackage{graphicx}
\usepackage{hyperref}

\usepackage{import}
\usepackage{xifthen}
\usepackage{pdfpages}

\newcommand{\incfig}[1]{%
    \def\svgwidth{\columnwidth}
    \import{./figures/}{#1.pdf_tex}
}


\usepackage{xifthen}
\makeatother
\def\@lecture{}%
\newcommand{\lecture}[3]{
    \ifthenelse{\isempty{#3}}{%
        \def\@lecture{Лекция #1}%
    }{%
        \def\@lecture{Лекция #1: #3}%
    }%
    \subsection*{\@lecture}
    \marginpar{\small\textsf{\mbox{#2}}}
}
\makeatletter


\usepackage{xcolor}
\definecolor{Aquamarine}{cmyk}{50, 0, 17, 100}
\definecolor{ForestGreen}{cmyk}{76, 0, 76, 45}
\definecolor{Pink}{cmyk}{0, 100, 0, 0}
\definecolor{Cyan}{cmyk}{56, 0, 0, 100}
\definecolor{Gray}{gray}{0.3}


\usepackage{mdframed}
\mdfsetup{skipabove=3pt,skipbelow=3pt}
\mdfdefinestyle{defstyle}{%
    linecolor=red,
	linewidth=3pt,rightline=false,topline=false,bottomline=false,%
    frametitlerule=false,%
    frametitlebackgroundcolor=red!0,%
    innertopmargin=4pt,innerbottommargin=4pt,innerleftmargin=7pt
    frametitlebelowskip=1pt,
    frametitleaboveskip=3pt,
}
\mdfdefinestyle{thmstyle}{%
    linecolor=cyan!100,
	linewidth=2pt,topline=false,bottomline=false,%
    frametitlerule=false,%
    frametitlebackgroundcolor=cyan!20,%
    innertopmargin=4pt,innerbottommargin=4pt,
    frametitlebelowskip=1pt,
    frametitleaboveskip=3pt,
}
\theoremstyle{definition}
\mdtheorem[style=defstyle]{defn}{Определение}

\newmdtheoremenv[nobreak=true,backgroundcolor=Aquamarine!10,linewidth=0pt,innertopmargin=0pt,innerbottommargin=7pt]{cor}{Следствие}
\newmdtheoremenv[nobreak=true,backgroundcolor=CarnationPink!20,linewidth=0pt,innertopmargin=0pt,innerbottommargin=7pt]{desc}{Описание}
\newmdtheoremenv[nobreak=true,backgroundcolor=Gray!10,linewidth=0pt,innertopmargin=0pt,innerbottommargin=7pt,font={\small}]{ex}{Пример}
\newmdtheoremenv[nobreak=false,backgroundcolor=Cyan!10,linewidth=0pt,innertopmargin=0pt,innerbottommargin=7pt]{thm}{Теорема}
\newmdtheoremenv[nobreak=true,backgroundcolor=Pink!10,linewidth=0pt,innertopmargin=0pt,innerbottommargin=7pt]{lm}{Лемма}

\newtheorem*{st}{Утверждение}
\newtheorem*{prop}{Свойства}

\theoremstyle{plain}
\newtheorem*{name}{Обозначение}

\theoremstyle{remark}
\newtheorem*{rem}{Ремарка}
\newtheorem*{com}{Комментарий}
\newtheorem*{note}{Замечание}
\newtheorem*{prac}{Упражнение}
\newtheorem*{probl}{Задача}


\renewcommand{\proofname}{Доказательство}
\renewenvironment{proof}
{ \hspace{\stretch{1}}\\ \faSquareO\quad \small  }
{ \hspace{\stretch{1}}  \faSquare \normalsize }


\numberwithin{ex}{section}
\numberwithin{thm}{section}
\numberwithin{equation}{section}



\newcommand{\K}{\mathcal{K}}
\newcommand{\Z}{\mathbb{Z}}
\newcommand{\N}{\mathbb{N}}
\newcommand{\Real}{\mathbb{R}}
\newcommand{\Q}{\mathbb{Q}}
\newcommand{\Cm}{\mathbb{C}}
\newcommand{\Pm}{\mathbb{P}}
\newcommand{\ord}{\operatorname{ord}}
\newcommand{\lcm}{\operatorname{lcm}}
\newcommand{\sign}{\operatorname{sign}}
\newcommand{\E}{\mathbb{E}}

\renewcommand{\o}{o}
\renewcommand{\O}{\mathcal{O}}
\renewcommand{\le}{\leqslant}
\renewcommand{\ge}{\geqslant}

\def\mybf#1{\textbf{#1}}
\def\selectedFont#1{\textbf{#1}}
\def\ComplexityFont#1{\textmd{\textbf{\textsf{#1}}}}
\def\LanguageFont#1{{\textbf{\texttt{#1}}}}


\newcommand{\Cclass}{\mathcal{C}}
\newcommand{\Dclass}{\mathcal{D}}


\renewcommand{\P}{\ComplexityFont{P}}
\newcommand{\DTIME}{\ComplexityFont{DTime}}
\newcommand{\DTime}{\ComplexityFont{DTime}}
\newcommand{\DSpace}{\ComplexityFont{DSpace}}
\newcommand{\PSPACE}{\ComplexityFont{PSPACE}}
\newcommand{\NTIME}{\ComplexityFont{NTime}}
\newcommand{\NSpace}{\ComplexityFont{NSpace}}
\newcommand{\coNSpace}{\ComplexityFont{coNSpace}}
\newcommand{\NPSPACE}{\ComplexityFont{NPSPACE}}
\newcommand{\poly}{\ComplexityFont{poly}}
\newcommand{\RP}{\ComplexityFont{RP}}
\newcommand{\coRP}{\ComplexityFont{co-RP}}
\newcommand{\ZPP}{\ComplexityFont{ZPP}}
\newcommand{\BPP}{\ComplexityFont{BPP}}
\newcommand{\BQP}{\ComplexityFont{BQP}}
\newcommand{\coBPP}{\ComplexityFont{co-BPP}}
\newcommand{\NP}{\ComplexityFont{NP}}
\newcommand{\NL}{\ComplexityFont{NL}}
\newcommand{\coNL}{\ComplexityFont{co-NL}}
\renewcommand{\L}{\ComplexityFont{L}}
\newcommand{\NPcomp}{\ComplexityFont{NP-complete}}
\newcommand{\tP}{\widetilde{\P}}
\newcommand{\tNP}{\widetilde{\NP}}
\newcommand{\tBH}{\widetilde{\BH}}
\newcommand{\Class}{{\ComplexityFont{C}}}
\newcommand{\coC}{\ComplexityFont{co-}\mathcal{C}}
\newcommand{\coNP}{\ComplexityFont{co-NP}}
\newcommand{\PH}{\ComplexityFont{PH}}
\newcommand{\EXP}{\ComplexityFont{EXP}}
\newcommand{\Size}{\ComplexityFont{Size}}
\newcommand{\Ppoly}{\ComplexityFont{P}/\ComplexityFont{poly}}
\newcommand{\NC}{\ComplexityFont{NC}}


\newcommand{\FACTOR}{\LanguageFont{FACTOR}}
\newcommand{\kQBF}{{\LanguageFont{QBF{\tiny k}}}}
\newcommand{\QBFk}{{\LanguageFont{QBF{\tiny k}}}}
\newcommand{\QBF}{{\LanguageFont{QBF}}}
\newcommand{\STCON}{\LanguageFont{STCON}}
\newcommand{\USTCON}{\LanguageFont{USTCON}}
\newcommand{\CircuitSat}{{\LanguageFont{CIRCUIT\_SAT}}}
\newcommand{\tCircuitSat}{\widetilde{{\LanguageFont{CIRCUIT\_SAT}}}}
\newcommand{\SAT}{\LanguageFont{SAT}}
\newcommand{\tSAT}{\widetilde{{\LanguageFont{SAT}}}}
\newcommand{\UNSAT}{{\LanguageFont{UNSAT}}}
\newcommand{\tThreeSAT}{\widetilde{{\LanguageFont{3\text{-}SAT}}}}
\newcommand{\ThreeSAT}{{\LanguageFont{3\text{-}SAT}}}
\newcommand{\BH}{\LanguageFont{BH}}
\newcommand{\CircuitEval}{{\LanguageFont{CIRCUIT\_EVAL}}}


\newcommand{\const}{\textmd{const}}
\newcommand{\logspace}{\textmd{logspace}}
\newcommand{\PATH}{\textmd{PATH}}


\newcommand{\readonly}{\textsf{read-only}}
\newcommand{\writeonly}{\textsf{write-only}}


\usepackage{ upgreek }
\newcommand{\PI}{\Uppi}
\newcommand{\SIGMA}{\Upsigma}
\newcommand{\DELTA}{\Updelta}


\title{Билеты к экзамену по теории графов\\2022 год\\
    факультет Математики и Компьютерных наук, СПбГУ\\
(лекции Карпова Дмитрия Валерьевича)}
\date{\today}
\author{Вячеслав Тамарин}

\begin{document}
\maketitle
\tableofcontents
\hspace{1em}
\begin{center}
	Исходный код на \url{https://github.com/tamarinvs19/theory_university}
\end{center}
\chapter{Пути и циклы} 
\section{Существование гамильтонова пути и цикла: теорема Оре}

\begin{lemma}\label{lm:circle_1}
    Пусть $n > 2$, $a_1 \ldots a_n$ --- максимальный путь (по ребрам) в графе $G$, причем $d_{G}(a_1) + d_{G}(a_{n}) \ge n$. Тогда в графе есть цикл длины $n$.
\end{lemma}
$N_G(v)$ --- все вершины смежные с вершиной $v$ в графе $G$.

$d_G(v)$ --- степень вершины  $v$ в графе $G$.

\begin{proof}
	Разберем несколько случаев:
	\begin{itemize}
		\item Если $a_1$ и $a_{n}$ смежны, то $a_1a_2\ldots a_{n}$ --- искомый цикл.
		\item Иначе $N_{G}(a_1), N_{G}(a_{n}) \subset \{a_2, \ldots a_{n-1}\}$, так как удлинить путь нельзя.

			\begin{minipage}{0.6\textwidth}
				Если есть вершина $a_{k}$ смежная с $a_{n}$ и вершина $a_{k+1}$ смежная с $a_1$, то в графе есть цикл из $n$ вершин
					\[
					a_1a_2\ldots a_{k}a_{n}a_{n-1}\ldots a_{k+1}
					.\] 
			\end{minipage}
			\hfill
			\begin{minipage}{0.25\textwidth}
				\centering
				\incfig{lm-circle}
				\label{fig:lm-circle}
			\end{minipage}
			Пусть $N_G(a_{n}) = \{a_{i_1}, \ldots , a_{i_l}\}$.

			Если хотя бы одна из вершин $a_{i_{1}+1}, \ldots , a_{i_{l}+1}$ лежит в $N_{G}(a_1)$, то, согласно утверждению выше, в графе есть цикл длины $n$. 

			Иначе $d_G(a_1) \le n-1 - d_G(a_n)$, а это противоречит условию.
	\end{itemize}
\end{proof}

\begin{theorem}[Критерий Оре, 1960]
	\begin{enumerate}
		\item Если для любых двух несмежных вершин $u, v \in V(G)$ выполняется 
			\[
			d_G(u) + d_{G}(v) \ge v(G) - 1
			,\] 
			то в графе $G$ есть гамильтонов путь.
		\item Если  $v(G) > 2$ и для любых двух несмежных вершин  $u, v \in V(G)$ выполняется
			\[
			d_{G}(u) + d_{G}(v) \ge v(G)
			,\] 
			то в графе $G$ есть гамильтонов цикл.
	\end{enumerate}
\end{theorem}
\begin{proof}
\begin{enumerate}
	\item Докажем первое утверждение
	\begin{itemize}
	\item Для двух вершин все очевидно. Далее предположим, что  $v(G) > 2$.
	\item Рассмотрим две вершины $a$ и $b$ и  предположим, что они несмежные. По условию $d_G(a)+ d_{G}(b) \ge v(G) -1 $, поэтому $N_G(a) \cap N_G(b) \neq \varnothing$, следовательно, $a$ и $b$ связаны. Тогда граф $G$ связен.
	\item Теперь найдем наибольший простой путь $a_1\ldots a_{n}$ в графе $G$. Так как вершин больше двух, и граф связен, $n \ge 3$. Предположим, что это не гамильтонов путь, то есть $n \le v(G) - 1$.
	\item Если $a_1\ldots a_{n}$ не цикл, то по лемме \ref{lm:circle_1} существует цикл $Z$ из $n$ вершин, так как 
		\[
		d_G(a_1) + d_G(a_{n}) \ge v(G) - 1 \ge n
		.\] 
	\item Так как граф связен, существует не вошедшая в этот цикл вершина, смежная с хотя бы одной из вершин цикла. Тогда из нее и цикла можно получить путь длиной $n+1$, противоречие.
	\end{itemize}
\item По первому пункту уже есть гамильтонов путь, обозначим его за $a_1\ldots a_{n}$, где $n = v(G)$.

	Если $a_1$ и $a_n$ смежны, то мы нашли гамильтонов цикл. Иначе
	\[
	d_G(a_1) + d_G(a_{n}) \ge v(G) = n
	.\] 
	А тогда по лемме \ref{lm:circle_1} в графе есть гамильтонов цикл.
\end{enumerate}    
\end{proof}

\begin{corollary}[Критерий Дирака, 1952]
	\begin{enumerate}
		\item Если $\delta(G) \ge \frac{v(G)-1}{2}$, то в графе $G$ есть гамильтонов путь.
		\item  Если $\delta(G) \ge \frac{v(G)}{2}$, то в графе $G$ есть гамильтонов цикл.
	\end{enumerate}
\end{corollary}
 
\section{Существование гамильтонова пути и цикла: замыкание Хватала}

\begin{lemma}\label{lm:circle_2}
    Пусть вершины $a$ и $b$ не смежны и $d_G(a) + d_G(b) \ge v(G)$. Тогда граф $G$ гамильтонов, согда граф $G + ab$ тоже гамильтонов.
\end{lemma}
\begin{proof}
	\begin{itemize}
		\item Если $G$ гамильтонов, то и граф с дополнительным ребром $ab$ тоже гамильтонов.
		\item Докажем следствие в обратную сторону. Пусть граф $G + ab$ гамильтонов. 
			\begin{itemize}
				\item Если гамильтонов цикл не проходит по ребру $ab$, то он есть и в графе $G$. 
				\item Если проходит по $ab$, то в $G$ есть гамильтонов путь, причем сумма степеней его концов не меньше $v(G)$, тогда по лемме \ref{lm:circle_1} в графе $G$ есть гамильтонов цикл.
			\end{itemize}
	\end{itemize}    
\end{proof}

\begin{definition}[Замыкание графа]
	Рассмотрим произвольный граф $G$. Пока существуют две вершины $a, b \in V(G)$, для которых $d_G(a) + d_G(b) \ge v(G)$, добавим в граф соответствующее ребро $ab$. Полученный граф называется \selectedFont{замыканием} графа $G$, обозначается $C(G)$.
\end{definition}

\begin{corollary}[Хватал, 1974]
    Граф $G$ гамильтонов, согда его замыкание $C(G)$ --- гамильтонов граф.
\end{corollary}

\begin{lemma}[о единственности замыкания]\label{lm:circle_3}
    Замыкание графа $G$ определено однозначно, то есть не зависит от порядка добавления ребер.
\end{lemma}
\begin{proof}
    Пусть в результате двух различных цепочек добавления ребер были получены различные графы $G_1$ и $G_2$.

	Тогда есть ребра, добавленные при построении $G_1$, которых нет в $G_2$. Найдем такое ребро $ab$, которое было добавлено первым. 

	Обозначим граф, к которому мы добавили $ab$, за $G_0$. Тогда $d_{G_0}(a) + d_{G_0}(b) \ge v(G)$.

	С другой стороны, все ребра, добавленные к $G$ при построении $G_0$, добавлены и $G_2$. Поэтому, $d_{G_2}(a) + d_{G_2}(b) \ge v(G)$, следовательно, в $G_2$ нет ребра, которое мы должны были добавить. Противоречие. 
\end{proof}
 
\section{Критерий существования гамильтонова цикла через связность.}
\begin{lemma}\label{lm:circle_3}
    Пусть граф $G$ гамильтонов. Тогда для любого множества $S \subset V(G)$
	выполняется неравенство $c(G - S) \le \lvert S \rvert$\footnote{$c(G)$ --- число компонент связности в графе $G$}.
\end{lemma}
\begin{proof}
	Пусть $c(G-S) = c$ и $U_1, \ldots , U_c$ --- компоненты связности графа $G - S$, $Z$ --- гамильтонов цикл графа $G$.

	Начнем обходить цикл $Z$, начиная с вершины из множества $S$. Пусть $s_i$ --- вершина, которая предшествует первому входу цикла в компоненту $U_i$.

	Все $s_i$ различны, причем принадлежат $S$, так как не могут входить ни в одну из компонент (иначе это одна компонента, а тогда вершина неправильная). Отсюда следует требуемое неравенство.
\end{proof}

\begin{theorem}[Хватал, Эрдёшь, 1972]
	Пусть $v(G) \ge 3$ и $\kappa(G) \ge \alpha(G)$\footnote{$\kappa(G)$ --- вершинная связность, $\alpha(G)$ --- размер максимального независимого множества}, тогда $G$ гамильтонов.
\end{theorem}
\begin{proof}
    \begin{itemize}
		\item Если в графе нет циклов, то есть $\kappa(G) == 1$. Тогда точно $\alpha(G) \ge 2$, так как вершин не меньше трех. По условию такой случай невозможен.
		\item Пусть $\kappa(G) = k$. Выберем цикл $C$ максимальной длины в графе $G$. Пусть  $C = v_1v_2\ldots v_n$\footnote{Считаем, что нумерация циклическая}.
		\item Пусть $C$ не гамильтонов. Рассмотрим компоненту связности $W$ графа $G - V(C)$. Заметим, что $N_G(W) \subset V(C)$.
				\begin{figure}[ht]
					\centering
					\begin{subfigure}{0.48\linewidth}
						\centering
						\incfig{thm-khvatal-ehrdesh}
						\caption{}
						\label{fig:thm-khvatal-ehrdesh}
					\end{subfigure}
					\begin{subfigure}{0.48\linewidth}
						\centering
						\incfig{thm-khvatal-ehrdesh-2}
						\caption{}
						\label{fig:thm-khvatal-ehrdesh-2}
					\end{subfigure}
				\end{figure}
		\item (a) Обозначим за $M = \{v_{i+1}\colon v_i \in N_{G}(W)\}$. Докажем, что $M \cap N_G(W) = \varnothing$.
			\begin{itemize}
				\item Пусть $v_i, v_{i+1} \in  N_G(W)$ и $w, w' \in  W$, $v_iw, v_{i+1}w' \in  E(G)$, $P$ --- $ww'$-путь по вершинам из $W$.
				\item Тогда можно удлинить цикл $C$ хотя бы на одно ребро, заменив ребро $v_iv_{i+1}$ на ребро $v_iw$, далее путь $P$, потом ребро $w'v_{i+1}$. Но цикл $C$ должен был быть максимальным. Противоречие. 
			\end{itemize}
			Это означает, что $N_G(W)$ отделяет непустое множество $M$ от $W$, следовательно $\lvert M \rvert = \lvert N_G(W) \rvert \ge k$
		\item (b)
			\begin{itemize}
				\item Теперь предположим, что вершины $v_{i+1}, v_{j+1} \in  M$ смежны. Пусть $w, w' \in W$ и $v_iw, v_jw' \in E(G)$ и $P$ --- $ww'$-путь по вершинам компоненты $W$.
				\item Рассмотрим цикл $Z$, проходящий сначала участок $v_{j+1}v_{j+2}\ldots v_i$ цикла $C$, затем ребро  $v_iw$, далее путь $P$ и ребро $w'v_j$, потом участок $v_jv_{j-1}\ldots v_{i+1}$ по циклу $C$ и ребро $v_{i+1}v_{j+1}$. Построенный цикл $Z$ длиннее $C$, противоречие.
			\end{itemize}
			Из этого следует, что $M \cup \{w\}$ --- независимое множество с $\lvert M \rvert + 1 > k$ вершин. А это противоречит условию $\alpha(G) \le k$. 
		\item В итоге, цикл $C$ должен быть гамильтоновым.
    \end{itemize}
\end{proof}
 
\section{Теорема Хватала о гамильтоновых последовательностях.}
\begin{definition}
	\begin{enumerate}
		\item Пусть $a_1 \le a_2 \le  \ldots a_n$ и $b_1 \le b_2 \le \ldots b_n$ --- две упорядоченные последовательности. Последовательность $\{a_i\}_{i \in [1..n]}$ \selectedFont{мажорирует} последовательность $\{b_i\}_{i \in [1..n]}$, если $ \forall i \in  [1..n]\colon a_i \ge b_i$.
		\item Пусть $G$ --- граф на $n$ вершинах. \selectedFont{Степенная последовательность} графа $G$ --- упорядоченная последовательность степеней его вершин $d_1 \le \ldots d_n$. 
		\item Будем говорить, что граф $G$ \selectedFont{мажорирует} граф $H$, если $v(G) = v(H)$ и степенная последовательность графа $G$ мажорирует степенную последовательность графа $H$.
		\item Последовательность  $a_1 \le a_2 \le \ldots a_n$ называется \selectedFont{гамильтоновой}, если $a_n \le n-1$ и любой граф на $n$ вершинах, степенная последовательность которого мажорирует $a_1, \ldots a_n$, имеет гамильтонов цикл.
	\end{enumerate}
\end{definition}

\begin{theorem}[Критерий Хватала, 1972]
    Пусть $0 \le a_1 \le a_2 \le \ldots \le a_n \le n-1$, $n \ge 3$. Следующие два утверждения равносильны:
	\begin{enumerate}
		\item Последовательность $a_1, \ldots a_n $ гамильтонова.
		\item Для каждого $s < \frac{n}{2}$ из $a_s \le s$ следует $a_{n-s} \ge  n-s$.
	\end{enumerate}
\end{theorem}
\begin{proof}
	\begin{description}
		\item[$2 \implies 1$] Предположим, что наша последовательность негамитльтонова. Рассмотрим негамильтонов граф $G$ на $n$ вершинах с максимальным числом ребер, степенная последовательность $\{d_i\}_{i \in [1..n]}$ которого мажорирует $\{a_i\}_{i \in [1..n]}$. 

			По лемме \ref{lm:circle_2} граф $G$ совпадает со своим замыканием, так как граф максимальный, и сумма степеней любых двух несмежных вершин менее $n$.

			Рассмотрим две несмежные вершины $x, y \in V(G)$ c максимальной суммой $d_G(x) + d_G(y)$, такие есть, иначе граф полный, и точно гамильтонов. Не умаляя общности $d_G(x) \le d_G(y)$.

			Так как $d_G(x) + d_G(y) < n$, имеем $d_G(x) = s < \frac{n}{2}$, поэтому $d_G(y) \le n-1 -s$.

			Пусть $W_x$ --- множество всех вершин графа $G$, отличных от $x$ и не смежных с $x$, $W_y$ --- аналогично для $y$.
			\[
			\lvert W_x \rvert = n-1-d_G(x) = n-1-s; \qquad \lvert W_y \rvert = n-1-d_G(y) \ge d_G(x) = s
			.\] 

			Степени всех вершин множества $W_y$ не превосходят $s $, так как $s$ дает максимальную сумму с $d_G(y)$. Поэтому  $a_s \le d_s \le s$. В множестве $W_x \cup \{x\}$ будет $n-s$ вершин, причем их степени не превосходят $d_G(y) \le n-1-s$, поэтому $a_{n-s} \le d_{n-s} \le n-s-1$.

			Но это противоречит условию. Следовательно, последовательность $\{a_i\}$ гамильтонова.
		\item[$1 \implies 2$] Докажем, что последовательность  $\{a_i\}$ не может быть гамильтоновой, если не выполнено второе условие.
			
			Пусть $h < \frac{n}{2}, a_h \le h$ и $a_{n-h} \le n - h -1$. Построим негамильтонов граф $G_{n, h}$, степенная последовательность которого мажорирует $\{a_i\}$.

			Пусть $A = \{v_1, \ldots , v_h\}$, $B = \{v_{n-h+1}, \ldots , v_n\}$, $D = \{v_{h+1}, \ldots, v_{n-h} \}$. Граф $G_{n, h}$ будет объединением $K_{h, h}$ с долями $A$ и $B$ и $K_{n-h}$ на вершинах $B \cup D$ :
		\begin{figure}[ht]
			\centering
			\incfig{existing-gamiltonov-cycle}
			\caption{Граф $G_{n,h}$}
			\label{fig:existing-gamiltonov-cycle}
		\end{figure}
	\end{description}

	Здесь все степени в $A$ равны $h$, в $B$ --- $n-1$, в $D$ --- $n-h-1$. Степенная последовательность выглядит следующим образом:
	 \[
		 \underbrace{h, \ldots , h}_{h}, \underbrace{n-h-1, \ldots , n-h-1}_{n-2h}, \underbrace{n-1, \ldots , n-1}_{h}
	.\] 

	Эта последовательность мажорирует $a_1, \ldots , a_n$.
	
	Всего компонент связности $c(G_{n,h} - B) = h+1 = \lvert B \rvert$, это $D$ и отдельные вершины в $A$.

	Так как $c(G_{n,h}-B) >h = \lvert B \rvert$, можем применить лемму \ref{lm:circle_4} и получить, что $G_{n, h}$ не является гамильтоновым.

\end{proof}
 
\section{Гамильтонов цикл в кубе связного графа.}
\begin{definition}
    Для графа $G$ и натурального числа  $d$ обозначим через $G^{d}$ граф на вершинах из  $V(G)$, в котором вершины $x$ и $y$ смежны, согда $\dist_G(x, y) \le d$.
\end{definition}
\begin{theorem}[Чартранд, Капур, 1969]
	Для любого связного графа $G$ с $v(G) \ge 3$ и ребра $e \in E(G)$ в графе $G^3$ существует гамильтонов цикл, содержащий ребро $e$.
\end{theorem}
\begin{proof}
    Достаточно доказать теорему для дерева, так как иначе можем просто выделить остовное.

	Будем доказывать индукцией по количеству вершин. 

	\begin{description}
		\item[База:] для трех или четырех вершин очевидно, так как $G^3$ --- полный граф.
		\item[Переход:] пусть для меньших деревьев теорема доказана. 

			Рассмотрим ребро $uv$. $G$ --- дерево, поэтому в $G-uv$ разбивается на две компоненты связности $U \ni u$ и $V \ni v$. Пусть $G_u = G(U)$, $G_v = G(V)$. НУО  $\lvert U \rvert \ge 3$. Тогда в $G_u^3$ по предположению индукции есть гамильтонов цикл, содержащий ребро $ux \in E(G(U))$.

			\begin{enumerate}[label=(\alph*)]
				\item Если $\lvert V \rvert \ge 3$, аналогично строим гамильтонов цикл в $G_v^3$, содержащий инцидентное вершине $v$ ребро $vy \in E(G(V))$, и соединяем эти циклы в один, заменив $ux$ и $vy$ на $uv$ и $xy$ (ребро $xy \in E(G^3)$, так как $\dist_G(x, y) \le 3$). 
				\item Если $\lvert V \rvert =2$, точно есть ребро $vy \in E(G)$, которое мы просто присоединяем к циклу из $U$ вместо ребра $ux$.
				\item Если  $\lvert V \rvert = 1$, заменяем ребро $ux$ на $uv$ и $vx$.
			\end{enumerate}
	\begin{figure}[ht]
		\centering
		\begin{subfigure}{0.32\linewidth}
			\centering
			\incfig{capoor-thm}
			\caption{}
			\label{fig:capoor-thm}
		\end{subfigure}
		\begin{subfigure}{0.32\linewidth}
			\centering
			\incfig{capoor-thm-b}
			\caption{}
			\label{fig:capoor-thm-b}
		\end{subfigure}
		\begin{subfigure}{0.32\linewidth}
			\centering
			\incfig{capoor-thm-c}
			\caption{}
			\label{fig:capoor-thm-c}
		\end{subfigure}
	\end{figure}
	\end{description}
\end{proof}
 
\section{Теорема Татта о существовании регулярного графа степени k с обхватом g.}
\begin{definition}
	\selectedFont{Обхват графа} $G$ (обозначение  $g(G)$ ) --- длина наименьшего цикла в графе $G$.
\end{definition}
\begin{theorem}[Тaтт]
	Пусть $k, g, n \in \N$, причем $k, g \ge 3$, $kn$ четно и 
	 \[
	n > \frac{k(k-1)^{g-1} -2}{k-2}
	.\] 
	Тогда существует регулярный граф $G$ степени $k$ с $g(G) = g$ и $v(G) = n$.
\end{theorem}
\begin{proof}
		Пусть $\G(n, g, k)$ --- множество всех графов на $n$ вершинах с обхватом $g$ и максимальной степенью вершин не более $k$.

		Пусть $v_{<k}(G)$ --- количество вершин степени менее $k$ в графе $G$, $\dist_{<k}(G)$ --- максимальное из расстояний между парами вершин степени менее $k$ в графе $g$, при $v_{<k}(G) < 2$ положим $\dist_{<k}(G) = 0$.

		Если $n > g$, $\G(n, g, k) \neq \varnothing$, например, есть граф из цикла на $g$ вершинах и нескольких изолированных вершинах.

		Будем выбирать в $\G(n, g, k)$ граф следующим образом:
			\begin{enumerate}
				\item сначала возьмем все графы с максимальным количеством ребер,
				\item затем из них выберем графы с максимальным $v_{<k}$,
				\item из оставшихся выберем граф $G$ с максимальным $\dist_{<k}(G)$.
			\end{enumerate}

		{\bf Докажем, что $G$ --- регулярный граф степени $k$.}
		\begin{itemize}
			\item Пусть не так. Рассмотрим пару его максимально удаленных вершин степени менее $k$. Пусть это $x$ и  $y$ (возможно $x = y$).
			\item Если $\dist_G(x, y) \ge g-1$, то соединим $x$ и $y$ и получим граф $G' \in \G(n, g, k)$  с $e(G') > e(G)$, а такого не должно быть.
				Следовательно, $\dist_{G}(x, y) \le g-2$.
			\item Так как степени $x$ и $y$ меньше $k$, а степени всех остальных не больше $k$, то на расстоянии не более $g-1$ от $y$ находится не более чем $\frac{(k-1)^{g}-1}{k-2}$ вершин, а на расстоянии не более $g-2$ от $x$ не более $\frac{(k-1)^{g-1}-1}{k-2}$ вершин.
			\item По условию теоремы существует такая вершина $z$, что $\dist(x, z) \ge g-1$ и $\dist(y,z) \ge g$. % TODO: почему?
			\item Так как $\dist_{G}(x, y) \le g-2$, степень $d_G(z) = k \ge 3$. Следовательно, есть ребро $zu \in  E(G)$, через которое проходят не все простые циклы длины $g$ графа $G$. Тогда $g(G-zu) = g(G) = g$. % TODO: почему?
			\item $d_G(u) = k$, так как:
				 \[
				\dist_G(y, u) \ge \dist_G(y, z) -1 \ge g-1 > \dist(x,y) = dist_{<k}(G)
				.\] 
			\item Пусть $G' = G - zu + zx$.  $g(G') = g$, $e(G') = e(G)$, $d_{G'}(x) = d_{G}(x) + 1$, $d_{G'}(u) = d_{G}(u) -1 = k-1$, степени остальных вершин совпадают. Итого, $G' \in \G(n, g, k)$.
			\item Заметим, что $v_{<k}(G') \ge v_{<k}(G)$. По алгоритму выбора графа $G$ должно быть равенство, поэтому $d_{G'}(x) = k$ и $d_G(x) = k-1$.
			\item Так как $kn$ четно, вершина $x$ не может быть единственной вершиной степени меньше $k$ в графе $G$, следовательно, $x \neq  y$.
		\end{itemize}
		{\bf Докажем, что $\dist_{G'}(y,u) > \dist_{G}(y,x)$.}
		\begin{itemize}
			\item Найдем $yu$-путь  $P$, который реализует расстояние между $y$ и $u$ в $G'$.
			\item Если $P$ проходит  только про ребрам $G$, то
				\[
				\dist_{G'}(y,u) = \dist_G(y,u) \ge g-1 > \dist_G(y,x)
				.\] 
			\item Следовательно, $P$ проходит по новому ребру $zx$. Тогда $P$ содержит путь по ребрам графа $G$ от $y$ до  $x$ или $z$ и само ребро $zx$. 

				И, так как $\dist_G(y,z) \ge g > \dist_G(y,z)$:
				\[
				\dist_{G'}(y,u) \ge \min\bigl( \dist_G(y, x) + 1, \dist_G(y,z) +1) > \dist_G(y,x)
				.\] 
			\item Таким образом,
				\[
				\dist_{<k}(G') \ge \dist_{G'}(y,u) > \dist_G(y,x_ = \dist_{<k}(G)
				.\] 
				Противоречие. Значит, $G$ --- $k$-регулярный граф.
				% TODO: было бы круто добавить картинку
		\end{itemize}
\end{proof}
 
\chapter{Паросочетания}
\section{Независимые множества, паросочетания и покрытия в графе. Теорема Галлаи.}
\begin{definition}
	Множество вершин $U \subset V(G)$ называется \selectedFont{независимым}, если никакие две его вершины не смежны. Обозначим через $\alpha(G)$ количество вершин в максимальном независимом множестве графа $G$.
\end{definition}

\begin{definition}
    Множество ребер $M \subset E(G)$ называется \selectedFont{паросочетанием}, если никакие его два ребра не имеют общей вершины. Обозначим через $\alpha'(G)$ количество ребер в максимальном паросочетании графа $G$.
\end{definition}

\begin{definition}
    Паросочетание $M$ графа $G$ называется \selectedFont{совершенным}, если оно покрывает все вершины графа.
\end{definition}

\begin{definition}
    Будем говорить, что множество вершин $W \subset V(G)$ \selectedFont{покрывает} ребро $e \in E(G)$, если существует вершина $w \in W$, инцидентная $e$. Будем говорить, что множество ребер $F \subset E(G)$ \selectedFont{покрывает} вершину $v \in V(G)$, если существует ребро $f \in F$, инцидентное $v$.
\end{definition}

\begin{definition}
    Множество вершин $W \subset V(G)$ называется \selectedFont{вершинным покрытием}, если оно покрывает все ребра графа. Обозначим через $\beta(G)$ количество вершин в минимальном вершинном покрытии графа $G$.
\end{definition}

\begin{definition}
    Множество ребер $F \subset E(G)$ называется \selectedFont{реберным покрытием}, если оно покрывает все вершины графа. Обозначим через $\beta'(G)$ количество ребер в минимальном реберном покрытии графа $G$.
\end{definition}

\begin{lemma}\label{lm:matching_1}
    \begin{enumerate}
		\item $U \subset V(G)$ --- независимое множество, согда $V(G) \setminus U$ --- вершинное покрытие.
		\item $\alpha(G) + \beta(G) = v(G)$.
    \end{enumerate}
\end{lemma}
\begin{proof}
    \begin{enumerate}
		\item Если $U$ --- независимое множество, то все ребра из этих вершин выходят в $V(G) \setminus U$, значит все ребра покрываются $V(G) \setminus U$. Если $V(G) \setminus U$ --- вершинное покрытие, ребер внутри $U$ быть не может, следовательно $U$ --- независимое множество.
		\item Применяем первый пункт для максимального независимого множества и минимального вершинного покрытия.
    \end{enumerate}
\end{proof}

\begin{theorem}[Галлаи, 1959]
	Пусть $G$ --- граф с $\delta(G) > 0$. Тогда $$\alpha'(G) + \beta'(G) = v(G)$$.
\end{theorem}
\begin{proof}
    Докажем неравенство в обе стороны.
	\begin{itemize}
		\item Пусть $M$ --- максимальное паросочетание, $U$ --- множество не покрытых $M$ вершин графа. $\lvert U \rvert = v(G) - 2 \alpha'(G)$.

			Так как $\delta(G) > 0$, можно выбрать множество $F$ из $\lvert U \rvert $ ребер, покрывающее $U$. 

			Тогда $M \cup F$ --- покрытие,
			\[
			\beta'(G) \le \lvert M \cup F \rvert \alpha'(G) + v(G) - 2 \alpha'(G)
			.\] 
			Из этого получаем неравенство $\alpha'(G) + \beta'(G) \le v(G)$.
		\item Пусть $L$ --- минимальное реберное покрытие, $\lvert L \rvert = \beta'(G)$, Рассмотрим подграф $H = G(L)$, порожденный ребрами покрытия.

			Все компоненты связности в $H$ --- звезды, иначе $L$ не минимально. В каждой компоненте можем выбрать только одно ребро в паросочетание.

			Следовательно, $\alpha'(G) \ge c(H)$ и $\beta'(G) = \lvert L \rvert = e(H) \ge v(H) - c(H) = v(G) - c(H)$. Сложим два неравенства и получим $\alpha'(G) + \beta'(G) \ge v(G)$.
	\end{itemize}
\end{proof}
 
\section{Максимальное паросочетание и дополняющие пути: теорема Бержа.}
\begin{definition}
    Пусть $M$ --- паросочетание в графе $G$.
	\begin{itemize}
		\item Назовем путь \selectedFont{М-чередующимся}, если в нем чередуются ребра из $M$ и ребра, не входящие в  $M$.
		\item Назовем $M$-чередующийся путь \selectedFont{M-дополняющим}, если его начало и конец не покрыты паросочетанием $M$.
	\end{itemize}
\end{definition}

\begin{theorem}[Берж, 1957]
    Паросочетание $M$ в графе $G$ является максимальным, согда нет $M$-дополняющих путей.
\end{theorem}
\begin{proof}
    \begin{description}
		\item[$ \implies$ ] Пусть в графе  $G$ существует $M$-дополняющий путь $S = a_1a_2\ldots a_{2k}$.

			Тогда мы можем заменить все входящие в $M$ ребра $a_2a_3, \ldots , a_{2k-2}a_{2k-1}$ на не входящие в $M$ ребра $a_1a_2, \ldots , a_{2k-1}a_{2k}$, увеличив паросочетание. Противоречие. 
		\item[$ \impliedby$] Пусть $M$ --- не максимальное паросочетание, тогда рассмотрим максимальное $M'$.

		Пусть $N = M \vartriangle M'$ и подграф $H = G(N)$. Для любой вершины $v \in  H$ имеем $d_H(v) \in \{1, 2\}$, поэтому $H$ --- объединение путей и циклов. 

		Причем в каждом пути или цикле ребра из $M$ и $M'$ чередуются. Так как ребер из $M'$ больше, есть хотя бы одна компонента $P$ графа $H$ --- путь нечетной длины, где ребер из $M'$ больше. Получается, что мы нашли $M$-дополняющий путь. Противоречие.
    \end{description}
\end{proof}
 
\section{Теорема Татта о совершенном паросочетании.}
\begin{definition}
    Для произвольного графа $G$ через $o(G)$ обозначим количество нечетных компонент связности графа $G$.
\end{definition}

\begin{theorem}[Татт, 1947]
	В графе $G$ существует совершенное паросочетание, согда для любого $S \subset V(G)$ выполняется условие $o(G-S) \le \lvert S \rvert$.
\end{theorem}
\begin{proof}
    \begin{description}
		\item[$ \implies$ ] Пусть $S \subset V(G)$, $M$ --- совершенное паросочетание. Тогда одна из вершин каждой нечетной компоненты связности графа $G-S$ должна быть соединена с вершиной из $S$ ребром паросочетания $M$, при этом все такие вершины различны, так как входят в паросочетание только один раз.
		\item[$ \impliedby$ ] Предположим, что граф удовлетворяет условию, но не имеет совершенного паросочетания. 

			Подставим пустое $S$ в условие: $o(G) \le \lvert \varnothing \rvert = 0$, то есть $v(G)$ четно.

			Пусть $G^*$ --- максимальный надграф $G$ на том же множестве вершин, не имеющий совершенного паросочетания. Хотим построить совершенное паросочетание в $G^*$, тем самым получив противоречие.

			Для любого $S \subset V(G)$ выполняется неравенство
			\[
			o(G^*-S) \le o(G-S) \le \lvert S \rvert
			.\] 

			Пусть $U = \{u \in V(G) \colon d_{G^*}(u) = v(G) -1\}$. Очевидно, что $G^*$ не может быть полным, поэтому $U \neq V(G)$.
			\begin{lemma}\label{lm:matching_2}
				Граф $G^* - U$ представляет собой объединение нескольких несвязных друг с другом полных подграфов.
			\end{lemma}
			\begin{proof}
			    \begin{itemize}
					\item Предположим противное. Тогда существуют такие вершины $x, y, z \in V(G) \setminus U$, что $xy, yz \in E(G^*)$, но $xz \notin E(G^*)$.
					\item Так как $y \notin U$, существует такая вершина $w \notin U$,  что $yw \notin E(G^*)$.
					\item Так как граф $G^*$ максимален, в графе $G^*+xz$ существует паросочетание $M_1$, а в графе $G^* + yw$ --- $M_2$. При этом $xz \in M_1$ и $yw \in M_2$, иначе в $G^*$ будет совершенное паросочетание.
					\item Пусть $H = (V(G), M_1 \vartriangle M_2)$. Граф  $H$ --- несвязное объединение четных циклов, в каждом из которых чередуются ребра из $M_1$ и $M_2$, поэтому в каждой компоненте есть совершенные паросочетания на ребрах $M_1$ и на ребрах $M_2$.
					\item Ребра $xz$ и $yw$ принадлежат ровно одному паросочетание, поэтому лежат и  $E(H)$.
			    \end{itemize}
				Разберем два случая:
				\begin{enumerate}
					\item Ребра $xz$ и $yw$ лежат в разных компонентах $C_1$ и $C_2$ графа $H$.

						Тогда можем выбрать на вершинах $C_1$ выбрать паросочетание из $M_2$, в $C_2$ из $M_1$, в остальных из любых. Так мы получили совершенное паросочетание в графе $G^*$. Противоречие. 
					\item Ребра $xz$ и $yw$ лежат в одной компоненте $C$ графа $H$.

						НУО, считаем, что в цикле $C$ вершины расположены в порядке $ywzx$.
\begin{figure}[ht]
    \centering
    \incfig{tutta-thm}
    \caption{Случай 2}
    \label{fig:tutta-thm}
\end{figure}
					Рассмотрим простой путь $P=xC''yzC'w$. Заметим, что $V(P) = V(C)$ и $E(P) \subset E(G^*)$. Следовательно, существует совершенное паросочетание $M_C \subset E(G^*)$ на вершинах компоненты связности $C$.

					В остальных компонентах можем выбрать ребра любого из $M_1$ и $M_2$. Так мы построили совершенное паросочетание графа $G^*$. Противоречие. 
				\end{enumerate}
			\end{proof}

			Будем использовать лемму. 

			\begin{itemize}
				\item Среди несвязных полных графов не более $\lvert U \rvert$ имеет нечетное число вершин по условию теоремы.
				\item В каждой четной компоненте графа $G^*-U$ мы построим полное паросочетание, в каждой нечетной --- паросочетание на всех вершинах кроме одной, оставшуюся мы соединим с вершиной из $U$. Мы используем различные вершины из $U$, их хватит.
				\item Наконец, разобьем на пары оставшиеся вершины из $U$ : это можно сделать, так как каждая из них смежна в $G^*$ со всеми остальными. 
			\end{itemize}
			Так мы построили совершенное паросочетание в графе $G^*$. Противоречие. 
    \end{description}
\end{proof}
 
\section{Теорема Петерсена о паросочетании в кубическом графе.}
\begin{definition}
    \selectedFont{Кубический граф} --- граф, все вершины которого имеют степень $3$.
\end{definition}
\begin{definition}
    \selectedFont{Мост графа} --- ребро, не входящее ни в один цикл.
\end{definition}
\begin{theorem}[Петерсон, 1891]
	Пусть $G$ --- связный кубический граф, в котором не более двух мостов. Тогда в графе есть совершенное паросочетание.
\end{theorem}
\begin{proof}
    Пусть совершенного паросочетания нет. Тогда по Теореме Тaтта существует такое множество $S \subset V(G)$, что $o(G-S) > \lvert S \rvert$.

	Так как в кубическом графе четное число вершин, $S \neq \varnothing$ и $o(G-S) \equiv \lvert S \rvert \pmod 2$.

	Пусть $U_1, \ldots , U_n$ --- все нечетные компоненты связности графа $G - S$. $n \ge \lvert S \rvert + 2$.

	Пусть $m_i = e_G(U_i, S)$. Это число нечетное, так как:
	\[
	m_i = \sum_{v \in U_i}^{} d_G(v) - 2 e(G(U_i)) = 3 \lvert U_i \rvert - 2e(G(U_i))
	.\] 

	В $G$ не больше двух мостов, поэтому не более, чем два числа из $m_i$ равны $1$, а остальные не меньше $3$, так как нечетные.
	\[
	3 \lvert S \rvert = \sum_{v \in S}^{} d_G(v) \ge  \sum_{i=1}^{n} m_i \ge 3n-4 \ge 3(\lvert S \rvert+2) -4 > 3 \lvert S \rvert
	.\] 
	Противоречие. 
\end{proof}
 
\section{Теорема Плесника о совершенном паросочетании в регулярном графе.}
\begin{theorem}[Плесник, 1972]
    Пусть $G$ --- регулярный граф степени  $k$ с четным числом вершин, причем $\lambda(G) \ge k-1$, а граф $G'$ получен из $G$ удалением не более, чем $k-1$ ребра. Тогда в графе $G'$ есть совершенное паросочетание.
\end{theorem}
\begin{proof}
    Пусть множество $F \subset E(G)$ таково, что $G' = G - F$. Тогда $\lvert F \rvert \le k-1$.

	Предположим, что условие теоремы Татта не выполняется. Рассмотрим множество Татта $S \subset V(G')$. Так как  \[
		o(G'-S) + \lvert S \rvert  \equiv v(G) \pmod 2
	,\] 
	из $o(G'-S) > \lvert S \rvert$ следует, что $o(G'-S) \ge \lvert S \rvert + 2$.

	Пусть $U_1, \ldots , U_n$ --- нечетные, а $U_{n+1}, \ldots , U_t$ --- четные компоненты связности графа $G'-S$.

	Для каждого $i \in [1..t]$ пусть:
	\begin{itemize}
		\item $\alpha_i$ --- количество ребер из $E(G')$, соединяющих $U_i $ c $S$ ;
		\item $\beta_i$ --- количество ребер из $F$, соединяющих с $U_i$ с $S$;
		\item $\gamma_i$ --- количество ребер из $F$, соединяющих $U_i$ с остальными компонентами связности $G'-S$;
		\item  $m_i = \alpha_i + \beta_i + \gamma_i$  --- количество ребер графа $G$, соединяющих $U_i$ c $V(G) \setminus U_i$.
	\end{itemize}
	Для нечетных компонент связности имеем $m_i \equiv k \pmod 2$. Также $m_i \ge \lambda(G) \ge k-1$, поэтому $m_i \ge k$. Отсюда:
	\begin{equation}\label{eq:1}
		\sum_{i=1}^{n} \alpha_i + \sum_{i=1}^{n} \beta_i + \sum_{i=1}^{n} \gamma_i \ge kn
	\end{equation}

	Очевидно, что
	\[
	\sum_{i=1}^{t} \alpha_i + \sum_{i=1}^{t} \beta_i \le k \cdot \lvert S \rvert
	\] 
	и 
	\[
	2 \sum_{i=1}^{t} \beta_i + \sum_{i=1}^{t} \gamma_i \le 2 \cdot \lvert F \rvert \le 2k-2
	.\] 
	Сложим:
	\begin{equation}\label{eq:2}
		\sum_{i=1}^{t} \alpha_i + 3 \sum_{i=1}^{t} \beta_i + \sum_{i=1}^{t} \gamma_i \le k(\lvert S \rvert + 2) - 2
	\end{equation}

	Из неравенств \ref{eq:1} и \ref{eq:2} получаем $
	kn \le k(\lvert S \rvert+2)-2
	$,
	следовательно, $o(G'-S) < \lvert S \rvert + 2$. А мы выше доказали противное. Противоречие. 
\end{proof}
\begin{corollary}\label{cor:matching_2}
    Пусть $G$ --- регулярный граф степени $k$ с четным числом вершин, причем  $\lambda(G) \ge k-1$. Тогда для любого ребра $e \in E(G)$ существует совершенное паросочетание графа, содержащее $e$.
\end{corollary}
\begin{proof}
    Пусть $e = ab$, $e_1, \ldots , e_{k-1}$ --- остальные ребра, инцидентные вершине $a$.

	По теореме Плесника в графе $G - \{e_1, \ldots , e_{k-1}\}$ есть совершенное паросочетание, которое должно содержать $e$, так как содержит $a$.
\end{proof}
 
\section{Теорема Петерсена о выделении 2-фактора в 2k-регулярном графе и ее следствие о регулярных факторах.}
\begin{definition}
    \selectedFont{$k$-фактором} графа $G$ называется его остовный регулярный подграф степени $k$.
\end{definition}

\begin{theorem}[Петерсон, 1891]
    У регулярного графа степени $2k$ есть 2-фактор.
\end{theorem}

\begin{proof}
    Так как все степени четные, есть эйлеров цикл. Обойдем его в некотором направлении и ориентируем каждое ребро в направлении обхода. Теперь в каждую вершину $\overline{G}$ входит и выходит ровно по $k$ стрелок.

	Построим граф $G^*$ следующим образом: разделим каждую вершину $v \in V(G)$ на две вершины $v_1$ и $v_2$, если ребро $xy \in E(G)$ было ориентировано от $x$ к $y$, то проведем в графе $G^*$ ребро $x_1y_2$.

	Таким образом, существует биекция $\varphi\colon E(G) \to E(G^*)$, заданная правилом $\varphi(xy) = x_1y_2$.

	$G^*$ --- регулярный двудольный граф степени $k$ с долями $\{v_1\}_{v \in V(G)}$ и $\{v_2\}_{v \in  V(G)}$.

	По следствию \ref{cor:matching_2} в графе $G^*$ есть совершенное паросочетание $M^*$.

	Пусть $M = \varphi^{-1}(M^*)$. Для любой вершины $x \in V(G)$ каждая из вершин $x_1, x_2 \in  V(G^*)$ инцидентна ровно одному ребру из $M^*$.

	Поэтому $x$ инцидентна ровно двум ребрам из $M$, то есть $M$ --- 2-фактор графа $G$.
\end{proof}

\begin{corollary}
    \begin{enumerate}
		\item Регулярный граф степени $2k$ есть объединение $k$ своих $2$-факторов.
		\item Для любого $r \le k$ регулярный граф степени $2k$ имеет $2r$-фактор.
    \end{enumerate}
\end{corollary}
 
\section{Теорема Томассена о почти регулярном факторе почти регулярного графа.}
\begin{theorem}[Томассен, 1981]
    Пусть $G$ --- граф, степени всех вершин которого равны или $k$ или $k+1$, а $r \le k$. Тогда существует остовный подграф $F$ графа $G$, степени всех вершин которого равны либо $r$, либо $r+1$.
\end{theorem}
\begin{proof}
    Индукция по $r$. 
	\begin{description}
		\item[База:] $r = k$, очевидно, подойдет $H = G$.
		\item[Переход:] от $k$ к $k-1$. Пусть граф имеет остовный подграф $F$, степени вершин которого равны $r$ или $r+1$.

			Будем удалять из графа $F$ по очереди ребра, соединяющие вершины степени $r+1$. В какой-то момент мы получим граф  $F'$, степени вершин которого равны $r$ или $r+1$, при этом любые две вершины степени $r+1$ несмежны.
			Пусть $V_{r+1}$ --- множество всех вершин степени  $r+1$ в $F'$. Если $V_{r+1} = \varnothing$, то $F'$ уже подходит.

			Пусть $V' = V(G) \setminus V_{r+1}$, $B$ --- двудольный граф с долями $V_{r+1}$ и $V'$, ребра которого --- $E_{F'}(V_{r+1}, V')$.

			Для каждой вершины $x \in  V_{r+1}$ мы имеем $d_B(x)= r+1$, для каждой $y \in V'$ имеем $d_B(y) \le y$.

			По следствию из теоремы Холла в графе $B$ есть паросочетание $M$, покрывающее все вершины из $V_{r+1}$.

			Тогда удалим его и все вершины степени $r+1$ потеряют по одному ребру, а степени $r$ не более одного. Итого получится граф $H = F' -M$, где степени равны $r$ или $r-1$.
	\end{description}
\end{proof}
 
\section{Теорема Ловаса о разбиении графа.}
\begin{corollary}[Ловас, 1970]
    Пусть $s, t \in \N$. Тогда любой граф максимальной степени $s+t-1$ представляется в виде объединения  графа максимальной степени не более $s$ и графа максимальной степени не более $t$.
\end{corollary}
\begin{proof}
    Пусть $G$ --- граф с $\Delta(G) = s+t-1$. Добавим в граф вершины и ребра, чтобы он стал регулярным степени $k=s+t-1$.

	По теореме Томассена граф  $H$ имеет остовный подграф $H_1$, степени вершин которого равны $t$ или $t-1$.

	Тогда оставшиеся ребра графа $H$ образуют подграф $H_2$, степени вершин которого равны $s-1$ или $s$.

	Теперь удалим из подграфов $H_1$ и $H_2$ добавленные вершины и ребра, получим подграфы $G_1$ и $G_2$ графа $G$ с $\Delta(G_1) \le t$ и $\Delta(G_2) \le s$. При этом $G = G_1 \cup G_2$.
\end{proof}
 
\chapter{Связность}
\section{Блоки и точки сочленения. Лемма о пересечении блоков.}
Здесь граф $G$ связен.
\begin{definition}
	Вершина $a \in V(G)$ называется \selectedFont{точкой сочленения}, если граф $G-a$ несвязен.
\end{definition}
\begin{definition}
    \selectedFont{Блок} --- максимальный по включению подграф графа $G$.
\end{definition}
\begin{definition}
    Блоки и точки сочленения несвязного графа --- блоки и точки сочленения его компонент.
\end{definition}

\begin{lemma}\label{lm:connectivity_1}
    Пусть $B_1$ и $B_2$ --- два разных блока графа $G$, причем $V(B_1) \cap v(B_2) \neq \varnothing$. Тогда $V(B_1) \cap V(B_2)$ состоит из одной точки сочленения $a$ графа $G$, причем $a$ --- единственная  точка сочленения, отделяющая $B_1$ от $B_2$.
\end{lemma}
\begin{proof}
    \begin{description}
		\item[Единственность] Пусть $\lvert V(B_1) \cap V(B_2) \rvert \ge 2$. Тогда для любой вершины $x \in  V(B_1 \cup B_2)$ граф $B_1 \cup B_2 - x$ связен, так как $B_1 - x$ связен, $B_2-x$ связен, плюс остается хотя бы одна общая вершина. Следовательно, $B_1 \cup B_2$ содержится в блоке $B$ графа $G$, но тогда $B_1$ и $B_2$ не максимальные по включению.

			Пусть $V(B_1) \cap V(B_2) = \{a\}$. Так как $a $ --- общая вершина блоков $B_1$ и $B_2$, отделять $B_{1}$ от $B_2$ в графе $G$ может только $a$.
		\item[Точка сочленения] Если $a$ не отделяет $B_1$ от $B_2$, в графе $G$ должен быть $V(B_1)V(B_2)$-путь $P$.
			
			Пусть $H = B_1 \cup B_2 \cup P$. Граф $H-x$ связен для любой вершины $x \in V(H)$. Поэтому $H$ содержится в одном блоке графа $G$. Но блок $B_1$  --- его собственный подграф. Противоречие. 

		В итоге $a$ --- единственная вершина, отделяющая $B_1$ от $B_2$, следовательно, граф $G - a$ несвязен, поэтому $a$ --- точка сочленения.
    \end{description}
\end{proof}
 
\section{Дерево блоков и точек сочленения и его свойства.}
\begin{definition}
    Построим двудольный граф $B(G)$, вершины которого --- точки сочленения $a_1, \ldots , a_n$ графа $G$, а вершины другой доли --- его блоки $B_1, \ldots, B_m$. Вершины $a_i$ и $B_j$ будут смежны, если $a_i \in V(B_j)$.

	Такой граф называется \selectedFont{деревом блоков и точек сочленения}.
\end{definition}

\begin{lemma}\label{lm:connectivity_2}
	Пусть $B_1$ и $B_2$ --- два разных блока графа $G$, а $P$ --- путь между ними в графе $B(G)$. Тогда точки сочленения графа $G$, отделяющие $B_1$ от $B_2$ --- это в точности те точки сочленения, что лежат на пути $P$. Остальные не разделяют даже объединение блоков пути $P$. 
\end{lemma}
\begin{itemize}
	\item
    Пусть $x$ ---  точка сочленения графа $G$, не лежащая на пути $P$, $H$ --- объединение всех блоков на пути $P$.

	Для любого блока $B$ на пути $P$ граф $B-x$ связен. Если $B$ --- не $B_1$ и не $B_2$, то в нем можно пройти между двумя точками сочленения, входящими в $P$, так как $x$ не входит в $P$. Поэтому $H-x$ --- связный граф.
\item

	Пусть $a$ ---  точка сочленения, лежащая на  $P$, и она входит в блоки $B_1'$ и $B_2'$ на пути $P$.

	Обозначим через $H_1$ объединение всех блоков на пути $P$ до $a$, через $H_2$ --- объединение всех блоков после $a$.

\begin{figure}[ht]
    \centering
    \incfig{block-tree}
    \caption{}
    \label{fig:block-tree}
\end{figure}

	Применим рассуждения первого пункта отдельно к $H_1$ и к $H_2$. Получаем, что $a$ не разделяет ни одного из них.

	С другой стороны, по лемме \ref{lm:connectivity_1} точка сочленения $a$ отделяет блок $B_1'$ от $B_2'$, а значит, $a$ отделяет $H_1$ от $H_2$, следовательно и $B_1$ от $B_2$.
\end{itemize}

\begin{theorem}
    \begin{enumerate}
		\item  Дерево блоков и точек сочленения --- это дерево, все листья которого соответствуют блокам.
		\item Точка сочленения $a$ разделяет два блока $B_1$ и $B_2$, согда $a$ разделяет $B_1$ и $B_2$ в $B(G)$.
    \end{enumerate}
\end{theorem}
\begin{proof}
    \begin{enumerate}
		\item Докажем первый пункт.
			\begin{description}
				\item[Связность.] Для любых двух вершин $B(G)$ рассмотрим путь $Q$ в $G$ между ними.

					Перестроим его в путь в $B(G)$ : участок пути $Q$, проходящий по одному блоку графа $G$, заменим на соответствующую блоку вершину в $B(G)$, переход $Q$ между различными блоками по лемме \ref{lm:connectivity_1} осуществляется через их общую точку сочленения --- вершину $B(G)$.
				\item[Дерево.] Пусть в $B(G)$ есть простой цикл $Z$. Рассмотрим подграф $H$ --- объединение всех блоков этого цикла. 

					Между любыми двумя входящими в  $ Z$ блоками есть два независимых пути в $B(G)$.

					По лемме \ref{lm:connectivity_2} граф $H$ не имеет точек сочленения, иначе они должны лежать на одновременно на двух путях по циклу.

					Следовательно, существует блок $B$, содержащий $H$, блоки цикла $Z$ --- собственные подграфы $B$, что невозможно.
				\item [Листья.] Если лист соответствует точкe сочленения $a$, то по лемме \ref{lm:connectivity_2} граф $G-a$ связен. Противоречие. 
			\end{description}
		\item Докажем второй пункт: в дереве $B(G)$ есть единственный путь между блоками $B_1$ и $B_2$, по лемме \ref{lm:connectivity_2} в точности точки сочленения с этого пути отделяют $B_1$ от $B_2$ в исходном графе $G$.
    \end{enumerate}
\end{proof}
 
\section{Крайние блоки}

\begin{definition}
    Назовем блок $B$ \selectedFont{крайним}, если он соответствует листу дерева блоков и точек сочленения.
\end{definition}
\begin{definition}
    \selectedFont{Внутренность} $\Int(B)$ блока $B$ --- множество всех его вершин, не являющихся точками сочленения в графе.
\end{definition}
\begin{itemize}
	\item Блок недвусвязного графа крайний, согда он содержит ровно одну точку сочленения.
	\item Внутренность некрайнего блока может быть пустой, а крайнего всегда непуста.
	\item Если у связного графа есть точки сочленения, то он имеет хотя бы два крайних блока.
	\item Если $B$ ---  блок графа $G$, $x \in \Int(B)$, то граф $G-x$ связен.
\end{itemize}
\begin{lemma}\label{lm:connectivity_3}
    Пусть $B$ --- крайний блок связного графа $G$ c $v(G) \ge 2$, $G' = G - \Int(B)$. Тогда граф $G'$ связен, а блоки $G'$ --- все блоки $G$, кроме $B$.
\end{lemma}
\begin{proof}
    Пусть $a \in V(B)$ --- точка сочленения, отрезающая крайний блок $B$ от остального графа $G$. Тогда $\Int(B)$ --- это одна из компонент связности графа  $G-a$, следовательно, сам граф $G'$ будет связен.

	Отличные от $B$ блоки графа $G$ --- подграфы $G'$, не имеют точек сочленения и являются максимальными подграфами $G'$ с таким свойством, так как были максимальными в $G$. Следовательно, они все --- блоки $G'$.

	Пусть $B'$ --- блок $G'$. Очевидно, что $v(G') \ge 2$, поэтому $B'$ содержит хотя бы одно ребро $e$, которое в графе $G$ лежит в некотором блоке $B^* \neq B$, так как блок максимальный по включению, $B^* = B'$.
\end{proof}
 
\section{Алгоритм разбиения графа на блоки.}
Пусть $U_1, \ldots , U_k$ --- все  компоненты связности графа $G-a$, $G_i = G(U_i \cup \{a\})$. Разрежем граф  $G$ на графы $G_1, \ldots G_k$.

\begin{lemma}\label{lm:connectivity_4}
	\begin{enumerate}
		\item Пусть $b \in U_i$. Тогда $b$ разделяет вершины $x, y \in  U_i$ d $G_i$, согда $b$ разделяет их в $G$.
		\item Все точки сочленения графов $G_1, \ldots , G_k$ --- в точности все точки сочленения графа $G$, кроме $a$.
	\end{enumerate}
\end{lemma}
\begin{proof}
    \begin{enumerate}
		\item Если $G-b$ не содержит $xy$-путь, то его нет и в $G_i - b$.
		
			Наоборот, пусть $x$ и $y$ лежат в разных компонентах связности графа $G_i -b$. НУО можно считать, что компонента связности $W \ni x$ не содержит $a$. Тогда $W$ --- компонента связности графа $G-b$, следовательно, в $G-b$ тоже не было $xy$-пути.
		\item Так как $G_i - a$ --- компонента графа $G-a$, вершина $a$ не является точкой сочленения ни в одном  из графов $G_1, \ldots , G_k$.

			Любая другая точка сочленения графа $G$ лежит ровно в одном из графов $G_1, \ldots , G_k $ и является в нем точкой сочленения по прошлому пункту.

			Так же из прошлого пункта следует, что других точек сочленения в графах $G_1, \ldots G_k $ нет.
    \end{enumerate}
\end{proof}
\subsection{Алгоритм разбиения связного графа на блоки}
\begin{itemize}
	\item Выберем точку сочленения $a$ и разрежем по ней $G$: заменим граф $G$ на полученные $G_1, \ldots , G_k$.
	\item Каждый следующий шаг берем один из имеющихся графов, выбирает точку сочленения и разрезаем по ней.
	\item И так далее, пока хотя бы один из полученных графов имеет точку сочленения.
\end{itemize}

\begin{theorem}
    В результате описанного алгоритма вне зависимости от порядка действий получатся блоки графа $G$.
\end{theorem}
\begin{proof}
	По лемме \ref{lm:connectivity_4} мы вне зависимости от порядка проведем разрезы только по всем точками сочленения и только по ним.

	Пусть $B$ --- блок графа $G$. Тогда в графе $G$ множество $V(B) $ не было разделено ни одной из точек сочленения. Тогда по первому пункту леммы \ref{lm:connectivity_4} множество $V(B)$ не было разрезано алгоритмом.

	Так как в результате алгоритма получились индуцированные подграфы графа $G$, один из них обозначим за $H$ --- надграф  $B$.

	Если $H \neq B$, то рассмотрим вершину $c \in V(H) \setminus V(B)$. 
	В графе $G$ существует точка сочленения $a$, отделяющая $c$ от $V(B)$. Тогда по лемме  \ref{lm:connectivity_4} при разрезе по $a$ вершина $c$ была отделена от блока $B$. Противоречие. 
    
\end{proof}
 
\section{Следствие о веере путей из теоремы Менгера. Теорема Дирака о цикле, содержащем заданные k вершин.}
\begin{definition}
    Пусть $X, Y \subset V(G)$, $R \subset V(G) \cup E(G)$.
	\begin{enumerate}
		\item Через $G - R$ обозначим граф, полученный из $G$ в результате удаления всех вершин и ребер из $R$, а также всех ребер инцидентных вершинам из $R$.
		\item Назовем множество $R$ \selectedFont{разделяющим}, если граф $G -R$ несвязен. Обозначим за $\RR(G)$ множество всех разделяющих множеств. 
	\end{enumerate}
\end{definition}
\begin{definition}
    Граф $G$ является \selectedFont{$k$-связным}, если  $v(G) \ge k+1$ и минимальное вершинное разделяющее множество в графе $G$ содержит хотя бы $k$ вершин.
\end{definition}
\begin{definition}
    \begin{enumerate}
		\item Пусть $x, y \in V(G)$ --- несмежные вершины. Обозначим за $\kappa_G(x, y)$ размер минимального множества $R \subset V(G)$ такого, что $R$ разделяет $x$ и $y$. Если $x$ и $y$ смежны, положим $\kappa_G(x, y) = + \infty$. Назовем $\kappa_G(x, y) $ \selectedFont{связностью} вершин $x$ и $y$.

			Пусть $X, Y \subset V(G)$. Обозначим через $\kappa_G(X, Y)$ размер минимального множества $R \subset V(G)$ такого, что $R$ разделяет $X$ и $Y$. Если такого множества нет, положим $\kappa_G(X, Y) = + \infty$.
    \end{enumerate}
\end{definition}
В $k$-связном графе $G$ для любых двух множество вершин $X, Y \subset V(G)$ выполнено $\kappa_G(X, Y) \ge k$.

\begin{theorem}[Менгер, 1927]
    Пусть $X, Y \subset V(G)$, $\kappa_G(X, Y) \ge k$, $\lvert X \rvert \ge k$, $\lvert Y \rvert \ge k$. Тогда в графе $G$ существуют $k$ непересекающихся $XY$-путей.
\end{theorem}

\begin{corollary}\label{cor:connectivity_2}
    Пусть $x \in  V(G)$, $Y \subset V(G)$, $x \notin Y$, $k = \min(\lvert Y \rvert, \kappa_G(x, Y))$. Тогда существуют $k$ путей от $x$ до различных вершин множества $Y$, не имеющих  общих внутренних вершин.
\end{corollary}
\begin{proof}
    Пусть $X = N_{G}(x)$. Так как $\kappa_G(x, Y) \ge k$, $\lvert N_{G}(x) \ge k \rvert$.

	Так как $x\notin Y$, любое множество вершин $R$, отделяющее $X$ от $Y$ отделяет вершину $x$ от множества $Y$. Следовательно, $\lvert R \rvert \ge k$.

	Так как $\lvert Y \rvert \ge k$ по теореме Менгера существует  $k$ непересекающихся путей от $x$ до различных вершин множества $Y$.
\end{proof}

\begin{theorem}[Уитни, 1932]
	Пусть $G$ --- $k$-связный граф. Тогда для любых двух вершин $x , y \in V(G)$ существует $k$ независимых $xy$-путей.
\end{theorem}
\begin{proof}
    Индукция по $k$. 
	\begin{description}
		\item[База:] $k=1$, очевидно
		\item[Переход:] пусть мы доказали для меньших $k$. Если вершины $x$ и $y$ несмежны, то утверждение следует из следствия \ref{cor:connectivity_2}.

			Разберем случай смежных $x$ и $y$.

			Если $G - xy$  --- это $(k-1)$-связный граф, то про индукционному предположению существует $k-1$ независимый $xy$-путь в графе $G-xy$ и еще один путь --- ребро $xy$.

			Теперь предположим, что в $G-xy$  существует разделяющее множество $T$, $\lvert T \rvert \le k-2$. Так как $T$ не разделяет $G$, в графе $G - (T \cup \{xy\})$ ровно две компоненты связности: $U_x \ni x$ и $U_y \ni y$.

			Пусть $T_x = T \cup \{x\}$. Если $U_x \neq \{x\}$, то $T_x$ отделяет $U_x \setminus \{x\}$ от $U_y$ в $G$, но это невозможно, так как $\lvert T_x \rvert \le  k-1$.

			Тогда $U_x = \{x\}$ и, аналогично, $U_y = \{y\}$. Получается, что в графе $G$ не более $k$ вершин: $T$, $x$ и $y$. Но такой граф не может быть $k$-связным.
	\end{description}
\end{proof}

\begin{theorem}[Дирак]
	Пусть $k \ge 2$. В $k$-связном графе для любых $k$ вершин существует простой цикл, содержащий все эти вершины.
\end{theorem}
\begin{proof}
    Индукция по $k$.
	\begin{description}
		\item[База:] $k=2$, следует из теоремы Уитни.
		\item[Переход:] $k-1 \to k$. Рассмотрим $k$-связный граф $G$ и его вершины $v_1, \ldots , v_{k}$. 
			Так как $G$ и  $k-1$-связный тоже, по индукционному предположению  есть простой цикл $Z$, содержащий вершины $v_1, \ldots , v_{k-1}$.
			
			Разберем два случая:
			\begin{enumerate}
				\item Пусть $v(Z) <k$. Тогда $V(Z) = \{v_1, \ldots , v_{k-1}\}$, по следствию \ref{cor:connectivity_2} существуют непересекающиеся пути от $v_k$ до всех вершин $Z$.

					Теперь можно вставить в $Z$ еще одну вершину $v_k$ между двумя соседними.
				\item $v(Z) > k$. По следствию \ref{cor:connectivity_2} существует  $k$ непересекающихся путей от $v_k$ до $Z$. 

					Обозначим концы этих путей $x_1, \ldots , x_k \in  V(Z)$. Они делят круг на $k$-дуг и внутренность еще одной из этих дуг. Поэтому хотя бы одна не содержит ни одной из вершин $v_1, \ldots , v_{k-1}$. Ее мы можем заменить на путь от начала до $v_k$  и от $v_k$ до конца, тем самым получив искомый цикл.
			\end{enumerate}
	\end{description}
\end{proof}
 
\section{Разделяющие множества в k-связном графе, части разбиения. Внутренность и граница части разбиения.}
Пусть $\S \subset \RR(G)$.
\begin{definition}
    Множество $A \subset V(G)$ --- \selectedFont{часть $\S$-разбиения}, если никакие две вершины из $A$ нельзя разделить никаким множеством из $\S$, но любая другая вершина графа $G$ отделена от множества $A$ хотя бы одним из множеств набора $\S$.

	Множество всех частей разбиения графа $G$ набором разделяющих множеств  $\S$ мы будем обозначать через $\Part(\S)$. Если граф не очевиден  $\Part(G; \S)$.

	Вершину части $A \in \Part(\S)$ назовем \selectedFont{внутренней}, если она не входит ни в одно из множеств набора $\S$. Множество таких вершин --- $\Int(A)$ --- \selectedFont{внутренность} части $A$.

	Вершины, входящие в какие-либо множества из $\S$, будем называть \selectedFont{ граничными}, а все их множество \selectedFont{границей} и обозначать через $\Bound(A)$.
\end{definition}

Внутренняя вершина части $A \in \Part(\S)$ может быть концом ребра , входящего в множество $S \in \S$.

Пусть $A, B \in \Part(\S)$, $A \neq B$, $A \cap B \neq \varnothing$. Тогда существует такое $S \in \S$, что $A \cap B \subset S$.

Разделяющее множество $S \subset V(G)$ в  $k$-связном графе $G$ должно содержать на менее $k$ вершин. Мы обозначим через $\RR_{k}(G)$ множество всех $k$-вершинных разделяющих множеств графа $G$.

Пусть $S \in \RR_{k}(G)$, $A \in \Part(\S)$. Тогда $\Int(A) \neq \varnothing$, $G(\Int(A))$ связен ---  это компонента графа $G - S$. Для любой вершины $x \in S$ существует вершина $y \in \Int(A)$, смежная с $x $ (иначе $S \setminus \{x\}$ отделяет $\Int(A)$ от $G - A$ ).

Однако, если $\S \subset\RR_K(G)$, $B \in \Part(\S)$, то возможно, что $\Int(B) = \varnothing$. Кроме того, при $\Int(B) \neq \varnothing$ индуцированный подграф $G(\Int(B))$ не обязательно связен.

\begin{lemma}\label{lm:connectivity_5}
    Пусть $\S \subset \RR_k(G)$, $A \in \Part(\S)$. Тогда верно:
	\begin{enumerate}
	    \item  Вершина $x \in \Int(A)$ не смежна ни с одной другой из вершин множества $V(G)\setminus A$.
		\item Если $\Int(A) \neq \varnothing$, то $\Bound(A)$ отделяет $\Int(A)$ от $V(G) \setminus A$.
	\end{enumerate}
\end{lemma}
\begin{proof}
    \begin{enumerate}
		\item Пусть вершина $x \in  \Int(A)$ смежна с вершиной $y \in V(G) \setminus A$. Существует множество $S \in \S$, отделяющее $ y$ от $\Int(A)$ в $G$. Тогда $x, y \notin S$, причем они смежны. Противоречие. 
		\item Следует из прошлого пункта.
    \end{enumerate}
\end{proof}

\begin{theorem}
    Пусть $G$ --- $k$-связный граф, $\S, \T \subset\RR_k(G)$.
	\begin{enumerate}
		\item Пусть $A \in  \Part(\S)$. Тогда $\Bound(A)$ --- множество всех вершин части $A$, смежных хотя бы с одной из $V(G) \setminus A$.
		\item Пусть $A \in \Part(\S)$ и $A \in \Part(\T)$. Тогда граница $A$ как части $\Part(\S)$ совпадает с границей $A$ как части $\Part(\T)$.
	\end{enumerate}
\end{theorem}
\begin{proof}
    \begin{enumerate}
		\item Пусть  $x \in \Bound(A)$. Существует такое множество $S \in \S$, что $x \in S$.

			Множество вершин $S$ не разделяет $A$, следовательно $A$ может пересекать внутренность не более чем одной части $\Part(S)$. Тогда существует такая часть $B \in \Part(S)$, что $\Int(B) \cap A = \varnothing$. Тогда существует вершина $y \in \Int(B)$, смежная с $x$.  

			По следствию \ref{cor:connectivity_1} ни одна из вершин множества $\Int(A)$ не может быть смежна с вершиной из $V(G) \setminus A$.
		\item  В первом пункте мы построили $\Bound(A)$ вне зависимости от $\S$ или $\T$, поэтому совпадать с границей обоих будет совпадать.
    \end{enumerate}
\end{proof}
 

\chapter{Раскраски}
\section{Лемма о галочке}
\begin{definition}
	\selectedFont{Раскраска вершин} графа $G$ в  $k$ цветов --- функция $\rho\colon V(G) \to M$, где $\lvert M \rvert = k$. Раскраска $\rho$ называется \selectedFont{правильной}, если $\rho(v) \neq \rho(u)$ для любой пары смежных вершин   $u$ и $v$ .

	Через $\chi(G)$ обозначим \selectedFont{хроматическое число} графа $G$ --- наименьшее натуральное число, для которого существует правильная раскраска вершин графа $G$ в такое количество цветов.

	\selectedFont{Раскраска ребер} графа $G$ в $k$ цветов --- функция $\rho\colon E(G) \to M$, где $\lvert M \rvert = k$. Раскраска называется \selectedFont{правильной}, если $\rho(v) \neq \rho(u)$ для любой пары смежных ребер $u$ и $v$.

	Через $\chi'(G)$ обозначим \selectedFont{хроматический индекс} графа $G$ --- наименьшее натуральное число, для которого существует правильная раскраска ребер графа $G$ в такое количество цветов.
\end{definition}

\begin{lemma}\label{lm:coloring_1}
	Пусть $G$ --- связный граф, $\Delta(G) \le d$, причем хотя бы одна из верши графа имеет степень менее $d$. Тогда $\chi(G) \le d$.
\end{lemma}
\begin{proof}
    Индукция по количеству вершин.
	\begin{description}
		\item [База:] Если в графе не более $d$ вершин, его точно можно покрасить в $d$ цветов.
		\item [Переход:] Пусть мы уже доказали утверждение для любого меньшего связного графа с меньшим числом вершин.

			Пусть $u \in V(G)$ --- вершина степени менее $d$. Рассмотрим граф $G - u$. Пусть $G_1, \ldots , G_k  $ --- компоненты $G-u$.

			В каждом из графов $G_i$ есть вершина $u_i$, смежная с $u$ в графе $G$. Тогда у $u_i$ в $G_i$ степень не более $d-1$, и $\Delta(G_i) \le d$.

			По индукционному предположению, каждый $G_i$ можно покрасить в $d$ цветов. Далее докрашиваем $u$, мы можем это сделать, так как у нее только $d-1$ ребро.
	\end{description}
\end{proof}

\begin{lemma}\label{lm:coloring_2}
    Если $G$ --- двусвязный неполный граф с $\delta(G) \ge 3$, существуют такие вершины $a, b, c \in V(G)$, что $ab, bc \in E(G)$, $ac \notin E(G)$ и граф $G-a-c$ связен.
\end{lemma}

\begin{proof}
	\begin{itemize}
		\item Пусть $G$ трехсвязен.

			Так как $G$ неполный, существуют такие вершины $a, b, c \in V(G)$, что $ab, bc \in E(G)$ и $ac \notin E(G)$. Граф $G-a-c$ точно связен. 
		\item Путь $G$ не трехсвязен, тогда существует вершина $b \in V(G)$, что граф $G' = G - b$ не двусвязен.

			Граф $G'$ имеет хотя бы два крайних блока. Так как исходный $G$ двусвязен, вершина  $b$ должна быть смежна хотя бы с одной внутренней вершиной каждого крайнего блока $G'$. Пусть $a$ и $c$ --- смежные с $b$ внутренние вершины двух разных крайних блоков $B_a$ и $B_c$ графа $G'$.
		\begin{figure}[ht]
			\centering
			\incfig{checkmark-lemma}
			\caption{}
			\label{fig:checkmark-lemma}
		\end{figure}
		
		Тогда графы $B_a - a$ и $B_c - c$ связны, поэтому и $G'- a -c $ связен.

		Так как $d_G(b) \ge 3$, вершина $b$ смежна с $G'-a-c$, а значит, и граф $G -a-c$ связен.
	\end{itemize}
\end{proof}

\section{Теорема Брукса}
\begin{theorem}[Брукс, 1941]
	Пусть $d \ge 3$, $G$ --- связный граф, отличный от $K_{d+1}$, $\Delta(G) \le d$. Тогда $\chi(G) \le d$.
\end{theorem}
\begin{proof}
	Достаточно рассмотреть случай регулярного графа степени $d$, иначе можно воспользоваться леммой \ref{lm:coloring_1}.
	Рассмотрим два случая.
	\begin{itemize}
		\item Пусть в графе $G$ есть точка сочленения $a$. Тогда $G = G_1 \cup G_2$, где $V(G_1) \cap V(G_2) = \{a\}$, а сами $G_1$ и $G_2$ связны.

			Так как $a$ смежна хотя бы с одной вершиной и из $G_1$ и из $G_2$, то $d_{G_1}(a) < d$ b $d_{G_2}(a) <d$. 
			По лемме \ref{lm:coloring_1} можем покрасить $G_1$ и $G_2$ в $d$ цветов. 

			Согласуем раскраски, чтобы цвет вершины $a$ был одинаковый, и получим правильную раскраску всего $G$.
		\item Теперь пусть $G$ двусвязен.
			По лемме \ref{lm:coloring_2} существуют такие $a, b, c \in V(G)$, что $ab, bc \in E(G)$, $ac \notin E(G)$ и граф $G-a-c = G'$ связен.

			Рассмотрим такой $G'$ и его остовное дерево $T$.

			Подвесим дерево за $b$. Пронумеруем уровни так, чтобы номер совпадал с расстоянием от корня. 

			Пусть $\rho(a) = \rho(c) = 1$. Будем красить остальные вершины дерева в порядке убывания номеров их уровней, начиная с листьев.

			Пусть $x \neq b$ --- очередная вершина, причем на момент ее рассмотрения мы покрасили все вершины больших уровней, но не красили вершины меньших. Тогда ее предок еще не имеет цвета, поэтому соседи покрашены максимум в $d-1$ цвет. Следовательно, хотя бы один свободный останется.

			Посмотрим на момент, когда осталась только вершина $b$. У нее два соседа $a$ и $c$ имеют один цвет, поэтому опять есть свободный цвет.

			Так мы покрасили все вершины графа $G$ в $d$ цветов.
\end{itemize}
\end{proof}

\section{Списочное хроматическое число $k$-редуцируемого графа.}

\begin{definition}[Списочные раскраски]
	Каждой вершине $v \in V(G)$ сопоставляется \selectedFont{ список} $L(v)$, после чего раскраска считается правильной, если цвет каждой вершины входит в ее список.

	Минимальное такое  $k \in \N$, что для любых списков из $k$ цветов существует правильная раскраска вершин графа $G$, обозначается через $\ch(G)$  и называется \selectedFont{ списочное хроматическое число}.

	Аналогично определяются \selectedFont{списочная раскраска ребер} и \selectedFont{списочный хроматический индекс}.
\end{definition}
\begin{note}
	Известны графы, где $\ch(G) > \chi(G)$, но не известны такие, где $\ch'(G) > \chi'(G)$.
\end{note}
 \begin{definition}
    Пусть $k \in \N$. Граф называется \selectedFont{$k$-редуцируемым}, если его вершины можно занумеровать $v_1, \ldots , v_n $ так, что каждая вершины смежна менее чем с $k$ вершинами с б\'oльшим номером.
\end{definition}
\begin{lemma}\label{lm:coloring_3}
	Пусть $G$ --- $k$-редуцируемый граф. Тогда $\chi(G) \le \ch(G) \le k$.
\end{lemma}
\begin{proof}
    Пусть $v_1, \ldots , v_n $ --- нумерация вершин графа из определения, причем каждой вершине $v_i$ соответствует список $L(v_i)$ длины $l(v_1) \ge k$.

	Покрасим вершины в порядке, обратном нумерации. При покраске вершины $v_i$ количество запретов на цвет не превосходит количество соседей среди вершин с б\'oльшим номером, а таких не более $k-1$. Значит, мы можем покрасить вершину $v_i$ в цвет из ее списка.
\end{proof}
\begin{lemma}\label{lm:coloring_4}
    Граф $G$ является редуцируемым, согда для любого его подграфа $H$ выполняется $\delta(H) \le k-1$.
\end{lemma}
\begin{proof}
    \begin{description}
		\item[$ \implies$ ] Пронумеруем вершины графа $G$ как в определении.
			Пусть какой-то подграф  $ H$ имеет $\delta(H) \ge k$.

			Рассмотрим вершину с наименьшим номером $v_i \in V(H)$. Она смежна не менее чем с $d_H(v_i) \ge \delta(H) \ge k$ вершинами с б\'oльшими номерами. Противоречие. 
		\item [$ \impliedby$ ] Пусть $v_1$ --- вершина графа  $G$ наименьшей степени.
			Она смежна не более чем с $d_G(v_1)=\delta(G) \le k-1$ вершиной.

			Пусть вершины $v_1, \ldots , v_{i-1} $ уже построены. 

			Рассмотрим граф $G_i = G - \{v_1, \ldots , v_{i-1}\}$. В нем должна быть вершина степени не более $\delta(G_i) \le  k-1$, которую мы и возьмем в качестве $v_i$.
    \end{description}
\end{proof}

\section{Две леммы о d-раскрасках (о избыточной вершине и о удалении вершины с сохранением связности).}

\begin{definition}
    Граф $G$ называется \selectedFont{$d$-раскрашиваемым}, если для любого набора списков $L$, удовлетворяющего условию $l(v) \ge d_G(v)$ для каждой вершины $v \in V(G)$, существует правильная раскраска вершин в цвета из списков.

	Список цветов, удовлетворяющий указанному  условию, будем называть \selectedFont{$d$-списком}.
\end{definition}

\begin{definition}
    Назовем вершину $v \in V(G)$ \selectedFont{нормальной}, если $l(v) \ge d_G(v)$, \selectedFont{избыточной}, если $l(v) > d_{G}(v)$.
\end{definition}

\begin{lemma}\label{lm:coloring_5}
	Пусть $ G$ --- связный граф, $L$ --- $d$-список, в котором вершина $a$ избыточная. Тогда существует правильная раскраска вершин графа $G$ в соответствии со списком $L$. 
\end{lemma}
\begin{proof}
    Индукция по количеству вершин.
	\begin{description}
		\item [База:] граф с одной избыточной вершиной, очевидно.
		\item [Переход:]  пусть мы уже доказали утверждение для графов с меньшим числом вершин.

			Рассмотрим граф $G-a$. Пусть $G_1, \ldots , G_k$ --- все компоненты графа $G-a$. В каждом графе $G_i$ должна быть вершина $a_i$, смежная с $a$.

			Рассмотрим отдельно граф $G_i$ с исходными списками вершин. Тогда $a_i$ станет избыточной, так как $d_{G_i}(a) \le d_G(a_i) - 1 \le l(a_i) - 1$.

			По индукционному предположению вершины всех $G_i$ можно покрасить в соответствии со списками. Так как $a$ избыточная, мы можем раскрасить ее в какой-то цвет из списка $L(a)$, не нарушив правильности раскраски. 
	\end{description}
\end{proof}

\begin{lemma}\label{lm:coloring_6}
	Пусть $G$ --- связный граф, $L$ --- $d$-список. Предположим, что существуют две смежные вершины $a$ и $b$ такие, что граф $G-a$ связен и $L(A) \not\subset L(b)$. Тогда существует правильная раскраска вершин графа $G$ в соответствии со списком $L$.
\end{lemma}

\begin{proof}
    Пусть $1 \in L(A) \setminus L(B)$.

	В связном графе $G-a$ из всех списков вершин множества $N_{G}(A)$, содержащих цвет $1$, удалим этот цвет, остальные оставим без изменений. Получим новые списки $L'(v)$ графа $G-a$.

	Все вершины графа $G-a$ нормальны: вершины не из $N_G(a)$ не изменились, а для $v \in N_G(a)$ имеем
	\[
	l'(v) \ge l(v) - 1 \ge d_G(v) - 1 = d_{G-a}(v)
	.\] 

	Так как $1 \notin L(b)$, $l'(b) = l(b)$. Так как $d_{G-a)(b) = d_G(b)-1}$, вершина $b$ избыточная.

	По лемме \ref{lm:coloring_5} существует правильная раскраска вершин графа $G-a$ в цвета из $L'$. Далее докрашиваем $a$ в цвет $1$, получаем правильную раскраску вершин графа $G$ в цвета списка $L$.
\end{proof}


\section{Теорема Бородина о d-раскрасках}
\begin{definition}
    Граф, в котором каждый блок --- нечетный цикл или полный граф называется \selectedFont{лесом Галлаи}.
\end{definition}

\begin{theorem}[Бородин, 1977]
    Если $G$ не является лесом Галлаи, то $G$ $d$-раскрашиваем.
\end{theorem}
\begin{proof}
	Пусть каждой вершине $v$ соответствует список $L(v)$. НУО $l(v) = d_G(v)$ для каждой вершины $v \in V(G)$.  Считаем граф связными. 

	Индукция по размеру графа.
	\begin{description}
		\item[База:] $G$ двусвязен. 

			Если не все списки одинаковые, то существуют две смежные вершины $a$ и $b$ c $L(a) \neq L(b)$ и $G$ раскрашиваем по лемме \ref{lm:coloring_6}.

			Значит, все списки одинаковы, состоят из  $d$ цветов. Тогда и все степени вершин равны $d$.

			По условию граф отличается  от полного графа и нечетного цикла, поэтому, по теореме Брукса раскраска существует.
		\item [Переход:] $G$ недвусвязен.
			Пусть для меньшего чем $G$ графа теорема доказана. 

			Рассмотрим крайний блок $B$ графа $G$, отделяемый от остального графа точкой сочленения $a$. 

			Граф $B-a$ связен, все вершины нормальны по условию, а все смежные с $ a$ вершины избыточны, причем такие должны быть, иначе это не точка сочленения.
			По лемме \ref{lm:coloring_3} его вершины можно покрасить согласно спискам.

			Пусть $G' = G - \Int(B)$. Граф $G'$ имеет те же блоки, что и $G$, кроме $B$. Поэтому среди этих блоков должен быть еще один блок, отличный от нечетного цикла и полного графа.

			Списки отличных от $a$ вершин не менялись, степени --- тоже.

			Составим новый список $L'(a)$ из всех цветов $L(a)$,  кроме использованных для раскраски $\N_B(a)$. Таких цветов не более $d_B(a)$. Так как $d_G(a) = d_B(a) + d_{G'}(A)$, получаем $l'(a) \ge d_{G'}(a)$.

			По индукционному предположению существуют правильная раскраску вершин $G'$ в цвета из списка. Далее дополняем ее раскраской $B-a$ и получаем искомую раскраску вершин $G$.
	\end{description}
\end{proof}

\section{Списочная теорема Брукса.}
\begin{theorem}[Визинг, 1976]
    Пусть $d \ge 3$, $G$ --- связный граф, отличный от $K_{d+1}$, $\Delta(G) \le d$. Тогда $\ch(G) \le d$.
\end{theorem}
\begin{proof}
    Пусть каждой вершине $v \in V(G)$ соответствует список $L(v)$, причем $l(v) \ge d$.

	\begin{itemize}
		\item Если $G$ --- не лес Галлаи, по теореме Бородина он раскрашивается.
		\item Пусть $G$ --- лес Галлаи. По условию $G$ не двусвязен, поэтоому его блоки точно отличны от $K_{d+1}$.

			Посмотрим на крайний блок $B$ и его вершину $b$, не являющуюся точкой сочленения. Так как этот блок не является полным подграфом:
			\[
			l(b) \ge d > d_{B}(b) = d_{G}(b)
			.\] 
			Значит, вершина $b$ избыточна, по лемме \ref{lm:coloring_5} существует искомая раскраска.
	\end{itemize}

\end{proof}

\section{k-критические графы. Простейшие свойства.}
\begin{definition}
    Назовем граф \selectedFont{$k$-критическим}, если $\chi(G) = k$, но $\chi(H) < k$ для любого собственного подграфа $H$ графа $G$.
\end{definition}
\begin{lemma}\label{lm:coloring_7}
    Если $G$ ---  $k$-критический граф, то $\delta(G) \ge k-1$.
\end{lemma}
\begin{proof}
    Пусть $a \in V(G)$, $d_G(a) \le k-2$. По определению $\chi(G-a) \le k-1$.

	Покрасим граф $G-a$ в $k-1$ цвет, так как степень $a$ в исходном графе меньше $k-1$, мы сможем докрасить ее и получить раскраску в $k-1$ цвет. Противоречие. 
\end{proof}

\begin{lemma}\label{lm:coloring_8}
    Пусть $G$ --- $k$-критический граф, $S \subset V(G)$ --- разделяющее множество $\lvert S \rvert < k$. Тогда $G(S)$ --- не полный.
\end{lemma}
\begin{proof}
    Пусть $G(S)$ --- полный, $S = \{a_1, \ldots , a_m\}$, $\Part(S) = \{F_1, \ldots , F_n\}$, $G_i = G(F_i)$.

	Так как $G_i$ --- собственный подграф $G$, то $\chi(G_i) \le k-1$, пусть $\rho_i$ --- правильная раскраска $G_i$ в $k-1$ цвет.

	Так как вершины $S$ попарно смежны в $G_i$, то все цвета  $\rho(a_1), \rho(a_2), \ldots , \rho(a_m)$ различны. Теперь согласуем раскраски в $G_1, \ldots , G_n$ и получим общую раскраску в $k-1$ цвет для вершин графа $G$. Противоречие. 
\end{proof}

\section{Теорема Галлаи о k-критических графах.}
\begin{theorem}[Галлаи, 1963]
    Пусть $k \ge 3$, $G$ --- $k$-критический граф. Пусть $V_{k-1}$ --- множество всех вершин графа $G$, имеющих степень $k-1$, а $G_{k-1} = G(V_{k-1})$. Тогда $G_{k-1}$ --- лес Галлаи.
\end{theorem}
\begin{proof}
	По лемме \ref{lm:coloring_7} $\delta(G) \ge k-1$. Будем считать, что $G_{k-1} \neq \varnothing$, иначе доказывать нечего.

	Предположим, что $G_{k-1}$ не лес Галлаи. Тогда этот граф имеет компоненту $G'$, у которой есть блок, отличный от полного графа и нечетного цикла. Пусть $V(G') = V'$.

	Для собственного подграфа $H = G - V'$ графа $G$ мы имеем $\chi(H) \le k-1$, так как $G$ --- $k$-критический.

	Пусть $\rho$ --- раскраска графа $H$ в  $k-1$ цвет. Рассмотрим любую вершину $x \in  V'$, пусть она имеет $n_x$ соседей  в $V(H)$. Поместим в список $L(x)$ в точности те цвета из $[1..k-1]$, что не встречаются среди $n_x$. Тогда длина списка $l(x) \ge k-1-n_x = d_{G'}(x)$.

	По теореме Бородина граф $G'$ является $d$-раскрашиваемым, следовательно, существует правильная раскраска $\rho^*$  графа $G'$ в цвета из построенных списков.

	Вместе $\rho$ и $\rho^*$ дают правильную раскраску вершин $G$ в $k-1$ цвет. Противоречие. Значит $G_{k-1}$ --- лес Галлаи.
\end{proof}

\section{Лемма Дирака о разделяющем двухвершинном множестве в критическом графе.}
\begin{definition}
    Пусть $x, y \in V(G)$ --- две несмежные вершины. Определим операцию \selectedFont{слияния вершин} $x$ и $y$ графа $G$ следующим образом: эти вершины объединяются в одну новую $x\#y$, которая будет смежна со всеми вершинами, смежными в графе $G$ хотя бы с одной из вершин $x$ и $y$. Полученный граф обозначим через $G\#xy$.
\end{definition}
\begin{lemma}[Дирак, 1953]\label{lm:coloring_9}
    Пусть $G$ --- $k$-критический граф, $S = \{a, b\} \in \RR(G)$, $\Part(S) = \{F_1, \ldots , F_m\}$ и $G_i = F(F_i)$. Тогда $m = 2$, $ab \notin E(G)$ и части $\Part(S)$ можно занумеровать так, что графы $G_1 + ab$ и $G_2 \# ab$ --- $k$-критические.
\end{lemma}
\begin{proof}
    Пусть $U_i = \Int(F_i)$ ---  компонента связности графа $G - S$, $G_i' = G - U_i$, $M_j = V(G_i')$.

	Так как $G$ --- $k$-критический, то его подграф $G_i'$ имеет правильную раскраску в $k-1$ цвет. Пусть $\rho_i'$ --- такая раскраска.

	Назовем $\rho_i'$ раскраской \selectedFont{типа 1}, если $\rho_i'(a) = \rho_i'(b)$, и раскраской \selectedFont{типа 2}, иначе.
	\begin{statement} \label{st:coloring_1}
	    Если для некоторых различных $i, j$ существуют правильные раскраски в $k-1$ цвет одного типа $\rho_i'$ графа $G_i'$ и $\rho_j'$ графа $G_j'$, то $\chi(G) \le k-1$.
	\end{statement}
	\begin{proof}
	    Пока нет
	\end{proof}
	Предположим, что $m \ge 3$. Рассмотрим правильные раскраски $\rho_i'$ графов $G_i'$ в $k-1$ цвет для $i \in [1..3]$.

	Какие-то две имеют один тип, следовательно, по утверждению \ref{st:coloring_1} $\chi(G) \le k-1$, но это не так.

	Тогда, $m = 2$. $G_1 = G_2'$ и $G_2 = G_1'$.

	По утверждению \ref{st:coloring_1} у графов $G_1$ и $G_2$ не может быть правильных раскрасок в $k-1$ цвет одного и того же типа. НУО у всех тип $1$ в $G_1$ и $2$ в $G_2$.

	Так как существует правильная раскраска типа $1$, $ab \notin E(G)$.
	\begin{statement}\label{st:coloring_2}
		Граф $G_1^* = G_1 + ab$ --- $k$-критический.
	\end{statement}
	\begin{proof}
	    Пока нет
	\end{proof}
	\begin{statement}\label{st:coloring_3}
	    Граф $G_2^* = G_2 \# ab$ --- $k$-критический.
	\end{statement}
	\begin{proof}
	    Пока нет
	\end{proof}
\end{proof}

\section{Гипотеза Хайоша, случай $k = 4$.}
\begin{definition}
    Пусть $H$ --- произвольный граф. Назовем граф $H'$ \selectedFont{подразбиением} графа $H$, если $H'$ может быть получен из $H$ заменой нескольких ребер на простые непересекающиеся пути.
\end{definition}
\begin{theorem}[Дирак, 1953]
    Если $\chi(G) = 4$, то граф содержит в качестве подграфа подразбиение $K_4$.
\end{theorem}
\begin{proof}
    Достаточно доказать для $4$-критических графов. Рассмотрим такой граф $G$. Будем доказывать по индукции по количеству вершин в графе.

	\begin{description}
		\item[База:] $G$ --- трехсвязный граф. $\delta(G) \ge 3$, тогда в графе $G$ существует простой цикл $Z = a_1a_2\ldots a_n$, длины хотя бы $4$.

			Так как $G - \{a_1, a_3\}$ связен, существует простой путь $P$ от $a_2$ до $a_m$, не проходящий по другим вершинам цикла.

			Эти две вершины делят цикл на две непустые дуги: $B = \{a_3, \ldots , a_{m-1}\}$ и $B' = \{a_{m+1}, \ldots , a_1\}$.

			Так как $G - \{a_2, a_m\}$ связен, существует $BB'$-путь $Q$, не проходящий через $a_2$ и $a_m$. Пусть $a_x \in B$ и $a_y \in B'$ --- концы $Q$.

			Рассмотрим два случая
			\begin{itemize}
				\item $V(P) \cap V(Q) = \varnothing$.

					Рассмотрим подграф $H$ графа $G$ --- объединение цикла $Z$ и путей $P$ и $Q$. Этот граф --- подразбиение $K_4$.  См. рис. \ref{fig:dirac-lemma-1}.
				\item $V(P) \cap V(Q) \neq \varnothing$.

					Пусть $u$ --- первая точка пересечения c $Q$ на пути $P$ (от $a_2$ ). Тогда подграф $H$, равный объединению цикла $Z$, пути $Q$ и участка $a_2Pu$ --- подразбиение $K_4$. См. рис. \ref{fig:dirac-lemma-2}.
			\end{itemize}
		\begin{figure}[ht]
			\centering
			\begin{subfigure}{0.3\textwidth}
				\incfig{dirac-lemma}
				\caption{}
				\label{fig:dirac-lemma-1}
			\end{subfigure}
			\hfill
			\begin{subfigure}{0.3\textwidth}
				\incfig{dirac-lemma-2}
				\caption{}
				\label{fig:dirac-lemma-2}
			\end{subfigure}
			\hfill
			\begin{subfigure}{0.3\textwidth}
				\incfig{dirac-lemma-3}
				\caption{}
				\label{fig:dirac-lemma-3}
			\end{subfigure}
		\end{figure}
	\item[Переход:] граф $G$ не трехсвязен.

		Пусть $S $ --- минимальное разделяющее множество графа $G$. Тогда $\lvert S \rvert \le 2$. Пусть $S = \{a, b\}$.

		По лемме \ref{lm:coloring_9} вершины $a$ и $b$ несмежны,  $\Part(S) = \{F_1, F_2\}$, $G_i = G(F_i)$, причем части можно занумеровать так, что граф  $G_1^* = G_1 + ab$ --- $4$-критический.

		По индукционному предположению, в графе $G_1^*$ есть подграф $H$, являющийся подразбиением $K_4$. Если $H$ --- уже подграф $G$, то переход доказан.

		Предположим, что $H$ не подграф $G$, тогда $ ab \in H$ . 

		Граф $G_2$ связен, поэтому существует $ab$-путь $P$ c $\Int(P) \cap V(H) = \varnothing$. Тогда граф $H' = H - ab \cup P$ -- подграф $G$ и подразбиение $K_4$. См. рис. \ref{fig:dirac-lemma-3}.
	\end{description}
\end{proof}

\section{Конструкция графа с произвольным хроматическим числом без треугольников.}
\begin{definition}
    \selectedFont{Кликовое число} графа $G$ (обозначение $\omega(G)$ ) --- это количество вершин в максимальной клике.
\end{definition}

\begin{theorem}[Мычельский, 1955]
	Для любого $ k \in \N$ существует граф $G$, удовлетворяющий условиям $\chi(G) = k$, $g(G) \ge 4$.
\end{theorem}
\begin{proof}
	Для $k=1$ и $k=2$  подойдут полные графы $K_1$ и $K_2$.

	Построим следующие графы $G_3, G_4 , \ldots $ без треугольников с $\lambda(G_k) = k$.

	Пусть построен граф $G_k$, причем $V(G_k) = \{u_1, \ldots  ,u_n \}$.
	Этот граф будет частью графа $G_{k+1}$, в котором будут добавлены вершины $v_1 , \ldots  , v_n , w$. 

	Ребра между новыми вершинами проведем так: $v_i$ будет смежна со всеми вершинами из $N_{G_k}(ui)$ и только с ними, а $w$ --- со всеми вершинами $v_1 , \ldots  , v_n$ и только с ними (см. рис. \ref{}). 
	\begin{figure}[ht]
		\centering
		\incfig{mycielski-theorem}
		\caption{}
		\label{fig:mycielski-theorem}
	\end{figure}

	Понятно, что треугольников в графе $G_{k+1}$ нет. 

	Далее заметим, что $\chi(G_{k+1}) \le  k + 1$: если $\rho$ -- правильная раскраска вершин $G_{k+1}$ в $k$ цветов, то можно продолжить ее на $G_{k+1}$, использовав только один дополнительный цвет, для этого положим $\rho(v_i) = \rho(u_i)$ и $\rho(w)  = k+1$.

	Предположим, что $\chi(G_{k+1}) \le k_i$, и рассмотрим правильную раскраску $\rho$ вершин графа $G_{k+1}$ в $k$ цветов.

	НУО $\rho(w) = k$. Построим правильную раскраску $\rho'$ вершин $G_k$ в $k-1$ цвет, получим противоречие.

	 Для каждой вершины положим, $\rho' (u_i) = \rho(u_i)$, если $\rho(u_i) \neq k$, и $\rho'(u_i) = \rho(v_i)$, если $\rho(u_i) = k$. 

	 Так как вершины $v_1, \ldots , v_n$ смежны с вершиной $w$ цвета $k$, то их цвета отличны от $k$, следовательно, $\rho'\colon V(G_k) \to [1..k-1]$. 

	 Докажем правильность раскраски $\rho:$. Предположим противное, пусть $\rho' (u_i ) = \rho' (u_j )$, вершины $u_i$ и $u_j$ смежны.

	 Очевидно, хотя бы одна из них перекрашена, пусть это $u_i$, тогда $\rho'(u_i) = \rho(v_i)$. 

	 Мы перекрашивали только вершины, имеющие цвет $k$ в раскраске $\rho$, среди них не было смежных, следовательно, $\rho' (u_j ) = \rho(u_j )$.

	 По построению, из $u_j \in N_{G_k} (u_i )$ следует $u_j \in  N_{G_k} (v_i )$ и мы можем сделать вывод, что \[
	 \rho'(u_i) = \rho(v_i) \neq \rho(u_j) = \rho'(u_j)
	 .\] 
	 Противоречие. 

	 Таким образом, $\rho'$ --- правильная раскраска вершин графа $G_k$, противоречие. Следовательно, $\chi(G_{k+1}) = k + 1$.

\end{proof}


\section{Оптимальные раскраски ребер и их свойства. Xроматический и покрывающий индексы двудольного графа.}

\begin{definition}
     \selectedFont{Хроматический индекс} графа $G$ --- наименьшее натурально число $\chi'(G)$, для которого существует правильная раскраска ребер графа $G$ в такое количество цветов.

	 Назовем раскраску \selectedFont{покрывающей}, если ребра каждого цвета образуют покрытие, то есть покрывают все вершины.

	 \selectedFont{Покрывающий индекс} графа $G$ --- наибольшее натуральное число $\kappa'(G)$, для которого существует покрывающая раскраска ребер графа $G$. 
\end{definition}

\begin{definition}
    Пусть $\rho$ --- раскраска ребер графа $G$ в $k$ цветов. 

	Будем говорить, что в раскраске $\rho$ цвет $i$ \selectedFont{представлен} в вершине $v$, если существует инцидентное $v$ ребро $e$ такое, что $\rho(e) = i$. Обозначим за $\rho(v)$ количество цветов, представленных в вершине $v$.

	Введем обозначение $\rho(G) = \sum_{v \in V(G)}^{} \rho(v)$. Назовем раскраску $\rho$ \selectedFont{$k$-оптимальной}, если для любой другой раскраски $\rho'$ ребер графа $G$ в $k$ цветов $\rho(G) \ge \rho'(G)$.
\end{definition}
Пусть $\rho$ --- правильная раскраска ребер графа $G$ в не более чем  $k$ цветов. Тогда для каждой вершины $v \in V(G)$ имеем $\rho(v) = d_{G}(v) \ge \rho'(v)$  для любой  другой раскраски $\rho'$. Поэтому правильная раскраска всегда $k$-оптимальна.

\begin{lemma}\label{lm:coloring_10}
	Пусть $G$ --- связный граф, отличный от простого цикла нечетной длины. Тогда существует такая раскраска ребер $G$ в два цвета, что в каждой вершине степени не менее двух представлены оба цвета.
\end{lemma}
\begin{proof}
	\begin{itemize}
		\item Если все вершины имеют степень $2$, то это четный цикл, утверждение верно.
		\item Если  графе есть вершины нечетной степени, то добавим новую вершину $w$ и соединим со всеми вершинами нечетной степени. Получим граф со всеми четными степенями $\tilde{G}$.
		\item Если в графе $G$ есть вершины нечетной степени, то положим $a = w$.
		\item Если  все степени четные, то $\tilde{G} = G$, а в качестве  $a$ возьмем вершину степени хотя бы $4$, такая есть, так как $G$ не является четным циклом.
	\end{itemize}
	В графе $\tilde{G}$ есть эйлеров цикл. Покрасим ребра в порядке обхода по нему, начиная с $a$ и чередуя цвета.

	Пусть $x \neq a$. Если $d_G(x) \ge 2$, мы прошли через $x$ не меньше раза, поэтому у $x$ есть два ребра $G$ разных цветов. 

	Если $a = w$, то ничего проверять не нужно. Тогда остался случай, когда $a$ --- вершина степени хотя бы $4$. Тогда есть в $G$ есть два ребра инцидентных $a$ разных цветов.
\end{proof}

Любая раскраска $\rho$ ребер графа в цвета $[1 .. k]$ --- разбиение множества $E(G)$ в объединение непересекающихся множеств $E_1, \ldots , E_k$, где $\rho$ принимает значение $i$ на ребрах $E_i$.
\begin{lemma}\label{lm:coloring_11}
    Пусть $\rho$ --- $k$-оптимальная раскраска ребер графа $G$. Предположим, что вершина  $w$ и цвета $i$ и $j$ таковы, что в вершине $w$ хотя бы два раза представлен цвет $i$ и не представлен цвет  $j$. Пусть $H = G(E_{i} \cap E_j)$, а $H_w$ --- компонента графа $H$, содержащая  вершину $w$. Тогда $H_w$ --- простой цикл нечетной длины.
\end{lemma}
\begin{proof}
    Пусть $H_w$ не является простым циклом нечетной длины.

	Построим новую раскраску $\rho'$, отличающуюся только раскраской ребер $H_w$: раскрасим из в цвета $i$ и $j$ так, чтобы в каждой вершине $x$ степени $d_{H_w}(x) \ge 2$ были представлены оба цвета, это возможно по лемме \ref{lm:coloring_10}.

	Тогда $\rho'(w) \ge \rho(w) + 1$, а для любой другой вершины $x$, очевидно, что $\rho'(x) \ge \rho(x)$. Тогда $\rho'(G) > \rho(G)$, поэтому $\rho$ не может быть $k$-оптимальной. Противоречие. 
\end{proof}
\begin{note}
    Очевидно, что $\chi'(G) \ge \Delta(G)$ : все ребра, инцидентные одной вершине наибольшей степени должны быть разноцветными.
\end{note}
\begin{theorem}[Кениг, 1916]
    Пусть $G$ --- двудольный граф (возможно, с кратными ребрами). Тогда $\chi'(G) = \Delta(G)$.
\end{theorem}
\begin{proof}
    Пусть $\Delta = \Delta(G)$. Рассмотрим $\Delta$-оптимальную раскраску $\rho$ ребер графа $G$.

	Предположим, что $\rho$ --- неправильная. Тогда существует вершина $v$ и цвет $i$ такие, что $i$ дважды представлен в вершине $v$. 

	Так как $d_G(v) \le \Delta$, существует цвет $j$, который не представлен в вершине $v$. По лемме \ref{lm:coloring_11} в $G$ есть нечетный цикл, противоречие.
\end{proof}

\begin{theorem}[Гупта, 1966]
    Если граф $G$ двудольный, то $\kappa'(G) = \delta(G)$.
\end{theorem}
\begin{proof}
    Рассмотрим $\delta(G)$-оптимальную раскраску $\rho$ ребер графа $G$.

	Предположим, что $\rho$ не является покрывающей. Тогда существует вершина $v$ и цвет $i$ такие, что $i$ не представлен в вершине $v$. 

	Так как $d_G(v) \ge  \delta$, существует цвет $j$, который представлен в вершине $v$ дважды. По лемме \ref{lm:coloring_11} в $G$ есть нечетный цикл, противоречие.
\end{proof}

\section{Теорема Визинга.}
\begin{definition}
    Через $\mu(G)$ обозначим \selectedFont{максимальную кратность} ребра графа $G$, то есть максимум $e_G(\{x\}, \{y\})$ для всех пар $x, y \in V(G)$.
\end{definition}
\begin{theorem}[Визинг, 1964]
    Для любого графа $G$ выполнено $\Delta(G) \le \chi'(G) \le \Delta(G) + \mu(G)$.
\end{theorem}
\begin{proof}
    Пусть $\mu = \mu(G), \Delta= \Delta(G)$. Достаточно доказать существование правильной раскраски ребер $G$ в  $\Delta + \mu$ цветов.

	Рассмотрим $(\Delta+\mu)$-оптимальную раскраску $\rho$ ребер $G$.
	Предположим, что эта раскраска неправильная.

	Тогда существует вершина $u$ и цвет $ i_1$, который дважды представлен в вершине $u$. Так как $d_{G}(u) < \Delta + \mu$, существует цвет $j$, не представленный в $u$.

	Пусть $uv_1 \in E(G)$ и $\rho(uv_1) = i_1$. Так как $d_G(v_1) < \Delta + \mu$, существует цвет $i_2$, не представленный в $v_1$.
	
	{\bf  Опишем один шаг построения. }

	Пусть различные цвета $i_1, \ldots , i_l $ и ребра $e_1, \ldots , e_l \in E(G)$ таковы, что $e_t = uv_t$, $\rho(e_t) = i_t$, цвет $i_{t+1}$ не представлен в вершине $v_t$.

	Будем говорить, что цвет $i_{t+1}$ \textit{выбран для вершины} $v_t$. Также вершины $v_1, \ldots , v_t $ не обязательно различны.

	Рассмотрим вершину $v = v_l$. Пусть в наборе $v_1, \ldots , v_l $ она встречается $m$ раз. Очевидно, $m \le \mu$.
	
	Тогда на предыдущих шагах мы рассматривали вершину $v$ и $m-1$ раз выбирали цвет, не представленный в этой вершине. Поскольку \[
	d_G(v_l) + m - 1 < \Delta + \mu
	,\] 
	существует цвет $i_{l+1}$, не представленный в вершине $v_l$ и не выбранный для нее на предыдущих шагах, его мы и выберем.

	Определим раскраску данного шага $\rho_l\colon \rho_l(e_s) = i_{s+1}$ при $s \in [1..l]$ и $\rho_l(e) = \rho(e)$ на остальных ребрах.
	\begin{statement}
	    Раскраска $\rho_l$ $k$-оптимальна. Цвет $i_{l+1}$ представлен в вершине $u$.
	\end{statement}
	\begin{proof}
	    Для вершин $x \notin \{u, v_1, \ldots , v_l\}$ цвета ребер не менялись, поэтому $\rho_l(x) = \rho(x)$.

		Рассмотрим вершину $w$, которая входит в  $\{v_1, \ldots , v_l\}$ ровно $n$ раз. Пусть $W = v_{s_1} = \ldots = v_{s_n}$.

		По построению все выбранные для вершины $w$ цвета $i_{s_1 + 1}, \ldots i_{s_n+1}$ различны, не представлены в вершине $w$ в раскраске $\rho$ и представлены в раскраске $\rho_l$.

		Цвета $i_{s_1}, \ldots , i_{s_n}$ представлены в вершине $w$ в раскраске $\rho$.

		Все отличные от $e_{s_1}, \ldots , e_{s_n}$ ребра, инцидентные $w$, не изменили свой цвет, поэтому остальные цвета одинаково представлены в $w$ в раскрасках $\rho$ и $\rho_l$, поэтому $\rho_l(w) \ge \rho(w)$.

		Рассмотрим вершину $u$. В результате перекрашивания инцидентных $u$ ребер $e_1, \ldots , e_l$ из их цветов исчез $i_1$, и появился $i_{l+1}$. Но так как цвет $i_1$ был представлен в $u$ в раскраске $\rho$ хотя бы дважды, он представлен и в раскраске $\rho$. 

		Тогда $\rho_l(u) \ge  \rho(u) $ и $\rho_l(G) \ge \rho(G)$, следовательно, раскраска $p_l$ оптимальна. Так как $\rho$ тоже  оптимальная, $\rho(G) = \rho_l(G)$, поэтому цвет $i_{l+1}$ был представлен в вершине $u$ и в раскраске $\rho$.
	\end{proof}
	Пусть $e_{l+1} = uv_{l+1}$ --- ребро цвета $\rho(e_{l+1}) = \rho(G)$. 

	{ \bf Так мы завершили еще один шаг. }

	Поскольку у $u$ конечное число соседей, на некотором шаге мы получим $i_{m+1} = i_k$. То есть $v_m$ не совпадает с $v_{k-1}$ (иначе мы выбрали бы $i_{m+1} \neq i_k$ ). Так как в вершинах $v_{k-1}$ и $v_m$ в раскраске $\rho$ не представлен цвет $i_{k}$, а в $v_k$ представлен, все три вершины $v_{k-1}, v_k, v_m$ различны.

	Рассмотрим $(\Delta+\mu)$-оптимальные раскраски $\rho_{k-1}$ и $\rho_m$ (считаем, что $\rho_0 = \rho$):
\begin{itemize}
	\item В обеих раскрасках в вершине $u$ дважды представлен цвет  $i_k$.
	\item Цвет $j$ не представлен в вершине $u$ ни в одной из раскрасок.
\end{itemize}
\begin{figure}[ht]
    \centering
	\begin{subfigure}{0.3\textwidth}
		\centering
		\incfig{vising-theorem}
		\caption{Раскраска $\rho$}
		\label{fig:vising-theorem}
	\end{subfigure}
	\hfill
	\begin{subfigure}{0.34\textwidth}
		\centering
		\incfig{vising-theorem-2}
		\caption{Раскраска $\rho_k$}
		\label{fig:vising-theorem-2}
	\end{subfigure}
	\hfill
	\begin{subfigure}{0.34\textwidth}
		\centering
		\incfig{vising-theorem-3}
		\caption{Раскраска $\rho_m$}
		\label{fig:vising-theorem-3}
	\end{subfigure}
\end{figure}

Пусть $E_s$ --- множество всех ребер цвета $s$ в раскраске $\rho_{k-1}$, $E_s' $ --- множество всех ребер цвета $s$ в раскраске $\rho_m$. $H = G(E_{i_k} \cup E_j')$ и $H' = G(E_{i_k}' \cup E_j')$.

По лемме \ref{lm:coloring_11} из оптимальности раскрасок следует, что содержащие вершину $u$ компоненты связности графов $H$ и $H'$ --- простые циклы нечетной длины.

Тогда $d_{H}(v_k) = 2$ : из $v_k$ выходит ребро $uv_k$ цвета $\rho_{k-1}(uv_k) = i_k$ и ребро цвета $j$. Для всех ребер $e$ цикла $H$, кроме $uv_k$ цвет $\rho_{k-1}(e) = \rho_m(e)$, поэтому $d_{H'}(v_k) = d_{H}(v_k) - 1 = 1$. 

Вершины $v_k$ и $u$ лежать в одной компоненте связности $H'$, которая должна быть нечетным циклом. Противоречие. 

Следовательно, $\rho$ --- искомая правильная раскраска в $\Delta + \mu$ цветов.
\end{proof}

\section{Теорема Гупты.}
\begin{theorem}[Гупта, 1974]
    Для любого графа $G$ выполняется неравенство $\kappa'(G) \ge  \delta(G) - \mu(G)$.
\end{theorem}
\begin{proof}
    Пока нет
\end{proof}

\section{Хроматический многочлен графа.}
\begin{definition}
    Для любого натурального числа $k$  обозначим через $\chi_{G}(k)$ количество правильных раскрасок вершин графа $G$ в $k$ цветов.

	Функция  $\chi_G(k)$ называется \selectedFont{хроматическим числом} графа $G$.
\end{definition}
\begin{note}
    \begin{itemize}
		\item $\chi_G(\chi(G)) \neq 0$ 
		\item $ \forall k < \chi(G) \colon\chi_G(k) = 0$
    \end{itemize}
\end{note}

\begin{lemma}\label{lm:coloring_12}
	Пусть $G$ --- непустой граф, а $e = uv$ --- его ребро. Тогда  \[
	\chi_{G-uv}(k) = \chi_G(k) + \chi_{G \cdot uv}(k)
	.\] 
\end{lemma}
\begin{proof}
	Разобьем правильные раскраски графа $G-e$ в $k$ цветов на два типа:
	\begin{enumerate}
		\item где вершины $u$ и $v$ разного цвета;
		\item где вершины $u$ и $v$ одного цвета.
	\end{enumerate}
	Количество раскрасок первого типа равно $\chi_{G}(k)$, а второго --- $\chi_{G \cdot  ab}(k)$.
\end{proof}

\begin{theorem}
    Для любого графа $G$ без петель выполнены следующие утверждения:
	\begin{enumerate}
		\item Функция $\chi_G(k) \in \Z[k]$ --- унитарный многочлен с целыми коэффициентами степени $v(G)$ ;
		\item Знаки коэффициентов $\chi_G(k)$ чередуются, причем старший не меньше нуля.
	\end{enumerate}
\end{theorem}
\begin{proof}
    Индукция по размеру графа $G$ и количеству ребер.
	\begin{description}
		\item[База:] Для пустого графа на $n$ вершинах $\overline{K_n}$, очевидно, 
			$\chi_{\overline{K_n}}(k) = k^{n}$, поэтому все утверждения верны.
		\item[Переход:] Пусть $G$ --- непустой граф, $e $ --- его ребро. По лемме \ref{lm:coloring_12} 
			\[
			\chi_G(k) = \chi_{G-e}(k) - \chi_{G \cdot e}(k)
			.\] 
			Для меньших графов $G \cdot e$ и $G-e$ утверждения доказаны:
			\begin{itemize}
				\item $\chi_{G-e}(k)$ --- многочлен степени $v(G)$;
				\item $\chi_{G \cdot e}(k)$ --- многочлен степени $v(G \cdot e) = v(G) - 1$.
			\end{itemize}
			
			Старший коэффициент $\chi_G(k)$ равен старшему коэффициенту $\chi_{G - e}(k)$, то есть $1$.

			Так как $\deg(\chi_{G \cdot e}) = \deg(\chi_{G-e}) - 1$, в $\chi_G$ чередование знаков сохранится.
	\end{description}
\end{proof}

\section{Хроматический многочлен и компоненты связности. Кратность корня 0 хроматического многочлена графа.}
\begin{lemma}\label{lm:coloring_13}
    Пусть  $G_1, \ldots , G_{n} $ --- все компоненты графа $G$. Тогда $\chi_{G}(k) = \prod_{i=1}^{n}\chi_{G_i}(k)$.
\end{lemma}
\begin{proof}
    Очевидно
\end{proof}

\begin{theorem}
    Для любого графа $G$ число $0$ является корнем $\chi_G(k)$ кратности, равной количеству компонент связности.
\end{theorem}
\begin{proof}
    $0$ --- корень любого хроматического многочлена, так как раскрасок в $0$ цветов быть не может.

	Докажем, что для связного графа $G$ кратность корня $0$ у $\chi_G(k)$ равна $1$. Далее по лемме \ref{lm:coloring_13} получим утверждение теоремы.

	Пусть $v(G) = n$. Индукцией по количеству вершин докажем для связного графа $G$, что коэффициент при $k$ многочлена $\chi_G(k)$ не равен $0$ и имеет такой же знак как $(-1)^{n}$.
	\begin{description}
		\item[База:] 	$n=0$, очевидно.
		\item[Переход:] пусть $G$ --- связный граф c $v(G) = n  \ge 2$, для меньшего $n$ утверждение доказано, $T$ --- остовное дерево графа $G$. $\chi_T(k) = k(k-1)^{n-1}$.

			Существует последовательность графов $ G_0 = T, \ldots , G_n = G$, в которой $G_{i+1} = G_i + e$, где $e \notin E(G_i)$.

			Пусть $a_i$ --- коэффициент при $k$ многочлена $\chi_{G_i}(k)$. Докажем по индукции, что $a_i \neq 0$ и имеет такой же знак, что и $(-1)^{n-1}$. 
			\begin{description}
				\item[Бaза:] $i=0$, очевидно, по формуле для дерева.
				\item [Переход:] пусть $a_i \neq 0$ и имеет знак $(-1)^{n-1}$. По лемме \ref{lm:coloring_122} $\chi_{G_{i+1}}(k) = \chi_{G_i}(k) - \chi_{G_i \cdot e}(k)$.
					
					Граф $G_i \cdot e_i$ связен. По индукционному предположению у многочлена $\chi_{G_i \cdot e_i}(k)$ знак коэффициента $b$ при $k$ такой же, как $(-1)^{n-2}$, то есть отличается от знака $a_{i}$.

					Поэтому $a_{i+1} = a_i - b$ имеет такой же знак, что и $a_i$ и отличный от $0$.
			\end{description}
	\end{description}
\end{proof}


\section{Хроматический многочлен и блоки. Кратность корня 1 хроматического многочлена графа.}

\begin{lemma}\label{lm:coloring_14}
    Пусть $G$ --- связный граф с $n$ блоками $B_1, \ldots , B_n$. Тогда 
	\[
	\chi_G(k) = \left( \frac{1}{k} \right) ^{n-1} \cdot \prod_{i=1}^{n} \chi_{B_i} (k)
	.\] 
\end{lemma}
\begin{proof}
    Индукция по количеству блоков.
	\begin{description}
		\item[База:] для двусвязного графа, очевидно, это один блок.
		\item[Переход:] Пусть $n \ge 2$. НУО, $B_n$ --- крайний блок, содержащий ровно одну точку сочленения $a$.

			В графе $G' = G - \Int(B_n)$ ровно на один блок меньше, так как нет $B_n$. По индукционному продолжению для $G'$ :
			\[
			\chi_{G'}(k) = \left( \frac{1}{k} \right) ^{n-2} \cdot \prod_{i=1}^{n-1} \chi_{i}(k)
			.\] 
			Рассмотрим любую правильную раскраску $\rho$ графа $G'$ в $k$ цветов. Попробуем покрасить вершины $B_n$ c соблюдением правильности.

			Единственное ограничение --- цвет вершины $a$ уже зафиксирован, поэтому раскрасок в $k$ раз меньше.

			Следовательно, $\chi_G(k) = \frac{1}{k} \cdot  \chi_{G'}(k) \cdot \chi_{B_n}(k)$.
	\end{description}
\end{proof}
\begin{theorem}
    Пусть  $G$ --- связный граф с более чем одной вершиной. Тогда $1$ --- корень многочлена $\chi_G(k)$ кратности, равной количеству блоков графа $G$.
\end{theorem}
\begin{proof}
	Так как в каждом блоке хотя бы две вершины, достаточно доказать, что у хроматического многочлена графа без точек сочленения число $1$ является корнем кратности $1$, а далее применить лемму \ref{lm:coloring_14}. 

	Для $H \simeq K_2$ утверждение очевидно. Разберем второй вариант -- двусвязный граф. 

	$1$ точно корень,  так как раскрасить в один цвет двусвязный граф невозможно.

	Докажем, что $\chi'_{H}(1) \neq 0$, тогда мы покажем, что $1$ имеет кратность $1$.
	Для этого докажем, что для двусвязного графа $H$ на $m$ вершинах $\chi'_H(1) \neq 0$ и имеет такой же знак, как $(-1)^{m}$.

	\textbf{Индукция по $v(H)$.}
	
	\begin{description}
		\item[База:]
			Если $H$ --- полный граф на трех вершинах, то 
			\[
			\chi_{K_3}(k) = k(k-1)(k-2) \qquad \chi'_{K_3}(1) = 1(1-2) = -1
			.\] 
		\item[Переход:] Пусть $v(H)>3$. Тогда по теореме \ref{thm:connectivity_11} существует такое ребро $e \in E(H)$, что граф $H \cdot e$ двусвязен.

			По лемме \ref{lm:coloring_10} $\chi'_{H}(1) = \chi_{H-e}'(1) - \chi'_{H \cdot e} (1)$.

			Так как $v(H \cdot e) < e(H)$, если граф $H-e$ двусвязен, то уже доказано, что $\chi_{H-e}'(1)$ имеет тот же знак, что и $(-1)^{m}$.

			Если $H-e$ односвязен, то он имеет хотя бы два блока. Тогда для него верна лемма \ref{lm:coloring_12}.

			Так как хроматический многочлен каждого блока имеет корень $1$, причем для недвусвязного графа $H-e$ его хроматический многочлен имеет $1$ корнем кратности хотя бы $2$. И тогда $\chi_H'(1) = 0$, разность тоже не может быть равна нулю, а знак сохраняется из $H \cdot e$.
	\end{description}
\end{proof}


\chapter{Планарные графы}
\section{Теорема Жордана для ломаной.}
\begin{theorem}[Жордан, 1887]
    Замкнутая несамопересекающаяся ломаная $P$ делит точки плоскости, не лежащие на $P$, на две такие части, что выполнены следующие условия:
	\begin{enumerate}[label=(\arabic*)]
		\item любые две точки из одной части можно соединить ломаной, не пересекающей $P$ ;
		\item любая ломаная, соединяющая две точки из разных частей пересекает $P$.
	\end{enumerate}
\end{theorem}
\begin{proof}
    Пусть $ P_1 \ldots P_m$ --- вершины $P$ в порядке обхода по часовой стрелке. Обозначим через $M$ множество всех точек плоскости, не лежащих на $P$.

	Зафиксируем на прямой вектор $l$, не параллельный ни одной из сторон $P$. Из каждой точки $A \in M$ выпустим луч $l(A)$ в направлении $l$.

	Если $l(A)$ содержит вершину $P_i$ многоугольника $P$, то стороны $P_{i-1}P_i$ и $P_iP_{i+1}$ лежат в одной полуплоскости относительно $l(A)$, будем говорить, что многоугольник $P$ \selectedFont{касается } $l(A) $ в вершине $P_i$.

	Посчитаем число $p(A)$ точек пересечения $l(A) $ c $P$, не являющихся касаниями. Оно точно конечное.

	Обозначим за $M_0$ ту часть, которая состоит из всех точек $A \in M$, для которых $p(A)$ четно, и за $M_1$ --- нечетно.

	\begin{statement}\label{st:planar_1}
	    $M_0$ и $M_1$ непусты.
	\end{statement}
	\begin{proof}
	    Рассмотрим прямую $l_0$, параллельную вектору $l$, проходящую через внутреннюю точку ломаной $P$.

		Найдем последнее пересечение во внутренней точке прямой $l_0$ и $P$ в направлении вектора $l$ --- обозначим за ее $X$.

		Рассмотрим содержащий $X$ малый отрезок $[Y, Z]$ на $l_0$, не пересекающий $P$ в отличных от $X$ точках. Пусть $Y$ лежит перед $X$ при движении в направлении $l$. Тогда $p(Y) = 1$, а $p(Z) = 0$.
	\end{proof}
	\begin{statement}\label{st:planar_2}
		Пусть $A, B \in M$ и отрезок $[A, B]$ не пересекает $P$. Тогда $p(A) \equiv p(B) \pmod 2$.  В частности, выполнено второе условие теоремы.
	\end{statement}
	\begin{proof}
	    Если $AB \parallel l$, то утверждение очевидно.

		Если нет, отметим на отрезке $AB$ все такие точки $A_1, \ldots , A_k$ в направлении от $A$ к $B$, что $l(A_i)$ касается $P$ (если такие есть). И обозначим $A_0 = A$, $A_{k+1}  = B$.

		Тогда для каждого $i \in [0..k]$ все точки отрезка $[A_i, A_{i+1}]$ имеют одинаковое значение функции $p$, при переходе на соседний отрезок значение может изменится на четное число (см. рис. \ref{fig:jordan-theorem}).

		В любом случае, на всем отрезке $[A, B]$ четность одинаковая.
	\end{proof}
	\begin{figure}[ht]
		\centering
		\begin{subfigure}{0.48\textwidth}
			\centering
			\incfig{jordan-theorem}
			\caption{}
			\label{fig:jordan-theorem}
		\end{subfigure}
		\hfill
		\begin{subfigure}{0.48\textwidth}
			\centering
			\incfig{jordan-theorem-2}
			\caption{}
			\label{fig:jordan-theorem-2}
		\end{subfigure}
			\caption{}
	\end{figure}
	\textbf{Докажем первое утверждение теоремы}

	Пусть $A, B \in M_i$. Если отрезок $[A, B]$ не пересекает $P$, то все уже доказано. Тогда найдем ближайшие к $A$ и к $B$ точки пересечения $A_1$ и $B_1$ соответственно.

	Отметим на отрезке $[A, A_1]$ точку $A'$ очень близко к $A_1$, на отрезке $[B_1, B]$ --- точку $B'$ очень близко к $B$, обозначим <<очень близко>> за $\delta$. Тогда $p(A) = p(A')$ и $p(B) = p(B')$. См. рис. \ref{fig:jordan-theorem-2}.

	Проведем вдоль каждой стороны многоугольника две параллельных прямых на расстоянии $\delta$ с обоих сторон от $P$. Получим два новых многоугольника $P'$ и $P''$. Подбираем $\delta$ так, чтобы эти многоугольники не пересекали сторон исходного.

	НУО $A'$ лежит на $P'$. Если $B'$ тоже лежит на $P'$, то мы можем дополнить ее до точек $A$ и $B$, тем самым получив ломаную от $A$ до $B$, не пересекающую $P$.

	Пусть $B'$ лежит на $P''$. Тогда обозначим за $B^*$ точку пересечения $P'$ c $AB$ около $B$ на расстоянии $\delta$.

	Тогда $p(B^*) - p(B') = \pm 1$. Но по утверждению \ref{st:planar_2} должно выполнятся сравнение 
	\[
	p(B^*) \equiv p(A') \equiv p(A) \equiv p(B) \equiv p(B') \pmod 2
	.\] 
	Противоречие. 
\end{proof}

\section{Грань плоского графа и ее граница. Свойства.}
\begin{definition}
    Граф называется \selectedFont{планарным}, если его можно изобразить на плоскости так, чтобы его ребра не пересекались во внутренних точках. 
\end{definition}
\begin{definition}
    \selectedFont{Плоский граф} --- конкретное изображение планарного графа бед пересечений и самопересечений ребер.
\end{definition}
\begin{definition}[Грань]
	Пусть $M$ --- множество всех точек плоскости, не входящих в изображение $G$. Запись $A \sim B $ означает, что точки $A, B \in M$  можно соединить ломаной, не пересекающей изображение графа $G$. $\sim$ --- отношение эквивалентности.

	Назовем классы эквивалентности по $\sim$ \selectedFont{гранями}. Обозначим множество всех граней $F(G)$ и  $f(G) = \lvert F(G) \rvert$.
\end{definition}

\begin{definition}
    Рассмотрим ребро $e$ плоского графа $G$. Если по разные стороны $e$ расположены разные грани, то это ребро \selectedFont{граничное}, если одна и та же, что \selectedFont{внутреннее}.
	Обозначим через $E_d$ множество всех граничных и внутренних ребер.
\end{definition}
\begin{definition}
    \selectedFont{Граничные вершины} грани $d$ --- вершины, до которых можно дойти по ломаной от внутренних точек этой грани, не пересекая изображение графа $G$. Обозначим их множество через $V_d$.
\end{definition}
\begin{definition}
    \selectedFont{Граница} грани $d$ --- подграф $B(d)$ графа $G$ с множеством вершин $V_d$ и множеством ребер $E_d$.
\end{definition}
\begin{definition}
    \selectedFont{Размер границы} грани $d$ --- количество граничный ребер этой грани плюс удвоенное количество внутренних. Обозначение: $b(d)$.

	$\sum_{d \in F(G)}^{} b(d) = 2 e(G)$.
\end{definition}
\begin{lemma}\label{lm:planar_1}
    \begin{enumerate}
		\item Любые две точки на границе грани $d$ можно соединить ломаной, проходящей в $d$.
		\item Если две точки $A$ и $B$ на изображении графа $G$ можно соединить ломаной $L$, не пересекающей изображения $G$, то $A$ и $B$ лежат на границе некоторой грани.
    \end{enumerate}
\end{lemma}
\begin{proof}
    \begin{enumerate}
		\item Пусть $A$ --- внутренняя точка грани $d$. От нее можно провести ломаные, не пересекающие изображение $G$, до любых двух граничных. Все точки на этих ломаных лежат в $d$.
		\item $A$ и $B$ точно лежат на границе грани $d$, содержащей все внутренние точки $L$.
    \end{enumerate}
\end{proof}

\section{Циклический обход границы.}
\begin{definition}
    Рассмотрим любую вершину $a$ плоского графа $G$ и упорядочим выходы ребер из $a$ по часовой стрелке. Два ребра, выходы которых соседние в 
	этом порядке, будем называть \selectedFont{соседними в вершине $a$ }. 
\end{definition}
\begin{lemma}\label{lm:planar_2}
    Пусть $ab_1$ и $ab_2$ --- два соседних ребра в вершине $a$. Тогда $ab_1$ и $ab_2$ лежат в границе некоторой грани.
\end{lemma}
\begin{proof}
    Вершины $b_1$ и $b_2$ можно соединить ломаной вдоль $b_1ab_2$, не пересекающей изображения $G$. Поэтому, ребра $ab_1$ и $ab_2$ лежат на границе некоторой грани.
\end{proof}
\subsection{Циклический обход границы}
Пусть $G$ --- плоский граф, $d \in F(G)$, $x_1x_2 \in E_d$.

Пройдем по ребру $x_1x_2$ от $x_1$ до $x_2$. НУО справа по ходу движения расположена грань $d$.

Повернем в вершине  $x_2$ направо до выхода соседнего ребра $x_2x_3$. Если $d_G(x_2) = 1$, то $x_1 = x_3$, это не проблема. Также $x_2x_3 \in E_d$.

Пройдем по этому ребру от $x_2$ к $x_3$, справа опять будет расположена грань $d$. И так далее. В итоге мы вернемся на ребро $x_1x_2$, при этом в вершину $x_1$ мы могли приходить и по другому ребру.

Мы получили замкнутый циклический путь, см. рис. \ref{fig:cycle}.
\begin{figure}[ht]
    \centering
	\begin{subfigure}{0.48\textwidth}
		\centering
		\incfig{cycle}
		\caption{}
		\label{fig:cycle}
	\end{subfigure}
	\hfill
	\begin{subfigure}{0.48\textwidth}
		\centering
		\incfig{cycle-2}
		\caption{cycle-2}
		\label{fig:cycle-2}
	\end{subfigure}
\end{figure}

Пусть получился циклический маршрут $Z = x_1 x_2 \ldots x_k$. Рассмотрим вершину $x_i$. По построению $Z$ обходит вокруг $x_i$ --- пусть против часовой стрелки. 

Пусть мы вышли из $x_i$ по ребру $x_ix_{i+1}$, вернулись по ребру $x_{j-1} x_j  = x_{j-1}x_i$, см. рис. \ref{fig:cycle-2}.

Тогда сектор между выходами ребер $x_i x_{i+1}$ b $x_i x_{j-1}$ не принадлежит грани $d$.

Следовательно, $Z$ проходит все ребра из $E_d$, инцидентные вершине $x_i$. Поскольку это верно для любой входящей в $Z$ вершины, этот маршрут обходит в точности все ребра одной из компонент графа $B(d)$.

Обозначим за $Z(U)$ такой маршрут для компоненты $U$, а через $Z(d)$ --- объединение построенных маршрутов для всех компонент $B(d)$.

Если маршрут $Z(d)$ проходит ребро $e$ дважды, то в разных направлениях. Значит, по обе стороны от $e$ расположена грань $d$, то есть $e$ --- внутренне ребро $d$.

Пусть $e$ --- внутреннее ребро грани $d$. Тогда при проходе по $e$ в любом из направлений справа будет расположена грань $d$. Поэтому, маршрут $Z)d)$ дважды пройдет $e$ в обоих направлениях.

\section{Лемма о несвязной границе грани несвязного графа.}
\begin{lemma}\label{lm:planar_3}
	Для плоского графа $G$ выполнены следующие утверждения:
    \begin{enumerate}
		\item Если  $d \in F(G) $ и $B(d)$ несвязна, то разные компоненты связности графа $B(d) $ лежат в разных компонентах связности графа $G$.
		\item Граф $G$ несвязен, согда он имеет грань с несвязной границей.
    \end{enumerate}
\end{lemma}
\begin{proof}
    
\end{proof}


\end{document}
 
\chapter{Ответы}
% \documentclass[11pt,dvipsnames]{report}
% \usepackage[english, russian]{babel}
\usepackage{xltxtra}
\usepackage{polyglossia}

\usepackage{mathpazo}

\defaultfontfeatures{Ligatures=TeX,Mapping=tex-text}

\setmainfont{STIX2Text-Regular.otf}[
ExternalLocation={/home/vyacheslav/builds/STIXv2.0.2/OTF/},
BoldFont=STIX2Text-Bold.otf,
ItalicFont=STIX2Text-Italic.otf,
BoldItalicFont=STIX2Text-BoldItalic.otf
]
\setmathrm{STIX2Math.otf}[
ExternalLocation={/home/vyacheslav/builds/STIXv2.0.2/OTF/}
]


\usepackage{makeidx}
\usepackage{amssymb, amsthm}
\usepackage{amsmath}
\usepackage{mathtools}
\usepackage{needspace}
\usepackage{enumitem}
\usepackage{cancel}
\usepackage{fdsymbol}
\usepackage{fontawesome}


% разметка страницы и колонтитул
\usepackage[left=2cm,right=2cm,top=1cm,bottom=1.1cm,bindingoffset=0cm]{geometry}
\usepackage{fancybox,fancyhdr}
\fancyhf{}
\fancyhead[R]{\thepage}
\fancyhead[L]{\rightmark}
\fancyfoot{}
\fancyhfoffset{0pt}
\addtolength{\headheight}{13pt}
\pagestyle{fancy}

% Отступы
\setlength{\parindent}{3ex}
\setlength{\parskip}{3pt}

\usepackage{graphicx}
\usepackage{hyperref}

\usepackage{import}
\usepackage{xifthen}
\usepackage{pdfpages}

\newcommand{\incfig}[1]{%
    \def\svgwidth{\columnwidth}
    \import{./figures/}{#1.pdf_tex}
}


\usepackage{xifthen}
\makeatother
\def\@lecture{}%
\newcommand{\lecture}[3]{
    \ifthenelse{\isempty{#3}}{%
        \def\@lecture{Лекция #1}%
    }{%
        \def\@lecture{Лекция #1: #3}%
    }%
    \subsection*{\@lecture}
    \marginpar{\small\textsf{\mbox{#2}}}
}
\makeatletter


\usepackage{xcolor}
\definecolor{Aquamarine}{cmyk}{50, 0, 17, 100}
\definecolor{ForestGreen}{cmyk}{76, 0, 76, 45}
\definecolor{Pink}{cmyk}{0, 100, 0, 0}
\definecolor{Cyan}{cmyk}{56, 0, 0, 100}
\definecolor{Gray}{gray}{0.3}


\usepackage{mdframed}
\mdfsetup{skipabove=3pt,skipbelow=3pt}
\mdfdefinestyle{defstyle}{%
    linecolor=red,
	linewidth=3pt,rightline=false,topline=false,bottomline=false,%
    frametitlerule=false,%
    frametitlebackgroundcolor=red!0,%
    innertopmargin=4pt,innerbottommargin=4pt,innerleftmargin=7pt
    frametitlebelowskip=1pt,
    frametitleaboveskip=3pt,
}
\mdfdefinestyle{thmstyle}{%
    linecolor=cyan!100,
	linewidth=2pt,topline=false,bottomline=false,%
    frametitlerule=false,%
    frametitlebackgroundcolor=cyan!20,%
    innertopmargin=4pt,innerbottommargin=4pt,
    frametitlebelowskip=1pt,
    frametitleaboveskip=3pt,
}
\theoremstyle{definition}
\mdtheorem[style=defstyle]{defn}{Определение}

\newmdtheoremenv[nobreak=true,backgroundcolor=Aquamarine!10,linewidth=0pt,innertopmargin=0pt,innerbottommargin=7pt]{cor}{Следствие}
\newmdtheoremenv[nobreak=true,backgroundcolor=CarnationPink!20,linewidth=0pt,innertopmargin=0pt,innerbottommargin=7pt]{desc}{Описание}
\newmdtheoremenv[nobreak=true,backgroundcolor=Gray!10,linewidth=0pt,innertopmargin=0pt,innerbottommargin=7pt,font={\small}]{ex}{Пример}
\newmdtheoremenv[nobreak=false,backgroundcolor=Cyan!10,linewidth=0pt,innertopmargin=0pt,innerbottommargin=7pt]{thm}{Теорема}
\newmdtheoremenv[nobreak=true,backgroundcolor=Pink!10,linewidth=0pt,innertopmargin=0pt,innerbottommargin=7pt]{lm}{Лемма}

\newtheorem*{st}{Утверждение}
\newtheorem*{prop}{Свойства}

\theoremstyle{plain}
\newtheorem*{name}{Обозначение}

\theoremstyle{remark}
\newtheorem*{rem}{Ремарка}
\newtheorem*{com}{Комментарий}
\newtheorem*{note}{Замечание}
\newtheorem*{prac}{Упражнение}
\newtheorem*{probl}{Задача}


\renewcommand{\proofname}{Доказательство}
\renewenvironment{proof}
{ \hspace{\stretch{1}}\\ \faSquareO\quad \small  }
{ \hspace{\stretch{1}}  \faSquare \normalsize }


\numberwithin{ex}{section}
\numberwithin{thm}{section}
\numberwithin{equation}{section}



\newcommand{\K}{\mathcal{K}}
\newcommand{\Z}{\mathbb{Z}}
\newcommand{\N}{\mathbb{N}}
\newcommand{\Real}{\mathbb{R}}
\newcommand{\Q}{\mathbb{Q}}
\newcommand{\Cm}{\mathbb{C}}
\newcommand{\Pm}{\mathbb{P}}
\newcommand{\ord}{\operatorname{ord}}
\newcommand{\lcm}{\operatorname{lcm}}
\newcommand{\sign}{\operatorname{sign}}
\newcommand{\E}{\mathbb{E}}

\renewcommand{\o}{o}
\renewcommand{\O}{\mathcal{O}}
\renewcommand{\le}{\leqslant}
\renewcommand{\ge}{\geqslant}

\def\mybf#1{\textbf{#1}}
\def\selectedFont#1{\textbf{#1}}
\def\ComplexityFont#1{\textmd{\textbf{\textsf{#1}}}}
\def\LanguageFont#1{{\textbf{\texttt{#1}}}}


\newcommand{\Cclass}{\mathcal{C}}
\newcommand{\Dclass}{\mathcal{D}}


\renewcommand{\P}{\ComplexityFont{P}}
\newcommand{\DTIME}{\ComplexityFont{DTime}}
\newcommand{\DTime}{\ComplexityFont{DTime}}
\newcommand{\DSpace}{\ComplexityFont{DSpace}}
\newcommand{\PSPACE}{\ComplexityFont{PSPACE}}
\newcommand{\NTIME}{\ComplexityFont{NTime}}
\newcommand{\NSpace}{\ComplexityFont{NSpace}}
\newcommand{\coNSpace}{\ComplexityFont{coNSpace}}
\newcommand{\NPSPACE}{\ComplexityFont{NPSPACE}}
\newcommand{\poly}{\ComplexityFont{poly}}
\newcommand{\RP}{\ComplexityFont{RP}}
\newcommand{\coRP}{\ComplexityFont{co-RP}}
\newcommand{\ZPP}{\ComplexityFont{ZPP}}
\newcommand{\BPP}{\ComplexityFont{BPP}}
\newcommand{\BQP}{\ComplexityFont{BQP}}
\newcommand{\coBPP}{\ComplexityFont{co-BPP}}
\newcommand{\NP}{\ComplexityFont{NP}}
\newcommand{\NL}{\ComplexityFont{NL}}
\newcommand{\coNL}{\ComplexityFont{co-NL}}
\renewcommand{\L}{\ComplexityFont{L}}
\newcommand{\NPcomp}{\ComplexityFont{NP-complete}}
\newcommand{\tP}{\widetilde{\P}}
\newcommand{\tNP}{\widetilde{\NP}}
\newcommand{\tBH}{\widetilde{\BH}}
\newcommand{\Class}{{\ComplexityFont{C}}}
\newcommand{\coC}{\ComplexityFont{co-}\mathcal{C}}
\newcommand{\coNP}{\ComplexityFont{co-NP}}
\newcommand{\PH}{\ComplexityFont{PH}}
\newcommand{\EXP}{\ComplexityFont{EXP}}
\newcommand{\Size}{\ComplexityFont{Size}}
\newcommand{\Ppoly}{\ComplexityFont{P}/\ComplexityFont{poly}}
\newcommand{\NC}{\ComplexityFont{NC}}


\newcommand{\FACTOR}{\LanguageFont{FACTOR}}
\newcommand{\kQBF}{{\LanguageFont{QBF{\tiny k}}}}
\newcommand{\QBFk}{{\LanguageFont{QBF{\tiny k}}}}
\newcommand{\QBF}{{\LanguageFont{QBF}}}
\newcommand{\STCON}{\LanguageFont{STCON}}
\newcommand{\USTCON}{\LanguageFont{USTCON}}
\newcommand{\CircuitSat}{{\LanguageFont{CIRCUIT\_SAT}}}
\newcommand{\tCircuitSat}{\widetilde{{\LanguageFont{CIRCUIT\_SAT}}}}
\newcommand{\SAT}{\LanguageFont{SAT}}
\newcommand{\tSAT}{\widetilde{{\LanguageFont{SAT}}}}
\newcommand{\UNSAT}{{\LanguageFont{UNSAT}}}
\newcommand{\tThreeSAT}{\widetilde{{\LanguageFont{3\text{-}SAT}}}}
\newcommand{\ThreeSAT}{{\LanguageFont{3\text{-}SAT}}}
\newcommand{\BH}{\LanguageFont{BH}}
\newcommand{\CircuitEval}{{\LanguageFont{CIRCUIT\_EVAL}}}


\newcommand{\const}{\textmd{const}}
\newcommand{\logspace}{\textmd{logspace}}
\newcommand{\PATH}{\textmd{PATH}}


\newcommand{\readonly}{\textsf{read-only}}
\newcommand{\writeonly}{\textsf{write-only}}


\usepackage{ upgreek }
\newcommand{\PI}{\Uppi}
\newcommand{\SIGMA}{\Upsigma}
\newcommand{\DELTA}{\Updelta}

% \begin{document}
\section{Аксиоматизация объема параллелепипеда. Полилинейное отображение, кососимметричность. Свойства.}
\begin{defn}[Параллелепипед]
Пусть $ V$ --- векторное пространство размерности  $ n$ над полем $ \R$.  Тогда для набора  $ v_1, \ldots v_n \in V$ определим {\sf параллелепипед}
 \[
     D(v_1, \ldots v_n ) = \left\{ \sum_{i=1}^{n} \lambda_i v_i \Biggm| \lambda_i \in  [\;0, 1] \right\} 
.\] 
\end{defn}
\begin{prop}[Аксиоматизация в $ \R^{n}$ ]
    Будем записывать векторы в матрицу.
    \begin{enumerate}[noitemsep,start=0]
	\item $ \Vol(E_n)$ = 1
	\item $ \Vol(\ldots , \lambda v, \ldots ) = \lvert \lambda \rvert \Vol (\ldots , v, \ldots )$
	\item $ \Vol(\ldots , v , \ldots , u , \ldots ) = \Vol(\ldots , v, \ldots , u + \lambda v, \ldots )$ (исходя из принципа Кавальери)
	\item $ \Vol(\ldots , v, \ldots , v, \ldots ) = 0$
    \end{enumerate}
\end{prop}
\begin{prop}[Аксиоматизация в поле $ K$ ]

    $ $
    \begin{enumerate}[noitemsep]
	\item $ w(\ldots , \lambda v, \ldots ) = \lambda w (\ldots , v, \ldots )$
	\item $ w(\ldots , u+v, \ldots ) = w(\ldots , u, \ldots ) + w(\ldots , v, \ldots )$
	\item $ w(\ldots , v, \ldots , v, \ldots ) = 0$
    \end{enumerate}
\end{prop}
\begin{defn}[Полилинейное отображение]
Пусть $ U_1, \ldots U_l, V$ --- векторные пространства над полем $ K$. Отображение  $ w\colon U_1 \times \ldots \times U_l \to  V$ называется {\sf полилинейным}, если 
\[
    w(v_1, \ldots v_i + \lambda u_i , \ldots v_l) = w(v_1, \ldots , v_i, \ldots  v_l) + \lambda w(v_1, \ldots , u_i, \ldots v_l)
.\] 
\begin{name}
    $ \Hom_K(U_1, \ldots U_l; V) $ --- множество всех полилинейный отображений.
\end{name}
\end{defn}

\begin{defn}[Форма]
    Полилинейное отображение $ w \colon V^{l} \to  K$ называется {\sf полилинейной формой степени $ l$ на  $ V$}.  
\end{defn}
\begin{defn}
    Полилинейная форма $ w\colon V^{l} \to  K$ на пространстве $ V$ над полем  $ K$ называется 
    \begin{itemize}[noitemsep]
	\item {\sf антисимметричной}  или {\sf кососимметричной}, если $ w(v_1, \ldots v, \ldots , v, \ldots , v_l) = 0$;
	\item {\sf симметричной}, если $ w(v_1, \ldots , v_i, \ldots , v_j, \ldots , v_l) = w(v_1, \ldots , v_j, \ldots v_i, \ldots , v_l)$.  
    \end{itemize}
\end{defn}
\begin{lm}
    Пусть $ V$ --- векторное пространство размерности $ n$.
    Для полилинейного отображения $ w: V^{l} \to  K$ и любого $ e_1, \ldots e_n$ базиса $ V$ выполнено
    \[
    w(v_1, \ldots v_l) = \sum_{1 \le i_1, \ldots i_l \le n}^{} w(e_{i_1}, \ldots , e_{i_l}) \prod _{j=1}^{l} a_{i_j, j}, \qquad \text{где }  a_{ij} \text{ --- }  i\text{-ая координата вектора } v_j \text{ в базисе } e 
    .\] 
\end{lm}
\begin{lm}
    Пусть  $V$ --- векторное пространство размерности $n$. Для полилинейного отображения $ w\colon V^{l} \to  K$ выполнено:
    \begin{enumerate}[noitemsep]
	\item если $ w$ кососимметрично, то  $ w(\ldots , u, \ldots , v, \ldots ) = - w(\ldots , v, \ldots , u, \ldots )$;
	    \item если $ \Char K \ne 2$, из результата первого свойства следует кососимметричность;
	    \item если $ w$ кососимметрично, то для любой перестановки  $ \sigma  \in S_{l} $ верно $ w(v_{\sigma (1)}, \ldots v_{\sigma (l)}) = \sgn (\sigma ) w(v_1, \ldots v_l)$;
	    \item если $ w$ кососимметрично,  $ w(\ldots v, \ldots , u, \ldots ) = w(\ldots , v, \ldots , u + \lambda v, \ldots )$;
	    \item если $ w $ кососимметрично и $ l  = n$, для набора векторов $ v_1, \ldots v_n$ и базиса $ e_1, \ldots e_n$ выполнено
		\[
		    w(v_1, \ldots v_n) = w(e_1, \ldots e_n) \sum_{\sigma \in S_n}^{} \sgn(\\sigma ) \prod_{j=1}^{n} a_{\sigma (j), j} = w(e_1, \ldots e_n) \sum_{\sigma \in S_{n} }^{} \sgn(\sigma ) \prod _{i=1}^{n} a_{i, \sigma (i)}  
		.\] 
    \end{enumerate} 
\end{lm}
% \end{document}
 

\end{document}
