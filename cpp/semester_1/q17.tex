\section{Наследование: детали}
\begin{itemize}[noitemsep]
    \item сортировка и структуры данных C vs ООП
    \item private/protected наследование
    \item C+11: final, override
\end{itemize}
\subsection{сортировка и структуры данных C vs ООП}
\subsubsection{сортировка в стиле С}
\begin{ccode}
struct point_s {
    int x, y;
};

void qsort (void* base, size_t num, size_t size,
	    int (*compar)(const void*, const void*)) {}

int cmp_point1 (const void* v1, const void* v2) { 
    // здесть нет никакой проверки типов со стороны компилятора
    const struct point_s *p1 = (const struct point*) v1;
    const struct point_s *p2 = (const struct point*) v2;
    ...
}

struct point_s points[10];
qsort(points, 10, sizeof(points[0]), cmp_point1);
\end{ccode}
\subsubsection{сортировка в стиле ООП}
\begin{cppcode}
class Comparable {
    virtual bool operator<(const Comparable* v)=0 const;
};
void nsort(Comparable** m, size_t size) {
    ...
    m[i] = ;
}

class Point: public Comparable {
private:
    int x, y;
public:
    virtual bool operator<(const Comparable* v) const { ... };
};

int N = 10;
Point** points = new Point[N];
for (size_t i = 0; i < N; ++i) 
    points[i] = new Point(i, i);
nsort(points, N);
\end{cppcode}
\subsubsection{связный список в стиле С}
\paragraph{неинтруизивный}
\begin{ccode}
struct node_s {
    void* user_data;
    struct node_s *next;
};
struct list_s {
    struct node_s *head;
};

void push_back(struct list_s *i, struct node_s *n);
\end{ccode}
\begin{ccode}
struct node_s *n = malloc(sizeof(struct node_s));
n->user_data = malloc(sizeof(struct point_s));
// два malloc
push_back(&l, n);
\end{ccode}
\paragraph{интруизивный}
\begin{ccode}
struct point_s {
    int x, y;
    struct node_s node;
};
list_t l;
struct point_s *pn = malloc(sizeof(*pn));
// malloc один, но требутся трюк для получения x, y
push_back(&l, &pn->node);
\end{ccode}
\subsubsection{связный список в стиле С++}
\begin{cppcode}
class Object {
public:
    virtual bool operator<(const Object* o) {
	return this < o;
    }
    virtual bool operator=(const Object* o) {
	return this == o;
    }
    virtual int hash(const Object* o) {
	return (int)this;
    }
};

class Node {
public:
    void setData(Object* o);
private:
    Object* o;
    Node* next;
};

class List {
    Node* head;
public:
    void push_back(Node *n);
};

class Point: public Object {
    int x, y;
    virtual bool operator<(const Object* o) {
	Point *p = (Point*) o;
	return this < o;
    }
};
\end{cppcode}
\begin{cppcode}
List l;
Node *n = new Node;
n->setData(new Point(3, 4));
// тоже два new
l.push_back(n);
\end{cppcode}
\subsubsection{private/protected/public наследование}
\begin{cppcode}
class P {
public:
    int x;
protated:
    int y;
private:
    int x;
};
\end{cppcode}
\begin{cppcode}
class A: public P {
    // x is public
    // y is protected
    // z in not accessible from P
}

class B: protected P {
    // x is protected
    // y is protected
    // z in not accessible from P
}

class C: private P {
    // x is private
    // y is private
    // z in not accessible from P
}
\end{cppcode}
\paragraph{Два типа отношений}
\begin{description}[noitemsep]
    \item[has a] у машины есть двигатель и тормоза
\begin{cppcode}
class Car {
    Engine e;
    Brakes b;
};
\end{cppcode}
    \item[is a] грузовик это  машина, автобус это машина 
\begin{cppcode}
class Track: public Car {};
class Bus: public Car {};
\end{cppcode}
\end{description}
Еще можно показать, что машина не двигатель:
\begin{cppcode}
class Engine {
    protected:
    void maintenanceCheck() { .. };
};

class Car: private Engine {
    void reset() {
	maintenanceCheck();
    }
};
\end{cppcode}

\subsection{C++11: final, override}
\subsubsection{override}
\begin{minipage}{0.45\textwidth}
\begin{cppcode}
class Base {
    public:
    virtual void f(int);
    virtual int g() const;
    void h(int);
};
\end{cppcode}
\begin{cppcode}
class Derived: public Base {
public:
    void f(int); // не знаем virtual?
    int g(); // перегрузка вместо перекрытия
    void h(int); // не была virtual
}
\end{cppcode}
\end{minipage}
\hfill
\begin{minipage}{0.45\textwidth}
    Это можно исправить, используя override;
\begin{ccode}
class Derived: public Base {
    public:
    void f(int) override; // ok
    int g() override; // CE
    void h(int) override; // CE
};
\end{ccode}
\end{minipage}
\subsubsection{final}
\begin{ccode}
struct Base {
    virtual void f();
};
struct Derived: public Base {
    void f() final; // дети не могут больше унаследовать f()
}
struct DerivedD: public Derived {
    void f(); // CE
}
\end{ccode}
