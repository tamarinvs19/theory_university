\section{Работа с кучей на C++}
\begin{itemize}[noitemsep]
    \item new/delete
    \item cоздание объектов в куче
    \item конструктор копий
    \item оператор присваивания
    \item C++11: =delete
\end{itemize}
\subsection{new/delete}
Это аналоги malloc и free. Но new и delete --- операторы, их транслирует компилятор, а не линкуется из библиотеки.
Нельзя смешивать две пары, так как они могут иметь внутренние различия.
\begin{ccode}
int* pi = new int[5]; // создаем массив
my_class* a = new my_class;
delete [] a;
delete my_class;
\end{ccode}
\subsection{Создание объектов в куче}
new и delete вызывают конструкторы и деструкторы классов соответственно. При создании массива через new не стоит удалять без [], может вызвать undefined behavior. Если вызываем delete [], он проходит по массиву и вызывает деструктор у каждого элемента.
\subsection{Конструктор копий}
Используется, когда нужен оператор присваивания. Есть один объект нашего классы, мы хотим проинициализировать новый с такими же значениями полей.
\begin{ccode}
class Array {
    Array (const Array& a) {
	size = a.size;
	data = new int[size];
	// если просто скопировать объекты через знак равно,
	//будет побайтовое копирование и будут ссылаться на одну память
	for (size_t i = 0; i < size; ++i) 
	    data[i] = a.data[i];
    }
}

Array a = b;
f(a); // здесь тоже копирование
\end{ccode}
В конструктор копий передается именно ссылка, так как иначе потребуется скопировать объект, то есть возникнет рекурсивный бесконечный вызов.
\subsection{Оператор присваивания}
Хотим создать два объекта, а потом присвоить первому второй. Для этого можно определить оператор ``=''.
\begin{ccode}
Array& Array::operator=(const Array& a) {
    if (&a == this) // для a = a;
	return *this;
    delete [] data;
    size = a.size;
    data - new int[size];
    for (size_t i = 0; i < size; ++i) 
	data[i] = a.data[i];
    return *this; // для a = b = c;
}
\end{ccode}
Еще можно сделать это же с помощью swap:
\begin{ccode}
Array& Array::operator=(Array a) { //copy-swap
    std::swap(_data, a._data);
    return *this;
}
\end{ccode}
