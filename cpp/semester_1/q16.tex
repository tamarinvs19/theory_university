\section{Разное}
\begin{itemize}[noitemsep]
    \item ключевое слово const (С/C++)
    \item перегрузка функций
    \item параметры функций по умолчанию
\end{itemize}
\subsection{ключевое слово const}
\subsubsection{в С}
Назначение:
$ $
\begin{enumerate}[noitemsep]
    \item избавить себя от описок
	\item сделать код более понятным для читающего
\end{enumerate}
\begin{cppcode}
const double pi = 3.14159;
char const * ptr1; // pointer to const
char * const ptr1; // const pointer

size_t strlen(const char *s);

const Array operator=(const Array a) {...}; //значение только rvalue

void f(const char* str) {
    char* s1 = (char*)str;
    s1[0] = 'x';
}
\end{cppcode}
\begin{cppcode}
char s1[] = "Hello, world!";
const char* s2 = "Hello, world";
f(s1);
f(s2); // RT error
\end{cppcode}
s2 --- строковая константа:
\begin{itemize}[noitemsep]
    \item храниться в отдельном блоке памяти рядом с глобальными;
    \item память, выделенная под такую константу, может быть помечена как read-only 
\end{itemize}
\subsubsection{в C++}
\begin{cppcode}
class Array {
private:
    const size_t size; 
    int* data;
public:
    Array(size_t s);
    Array(const Array& a); // a не будет изменен
    Array& operation=(const Array& a);
    ~Array();

    int get(int i) const;
    void get(int i, int v);
    size_t get_size() const; // защищает от изменения сам объект
}
\end{cppcode}
\begin{cppcode}
void print(const Array& a) {
    for (size_t i = 0; i < a.get_size(); ++i) 
	printf("%d", a.get(i));
}
\end{cppcode}
\subsection{перегрузка функций}
Можно создать несколько функций с одним именем, но разными параметрами, при этом линкер их сможет различить: во время компиляции  им присваиваются новые имена (name mangling). Так можно, например, создать одному классу несколько конструкторов.
\subsection{параметры функций по умолчанию}
\begin{minted}[fontsize=\footnotesize,numbersep=3pt,framesep=1mm,linenos,frame=single,label=player.h]{cpp}
class Player {
private:
    double life;
    int damage;
public:
    Player(double life = 100.0, int damage = 10);
    // Player(double life, int damage = 10); ok
    // Player(double life = 100.0, int damage); error
}
\end{minted}
\begin{minted}[fontsize=\footnotesize,numbersep=3pt,framesep=1mm,linenos,frame=single,label=main.cpp]{cpp}
#include "player.h'
int main() {
    Player p1(2, 3);
    Player p2(2);
    Player p2();
}
\end{minted}
Параметры по умолчанию должны быть указаны в  header, так как их необходимо знать при компиляции $main.cpp$, а другие файлы в этот момент не используются.
