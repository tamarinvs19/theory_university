\documentclass{article}
\usepackage{xltxtra}
\usepackage{polyglossia}
\usepackage{amsthm}
\usepackage{amsmath}
\usepackage{mathtools}
\usepackage{enumitem}
\usepackage{mathpazo}
\usepackage{fontspec}

\defaultfontfeatures{Ligatures=TeX,Mapping=tex-text}

\setmainfont[
ExternalLocation={/home/vyacheslav/builds/STIXv2.0.2/OTF/},
BoldFont=STIX2Text-Bold.otf,
ItalicFont=STIX2Text-Italic.otf,
BoldItalicFont=STIX2Text-BoldItalic.otf
]
{STIX2Text-Regular.otf}
\setmathrm{STIX2Math.otf}[
ExternalLocation={/home/vyacheslav/builds/STIXv2.0.2/OTF/}
]

% \setromanfont[ExternalLocation={/home/vyacheslav/builds/STIXv2.0.2/OTF/}]{STIX2Text-Regular.otf}
% \setsansfont[ExternalLocation={/home/vyacheslav/builds/STIXv2.0.2/OTF/}]{STIX2Text-Regular.otf}
\setmonofont[ExternalLocation={/home/vyacheslav/.fonts/}]{Iosevka Term Light Nerd Font Complete.ttf}

% \setromanfont{Times New Roman} 
% \setsansfont{Arial} 
% \setmonofont{Courier New} 

\begin{document}
Топологией на множестве $ T$ называется набор подмножеств $ A \subset B$,
обладающий следующими свойствами:

Привет Мир! 
$ f(x) = \int_{0}^{\infty} g(x) dx$ 
Γειά σου Κόσμε!

\texttt{123 iosevka}
\textbf{123 stix}
\textit{123 italic stix}
\textsf{123 sans stix}

	\textit{Топологией} на множестве $ X $ называется набор подмножеств $ \mathcal{T} \subset 2^X $, обладающий следующими свойствами: 
	\begin{enumerate}
		\item $ \emptyset $ и $ X $ лежат в $ \mathcal{T} $; 
		\item объединение всех элементов любого подмножества $ \mathcal{T} $ лежит в $ \mathcal{T} $; 
		\item пересечение элементов любого конечного подмножества $ \mathcal{T} $ лежит в $ \mathcal{T} $. 
	\end{enumerate}
	Множество $ X $ с заданной на нём топологией $ \mathcal{T} $ называется \textit{топологическим пространством.}

\end{document}
