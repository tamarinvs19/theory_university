\lecture{6}{18 march}{\dag}
\section{Арифметическая иерархия}

Вспомним следующее свойство:
	$ A \subset \N$ перечислимо тогда и только тогда, когда $\exists  B \subset \N \times \N$ разрешимое, такое что $ A$ --- проекция $ B$,
	или 
	\begin{equation}\label{eq:2.8.1}
	x \in A \Longleftrightarrow \exists  y ~ (x, y) \in  B
	\end{equation}
Можем считать $ A, B$ свойствами (предикатами), то есть  
$ A(x) \hookrightarrow x \in A  $.
Тогда можем переписать \ref{eq:2.8.1} так
\[
    A(x) \Longleftrightarrow \exists y ~ B(x, y)
.\] 
Какие множества представимы в виде $ \forall y \colon B(x, y)$? Это равносильно
\[
    \neg \left( \exists  y ~ \neg B(x, y) \right) 
.\] 
Это \selectedFont{коперечислимые} (то есть дополнение перечислимого)

\begin{defn}\index{$\Sigma_n$}\index{$\Pi_n$}
    $ A \in \Sigma_{n} $, если его можно представить в виде 
    \[
	A(x) \Longleftrightarrow \exists y_1 \forall y_2 \exists y_3 \ldots B(x, y_1, \ldots , y_{n})
    ,\] 
    где $ B$ разрешимо.

	\noindent
     $ A \in \Pi_{n}$, если
     \[
	 A(x) \Longleftrightarrow \forall y_1 \exists y_2 \forall y_3 \ldots B(x, y_1, \ldots , y_{n})
     .\] 
\end{defn}

\begin{prop}
    ~\begin{enumerate}
        \item Определение не изменится, если разрешить несколько одинаковых кванторов подряд, так как можем заменить повторные на один с помощью канторовой нумерации
	\item $ A(x) \in  \Sigma _{n} \Longleftrightarrow \neg A(x) \in \Pi_{n}$
    \end{enumerate}
\end{prop}

\begin{thm}
    Если $ A(x), B(x) \in \Sigma _n $ (или $ \Pi_n$), то
     \[
		 A(x) \cup B(x) \in \Sigma_n ~ \text{(или $ \Pi_n$)}
    \] 
     \[
		 A(x) \cap  B(x) \in \Sigma_n ~ \text{(или $ \Pi_n$)}
    .\] 
\end{thm}
\begin{proof}
	Запишем определения
    \begin{align*}
	A(x) & \Longleftrightarrow \exists y_1 \forall y_2 \exists y_3 \ldots P(x, y_1, \ldots y_{n}) \\
	B(x) & \Longleftrightarrow \exists z_1 \forall z_2 \exists z_3 \ldots Q(x, z_1, \ldots z_{n}) 
    \end{align*}
	Скомбинируем кванторы
    \[
	%A(x) \cap B(x) \Longleftrightarrow \underbrace{\exists y_1 \exists z_1}_{\exists c(y_1, z_1)} \forall y_2 \forall z_2 \ldots \underbrace{P(x, y_1, \ldots y_{n}) \cap Q(x, z_1, \ldots z_n)}_{l(c(y, z))}
	%A(x) \cap B(x) \Longleftrightarrow \underbrace{\exists y_1 \exists z_1}_{\exists s_1 = c(y_1, z_1)} \forall y_2 \forall z_2 \ldots \underbrace{P(x, y_1, \ldots y_{n}) \cap Q(x, z_1, \ldots z_n)}_{S(x, s_1, \ldots, s_n) = P(x, l(s_1), \ldots, l(s_n)) \cap Q(x, r(s_1), \ldots, r(s_n))}
	A(x) \cap B(x) \Longleftrightarrow \exists y_1 \exists z_1 \forall y_2 \forall z_2 \ldots P(x, y_1, \ldots y_{n}) \cap Q(x, z_1, \ldots, z_n).\]
	\[
	A(x) \cap B(x) \Longleftrightarrow \underbrace{\exists s_1}_{c(y_1, z_1)} \forall s_2 \ldots \underbrace{S(x, s_1, \ldots, s_n)}_{P(x, l(s_1), \ldots, l(s_n)) \cap Q(x, r(s_1), \ldots, r(s_n))}
    .\]
	Так как $ P$ и $ Q$ разрешимы их объединения и пересечения тоже разрешимы.
\end{proof}

\begin{note}
    Аналогично можно определить $ \Sigma _n, \Pi_n$ для подмножеств $ \N^{m}$.
\end{note}
\begin{prop}
	~\begin{itemize}
		\item $ \Sigma _n, ~ \Pi _n \subseteq \Sigma _{n+1}, ~\Pi_{n+1}$
	\item $ \Sigma _n \cup \Pi _n \subseteq \Sigma _{n+1} \cap \Pi _{n+1}$
	\end{itemize}
\end{prop}
\begin{proof}
    Аналогично прошлой теореме сворачиваем кванторы в группы, добавляем кванторы.
\end{proof}

В итоге получается следующая картина
\begin{align*}
	& \Sigma_0 \subseteq \Sigma _1 \subseteq  \Sigma _2 \subseteq \Sigma _3 \subseteq \ldots \\
	&\rotatebox[origin=c]{90}{=} \quad \rotatebox[origin=c]{-45}{$\subseteq$} \quad~  \rotatebox[origin=c]{45}{$\subseteq$}\\
	& \Pi_0 \subseteq \Pi _1 \subseteq  \Pi _2 \subseteq \Pi _3 \subseteq \ldots
\end{align*}

\begin{thm}
    $ A \le _m B ,  ~ B \in \Sigma _n$, то $ A \in \Sigma _{n}$ (аналогично с $ \Pi_n$)   
\end{thm}
\begin{proof}
	Распишем согласно определениям
	\[
		A \le _m B \stackrel{\operatorname{def}}{\Longleftrightarrow }  \exists f \in \F_{*}\colon x \in A \Longleftrightarrow f(x) \in  B
	\] 
	и
	\[
		B \in \Sigma _n \stackrel{\operatorname{def}}{\Longleftrightarrow } x \in B \Longleftrightarrow \exists y_1 \forall y_2 \ldots R(x, y_1, y_2, \ldots y_n)
	.\] 
	Тогда 
	\[
		x \in A \Longleftrightarrow f(x) \in B \Longleftrightarrow \exists y_1 \forall y_2 \ldots R(f(x), y_1, \ldots y_{n})
	.\] 
	Так как $ f\in \mathcal{F}_*$, то и $ R(f(x), y_1, \ldots y_n)$ разрешимо.
\end{proof}

\begin{defn}[Универсальное множество]
	Множество $ W \subset \N \times \N$ \selectedFont{универсальное} для перечислимых подмножеств $ \N$, если
	\begin{itemize}
		\item $ W $ перечислимое;
		\item для всех перечислимых $ B \subset \N$ существует номер $ n$,такой что $ W_n = B$.
	\end{itemize}
\end{defn}

\begin{defn}[Главное универсальное множество]
	Множество $ W \subset \N \times \N$ \selectedFont{главное универсальное} для всех перечислимых подмножеств $ \N$, если
для любого перечислимого $ V \in \N \times \N$ существует функция $ s \colon \N \to  \N$:
	\begin{itemize}
		\item $s$ вычислима и всюду определена;
		\item $ \forall x, n\colon  (n, x) \in  V \Longleftrightarrow (s(n), x) \in  W$
	\end{itemize}
\end{defn}

\begin{ex} 
Например, таким множеством будет область определения главной универсальной функции.
\end{ex}

\begin{thm}
    Для любого $ n>0$ в классе $ \Sigma_n $ (соответственно $ \Pi_n$ ) существует множество, универсальное для всех множеств в $ \Sigma _n$ (соответственно $ \Pi_n $ ) 
\end{thm}
\begin{note}
    Если $ A$ --- универсальное в $ \Sigma _n$, то $ \overline{A}$ --- универсальное в $ \Pi_n$
\end{note}

\begin{proof}
	Докажем по индукции.

    Для $ \Sigma _1$ -- перечислимое, уже строили.
    Для $ \Pi_2$
     \[
     \forall y \underbrace{\exists z \underbrace{R(x, y, z)}_{\text{разрешимо}}}_{P(x, y)} \Longrightarrow \forall y \underbrace{P(x, y)}_{\text{\tiny перечислимое свойство}}
    .\] 

    Рассмотрим $ U(n, x, y)$ --- универсальное множество для перечислимых. Тогда 
     \[
	 T(n, x) \coloneqq  \forall y U(n, x, y)
    .\] 

	$ T(n, x)$ получается универсальным для  $  \Pi_2$.

    Следовательно, 
    существует универсальное для $ \Sigma _2$ --- дополнение до $ T(n, x)$.

    Продолжаем далее по индукции: для $ 3$ начинаем с  $ \Sigma _3$
    \[
	\Sigma _3 \colon \exists y \forall z \exists t R(x, y, z, t) \Longleftrightarrow \exists y \forall z P(x, y, z)
    .\] 

    Универсальное для $ \Sigma _3$:
    \[
	T(n, x) = \exists y \forall z U(n, x, y, z), \text{где }  U \text{ --- универсальное перечислимое}
    .\] 
	И так далее
\end{proof}

\begin{thm}
    Универсальное множество для $ \Sigma_n$ не принадлежит $ \Pi_n$ и наоборот.
\end{thm}
\begin{proof}
	Пусть $ T(m, x)$  --- универсальное $  \Sigma _n$-свойство. И предположим, что $ T(m, x) \in \Pi_n$.  

	Рассмотрим $ D(x) = T(x, x) \in \Pi_n$, так как $ D \le _m T$.

	Поэтому $ \neg D(x) \in \Sigma_n$, но оно отличается от всех сечений $ T(m, x)$. Противоречие. 
\end{proof}
\begin{cor}
    $ \Sigma_n \subsetneq \Sigma_{n+1}$ и $ \Pi_n \subsetneq \Pi_{n+1}$. 
\end{cor}

\begin{proof}
    Знаем, что $ \exists x \in \Pi_n \setminus \Sigma_n$, а также что $ \Pi_n \subseteq \Sigma_{n + 1}$, то есть $ x \notin \Sigma_n$ и $ x \in \Sigma_{n + 1}$.
\end{proof}

\section{Еще про $ T$-сводимость}
Зафиксируем некоторую функцию-оракула $  \alpha $. Вся теория вычислимости может быть <<релятивизована>> относительно вычислений с оракулом $  \alpha $. То есть теперь все вычисления просто выполняются с оракулом $\alpha$.
\begin{defn}
    Функцию, вычислимую с оракулом $  \alpha $ будем называть $  \alpha $-вычислимой.
\end{defn}
\begin{name}
    $ \F_{\alpha}^{m}$ --- класс $  \alpha $-вычислимых функций от $ m $ аргументов.
\end{name}

    \begin{thm}
        Пусть $ \alpha $ --- всюду определенная функция. Тогда 
		$
			\exists U_{ \alpha } (n, x) \in \F_{ \alpha} ^2
			$
		--- универсальная для класса $ \F^{1}_{ \alpha }$
    \end{thm}
	\begin{proof}
		Пусть $ U_{ \alpha }(i, x)$ --- результат применения $ i$-ой МТ c оракулом $  \alpha $ к $ x$.
		Как устроен оракул нам не важно, поэтому можем просто вшить обращение к нему во входные данные.
	\end{proof}

Аналогично перечислимым множествам можем определить $  \alpha$-перечислимые множества как:
\begin{itemize}
	\item область определения $  \alpha $-вычислимой функции 
	\item область значений $  \alpha $-вычислимой функции
	\item проекция  $  \alpha $-разрешимого множества
\end{itemize}

\begin{thm}
	~\begin{enumerate}
	    \item Для любого $X \subset \N  $ существует универсальное $ X$-перечислимое множество для $ X$-перечислимых.
		\item Это множество будет $ m$-полным в классе $ X$-перечислимых.
	\end{enumerate} 
\end{thm}
	\begin{proof}
	    Доказательство полностью аналогично такой же теореме для обычной перечислимости.
	\end{proof}

\begin{defn}[]
	Множества $ P$ и $ Q$ являются \selectedFont{ $ m$-эквивалентными} ($ P \equiv _m Q$), если $ P \le _m Q$ и $ Q \le _m P$.

	\noindent
	\selectedFont{$ m$-степень} --- $ \deg_m(P) = \{Q \mid Q \equiv _m P\} $.
\end{defn}

\begin{defn}[]
	Множества $ P$ и $ Q$ являются \selectedFont{ $ T$-эквивалентными} ($ P \equiv _T Q$), если $ P \le _T Q$ и $ Q \le _T P$.

	\noindent
	\selectedFont{$ T$-степень} --- $ \deg_T(P) = \{Q \mid Q \equiv _T P\} $.
\end{defn}

\begin{note}
	Если $ P, Q \in \deg_T(X)$, то $ \F_P = \F_Q$ и  $ P$-перечислимость  эквивалентна  $ Q$-перечислимости.

	Поэтому можем говорить о $\deg_T X$-перечислимости.
\end{note}

\begin{defn}[Операция скачка]
	\selectedFont{Операция скачка} $ J\colon \deg_T \to \deg_T $ (или $ \deg_T \to \deg_m) $ выбирает для $ X$ какое-то $ m$-полное относительно $ X$-перечислимости.

	\noindent
	То есть по всем эквивалентным $ \deg_T(X)$ получаем $ X$-перечислимые и среди них выбираем полные. 
\end{defn}
\begin{figure}[ht]
    \centering
    \incfig{jump-img}
    \label{fig:jump-img}
\end{figure}

\paragraph{$ T$-степени}
\begin{itemize}
	\item $ \mathbb{O}$ --- степень, содержащая все разрешимые.
	\item $ \mathbb{O}' = J(\mathbb{O}) $ --- степень $ m$-полного перечислимого неразрешимого. Один из представителей --- область определения универсальной функции.
	\item $ \mathbb{O}^{(n+1)}= \left(\mathbb{O}^{(n)}\right)'$
\end{itemize}
Можно считать, что $ \mathbb{O}'$ --- множества, перечислимые с оракулом проблемы остановки. 

\section{Теорема об арифметической иерархии}

\begin{thm}[Об арифметической иерархии]\index{теорема об арифметической иерархии}\label{thm:main_thm}
	$  \forall n \ge 1 \colon \Sigma _n = \{\mathbb{O}^{(n-1)} \text{-перечислимые множества}\}$
\end{thm}
\begin{proof}
    Для $ n = 1$ уже знаем, это перечислимые множества.
	\begin{description}
	    \item \boxed{  \subset } 
			Рассмотрим $ X \in \Sigma _2$, тогда
			\[
				x \in X \Longleftrightarrow \exists y \forall z \underbrace{R(x, y,z)}_{ \text{разрешимо}}
			.\] 
			Навешиваем отрицание
			\[
				X' := \neg \left( \forall z R(x, y, z) \right)  \in \Sigma _1
			.\] 
			Принадлежность $  \Sigma _1$ дает $ m$-сводимость к $ m$-полному перечислимому множество. А так как $ m$-сводимость влечет $ T$-сводимость, то по определению $ \mathbb{O}'$ множество $ X'$ будет $ \mathbb{O}'$-разрешимо.
			
			Cледовательно, его дополнение тоже $ \mathbb{O}'$-разрешимо.

			А значит проекция $\neg X'$ ($X$) будет $ \mathbb{O}'$-перечислима, так как можно перебрать $ y$.

			Аналогично действуем для больших  $ n$: 
			\[ x \in \Sigma_n\colon x \in X \Longleftrightarrow \exists y \underbrace{R(x, y)}_{ \in \Pi _{n - 1}}\]
			Тогда $ \neg R \in \Sigma _{n-1} .$ 

		На предыдущем слое доказали, что тогда $ \neg R $ будет $ \mathbb{O}^{(n-2)}$-перечислимо, а тогда оно $ \mathbb{O}^{(n-1)}$-разрешимо. Тогда его проекция $ \mathbb{O}^{(n-1)}$-перечислима.
	    \item \boxed{\supset } 
			См. \hyperref[proof:main_thm]{далее} (страница \pageref{proof:main_thm}).
	\end{description} 
\end{proof}


\subsection{Утверждения для доказательства в обратную сторону}
\begin{defn}
    Рассмотрим $ c$ --- некоторую вычислимую нумерацию конечных множеств.

	Пусть $ D_x$ --- множество с номером  $ x$.

	Возьмем $ A \subset \N$ (не обязательно конечное).

	Определим $ \Subsetm(A) = \{x \mid D_x \subset A\}$ --- множество номеров конечных подмножеств $ A$.

	Аналогично $ \Disjointm(A) = \{x \mid D_x \cap A = \varnothing\}$.
\end{defn}
\begin{lm}[о $ \Subsetm$]
	Если $ A \in \Sigma_n$ (или $ \Pi_n$), то  $ \Subsetm(A) \in \Sigma_n$ (или $ \Pi_n$ соответственно).
\end{lm}
\begin{proof}
	Пусть $ A \in \Sigma_3$,
	\[
	x \in A \Longleftrightarrow \exists y \forall z \exists t \underbrace{R(x, y, z, t)}_{\text{разрешимо}}
	.\] 
	Для конечного набора
	\[
		\{x_1, \ldots x_m\} \subset A \Longleftrightarrow \exists (y_1, \ldots y_m) \forall (z_1, \ldots , z_m) \exists (t_1, \ldots t_m) \bigwedge_{i=1}^{m} R(x_i, y_i, z_i, t_i)
	.\] 
	А это равносильно $ c(x_1, \ldots x_m) \in \Subsetm(A)$.
\end{proof}

\begin{lm}[о $ \Disjointm$]
	Если $ A \in \Sigma_n$ (или $ \Pi_n$), то $ \Disjointm(A) \in \Pi_n$ (или $ \Sigma_n$ соответственно).
\end{lm}
\begin{proof}
    \[
		D_x \cap A = \varnothing  \Longleftrightarrow D_x \subset \overline{A} \Longleftrightarrow x \in \Subsetm (\overline{A})
	\]
	То есть $ \Disjointm(A) = \Subsetm(\overline{A})$.
	Тогда
	\begin{align*}
		A \in \Sigma_n &\Longleftrightarrow \overline{A} \in \Pi_n \Longrightarrow  \tag{по лемме о $ \Subsetm$} \\
		&\Subsetm(\overline{A}) \in \Pi_n \Longleftrightarrow \\
		&\Disjointm(A) \in \Pi_n
	\end{align*}
\end{proof}

\subsection{Относительная вычислимость: эквивалентные определения}
\begin{defn}[Образец]
	\selectedFont{Образец} --- функция $ \N \to \N$, область определения которой конечна. Задается конечным множеством пар, то есть  можем вычислимо пронумеровать образцы.
	
	\noindent
	Образцы \selectedFont{совместны}, если объединение их графиков есть график функции. Т.е. если оба определены для какого-то $ x$, то значение на этом $ x$ должно совпадать.
\end{defn}
\begin{defn}[]
	$ M$ --- множество троек $ (x, y, t)$, где  $ x, y \in \N$, а $ t$ --- образец.

	\noindent
	Две тройки $ (x_1, y_1, t_1), (x_2, y_2, t_2)$ \selectedFont{противоречат друг другу}, если $  x_1= x_2$, $  y_1 \ne  y_2$ и $  t_1$ и $  t_2$ совместны.

	\noindent
	Множество $ M$ \selectedFont{корректно}, если оно не содержит противоречащих троек.
\end{defn}
\begin{defn}
    Пусть  $  M$ --- корректное множество троек, $ \alpha $ --- функция.
	\[
	\begin{aligned}
		M_1& = \{(x, y, t) \mid (x, y, t) \in M, t \text{ --- подмножество графика } \alpha \} \\
		M_2&= \{(x, y) \mid \exists t\colon (x, y, t) \in M_1\}
	\end{aligned}
	\]
	Тогда $  M_2$ определяет график некоторой функции $ M[ \alpha ]$. Причем определение корректно, так как для всех $ x$ не больше одного значения.
\end{defn}

