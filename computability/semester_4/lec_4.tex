\lecture{4}{4 march}{\dag}

\subsection{Главные универсальные функции}
\begin{defn}[]
	Пусть $ U^{(n+1)} \in \F^{n+1}$ универсальна для $ \F^{n}$. $U$ называется \selectedFont{главной}, если для всех $   V \in \F^{n+1}$  существует \selectedFont{транслятор} $ s \in \F_{*}$ , такой что 
	\[
		\forall m, \overline{x}\colon V(m, \overline{x}) = U(s(m), \overline{x}) 
	.\] 
	% картинка
\end{defn}


\begin{thm}
    Существует универсальная функция $ U \in \F^{n+1} $ для $ \F^{n}$.
\end{thm}
\begin{proof}
	Пусть существует $ T \in \F^{n+2}$, универсальная для $ \F^{(n+1)}$. Построим $ U$.
	\begin{itemize}
		\item Определим $ U(m, \overline{x}) \coloneqq T(l(m), r(m), \overline{x})$, где $ l $ и $ r$ --- левый и правый обратные к канторовской нумерации.
			
			Можно  записать так:
			\[
				U(c(n, u), \overline{v}) = T(n, u, \overline{v})
			.\] 
		\item За счет универсальности $ T $ для всех $ V \in \F^{n+1}$ существует $ m$, такое что
			\[
				\forall x, \overline{y}\colon V(x, \overline{y}) = T(m, x, \overline{y})
			.\] 
		\item 
			По определению $ U$ выражение выше равно $ U(x(m, x), \overline{y})$. То есть можно взять транслятор $ s(x) = c(m, x)$.
	\end{itemize}
\end{proof}
\begin{proof}
    Универсальная функция, которую мы строим в прошлый раз, является главной.

	Все $ V(x, y)$ нужно <<превратить в константу>> $ V_{x}^{(y)}$.
	% расписать нормально
\end{proof}


\begin{thm}
    Пусть $ U \in \F^2 $ --- главная функция для $ F$ тогда и только тогда, когда существует $ f \in \F^2_{*}$, такая что
	\[
		U_p \circ U_q = U_{f(p, q)}
	.\] 
\end{thm}
\begin{proof}
    ~\begin{description}
        \item \boxed{ 1 \Longrightarrow 2} 
			Рассмотри $ V(n, x) = U(l(n), U(r(n), x))$, то есть  $ V(c(p, q), x) = U(p, U(q, x))$. Фактически $ V$ --- это $ U_q \circ U_p$.

			Так как $ U$ --- главная универсальная, 
			\[
				\exists s \in \F_{*} \colon V(n, x)  = U(x(n), x)
			.\] 
			Тогда
			\[
				U(p, U(q, x)) = V(c(p, q), x) = U(s(c(p, q)), x)
			.\] 
			Искомая $ f = s(c(p, q))$.
        \item \boxed{ 2 \Longrightarrow 1} 
			упражнение(*)
    \end{description} 
\end{proof}


\begin{thm}[Райса (Успенского)]
	Пусть $ A \subset \F$ --- некоторое нетривиальное \footnote{Имеется в виду, что $ A \ne \F$ и $ A \ne \varnothing$ } свойство вычислимой функции.

	Пусть $ U$ --- главная вычислимая функция для всех вычислимых функций $ \F$.

	Тогда не существует алгоритма, который по  $ U$-номеру вычислимой функции проверяет $ A$. То есть множество $ \{n \mid U_n \in A\}$ неразрешимо.
\end{thm}
\begin{proof}
    Покажем, что, если свойство $ A$ можно алгоритмически проверить, то любые два непересекающихся перечислимых множества можно отделить некоторым разрешимым.

	Пусть $ P$ и $ Q$ --- произвольные непересекающиеся множества. $ \xi$ --- какая-нибудь функция из $ A$, а $ \eta$ --- какая-нибудь не из $ A$.

	Рассмотрим следующую функцию:
	\[
		V(n, x) = 
		\begin{cases}
			\xi(x), & x \in P \\
			\eta(x), & x \in Q \\
			\uparrow, & n \notin P \cup Q
		\end{cases}
	\] 
	Заметим, что $ V$ вычислимая, так как можем запустить по шагам алгоритмы для $ P$ и $ Q$, если один из них завершается, то считаем соответствующую функцию, а иначе значение не определено.

	Теперь с помощью проверки  $ V_n(x) = U_{s(n)}(x) \in  A$ можем отделить $ P$ от $ Q$. Противоречие. 
\end{proof}
\begin{cor}
    Множество номеров некоторой заданной функции $ \varphi $ в главной нумерации неразрешимо.

	В частности, в главной нумерации множество МТ, вычисляющих одну функцию, бесконечно много.
\end{cor}
\begin{cor}\label{cor:5}
    Можно построить пример универсальной неглавной функции --- нигде не определенная функция имеет единственный номер.

	\noindent
	\textbf{Переформулировка:} для любой главной нумерации $ U$ и любой вычислимой функции $ f$  множество $ \{n \mid U_n = f\}$ неразрешимо.
\end{cor}
\begin{proof}
	Пусть $ U(n, x)$ --- произвольная универсальная и $ D$ --- множества номеров функций с непустой областью определения, которое  перечислимое.

	Поэтому существует алгоритм, перечисляющий $ D = \{f(0), f(1), \ldots \}$.

	Рассмотрим функцию
	\[
		V(i, x) = \begin{cases}
			\uparrow, & i = 0 \\
			U(f(i-1), x), & i \ne 0
		\end{cases}
	\] 
	Эта функция выполнима, универсальна, единственный номер нигде не определенной функции --- только $ \{0\}$,  это множество конечно, следовательно разрешимо, поэтому $ V$ не главная по \href{cor:5}{следствию \ref{cor:5}}. 
\end{proof}


\section{Теорема о неподвижной точке}

\begin{lm}\label{lm:3}
    Пусть $ \equiv $ --- отношение эквивалентности на $ \N$. Тогда следующие утверждения не выполняются одновременно:
	\begin{enumerate}
		\item $ \forall f \in \F ~ \exists g \in \F_{*}$, отношение $ \equiv  $ является продолжением функции $ f$, то есть, если $ f(x) $ определена, то  $ g(x)$ тоже определена и $ g(x) \equiv f(x)$.
		\item $ \exists h \in \F_{*}$, не имеющая неподвижной точки, то есть $ \forall n \colon n \not\equiv h\left( n \right) $.
	\end{enumerate} 
\end{lm}
\begin{proof}
	Рассмотрим $ f \in \F$, от которой никакая вычислимая функция не может отличаться всюду (например,  $ f(x) = U(x, x)$ ).

	Пусть выполняются оба пункта. Рассмотрим $ t(x) \coloneqq h(g(x))$ и предположим, что она всюду отличается от $ f$.
	\begin{itemize}
		\item Если $ f$ определена, то $ f(x) \equiv g(  x )  \not\equiv  h(g(x)) = t(x)$ 
		\item Если нет, то $ t$ определена
	\end{itemize}
\end{proof}
\begin{thm}[О неподвижной точке]
	Если $ U$ --- главная универсальная функция для $ \F$, а $ h \in \F_{*}$, то $ \exists n \colon U_n = U_{h(n)}$.
\end{thm}
\begin{proof}
    Возьмем в качестве отношения эквивалентности
	\[
	x \equiv x \Longleftrightarrow U_x = U_y
	.\] 
	Покажем, что выполняется первый пункт из \href{lm:3}{леммы \ref{lm:3}}.

	Пусть $ f \in \F$. Тогда можем рассмотреть $ V(n, x) \coloneqq U(f(u), x)$. 
	Так как $ U$ главная:

	\[
		\exists s \in \F^{*} \colon  \forall n, x ~V(n, x) = U(S(n), x)
	.\] 

	$ S $ и есть $ \equiv $-продолжение $ f$
	\begin{itemize}
		\item если $ f(n)$ определена, то $ U_{s(n) = U_{f(n)}}$, то есть $ (s(n) = f(n)$.
		\item если не определена, то $ U_{s(n)}$ нигде не определена.
	\end{itemize}

	В итоге первый пункт выполняется, поэтому второй не выполняется.
\end{proof}
\begin{cor}
	$ U(n, x)$ --- главная универсальная вычислимая функция. Тогда 
	\[
		\exists p \in \N \colon \forall x  ~ U(p, x) = p
	.\] 
\end{cor}
\begin{proof}
	Рассмотрим $ V \in \F^{2}$, такую что $ V(n, x) = n$.

	Теперь применим теорему о неподвижной точке к  $ S(n)$ : 
	\[
		\exists p \colon  U_p = U_{s(p)} = V_p = p
	.\] 
\end{proof}

\section{$ m$-сводимость}
\begin{defn}[$ m$-сводимость]
	Множество $ A \subset \N$  \selectedFont{$m$-сводится } ( $ A \le_{m} B$ к $ B \subset \N$, если существует $ f \in \F_{*}$, такая что
	\[
		\forall x \in \N \colon x \in A \Longleftrightarrow f(x) \in B
	.\] 
\end{defn}
\begin{prop}
	~\begin{itemize}
		\item Если $ A \le _{m} B$ и $ B$ разрешимо, то $ A$ разрешимо.
		\item Если $ A \le _{m} B$ и $ B$ перечислимо, то $ A$ перечислимо.
		\item Отношение $ \le _{m}$ рефлексивно и транзитивно.
		\item Если $ A \le _{m} B$, то $ \N \setminus A \le _{m} \N \setminus B$.
	\end{itemize}
\end{prop}
\begin{proof*}
    Упражнение
\end{proof*}
\begin{note}
    Разрешимое множество сводится к любому $ B \notin \{\infty, \N\}$
\end{note}
\begin{note}
    К пустому множеству сводится только пустое. к $ \N$ сводится только $ \N$.
\end{note}


\begin{defn}[$ m$-полнота]
	Перечислимое множество $ A$ называется \selectedFont{$ m$-полным} (в классе перечислимых мнжеств), если любое перечислимое $ B$  $ m$-сводится к $ A$.
\end{defn}

\begin{thm}
    Существует $ m$-полное перечислимое множество.
\end{thm}
