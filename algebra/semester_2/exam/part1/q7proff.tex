% \documentclass[11pt,dvipsnames]{report}
% \usepackage[english, russian]{babel}
\usepackage{xltxtra}
\usepackage{polyglossia}

\usepackage{mathpazo}

\defaultfontfeatures{Ligatures=TeX,Mapping=tex-text}

\setmainfont{STIX2Text-Regular.otf}[
ExternalLocation={/home/vyacheslav/builds/STIXv2.0.2/OTF/},
BoldFont=STIX2Text-Bold.otf,
ItalicFont=STIX2Text-Italic.otf,
BoldItalicFont=STIX2Text-BoldItalic.otf
]
\setmathrm{STIX2Math.otf}[
ExternalLocation={/home/vyacheslav/builds/STIXv2.0.2/OTF/}
]


\usepackage{makeidx}
\usepackage{amssymb, amsthm}
\usepackage{amsmath}
\usepackage{mathtools}
\usepackage{needspace}
\usepackage{enumitem}
\usepackage{cancel}
\usepackage{fdsymbol}
\usepackage{fontawesome}


% разметка страницы и колонтитул
\usepackage[left=2cm,right=2cm,top=1cm,bottom=1.1cm,bindingoffset=0cm]{geometry}
\usepackage{fancybox,fancyhdr}
\fancyhf{}
\fancyhead[R]{\thepage}
\fancyhead[L]{\rightmark}
\fancyfoot{}
\fancyhfoffset{0pt}
\addtolength{\headheight}{13pt}
\pagestyle{fancy}

% Отступы
\setlength{\parindent}{3ex}
\setlength{\parskip}{3pt}

\usepackage{graphicx}
\usepackage{hyperref}

\usepackage{import}
\usepackage{xifthen}
\usepackage{pdfpages}

\newcommand{\incfig}[1]{%
    \def\svgwidth{\columnwidth}
    \import{./figures/}{#1.pdf_tex}
}


\usepackage{xifthen}
\makeatother
\def\@lecture{}%
\newcommand{\lecture}[3]{
    \ifthenelse{\isempty{#3}}{%
        \def\@lecture{Лекция #1}%
    }{%
        \def\@lecture{Лекция #1: #3}%
    }%
    \subsection*{\@lecture}
    \marginpar{\small\textsf{\mbox{#2}}}
}
\makeatletter


\usepackage{xcolor}
\definecolor{Aquamarine}{cmyk}{50, 0, 17, 100}
\definecolor{ForestGreen}{cmyk}{76, 0, 76, 45}
\definecolor{Pink}{cmyk}{0, 100, 0, 0}
\definecolor{Cyan}{cmyk}{56, 0, 0, 100}
\definecolor{Gray}{gray}{0.3}


\usepackage{mdframed}
\mdfsetup{skipabove=3pt,skipbelow=3pt}
\mdfdefinestyle{defstyle}{%
    linecolor=red,
	linewidth=3pt,rightline=false,topline=false,bottomline=false,%
    frametitlerule=false,%
    frametitlebackgroundcolor=red!0,%
    innertopmargin=4pt,innerbottommargin=4pt,innerleftmargin=7pt
    frametitlebelowskip=1pt,
    frametitleaboveskip=3pt,
}
\mdfdefinestyle{thmstyle}{%
    linecolor=cyan!100,
	linewidth=2pt,topline=false,bottomline=false,%
    frametitlerule=false,%
    frametitlebackgroundcolor=cyan!20,%
    innertopmargin=4pt,innerbottommargin=4pt,
    frametitlebelowskip=1pt,
    frametitleaboveskip=3pt,
}
\theoremstyle{definition}
\mdtheorem[style=defstyle]{defn}{Определение}

\newmdtheoremenv[nobreak=true,backgroundcolor=Aquamarine!10,linewidth=0pt,innertopmargin=0pt,innerbottommargin=7pt]{cor}{Следствие}
\newmdtheoremenv[nobreak=true,backgroundcolor=CarnationPink!20,linewidth=0pt,innertopmargin=0pt,innerbottommargin=7pt]{desc}{Описание}
\newmdtheoremenv[nobreak=true,backgroundcolor=Gray!10,linewidth=0pt,innertopmargin=0pt,innerbottommargin=7pt,font={\small}]{ex}{Пример}
\newmdtheoremenv[nobreak=false,backgroundcolor=Cyan!10,linewidth=0pt,innertopmargin=0pt,innerbottommargin=7pt]{thm}{Теорема}
\newmdtheoremenv[nobreak=true,backgroundcolor=Pink!10,linewidth=0pt,innertopmargin=0pt,innerbottommargin=7pt]{lm}{Лемма}

\newtheorem*{st}{Утверждение}
\newtheorem*{prop}{Свойства}

\theoremstyle{plain}
\newtheorem*{name}{Обозначение}

\theoremstyle{remark}
\newtheorem*{rem}{Ремарка}
\newtheorem*{com}{Комментарий}
\newtheorem*{note}{Замечание}
\newtheorem*{prac}{Упражнение}
\newtheorem*{probl}{Задача}


\renewcommand{\proofname}{Доказательство}
\renewenvironment{proof}
{ \hspace{\stretch{1}}\\ \faSquareO\quad \small  }
{ \hspace{\stretch{1}}  \faSquare \normalsize }


\numberwithin{ex}{section}
\numberwithin{thm}{section}
\numberwithin{equation}{section}



\newcommand{\K}{\mathcal{K}}
\newcommand{\Z}{\mathbb{Z}}
\newcommand{\N}{\mathbb{N}}
\newcommand{\Real}{\mathbb{R}}
\newcommand{\Q}{\mathbb{Q}}
\newcommand{\Cm}{\mathbb{C}}
\newcommand{\Pm}{\mathbb{P}}
\newcommand{\ord}{\operatorname{ord}}
\newcommand{\lcm}{\operatorname{lcm}}
\newcommand{\sign}{\operatorname{sign}}
\newcommand{\E}{\mathbb{E}}

\renewcommand{\o}{o}
\renewcommand{\O}{\mathcal{O}}
\renewcommand{\le}{\leqslant}
\renewcommand{\ge}{\geqslant}

\def\mybf#1{\textbf{#1}}
\def\selectedFont#1{\textbf{#1}}
\def\ComplexityFont#1{\textmd{\textbf{\textsf{#1}}}}
\def\LanguageFont#1{{\textbf{\texttt{#1}}}}


\newcommand{\Cclass}{\mathcal{C}}
\newcommand{\Dclass}{\mathcal{D}}


\renewcommand{\P}{\ComplexityFont{P}}
\newcommand{\DTIME}{\ComplexityFont{DTime}}
\newcommand{\DTime}{\ComplexityFont{DTime}}
\newcommand{\DSpace}{\ComplexityFont{DSpace}}
\newcommand{\PSPACE}{\ComplexityFont{PSPACE}}
\newcommand{\NTIME}{\ComplexityFont{NTime}}
\newcommand{\NSpace}{\ComplexityFont{NSpace}}
\newcommand{\coNSpace}{\ComplexityFont{coNSpace}}
\newcommand{\NPSPACE}{\ComplexityFont{NPSPACE}}
\newcommand{\poly}{\ComplexityFont{poly}}
\newcommand{\RP}{\ComplexityFont{RP}}
\newcommand{\coRP}{\ComplexityFont{co-RP}}
\newcommand{\ZPP}{\ComplexityFont{ZPP}}
\newcommand{\BPP}{\ComplexityFont{BPP}}
\newcommand{\BQP}{\ComplexityFont{BQP}}
\newcommand{\coBPP}{\ComplexityFont{co-BPP}}
\newcommand{\NP}{\ComplexityFont{NP}}
\newcommand{\NL}{\ComplexityFont{NL}}
\newcommand{\coNL}{\ComplexityFont{co-NL}}
\renewcommand{\L}{\ComplexityFont{L}}
\newcommand{\NPcomp}{\ComplexityFont{NP-complete}}
\newcommand{\tP}{\widetilde{\P}}
\newcommand{\tNP}{\widetilde{\NP}}
\newcommand{\tBH}{\widetilde{\BH}}
\newcommand{\Class}{{\ComplexityFont{C}}}
\newcommand{\coC}{\ComplexityFont{co-}\mathcal{C}}
\newcommand{\coNP}{\ComplexityFont{co-NP}}
\newcommand{\PH}{\ComplexityFont{PH}}
\newcommand{\EXP}{\ComplexityFont{EXP}}
\newcommand{\Size}{\ComplexityFont{Size}}
\newcommand{\Ppoly}{\ComplexityFont{P}/\ComplexityFont{poly}}
\newcommand{\NC}{\ComplexityFont{NC}}


\newcommand{\FACTOR}{\LanguageFont{FACTOR}}
\newcommand{\kQBF}{{\LanguageFont{QBF{\tiny k}}}}
\newcommand{\QBFk}{{\LanguageFont{QBF{\tiny k}}}}
\newcommand{\QBF}{{\LanguageFont{QBF}}}
\newcommand{\STCON}{\LanguageFont{STCON}}
\newcommand{\USTCON}{\LanguageFont{USTCON}}
\newcommand{\CircuitSat}{{\LanguageFont{CIRCUIT\_SAT}}}
\newcommand{\tCircuitSat}{\widetilde{{\LanguageFont{CIRCUIT\_SAT}}}}
\newcommand{\SAT}{\LanguageFont{SAT}}
\newcommand{\tSAT}{\widetilde{{\LanguageFont{SAT}}}}
\newcommand{\UNSAT}{{\LanguageFont{UNSAT}}}
\newcommand{\tThreeSAT}{\widetilde{{\LanguageFont{3\text{-}SAT}}}}
\newcommand{\ThreeSAT}{{\LanguageFont{3\text{-}SAT}}}
\newcommand{\BH}{\LanguageFont{BH}}
\newcommand{\CircuitEval}{{\LanguageFont{CIRCUIT\_EVAL}}}


\newcommand{\const}{\textmd{const}}
\newcommand{\logspace}{\textmd{logspace}}
\newcommand{\PATH}{\textmd{PATH}}


\newcommand{\readonly}{\textsf{read-only}}
\newcommand{\writeonly}{\textsf{write-only}}


\usepackage{ upgreek }
\newcommand{\PI}{\Uppi}
\newcommand{\SIGMA}{\Upsigma}
\newcommand{\DELTA}{\Updelta}

% \begin{document}
\section{Представление перестановки в виде произведения независимых циклов. Порядок перестановки. Обратная перестановка и ее циклическая запись.}
\begin{defn}[Цикл]
    Пусть  $ \{a_1, \ldots a_k\} \subset \{1, \ldots n\}$.
    {\sf Цикл} $ (a_1, \ldots , a_k)$ --- такой элемент $ c $ из  $ S_n$, что  
    \[
	c(x) =
	\begin{cases}
	    x, & x \not\in \{a_1, \ldots a_k\}\\
	    a_{i+1}, & x = a_i \wedge 1 \le i < k\\
	    a_1, & x = a_k
	\end{cases}
    .\] 
    \begin{note}
	Порядок $ (a_1, \ldots , a_k)$ равен $ k$.
    \end{note}
\end{defn}
\begin{defn}[Неподвижная точка]
    Пусть $ \sigma \in S_{n}$. {\sf Неподвижная точка} --- такой $ x \in \{1, \ldots , n\}$, что $ \sigma (x) = x$. 
    \begin{name}
	$ \Fix(\sigma ) $ --- множество всех неподвижных точек относительно $ \sigma $.  
    \end{name}
\end{defn}
\begin{defn}[Носитель]
    {\sf Носитель перестановки $ \sigma \in  S_{n} $} --- множество $ \{1, \ldots , n\} \setminus \Fix( \sigma )$.  
    \begin{name}
        $ \supp \sigma $.
    \end{name}
\end{defn}
\begin{defn}[Независимость перестановок]
    Перестановки $\sigma _1 , \sigma _2 \in S_{n} $ называются {\sf независимыми}, если $ \supp \sigma _1 \cap \supp \sigma _2 = \varnothing$.  
    \begin{prop}
        Две независимые перестановки коммутируют.
    \end{prop}
\end{defn}
\begin{thm}[Разложение в произведение циклов]
    Пусть $ \sigma \in S_{n} $. Тогда существует единственный с точностью до порядка набор независимых циклов $ c_1, \ldots , c_k, ~ c_i \ne \id$, что $ \sigma  = c_1  \ldots c_k$.
\end{thm}
\begin{myproof*}
    Рассмотрим все различные орбиты $ \Omega_1, \Omega _2, \ldots \Omega _s$.
    Определим перестановки $ c_i, ~ i \in  \{1, \ldots s\}$:
    \[
    c_i = 
    \begin{cases}
	\sigma (x),& x \in \Omega _i\\
	x,& x \not\in \Omega _i
    \end{cases}
    .\] 
    Докажем, что $ c_i$ --- независимые циклы.
    \begin{itemize}[noitemsep]
	\item $ \supp c_i \subseteq \Omega _i$, поэтому все $ c_i$ различны.
	\item Докажем, что  $ c_i = \left( x, \sigma (x), \ldots \sigma ^{l-1}(x) \right), \quad  l = \left| \Omega _i \right|, x  \in \Omega _i $. 
	    \begin{itemize}[noitemsep]
		\item $ \sigma ^{k}(x) = x, ~ k >0 \Longrightarrow \left| \Omega _i \right| \le k$
		\item $ \sigma ^{k_1}(x)  = \sigma ^{k_2}(x), ~ 0 \le k_2 < k_1 < l \Longrightarrow \sigma ^{k_2-k_1}(x) = x \Longrightarrow \left| \Omega _i \right| \le k_2-k_1 < l$. Из чего следует, что все элементы $ \Omega _i$ различны.
		\item Рассмотрим элемент $ \sigma ^{l}(x) \in \{x, \sigma (x), \ldots \sigma ^{l-1}(x)\}$. По прошлому пункту он не может совпасть ни с кем кроме $ x$.
	    \end{itemize}
	Получили, что $ c_i$ --- цикл.
    \end{itemize}
    Докажем, что $ \sigma = c_1 \ldots c_s$.
    Пусть  $ x \in  \Omega _i$, тогда  $ \sigma (x) \in  \Omega _i$.
    \[
	c_1 \ldots c_s(x) = c_1 \ldots c_{i-1}c_i(x) = c_1 \ldots c_{i-1}(\sigma (x)) = \sigma (x)
    .\] 

    Теперь докажем единственность. Пусть $ \sigma  = c_1\ldots c_k$. $ \supp c_i = \Omega _j$. Порядок следования элементов в $ c_i$ определяется действием на этой орбите, так как остальные циклы независимы и не влияют на жту орбиту.
\end{myproof*}
\begin{thm}[Порядок перестановки]
    Пусть $ \sigma  \in S_n$ и $ \sigma  = c_1\ldots c_k$. Обозначим $ d_i$ за длину  $ c_i$. Тогда  $ \ord \sigma  = НОК\left( d_1, \ldots d_k \right) $
\end{thm}
\begin{myproof*}
    Так как независимые перестановки коммутируют, $ \sigma ^{d} = \prod c_i^{d}$. Так как $ c_i^{d}$ тоже независимы, чтобы $ \sigma ^{d} = id$, нужно $ c_i^{d} = id$. То есть требуется $ d \del d_i$.
\end{myproof*}
\begin{thm}[Обратная перестановка в циклической записи]
    Пусть $ c = (a_1, \ldots a_k)$. Тогда $ c^{-1} = (a_k, \ldots a_1)$.\\ Если $ \sigma  = c_1c_2\ldots c_s$, где $ c_i$ --- независимые циклы, то  $ \sigma^{-1} = c_1^{-1}c_2^{-1}\ldots c_s^{-1}$.
\end{thm}
\begin{myproof*}
    Так как $ c_i^{-1}$ тоже независимы, они коммутируют, поэтому можем поставить в нужном порядке.
\end{myproof*}
% \end{document}
