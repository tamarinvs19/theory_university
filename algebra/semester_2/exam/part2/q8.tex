\section{Многочлен от элемента. Минимальный многочлен. Нетривиальность минимального многочлена для элемента конечномерной алгебры. Дихотомия для элементов конечномерной алгебры.}
\begin{note}
    Пусть $ K$ --- поле, $ A$ --- алгебра над  $ K$. Заметим, что для  $ y \in A$ и многочлена $ p(x) = a_0+\ldots +a_n x^{n} \in K[x]$ можно определить элемент $ p(y) = a_0+\ldots a_ny^{n} \in A$. Соответствие $ p(x) \to p(y) \in A$ определяет единственный гомоморфизм $ K$-алгебр $ \varphi \colon K[x] \to  A$, $ \varphi (x) = y$. 
\end{note}
\begin{note}
    Пусть $ a, b$ --- два элемента алгебры  $ A$, которые не коммутируют между собой. Тогда не существует гомоморфизма $ K[t_1, t_2]$, переводящего $ t_1 \to  a$, $ t_2 \to  b$. 
\end{note}
\begin{st}
    Для любого элемента $ y $ конечномерной алгебры  $ A$ существует   $ p(x) \in K[x], ~p(x) \ne 0$ такой, что $ p(y) = 0$.
\end{st}
\begin{defn}[Аннуляторы]
    Ядро гомоморфизма $ K[x] \to  A$, переводящего $ x \to  y$, является идеалом $ Ann_y \le K[x]$. Его элементы называют {\sf аннуляторами} для элемента $ y \in A$. Если этот идеал  не 0 (есть нетривиальные многочлен, аннулирующий $ y$), то образующую этого идеала (со старшим коэффициентом 1) называют  {\sf минимальным многочленом}  для элемента $ y \in A$ и обозначают $ \mu_y(x)$. 

    По другому, это многочлен минимальной степени со старшим коэффициентом, аннулирующий $ y$.
\end{defn}
\begin{thm}
    Любой элемент конечной алгебры $ A$ над полем  $ K$ либо обратим, либо делитель нуля (с любой стороны).
\end{thm}
