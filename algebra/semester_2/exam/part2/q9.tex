\section{Матрица линейного оператора. Инвариантные подпространства и как заметить по матрице линейного оператора. Примеры.}
\begin{defn}
    Две матрицы $ A, B \in M_n(K)$ {\sf   подобны}, если существует матрица $ C \in \GL_n(K)$, что $ A = CBC^{-1}$.
    \begin{note}
        Матрицы одного оператора в разных базисах подобны.
    \end{note}
\end{defn}
\begin{defn}[Инвариантное подпространство]
    Пусть $ V$ --- пространство с опрератором $ L$. Пусть  $ U \le V$. Тогда $ U$ называется  {\sf инвариантным подпространством}, если $ L(U) \le V$.  
    \begin{note}
        Это условие позволяет сузить оператор $ L$ с  $ V$ на  $ U$. Наличие инвариантных подпространств не зависит от выбора системы координат.
    \end{note}
\end{defn}
\begin{lm}
    Пусть $ U \le V$ --- подпространство, $ L\colon V \to  V$ --- линейный оператор. Тогда $ U$ инвариантно относительно $ L$  тогда и только тогда, когда в базисе $ e_1, \ldots e_k, e_{k+1}, \ldots , e_n$, где $ e_1, \ldots e_k$ --- базис $ U$, матрица оператора имеет блочно диагональный вид
     \[
    \begin{pmatrix}
	A&B\\0&C
    \end{pmatrix}
    .\] 
\end{lm}

