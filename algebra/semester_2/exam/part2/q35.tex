\section{Сорпяженное линейное отображение: существование и единственность. Свойства. Примеры.}
\begin{defn}[Спряженное отображение]
    Пусть $ L$ --- линейное отображение  $ L \colon U \to  V$ между евклидовым и унитарным пространствами.  Тогда {\sf сопряженным отображением к $ L$} называется такое линейное отображение $ L^{*}$ , то $ \langle L^{*}x, y \rangle = \langle x, Ly \rangle$ для всех $ x \in V, ~ y \in U$. 
\end{defn}
\begin{thm}
    Сопряженное линейное отображение единственно. Более того, если в $ U$ и  $ V$ выбрать ортонормированные базисы  $ u$ и  $ v$, матрица  $ L$ в этих базисах есть $ A$, то матрица сопряженного отображения будет равна  $ \overline{A}^{\top}$.
\end{thm}
\begin{cor}
    Сопряженный оператор к оператору  $ L$ существует и единственен. Более того, если задан ортонормированный базис $ e_1, \ldots e_n$ и матрица $ A$ (матрица  $ L$), то матрица  $ L^{*} = \overline{A}^{\top}$.
\end{cor}
\begin{lm}[Общие свойства]
    \begin{enumerate}[noitemsep]
	\item $ (L + T)^{* } = L^{*}+T^{*}$
	\item $ (LT)^{*} = T^{*}L^{*}$
	\item $ (\lambda L)^{*} = \overline{ \lambda }L^{*}$ 
	\item $ (L^{-1})^{*} = (L^{*})^{-1}$
	\item $ L^{**} = L$
    \end{enumerate} 
\end{lm}
\begin{defn}[Самомопряженность]
    $ L$ --- оператор на евклидовом или унитарном пространстве  $ V$ называется {\sf самосопряженным}, если $ L = L^{*}$.  
\end{defn}
