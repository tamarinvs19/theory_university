\section{Матрица Грама и невырожденность. Метод наименьших квадратов. Пример с приближением многочленом фиксированной степени. Псевдообратная матрица.}
\begin{defn}[Матрица Грама]
    Пусть $ e_1, \ldots e_k$ --- набор векторов $ V$. Тогда {\sf матрица Грамма} --- матрица
    \[
	G_{ij}(e_1, \ldots , e_k)_{ij} = \langle e_i, e_j \rangle
    .\] 
\end{defn}
\begin{lm}
    Пусть $ v_1, \ldots v_n $ --- набор векторов в $ \R^{n} $. Тогда 
    \[
	\det G(v_1, \ldots v_n) = \left( \Vol(v_1, \ldots v_n) \right) ^2
    .\] 
\end{lm}
\begin{thm}[Метод наименьших квадратов]
    $ A \in M_{m \times n} (\R)$, $ b \in \R^{m} $. Ищем $ x\colon \| Ax - b \| $ --- минимально.

    \[
    A^{\top } A x = A^{\top} b
    .\] 
    \[
	x = (A^{\top} A)^{-1} A^{\top}
    .\] 
\end{thm}
\begin{defn}[Псевдообратная матрица]
    Если $ \ker A = 0$, матрица $ (A^{\top}A)^{-1} A^{\top}$ называется {\sf псевдообратной}.   
\end{defn}
