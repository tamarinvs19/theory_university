\section{Евклидовы пространства. Неравенство Коши-Буняковского. Неравенство треугольника. Ортогональное дополнение. Примеры евклидовых пространств.}
\begin{defn}[Евклидово пространство]
Векторное пространство $ V$ над $ \R$ вместе с заданной на волнительно определенной симметричной билинейной формой  $ \langle \cdot , \cdot  \rangle$ называется {\sf евклидовым пространством}. Форма называется {\sf скалаярным произведением}.  
\end{defn}

\begin{defn}[Норма]
    Определим норму на евклидовом пространстве как $ \| v \| = \sqrt{ \langle v, v \rangle} $. Норма задает расстояние по правилу $ \rho(u, v) = \| u-v \| $.
    \begin{note}
        Это действительно норма: $ \|  u + v \| \le  \|  u \| + \| v \| $.
    \end{note}
\end{defn}
\begin{lm}[Неравенство Коши-Буняковского]
    В евклидовом пространстве выполнено неравенство $ \langle u, v \rangle \le \| u \| \cdot \| v \| $
\end{lm}

\begin{lm}
Пусть $ V$ --- евклидово пространство. Тогда для всякого подпространства $ U$ имеет место ортогональное разложение  $ V = U \oplus U^{\top}$. Если есть такое разложение, то оператор проекции на $ U$ называется {\sf ортогональной проекцией}.
\end{lm}
