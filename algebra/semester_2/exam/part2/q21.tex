\section{Граф матрицы. Неприводимые и эрогодические (примитивные) матрицы. Связь этих понятий. Теорема Фробениуса. Следствие для неприводимых матриц.}
\begin{defn}[Граф матрицы]
    Пусть $ A$ ---  неотрицательная вещественная матрица размера   $ n$. Вершинами графа этой матрицы будут числа от  $ 1$ до  $ n$, а ребро между  $ j \to  i$ есть, если коэффициент $ A_{ij} \ne 0$.
\end{defn}
\begin{defn}[Неприводимая матрица]
    Неотрицательная матрица $ A$ называется  {\sf неприводимой}, если связанный с ней граф сильно связен.  
    \begin{note}
        Это равносильно тому, что нельзя так перенумеровать координаты, чтобы в новых координатах матрица имела блочно-верхнетреугольный  вид
	$
	\begin{pmatrix}
	    B & C \\ 0 &D
	\end{pmatrix}
	$.
    \end{note}
\end{defn}
\begin{lm}
    Пусть $ A$ --- неотрицательная неприводимая матрица размера  $ n$. Тогда  $ \forall \varepsilon >0 $ матрица $ A + \varepsilon E$ эрогодическая.
\end{lm}
\begin{thm}[Фробениус, 1912]
    Пусть $ A$ --- эрогодическая матрица. Тогда у $ A$ есть единственное максимальное по модулю собственное число  $ \lambda $ и оно вещественно и положительно. Кроме того, $ \lambda $ не является кратным для $ A$, этому числу соответствует положительный собственный вектор.
\end{thm}
\begin{cor}
    Пусть $ A$ --- неприводимая матрица. Тогда у  $ A$ есть вещественное собственное число  $ \lambda  >0$, которое не меньше всех остальных собственных чисел по модулю. Оно не кратно и соответствующий собственный вектор можно выбрать положительным.
\end{cor}

