\section{Положительная определенность. Единственность канонического вида над $ \R$. Критерий Сильвестра.}
\begin{defn}[Сигнатура формы]
    {\sf Сигнатура формы} над $ \R$ --- пара чисел  $ (k, l)$ --- число плюсов и минусов в каноническом виде.
     \begin{note}
        $ k + l  = \rk q$
    \end{note}
\end{defn}

\begin{defn}[Положительная определенность]
    Квадратичная форма называется {\sf положительно определенной}, если   $ \forall v \ne 0\colon q(v) > 0$.\\
    Симметричная билинейная форма называется {\sf положительно определенной}, если форма $ q(v) = h(v, v)$ положительно определена.\\
    Симметричная матрица называется {\sf положительно определенной}, если соответствующая форма положительно определена.  
\end{defn}
\begin{thm}
    Сигнатура формы $ q$ не зависит  от способа приведения формы к каноническому виду. Точнее --- число  $ k$ равно размерности наибольшего подпространства, ограничение формы на которое положительно определено.
\end{thm}
\begin{cor}
    Пусть $ q$ --- форма на вещественном пространстве $ V$ размерности  $ n$. Тогда канонический вид  $ q$ однозначно определяется числом  $ n$ и ее сигнатурой.
\end{cor}
\begin{thm}[Критерий Сильвестра]
    Пусть $ V$ --- векторное пространство над $ \R$,  $ q$ --- квадратичная форма,  $ A$ --- ее матрица в некоторым базисе $ e_1, \ldots e_n$. Пусть главные миноры $ d_i$ матрицы  $ A$ не все равны 0. Тогда число перемен знака в последовательности $ 1 = d_0, d_1, \ldots d_n$ равно числу отрицательных квадратов в каноническом виде.
\end{thm}
