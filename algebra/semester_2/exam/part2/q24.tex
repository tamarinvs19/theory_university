\section{Понятие ортогонального базиса. Существование ортогонального базиса. Алгоритм приведения квадратичной формы к сумме квадратов.}
\begin{defn}[Ортогональная система векторов]
    Пусть $ h$ --- симметричная билинейная форма на  $ V$.  
    Система векторов называется {\sf ортогональной}, если $ \forall i \ne j\colon h(e_i, e_j) = 0$. Если $ \{e_i\}$ --- базис, то его тоже  называют {\sf ортогональным}.  
    \begin{note}
        Матрица симметричной билинейной формы в ортогональном базисе имеет диагональный вид, а выражение для квадратичной формы --- сумма квадратов координат вектора с коэффициентами.
    \end{note}
\end{defn}
\begin{defn}[Эквивалентность]
    Будем говорить, что симметрические билинейные (или квадратичные) формы {\sf эквивалентны}, если в некоторых базисах они миеют одинаковые матрицы.
\end{defn}
\begin{thm}[о существовании ортогонального базиса]
    Пусть $ V$ --- пространство с симметричной билинейной формой  $ h$. Тогда в  $ V$ существует ортогональный относительно  $h $ базис.
\end{thm}


