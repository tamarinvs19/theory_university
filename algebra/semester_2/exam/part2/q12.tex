\section{Критерий диагонализируемости. Случай отсутствия кратный собственных чисел. Последовательности, удовлетворяющие линейному рекурентному соотношению.}
\begin{thm}[Критерий диагонализируемости]
    Пусть $ K$ --- поле и все корни $ \chi_L(t)$ лежат в  $ K$. Тогда оператор  $ L$ диагонализуем  тогда и только тогда, когда для любого собственного числа алгебраическая и геометрическая кратности равны.
\end{thm}
\begin{cor}[Случай  без кратных корней]
    Пусть $ K$ --- алгебраически замкнутое поле. Если $ \chi_L(t)$ не имеет кратных корней, то оператор  $ L$ диагонализируем.
\end{cor}
\begin{cor}
    Пусть дана последовательность $ x_{n} \in \Cm$, удовлетворяющая линейному рекурентному соотношению
    \[
	x_{n+k} + a_{k-1}x_{n+k-1}+ \ldots + a_0x_n = 0
    ,\] 
    где $ a_i \in \Cm$. Рассмотрим многочлен $ f(t) = t ^{k} + a_{k-1}t^{k-1} + \ldots + a_0$. Пусть у $ f(t)$ нет кратных корней. Тогда   $ x_n = c_1 \lambda _1^{n}+ \ldots + c_k \lambda _k^{n}$, где $ \lambda  _i$ --- корни $ f(t)$.
\end{cor}

