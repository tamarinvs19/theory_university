\section{Билинейные формы. Матрица билинейной формы. Ранг и нувырожденность билинейной формулы. Ортогональное дополнение. Размерность ортогонльного дополнения. Разложение в ортогональную сумму.}
\begin{defn}[Билинейная форма]
    Пусть $ V$ ---  векторное пространство над $ K$. Отображение $ h\colon V \times V \to  K$ называется  {\sf билинейной формой}, если
    \begin{enumerate}[noitemsep]
	\item $ \forall \lambda \in K~ \forall u,v, w \in V\colon \quad h(u+ \lambda v, w) = h(u, w) + \lambda h(v, w)$,
	\item $ h(w, u+ \lambda v) = h(w, u) = \lambda h(w, v)$
    \end{enumerate}
\end{defn}
\begin{defn}[Матрица билинейной формы]
    Пусть $ e_1, \ldots e_n $ --- базис $ V$,  $ h$ --- билинейная форма на $ V$. Тогда матрица  $ A$, составленная из элементов  $ h(e_i, e_j) $ называется  {\sf матрицей билинейной формы}.  
\end{defn}
\begin{lm}
    Пусть $ V$ ---  пространство с базисом $ e_1, \ldots e_n$. Тогда имеет место взаимооднозначное соответствие между билинейными формами $ h$ на  $ V$ и матрицами $ A \in M_n(K)$. 

    В частности, если вектор $ v$ имеет столбец координат  $ x$, а вектор  $ u$ --- столбец  $ y$, то  $ h(u, v ) = y^{\top}A x$.
\end{lm}
\begin{lm}
    Пусть $ V$ --- пространство с билинейной формой  $ h$ и базисом  $ e_1, \ldots e_n$.  Пусть матрица $ h$ в этом базисе --- это $ A$. Если выбрать другой базис $ f$ с матрицей  перехода $ C$, то в новом базисе матрица  $ A$ будет иметь вид  $ A' = C^{\top}AC$.
\end{lm}
\begin{defn}[Ранг]
    {\sf Ранг билинейной формы} --- это ранг ее матрицы. 
\end{defn}
\begin{defn}
    Будем говорить, что элемент $ u$  {\sf ортогонален (слева)} элементу $ v$, если  $ h(u, v) = 0$, и записывать так  $ u \perp v$. 
\end{defn}
\begin{defn}[Невырожденность]
    Билинейная форма $ h$ называется  {\sf невырожденной}, если $ \forall v \ne 0$ существует $ u \in V\colon h( u, v ) \ne  0$.  
\end{defn}
\begin{st}
    Билинейная форма невырождена тогда и только тогда, когда ее матрица в некотором базисе невырождена.
\end{st}
\begin{defn}[Ортогональное дополнение сверху]
    Пусть $ h$ --- билинейная форма на $ V$. Если  $ U$ --- подпространство  $ V$, то  {\sf правым ортогональным дополнением к $ U$} (внутри $ v$ относительно  $ h$) будет множество
    \[
    U^{\perp} = \{v \in V \mid \forall u \in U ~ u \perp v\}
    .\] 
    \begin{note}
        Аналогично есть левое дополнение $ \:^{\perp}U$
    \end{note}
    \begin{note}
        Если $ e_1, \ldots e_k$ базис $ U$, то условие  $ v \in U^{\perp}$ равносильно $ \forall i ~e_i \perp v$.
    \end{note}
\end{defn}

\begin{st}
    Пусть $ U$ --- подпространство  $ V$,  $ h $ --- билинейная форма на $ V$. Тогда  $ \dim U ^{\perp} \ge \dim V - \dim U$. Если форма невырождена, то 
    $ \dim U^{\perp} = \dim V - \dim U$ и верно, что $ \;^{\perp} \left( U^{\perp} \right) = U$
\end{st}
\begin{st}
    Пусть $ U \le V$ и $ h$ --- билинейная форма на  $ V$. Тогда  $ V = U \oplus U ^{\perp}$ тогда и только тогда, когда $ h \mid_U$ невырождена.
\end{st}
\begin{defn}[Разложение в ортогональную прямую сумму]
    Если пространство разложилось в виде прямой суммы подпространств $ V = U \oplus U'$, таких, что  $ U' \le U^{\perp}$, то будем говорить, что имеет место {\sf разложение в ортогональную прямую сумму} подпространств $ V = U \oplus^{\perp} U'$. 
    \begin{note}
        Если $ h$ невырождена, то для данного подпространства  $ U$ может найтись не более одного пространства  $ U'$, что  $ V = U \oplus^{\perp} U'$.  А именно $ U' = U^{\perp} $
    \end{note}
\end{defn}

