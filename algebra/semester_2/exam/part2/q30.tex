\section{Примеры унитраных пространств. Неравенство Коши-Буняковского и неравенство треугольника. Разложение в ортогональную прямую сумму. Понятие угла между векторами.}
\begin{lm}
    Пусть $ u, v \in V$ --- два вектора в унитарном пространстве. Тогда $ \lvert \langle u, v \rangle \rvert \le \| u \| \| v \| $
\end{lm}
\begin{cor}
    Отображение $ \| \cdot  \| \colon V \to  \R$, заданное по правилу $ v \to  \sqrt{ \langle v, v \rangle} $ задает норму на $ V$.
\end{cor}
\begin{defn}[Угол между векторами]
    Пусть $ x, y \ne 0$ --- два вектора в $ V$. Если  $ V$ --- евклидово, то {\sf углом между $ x$ и  $ y$} называется такое число $ 0 \le \varphi \le \pi$, что 
    \[
	\cos \varphi  = \frac{\langle x, y \rangle}{\| x \| \| y \| }
    .\] 
    В случае унитарного пространства $ V$ угол  $ \varphi \in [\:0, \frac{\pi}{2}]$ и 
    \[
	\cos \varphi  = \frac{\lvert \langle x, y \rangle \rvert }{\| x \| \| y \| }
    .\] 
\end{defn}
