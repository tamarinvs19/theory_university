\section{Главные миноры. Теорема Якоби. Канонический вид квадратичной формы над $ \Cm$ и $ \R$}
\begin{defn}[Главные миноры]
    Пусть $ A$ --- матрица. Числа  $ d_i = \det A_i$, где $ A_i$ --- подматрица  $ A$, составленная из первых  $ i$ строк и столбцов, называются {\sf главными минорами}. 
    \begin{note}
        $ d_0= 1$
    \end{note}
\end{defn}
\begin{thm}[Якоби]
    Пусть $ V$ ---  векторное пространство, $ q $ --- квадратичная форма,  $ A$ --- ее матрица в некотором базисе  $ e_1, \ldots e_n$. Пусть главные миноры $ d_i \ne 0$. Тогда матрица $ A$ --- невырождена и может быть приведена к диагональному виду с числами  $ \frac{d_i}{d_{i-1}}$ на диагонали.
\end{thm}
\begin{st}
    Канонический вид, к которому можно привести квадратичную форму над $ \Cm$:
     \[
	 q(x) = x_1^2+ \ldots + x^{2}_r
    .\] 
\end{st}
\begin{st}
    Пусть $ q$ --- квадратичная форма на вещественном пространстве $ V$. Тогда  существует линейная система координат, в которой форма имеет вид
    \[
	q(x) = x_1^2+ \ldots + x_k^2 - x_{k+1}^2 - \ldots - x_{k+l}^2
    .\] 
\end{st}
