\usepackage [utf8] {inputenc}
\usepackage [T2A] {fontenc}
\usepackage[english, russian]{babel}
\usepackage {amsfonts}
% \usepackage{eufrak}
\usepackage{amssymb, amsthm}
\usepackage{amsmath}
\usepackage{mathtools}
\usepackage{needspace}
\usepackage{etoolbox}
\usepackage{lipsum}
\usepackage{comment}
\usepackage{cmap}
\usepackage[pdftex]{graphicx}
\usepackage{hyperref}
\usepackage{epstopdf}
\usepackage{enumitem}
\usepackage{cancel}

% разметка страницы и колонтитул
\usepackage[left=2cm,right=2cm,top=2cm,bottom=2cm,bindingoffset=0cm]{geometry}
\usepackage{fancybox,fancyhdr}
\fancyhf{}
\fancyhead[R]{\thepage}
\fancyhead[L]{\rightmark}
% \fancyfoot[RO,LE]{\thesection}
\fancyfoot[C]{\leftmark}
\addtolength{\headheight}{13pt}

\pagestyle{fancy}

\usepackage{import}
\usepackage{xifthen}
\usepackage{pdfpages}
\usepackage{transparent}

\newcommand{\incfig}[1]{%
    \def\svgwidth{\columnwidth}
    \import{./figures/}{#1.pdf_tex}
}

\usepackage{xifthen}
\makeatother
\def\@lecture{}%
\newcommand{\lecture}[3]{
    \ifthenelse{\isempty{#3}}{%
        \def\@lecture{Лекция #1}%
    }{%
        \def\@lecture{Лекция #1: #3}%
    }%
    \subsection*{\@lecture}
    \marginpar{\small\textsf{\mbox{#2}}}
}
\makeatletter
\usepackage[]{xcolor}

\newcommand{\Z}{\mathbb{Z}}
\newcommand{\N}{\mathbb{N}}
\newcommand{\R}{\mathbb{R}}
\newcommand{\Q}{\mathbb{Q}}
\newcommand{\K}{\mathbb{K}}
\newcommand{\Cm}{\mathbb{C}}
\newcommand{\Pm}{\mathbb{P}}
\newcommand{\ilim}{\int\limits}
\newcommand{\slim}{\sum\limits}
\newcommand{\im}{{\mathop{\text{\rm Im}}}~}
\newcommand{\re}{{\mathop{\text{\rm Re}}}~}
\newcommand{\ke}{{\mathop{\text{\rm Ker}}}~}
\newcommand{\ord}{{\mathop{\text{\rm ord}}}~}
\newcommand{\lcm}{{\mathop{\text{\rm lcm}}}~}
\newcommand{\sign}{{\mathop{\text{\rm sign}}}}
\newcommand{\Hom}{{\mathop{\text{\rm Hom}}}}
\newcommand{\Stab}{{\mathop{\text{\rm Stab}}}}
\newcommand{\Poly}{{\mathop{\text{\rm Poly}}}}
\newcommand{\osc}{{\mathop{\text{\rm osc}}}}
\newcommand{\SO}{{\mathop{\text{\rm SO}}}}
\newcommand{\M}{{\mathop{\text{\rm M}}}}
\newcommand{\pivi}{\stackrel \circ }

\renewcommand{\o}{{\mathop{\text{\rm o}}}}
\renewcommand{\O}{{\mathop{\text{\rm O}}}}
\newcommand{\grad}{{\mathop{\text{\rm grad}}}}
\renewcommand{\le}{\leqslant}
\renewcommand{\ge}{\geqslant}

\def\mydef{\mathrel{\stackrel{\rm def}=}}

\usepackage{mdframed}
\mdfsetup{skipabove=5pt,skipbelow=5pt}
\mdfdefinestyle{defstyle}{%
    linecolor=red,linewidth=3pt,topline=false,bottomline=false,%
    frametitlerule=false,%
    frametitlebackgroundcolor=gray!10,%
    innertopmargin=7pt,innerbottommargin=5pt,
}
\theoremstyle{definition}
\mdtheorem[style=defstyle]{defn}{Definition}
\newmdtheoremenv[nobreak=true,backgroundcolor=SpringGreen!20,linewidth=0pt,innertopmargin=0pt,innerbottommargin=7pt]{ex}{Example}
\theoremstyle{plain}
\newmdtheoremenv[nobreak=true,backgroundcolor=cyan!10,linewidth=0pt,innertopmargin=0pt,innerbottommargin=7pt]{thm}{Theorem}
\newmdtheoremenv[nobreak=true,backgroundcolor=Mulberry!20,linewidth=0pt,innertopmargin=0pt,innerbottommargin=7pt]{lm}{Lemma}

\theoremstyle{plain}
\newtheorem*{st}{Statement}
\newtheorem*{prop}{Property}

\theoremstyle{definition}
\newtheorem*{cor}{Corollary}
\newtheorem*{name}{Designation}

\theoremstyle{remark}
\newtheorem*{rem}{Remark}
\newtheorem*{com}{Comment}
\newtheorem*{note}{Note}
\newtheorem*{prac}{Practice}
\newtheorem*{probl}{Exercise}

\renewcommand{\proofname}{Proof}

\numberwithin{ex}{section}
\numberwithin{thm}{section}
\numberwithin{equation}{section}

