\documentclass[a_4paper,11pt]{report}
\usepackage [utf8] {inputenc}
\usepackage [T2A] {fontenc}
\usepackage {amsfonts}
\usepackage[left=20mm,top=20mm,left=20mm,bottom=20mm,nohead,nofoot]{geometry}

\newcommand{\Z}{\mathbb{Z}}
\newcommand{\N}{\mathbb{N}}
\newcommand{\R}{\mathbb{R}}
\newcommand{\Q}{\mathbb{Q}}
\newcommand{\K}{\mathbb{K}}
\newcommand{\Cm}{\mathbb{C}}
\newcommand{\Pm}{\mathbb{P}}
\newcommand{\Zero}{\mathbb{O}}
\newcommand{\ilim}{\int\limits}
\newcommand{\slim}{\sum\limits}

\theoremstyle{plain}
\newtheorem{thm}{Теорема}[section]
\newtheorem*{lm}{Лемма}
\newtheorem*{st}{Утверждение}
\newtheorem*{prop}{Свойства}

\theoremstyle{definition}
\newtheorem*{defn}{Определение}
\newtheorem*{ex}{Пример}
\newtheorem*{cor}{Следствие}
\newtheorem*{name}{Обозначение}

\theoremstyle{remark}
\newtheorem*{rem}{Ремарка}
\newtheorem*{note}{Замечание}
\newtheorem*{probl}{Упражнение}

\begin{document}
\begin{center}
\title{Линейная алгебра

Векторные пространства}
\end{center}
Х - множество\\
$ *: X \times X \to X$\\
$ (x, y) \mapsto x * y$\\
{\bf Аксиомы:}
\begin{enumerate}
    \item $\forall x,y,z \: \in X: x*(y*z) = (x*y)*z$ $\;$ (ассоциативность)
    \item $\exists e \in X \; \forall a \in X: e*a = a*e = a \;$ (нейтральный элемент)
    \item $\forall a \in X \; \exists a' \in X: a*a' = a' * a = e \;$ (обратный элемент)
    \item $\forall a, b \in X: a * b = b * a \; $(коммутативность)
\end{enumerate}

{\bf Определение 1}

Множество $X$ с операцией $*$ , удовлетворяющее  аксиоме 1, называется {\bf полугруппой}

{\bf Определение 2}

Множество $X$ с операцией $*$ , удовлетворяющее  аксиомам 1-2, называется {\bf моноидом}

{\bf Определение 3}

Множество $X$ с операцией $*$ , удовлетворяющее  аксиомам 1-3, называется {\bf группой}

{\bf Определение 4}

Множество $X$ с операцией $*$ , удовлетворяющее  аксиомам 1-4, называется {\bf коммутативной} или {\bf абелевой группой}

 Примеры:
\begin{enumerate}
    \item $(\mathbb{Z}, + )$ -- группа
    \item $(\mathbb{N}, + )$ -- полугруппа
    \item $(\mathbb N_0 , +)$ -- моноид
    \item $(\mathbb R \backslash \{0\}, \cdot)$ -- группа
    \item Пусть $A$ - множество\\
	$X$ := множество биективныx отображений $A \to A$\\
	$id_A $-- нейтральный элемент\\
	Если $f(x) = y$, то $\tilde f (y) = x$ -- обратная функция ($f \circ \tilde f = \tilde f \circ f = id_A$).\\
	$f(x) = x+1,\; g(x) - 2x,\;id_A(x)=x$\\
	$f \circ g(x) = f(g(x)) = f(2x) = 2x + 1$\\
	$g \circ f(x) = g(f(x)) = g(x+1) = 2x + 2 \ne 2x+1$\\
	Следовательно, $(X, \circ)$ -- не коммутативная группа
\end{enumerate}

{\bf Обозначения}
\begin{itemize}
    \item $\cdot$ -- мультипликативность, $1$, $x^{-1}$
    \item $+$ -- аддитивность, $0$, $-x$
    \item $\circ$ -- относительно композиции, $id$, $x^{-1}$
    \item $*$ -- абстрактная операция, $e$, $x^{-1}$
\end{itemize}

 Пусть $(R, +)$ -- абелева группа\\
Определим отображение
$$\cdot: R \times R \to R$$
$$  (a,b) \mapsto a \cdot b$$\\
Для $(R, +, \cdot)$ могут быть верны следующие аксиомы:
\begin{enumerate}
    \setcounter{enumi}{+4}
    \item $a(b+c) = ab + ac\\ 
	(b+c)a = ba + ca$ (дистрибутивность)
    \item $a(bc) = (ab)c$ (ассоциативность)
    \item $\exists 1_R \: \forall a \in R: 1_R \cdot a = a \cdot 1_R = a$ (нейтральный элемент)
    \item ab = ba (коммутативность)
    \item $0_R \ne 1_R$
    \item $\forall a \ne 0_R \: \exists a^{-1}: a \cdot a^{-1} = a^{-1} \cdot a = 1_R$ (обратный элемент)
\end{enumerate}

{\bf Определение 5}

$(R, +, \cdot)$, удовлетворяющее аксиоме 5, называется {\bf не ассоциативным кольцом без единицы}.

{\bf Определение 6}

$(R, +, \cdot)$, удовлетворяющее аксиомам 5-6, называется {\bf ассоциативным кольцом без единицы}.

{\bf Определение 7}

$(R, +, \cdot)$, удовлетворяющее аксиоме 5-7, называется {\bf ассоциативным кольцом с единицей}.

{\bf Определение 8}

$(R, +, \cdot)$, удовлетворяющее аксиомам 5-8, называется {\bf коммутативным кольцом}.

 Примеры:
\begin{enumerate}
    \item $\mathbb Z$ --коммутативное кольцо
    \item $\mathbb {Q, R, C}$ -- поля
    \item Рассмотрим $\mathbb Z_n = {0, \ldots, n-1}$ с операциями $+_n, \cdot_n$ :\\
	$a +_n b = (a + b) \% n \\
	a \cdot_n b = (a \cdot b) \% n$\\
	Обратимые элементы:\\
	$ax = 1 + ny \\
	ax - ny = 1$\\
	Если $(a, n) = 1$, есть решение, иначе -- нет.
	$\mathbb Z_p $-- поле $\Leftrightarrow$ $ p \in \mathbb P$
\end{enumerate}

{\bf Определение 9}

$V$ -- векторное пространство над полем $F$ , если $(V, +)$ -- абелева группа, задано отображение $V\times F \to V \\ (x, \alpha) \mapsto x \cdot \alpha $ , удовлетворяющее аксиомам $\forall x, y \in V, \forall a, b \in F$:
\begin{enumerate}
    \setcounter{enumi}{+4}
    \item $x \cdot (\alpha \cdot \beta) = (x \cdot \alpha) \cdot \beta$
    \item$ (x + y) \cdot \alpha = x \cdot \alpha + y \cdot \alpha$\\
     $x \cdot (\alpha + \beta) = x \cdot \alpha + x \cdot \beta$
    \item $x \cdot 1_F = x$
\end{enumerate}

{Примеры:}
\begin{enumerate}
    \item Множество векторов в $\mathbb R ^3$
    \item $F^n = \left\{ \left( 
	\begin{array}{c} a_1 \\ a_2 \\ \vdots\\ a_n \end{array} \right) | a_i \in F \right\}$\\
	$\left(\begin{array}{c}
	    a_1 \\ \vdots \\ a_n 
	\end{array} \right) +
	\left( \begin{array}{c}
		b_1 \\ \vdots \\ b_n
	\end{array} \right)=
	\left( \begin{array}{c}
		a_1 + b_1 \\ \vdots \\ a_n + b_n
	\end{array} \right)$
\end{enumerate}
    
$A \in M_n(F), \alpha \in F \\ (A, \alpha)_{ij} = a_{ij} \cdot \alpha \\ (AB)\alpha = A(B\alpha)$

{\bf Определение 1}\\
$(G, *), (H, \#) $-- группа \\
$\varphi: G \to H $- гомоморфизм, если: 
$$\varphi (g_1 * g_2) = \varphi(g_1) \# \varphi(g_2)$$

{\bf Определение 2}\\
$R, S$ -кольца\\
$\varphi: R\to S$ - гомоморфизм, если:
$$\varphi(r_1 + r_2) = \varphi(r_1) + \varphi(r_2)$$ 
$$\varphi(r_1 \cdot r_2) = \varphi(r_1) \cdot \varphi(r_2)$$
Для колец с 1:$\varphi(1) = 1$

{\bf Определение 3}\\
$U, V$ - векторные пространства над $F$\\
$\varphi: U \to V$ - линейное отображение, если:
$$\varphi(u_1 + u_2) = \varphi(u_1) + \varphi(u_2)$$
$$\varphi(u \alpha) = \varphi(u) \alpha$$

{\bf Замечание}\\
Изоморфизм -- биективный гомоморфизм.

{\bf Определение 4}\\
$V$ - векторное пространство над полем $F$\\
$v$ - строка элементов "длины" $I$ над $V$\\
$a$ - столбец "высоты" $I$, почти все элементы которого равны 0\\
Тогда $va$ - линейная комбинация набора $v$ с коэффициентами $а$.\\


{\bf Определение 5}\\
$U anal \in V$\\
$U$ является векторным пространством относительно тех же операций, которые заданы в $V$.
Тогда $U$ - подпространство $V$\\

{\bf Лемма}\\
$U \subseteq V$\\
$\forall u_1, u_2 \in U, \alpha \in F:\\
u_1+u_2 \in U, u_1 \alpha \in U$
Тогда $U$ - подпространство.
Если $U$ - подпространство в $V$, то пишут $U \subseteq V$.\\


{\bf Определение 6}\\
$v = \{v_i | i \in I\}$, где $v_i \in V \: \forall i \in I$\\
$<v>$ - наименьшее подпространство, содержащее все $v_i$

{\bf Лемма}\\
$<v> = \{va | a - \mbox{ столбец высоты } I \mbox{ над } F \mbox{, где почти всюду элементы равны нулю }\} = U$\\

{\bf Доказательство}\\
$v_i \in <v> \Rightarrow v_i a_i \in <v>\\
\Rightarrow v_{i_1} a_{i_1}a+ ... + v_{i_k} a_{i_k} \in <v>\\
\Rightarrow <v> $ содержит все варианты комбинаций.
$va + vb = v(a+b) \in U \\
(va)\alpha = v(a\alpha )\in U\\
\Rightarrow$ множество линейных комбинаций  -- подпространство 
$U$ - подпространство, содержащее $v_i \forall i \in I$\\
$<v> $a -- наименьшее подпространство, содержащее $v_i$\\
$\Rightarrow <v> \subseteq U$
тогда $<v> = U$
\\

{\bf Определение 7}\\
Если $<v> = V$, то $v$ -- система образующих пространство $V$\\
Базис -- система образующих.\\

$F^I$ -- множество функций из $I$ в $F$ = множество столбцов высоты $I$\\
$\sideset{^I}{}V $--  множество строк длины $I$

Набор элементов из $V$ ,  заиндексирванных множеством $I$ -- это функция $f: I \to V \\ i \mapsto f_c$

{\bf Определение 8}\\
$v \in \sideset{^I}{}V$\\
$v$ -- {\bf линейно независим}, если $\forall a \in F^I, a \neq 0 \Rightarrow v a  \neq  0$\\

{\bf Теорема}\\
$v \subseteq V $(можно считать, что $v$ - строка длины $v$\\
Следующие утверждения эквивалентны:\\
\begin{enumerate}
    \item $v$ - линейно независимая система образующих\\
    \item $v$ - максимальная линейно-независимая система\\
    \item $v$ - минимальная система образующих\\
    \item $\forall x \in V \exists! a \in F^v : x = v a = \sum\limits_{t \in v} t \cdot a_t  \;$ (почти все элементы равны 0)\\
\end{enumerate}
{\bf Доказательство}\\
$(1) \Rightarrow (4) $ -- доказали ранее
$(1) \Rightarrow (2) $\\
$x \in V \setminus  v \\ x = v a (a \in F^v)$\\
$v a = x \cdot 1 = 0$ -- линейная зависимость набора $v \cup {x}$\\
Т.о. любой набор , строго содержащий $v$, линейно зависим $\Rightarrow v$ -- максимальный.
\\
$(1)\Rightarrow(2) $\\
$x \in V \setminus $\\
$v \subseteq V \cup {x} $--линейно зависим\\
$va + x a_x = 0 \\ a \ne 0$\\
Если $a_x = 0 \Rightarrow va = 0 \Rightarrow a = 0 \; ?!$\\
Значит $a_x \ne 0 \\ va = c\cdot (-a_x)$\\
$x = v \cdot \frac{a}{-a_x} \Rightarrow v$ --система образующих.\\

{\bf Лемма Цорна}\\
Пусть $\mathbb A $ -- набор подмножеств (не всех) множества $X$. \\
Если объединение любой цепи из $\mathbb A$ , принадлежащей $\mathbb A$, то в $\mathbb A$ существует максимальный элемент.\\
$M \in \mathbb C$ - максимальная, если $M \subseteq M' \subseteq \mathbb A \Rightarrow M =M'$\\

{\bf Теорема (о существовании базиса)}\\
$V $ -- векторное пространства \\
$X$  -- линейное независимое подмножество $V$\\
$Y$ -- система образующих $V$\\
$X \subseteq Y$\\
Тогда существует базис $Z$ пространства $V: X \subseteq Z \subseteq Y$

{\bf Доказательство}\\
$\mathbb A - $множество всех линейно независимых подмножеств, лежащих между $X$ и $Y$. $X \in \mathbb A$\\
$\mathbb C \subseteq \mathbb A $\\
$X \subseteq \cup {C \in \mathbb C} \subseteq Y$\\
Пусть $\cup {C \in \mathbb C}$ --  линейно зависимый. То есть$ \exists u_1, ...,  u_2 \in /...$

$\ldots$

Пусть $v$ - базис $V$.\\
$$\forall x \in V \: \exists! x_v \in F^v : x = v \cdot x_v$$
$$v = (v_1, \ldots , v_n), \; x_v = матрица столцов альфа и;$$\\
$$x = v_1 \alpha_1 + \ldots = v\cdot x_v$$
\end{document}
