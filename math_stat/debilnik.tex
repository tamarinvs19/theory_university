\documentclass[11pt]{book}
\usepackage[english, russian]{babel}
\usepackage{xltxtra}
\usepackage{polyglossia}

\usepackage{mathpazo}

\defaultfontfeatures{Ligatures=TeX,Mapping=tex-text}

\setmainfont{STIX2Text-Regular.otf}[
ExternalLocation={/home/vyacheslav/builds/STIXv2.0.2/OTF/},
BoldFont=STIX2Text-Bold.otf,
ItalicFont=STIX2Text-Italic.otf,
BoldItalicFont=STIX2Text-BoldItalic.otf
]
\setmathrm{STIX2Math.otf}[
ExternalLocation={/home/vyacheslav/builds/STIXv2.0.2/OTF/}
]


\usepackage{makeidx}
\usepackage{amssymb, amsthm}
\usepackage{amsmath}
\usepackage{mathtools}
\usepackage{needspace}
\usepackage{enumitem}
\usepackage{cancel}
\usepackage{fdsymbol}
\usepackage{fontawesome}


% разметка страницы и колонтитул
\usepackage[left=2cm,right=2cm,top=1cm,bottom=1.1cm,bindingoffset=0cm]{geometry}
\usepackage{fancybox,fancyhdr}
\fancyhf{}
\fancyhead[R]{\thepage}
\fancyhead[L]{\rightmark}
\fancyfoot{}
\fancyhfoffset{0pt}
\addtolength{\headheight}{13pt}
\pagestyle{fancy}

% Отступы
\setlength{\parindent}{3ex}
\setlength{\parskip}{3pt}

\usepackage{graphicx}
\usepackage{hyperref}

\usepackage{import}
\usepackage{xifthen}
\usepackage{pdfpages}

\newcommand{\incfig}[1]{%
    \def\svgwidth{\columnwidth}
    \import{./figures/}{#1.pdf_tex}
}


\usepackage{xifthen}
\makeatother
\def\@lecture{}%
\newcommand{\lecture}[3]{
    \ifthenelse{\isempty{#3}}{%
        \def\@lecture{Лекция #1}%
    }{%
        \def\@lecture{Лекция #1: #3}%
    }%
    \subsection*{\@lecture}
    \marginpar{\small\textsf{\mbox{#2}}}
}
\makeatletter


\usepackage{xcolor}
\definecolor{Aquamarine}{cmyk}{50, 0, 17, 100}
\definecolor{ForestGreen}{cmyk}{76, 0, 76, 45}
\definecolor{Pink}{cmyk}{0, 100, 0, 0}
\definecolor{Cyan}{cmyk}{56, 0, 0, 100}
\definecolor{Gray}{gray}{0.3}


\usepackage{mdframed}
\mdfsetup{skipabove=3pt,skipbelow=3pt}
\mdfdefinestyle{defstyle}{%
    linecolor=red,
	linewidth=3pt,rightline=false,topline=false,bottomline=false,%
    frametitlerule=false,%
    frametitlebackgroundcolor=red!0,%
    innertopmargin=4pt,innerbottommargin=4pt,innerleftmargin=7pt
    frametitlebelowskip=1pt,
    frametitleaboveskip=3pt,
}
\mdfdefinestyle{thmstyle}{%
    linecolor=cyan!100,
	linewidth=2pt,topline=false,bottomline=false,%
    frametitlerule=false,%
    frametitlebackgroundcolor=cyan!20,%
    innertopmargin=4pt,innerbottommargin=4pt,
    frametitlebelowskip=1pt,
    frametitleaboveskip=3pt,
}
\theoremstyle{definition}
\mdtheorem[style=defstyle]{defn}{Определение}

\newmdtheoremenv[nobreak=true,backgroundcolor=Aquamarine!10,linewidth=0pt,innertopmargin=0pt,innerbottommargin=7pt]{cor}{Следствие}
\newmdtheoremenv[nobreak=true,backgroundcolor=CarnationPink!20,linewidth=0pt,innertopmargin=0pt,innerbottommargin=7pt]{desc}{Описание}
\newmdtheoremenv[nobreak=true,backgroundcolor=Gray!10,linewidth=0pt,innertopmargin=0pt,innerbottommargin=7pt,font={\small}]{ex}{Пример}
\newmdtheoremenv[nobreak=false,backgroundcolor=Cyan!10,linewidth=0pt,innertopmargin=0pt,innerbottommargin=7pt]{thm}{Теорема}
\newmdtheoremenv[nobreak=true,backgroundcolor=Pink!10,linewidth=0pt,innertopmargin=0pt,innerbottommargin=7pt]{lm}{Лемма}

\newtheorem*{st}{Утверждение}
\newtheorem*{prop}{Свойства}

\theoremstyle{plain}
\newtheorem*{name}{Обозначение}

\theoremstyle{remark}
\newtheorem*{rem}{Ремарка}
\newtheorem*{com}{Комментарий}
\newtheorem*{note}{Замечание}
\newtheorem*{prac}{Упражнение}
\newtheorem*{probl}{Задача}


\renewcommand{\proofname}{Доказательство}
\renewenvironment{proof}
{ \hspace{\stretch{1}}\\ \faSquareO\quad \small  }
{ \hspace{\stretch{1}}  \faSquare \normalsize }


\numberwithin{ex}{section}
\numberwithin{thm}{section}
\numberwithin{equation}{section}



\newcommand{\K}{\mathcal{K}}
\newcommand{\Z}{\mathbb{Z}}
\newcommand{\N}{\mathbb{N}}
\newcommand{\Real}{\mathbb{R}}
\newcommand{\Q}{\mathbb{Q}}
\newcommand{\Cm}{\mathbb{C}}
\newcommand{\Pm}{\mathbb{P}}
\newcommand{\ord}{\operatorname{ord}}
\newcommand{\lcm}{\operatorname{lcm}}
\newcommand{\sign}{\operatorname{sign}}
\newcommand{\E}{\mathbb{E}}

\renewcommand{\o}{o}
\renewcommand{\O}{\mathcal{O}}
\renewcommand{\le}{\leqslant}
\renewcommand{\ge}{\geqslant}

\def\mybf#1{\textbf{#1}}
\def\selectedFont#1{\textbf{#1}}
\def\ComplexityFont#1{\textmd{\textbf{\textsf{#1}}}}
\def\LanguageFont#1{{\textbf{\texttt{#1}}}}


\newcommand{\Cclass}{\mathcal{C}}
\newcommand{\Dclass}{\mathcal{D}}


\renewcommand{\P}{\ComplexityFont{P}}
\newcommand{\DTIME}{\ComplexityFont{DTime}}
\newcommand{\DTime}{\ComplexityFont{DTime}}
\newcommand{\DSpace}{\ComplexityFont{DSpace}}
\newcommand{\PSPACE}{\ComplexityFont{PSPACE}}
\newcommand{\NTIME}{\ComplexityFont{NTime}}
\newcommand{\NSpace}{\ComplexityFont{NSpace}}
\newcommand{\coNSpace}{\ComplexityFont{coNSpace}}
\newcommand{\NPSPACE}{\ComplexityFont{NPSPACE}}
\newcommand{\poly}{\ComplexityFont{poly}}
\newcommand{\RP}{\ComplexityFont{RP}}
\newcommand{\coRP}{\ComplexityFont{co-RP}}
\newcommand{\ZPP}{\ComplexityFont{ZPP}}
\newcommand{\BPP}{\ComplexityFont{BPP}}
\newcommand{\BQP}{\ComplexityFont{BQP}}
\newcommand{\coBPP}{\ComplexityFont{co-BPP}}
\newcommand{\NP}{\ComplexityFont{NP}}
\newcommand{\NL}{\ComplexityFont{NL}}
\newcommand{\coNL}{\ComplexityFont{co-NL}}
\renewcommand{\L}{\ComplexityFont{L}}
\newcommand{\NPcomp}{\ComplexityFont{NP-complete}}
\newcommand{\tP}{\widetilde{\P}}
\newcommand{\tNP}{\widetilde{\NP}}
\newcommand{\tBH}{\widetilde{\BH}}
\newcommand{\Class}{{\ComplexityFont{C}}}
\newcommand{\coC}{\ComplexityFont{co-}\mathcal{C}}
\newcommand{\coNP}{\ComplexityFont{co-NP}}
\newcommand{\PH}{\ComplexityFont{PH}}
\newcommand{\EXP}{\ComplexityFont{EXP}}
\newcommand{\Size}{\ComplexityFont{Size}}
\newcommand{\Ppoly}{\ComplexityFont{P}/\ComplexityFont{poly}}
\newcommand{\NC}{\ComplexityFont{NC}}


\newcommand{\FACTOR}{\LanguageFont{FACTOR}}
\newcommand{\kQBF}{{\LanguageFont{QBF{\tiny k}}}}
\newcommand{\QBFk}{{\LanguageFont{QBF{\tiny k}}}}
\newcommand{\QBF}{{\LanguageFont{QBF}}}
\newcommand{\STCON}{\LanguageFont{STCON}}
\newcommand{\USTCON}{\LanguageFont{USTCON}}
\newcommand{\CircuitSat}{{\LanguageFont{CIRCUIT\_SAT}}}
\newcommand{\tCircuitSat}{\widetilde{{\LanguageFont{CIRCUIT\_SAT}}}}
\newcommand{\SAT}{\LanguageFont{SAT}}
\newcommand{\tSAT}{\widetilde{{\LanguageFont{SAT}}}}
\newcommand{\UNSAT}{{\LanguageFont{UNSAT}}}
\newcommand{\tThreeSAT}{\widetilde{{\LanguageFont{3\text{-}SAT}}}}
\newcommand{\ThreeSAT}{{\LanguageFont{3\text{-}SAT}}}
\newcommand{\BH}{\LanguageFont{BH}}
\newcommand{\CircuitEval}{{\LanguageFont{CIRCUIT\_EVAL}}}


\newcommand{\const}{\textmd{const}}
\newcommand{\logspace}{\textmd{logspace}}
\newcommand{\PATH}{\textmd{PATH}}


\newcommand{\readonly}{\textsf{read-only}}
\newcommand{\writeonly}{\textsf{write-only}}


\usepackage{ upgreek }
\newcommand{\PI}{\Uppi}
\newcommand{\SIGMA}{\Upsigma}
\newcommand{\DELTA}{\Updelta}


\title{Дебильник}
\date{\today}
\author{Вячеслав Тамарин}

\begin{document}
\chapter{Дебильник}
\section{Многомерное нормальное распределение}
\begin{definition}
	\selectedFont{Стандартный гауссовский вектор} --- случайный $n$-мерный вектор $ Z = (Z_1, Z_2, \ldots Z_{n})$, координаты которого независимы и имеют распределение $ \N(0, 1)$.
\end{definition}

\begin{definition}[]
	\selectedFont{Гауссовский вектор} (\selectedFont{Нормальный вектор}) --- вектор, для которого существует матрица $ {\bf A} \in \R^{n \times m} $, стандартный гауссовский вектор $ Z \in \R^{m}$, и вектор $ b \in \R^{n}$ такие, что $ X = {\bf A}Z + b $.
\end{definition}

\begin{definition}[]
	\selectedFont{Распределение нормального вектора} $ X \in \R^{n}$ --- $ \N(\mu, {\bf \Sigma}) $ или $ \N_n(\mu, {\bf \Sigma}) $, где $ \mu = \E X$ и $ {\bf \Sigma} = \cov(X) $.
\end{definition}

\begin{definition}[]
	\selectedFont{Распределение хи-квадрат} с $ n$ степенями свободы --- распределение $ \chi^2(n) $ величины $ \chi^2 = Z_1^2 + Z_2^2 + \ldots + Z_{n}^2 $, где $ Z_{1}, Z_2, \ldots Z_{n}$ --- независимы $ \N(0, 1)$ величины. 
\end{definition}

\begin{definition}[]
	\selectedFont{Распределение Стьюдента} с $n$ степенями свободы --- распределение $T(n)$ величины $\frac{\sqrt{n}X}{\sqrt{Y}}$, где $X \sim \N(0, 1)$, $Y \sim \chi^2(n)$ и независимы.
\end{definition}

\begin{definition}[]
	\selectedFont{Распределение Фишера} со степенями свободы $n$ и $m$ --- распределение $F(n, m)$ величины $\frac{X / n}{Y / m}$, где $X \sim \chi^2(n)$, $Y \sim \chi^2(m)$ и независимы.
\end{definition}

\section{Условное матожидание}
\begin{definition}[]
	\selectedFont{Условное матожидание} $\E (Y \mid X)$ случайной величины $Y$ при условии случайной величины $X$ --- такая измеримая функция $g_0$ величины $X$, при которой $\E(Y - g(X))^2$ минимально для всех измеримых функций $g$.
\end{definition}
Условное матожидание --- ортогональная проекция $Y$ на линейное пространство всех измеримых функций $X$. То есть УМО --- единственная измеримая функция, которая удовлетворяет условию ортогональности:
 \[
\forall g \colon\E(Y - \E(Y \mid X)) g(X) = 0
.\] 

\section{Статистическая модель, выборка}
\begin{definition}[]
	\selectedFont{Статистическая модель} --- множество распределений $\PP$, которое, по нашему мнению, адекватно приближает $\P_D$. 
\end{definition}
\begin{definition}[]
	\selectedFont{Данные} $d$ --- реализация случайного элемента $D$, имеющего распределение $\P_D$.
\end{definition}
Статистические модели делят на:
\begin{itemize}
	\item \selectedFont{параметрические}, если $\PP = \{\P_0 \mid \theta \in \Theta \subset \R^{k}\}$.

		Пример: $\PP = \{\N(\mu, \sigma^2) \mid \mu \in \R, \sigma^2 \ge 0\}$.
	\item \selectedFont{непараметрические}, если $\PP = \{\P_0 \mid \theta \in  \Theta \subset V\}$, где $V$ не обязательно конечномерное.

		Пример: $\PP = \{\P^{\otimes n} \mid \int_{\mathfrak{X}}^{} x \P(dx) = 0 \}$
	\item \selectedFont{семипараметрические}, если $\PP = \{\P_0 \mid \theta \in \Theta \subset \R^{k} \times V\}$.

		Пример: линейная регрессия $Y = X \beta + \varepsilon$, $\beta \in \R^{k}$, $\E \varepsilon = 0$, $\D \varepsilon = \sigma^2$.
\end{itemize}
Если $D = [X_1, \ldots X_n]$ и $X_i$ независимы и имеют одинаковое распределение $\P_{X}$, $D$ называется \selectedFont{выборкой объема} $n$ и обозначается $X_{[n]}$, $\P_{X}$ --- генеральная совокупность.
В этом случае модель приобретает вид
$ \PP = \{\P^{\otimes n} \mid \P \in \PP_X\} $, где $\PP_X$ --- модель для $\P_X$.

\section{Формула Байеса, априорное, апостериорное распределение}
\begin{itemize}
	\item \selectedFont{Априорное распределение} --- наше ощущение относительно значения параметра до проведения эксперимента.
	\item \selectedFont{Апостериорное распределение} --- ощущение после получения данных эксперимента.
\end{itemize}
\begin{definition}[Формула Байеса]
	Здесь $p$ --- вероятность, $d$ --- данные, $\theta$ --- параметры.
	\[
	p(\theta \mid d) = \frac{p(d \mid \theta) \cdot p(\theta)}{p(d)}
	.\] 
	\begin{itemize}
		\item $p(\theta \mid d)$ --- апостериорное распределение,
		\item $p(d \mid \theta)$ --- правдоподобие,
		\item $p(\theta)$ --- априорное распределение,
		\item $p(d)$ --- вероятность данных.
	\end{itemize}
\end{definition}

\section{Расстояние Кульбака-Лейблера, энтропия}
Пусть мы принимаем случайные символы $x_1, \ldots x_{k}$, вероятность появления $x_{i}$ равна $p_i$, записываем с помощью битовой строки длины $l_i$. Тогда средняя длина символа равна 
\[
l = \sum_{i=1}^{k} p_i \cdot l_i
.\] 
Чтобы минимизировать $l$, необходимо подобрать следующие $l_i = - \log_2{p_i}$. И тогда средняя длина будет равна $H(x) \coloneqq - \sum_{i=1}^{k} p_i \cdot  \log_2{p_i}$, эта величина называется \selectedFont{двоичной энтропией сообщения}. Аналогично можно брать любой другой логарифм, мы будем использовать натуральный.

Для непрерывной величины можно завести \selectedFont{дифференциальную энтропию}:
\[
H(X) = - \int_{}^{} p(x) \log{p(x)} dx 
.\] 
Пусть случайная величина $X$ имеет функцию вероятности $p$, но мы кодируем символы, как-будто она имеет функцию вероятности $q$. Тогда средняя длина сообщения будет равна $- \sum_{i=1}^{k} p_i \cdot \log{q_i}$, эта величина называется \selectedFont{кросс-энтропией} $H(p \mid q)$ распределений $p$ и $q$.

$H(p \mid q)$ всегда будет больше $H(p)$, так как $H(p)$ минимально.
\begin{definition}[]
	Величина потери информации из-за использования $q$ вместо $p$ называется \selectedFont{расстоянием Кульбака-Лейблера} между $p$ и $q$:
	\[
	D_{KL} (p, q) = H(p \mid q) - H(p) = - \sum_{i=1}^{k} p_i \cdot \log{\frac{q_i}{p_i}}
	.\] 
\end{definition}
Для непрерывных величин все обобщается следующим образом 
\[
D_{KL} = - \int p_i \cdot \log{\frac{q_{i}}{p_i}}
.\] 

\section{Статистика...}
\subsection{Статистика}
Параметр или характеристика распределения --- функционал от этого распределения.
\begin{definition}[]
	\selectedFont{Статистика} ---  функция $\theta^*$ от данных $d$.
\end{definition}

Пусть модель $\PP_{[n]} = \{\P ^{\otimes n} \mid \P \in \PP\}$, искомая характеристика $\theta\colon \PP \to \R^{k}$.

\subsection{Несмещенность}
Чему равна оценка как случайная величина в среднем, если она равна характеристике?

\begin{definition}[]
	Оценка $\Theta^*$ называется
	\begin{itemize}
		\item \selectedFont{несмещенной}, если $\forall \P \in \PP\colon \E \theta^*(X_{[n]}) = \theta(\P)$, где $X_{[n]} \sim \P^{\otimes n}$,
		\item \selectedFont{асимптотически несмещенной}, если $\forall \P \in \PP\colon \E \theta^*(X_{[n]}) \to \theta(\P)$.
	\end{itemize}

	\selectedFont{Смещение} --- величина $b(\theta^*) = \E(\theta^*(X_{[n]})) - \theta(\P)$.

	\selectedFont{Среднеквадратичная ошибка} --- величина $\MSE(\theta^*) = \E \left( \theta^*(X_{[n]}) - \theta(\P) \right) ^2$.
\end{definition}
В общем случае 
\[
\MSE(\theta^*) = \D \theta^*(X_{[n]} + b^2(\theta^*)
.\] 
\begin{itemize}
	\item Выборочное среднее как оценка матожидания --- несмещенная оценка,
	\item Выборочная дисперсия как оценка дисперсии --- асимптотически несмещенная,
	\item Исправленная выборочная дисперсия как оценка дисперсии --- несмещенная оценка.
\end{itemize}

\subsection{Состоятельность}
\begin{definition}[]
	Оценка $\theta^*$ называется
	\begin{itemize}
		\item \selectedFont{состоятельной}, если $\forall \P \in \PP\colon \theta^*(X_{[n]}) \xrightarrow{\mathbb{P}} \theta(\P)$, где $X_{[n]} \sim \P^{\otimes n}$,
		\item \selectedFont{сильно состоятельной}, если $\theta^*(X_{[n]}) \xrightarrow{\text{п. н.}} \theta(\P)$.
	\end{itemize}
\end{definition}

\subsection{Асимптотическая нормальность}
\begin{definition}[]
	Оценка $\theta^*$ называется \selectedFont{асимптотически нормальной} с коэффициентом рассеивания (или просто дисперсией) $\sigma^2\bigl(\theta(\P)\bigr > 0$, если 
		\[
		\sqrt{n} \left( \theta^*(X_{[n]}) - \theta(\P) \right) \xrightarrow{d} \eta \sim \N(0, \sigma^2(\theta^*(\P)))
		.\] 
	В многомерном случае рассматривается ковариационная матрица вместо дисперсии.
\end{definition}
\begin{itemize}
	\item Выборочная дисперсия и второй момент --- асимптотически нормальная оценка.
	\item Из асимптотической нормальности следует состоятельность.
\end{itemize}

\subsection{Эффективность}
Рассмотрим класс оценок $K = \{\hat{\theta}\}$ параметра $\theta$.
\begin{definition}[]
	Оценка $\theta^* \in K$ называется \selectedFont{эффективной в классе} $K$, если \it{для любой другой оценки} $\hat{\theta} \in  K$ и \it{для любого исследуемого параметра} $\theta \in \Theta$ выполняется
	\[
	\MSE_{\theta}(\theta^*) \le \MSE_{\theta}(\hat{\theta})
	.\] 
\end{definition}
Класс несмещенных оценок
 \[
K_0 = \{\hat{\theta} \mid \E {\hat{\theta}} = \theta, \forall \theta \in \Theta\}
.\] 
\begin{definition}[]
		\selectedFont{Эффективная оценка} $\theta^*$, если эффективна в классе $K_0$.
\end{definition}
\begin{definition}[]
		\selectedFont{Асимптотически эффективной в классе} $K$, если для любой оценки $\hat{\theta} \in K$ и для любого $\theta \in  \Theta$ выполняется
			\[
			\overline{\lim_{n \to \infty}} \frac{\MSE(\theta^*)}{\MSE(\hat{\theta}}
			.\] 
\end{definition}

\subsection{Робастность}
\begin{definition}[]
	\selectedFont{Робастность} --- свойство оценки быть устойчивой к хвостам распределения.
\end{definition}
	Пусть $F$ --- распределение,  $\{G_{n}\}$ --- последовательность распределений, что
	\[
	\lvert F - G_{n}\rvert \coloneqq \underset{x}{\sup} \lvert F(x) - G_n(x) \rvert \to 0
	.\] 
\begin{definition}[]
	Характеристика $\theta$ обладает \selectedFont{качественной робастностью}, если $\theta(G_{n}) \to \theta(F)$
\end{definition}
Пусть также $\delta_x$ --- вырожденное распределение в точке $x$.
\begin{definition}[]
	\selectedFont{Загрязненное распределение} --- смесь $F_{x, \varepsilon} = (1-\varepsilon) F + \varepsilon \delta_x$.	
\end{definition}
\begin{definition}[]
	\selectedFont{Функция влияния} характеристики $\theta$ --- величина $$IF(x) = \lim_{\varepsilon \to 0+} \frac{\theta(F_{x, \varepsilon}) - \theta(F)}{\varepsilon}.$$
\end{definition}
\begin{definition}[]
	Характеристика $\theta$ называется \selectedFont{$B$-робастной} или \selectedFont{инфинитезимально робастной}, если $IF(x)$ ограничена.
\end{definition}
\begin{definition}[]
	\selectedFont{Асимптотическая толерантность} характеристики $\theta$ ---
	\[
	\tau = \inf \bigl\{ \varepsilon \mid \underset{x}{\sup}\lvert \theta(F_{x, \varepsilon} - \theta(F) \rvert = \infty \bigr\}	
	.\] 
\end{definition}

\subsection{Достаточность}
\begin{definition}[]
	Статистика $T(x) = \{T_1(x), \ldots , T_m(x))\}$ называется \selectedFont{достаточной}, если для всех 
	\begin{itemize}[noitemsep]
		\item $\theta \in \Theta$,
		\item $B \in \PP(\R^{n})$ и 
		\item $t = (t_1, \ldots , t_m)$
	\end{itemize}
	 условная вероятность $\mathbb{P}(X_{[n]} \in B \mid T(X_{[n]}) = t)$ не зависит от $\theta$.

	 То есть информация о $\theta$ в выборке полностью содержится в значении $T(x_{[n]})$.
\end{definition}
\begin{theorem}[факторизации]
	$T(x)$ достаточна, согда существуют функции $g$ и  $h$, что
	\[
	p(X_{[n]} = x_{[n]} \mid \theta) = g(T(x_{[n]}), \theta) h(x_{[n]})
	,\] 
	где $p$ --- вероятность или плотность.
\end{theorem}

\subsection{Полнота}
\begin{definition}[]
	Статистика $T$ называется \selectedFont{полной}, если для любой \it{измеримой} $g$ верно следствие
	\[
	 \forall \theta \in  \Theta\colon\E {g (T(X_{[n]}))} \equiv 0 \quad \implies \quad g(T(X_{[n]})) \overset{\text{п.н.}}{=} 0
	.\] 
\end{definition}


\section{Теоремы Колмогорова-Блэкуэлла-Рао и Лемана-Шеффе}
\begin{theorem}[Колмогорова-Блэкуэлла-Рао]
    Пусть $\theta^*$ --- оценка параметра $\theta$, $T$ --- достаточная статистика. Тогда \[
    \MSE(\theta^*) \ge \MSE(\E(\theta^* \mid T))
    .\] 
\end{theorem}
\begin{theorem}[Лемана-Шеффе]
    Пусть $\theta^*$ --- оценка параметра $\theta$, $T$ --- достаточная и полная статистика. Тогда $\E(\theta^* \mid T)$ --- единственная эффективная оценка в классе оценок со смещением $b(\theta^*)$.
\end{theorem}

\section{Доверительный интервал}
Пусть есть модель $\PP_{[n]} = \{\P^{\otimes n} \mid \P \in \PP\}$ и $\theta\colon \PP \to \R^{k}$ --- искомая характеристика.

\begin{definition}[]
	\selectedFont{Доверительный интервал} (точный доверительный интервал) с уровнем доверия $\gamma$ --- пара статистик $(\theta^*_L, \theta^*_R)$, такая что для любого $\P \in \PP$  и $X_{[n]} \sim \P^{\otimes n}$
	\[
	\mathbb{P} \left( \theta^*_L(X_{[n]}) \le \theta(\P) \le \theta^*_R(X_{[n]}) \right) = \gamma 
	.\] 

	Интервал называется
	\begin{itemize}
		\item \selectedFont{асимптотическим}, если
	\[
	\mathbb{P} \left( \theta^*_L(X_{[n]}) \le \theta(\P) \le \theta^*_R(X_{[n]}) \right) \xrightarrow{n \to \infty} \gamma 
	.\] 
		\item \selectedFont{центральным}, если
			\[
			\mathbb{P} \left( \theta^*_L(X_{[n]}) > \theta(\P) \right) = \mathbb{P} \left( \theta^*_R(X_{[n]}) < \theta(\P) \right) 
			.\] 
		\item \selectedFont{левым}, если
			\[
			\mathbb{P} \left( \theta^*_L(X_{[n]}) > \theta(\P) \right) = 0
			.\] 
		\item \selectedFont{правым}, если
			\[
			\mathbb{P} \left( \theta^*_R(X_{[n]}) < \theta(\P) \right) = 0
			.\] 
	\end{itemize}
\end{definition}

\section{Бутстреп}
\subsection{Параметрический бутстреп}
Если работаем с параметрической моделью, можем заменить $X=X(\theta)$ не на $X^*$, а на $X(\theta^*)$ и сэмплировать из этого распределения.
\subsection{Непараметрический бутстреп }
\paragraph{Рецепт}
\begin{enumerate}
	\item изготовим $N$ выборок $x^*_{[n], 1}, \ldots , x^*_{[n], N}$ из эмпирического распределения (рандом с возвращением)
	\item вычисляем $\theta^{b}_{i} = \theta^*(x^*_{[n], i}$, получаем бутстреповскую выборку $\theta^{b}_{[N]}$,
	\item по бутстреповской выборке оцениваем, что нужно.
\end{enumerate}
\paragraph{Ограничения}
\begin{itemize}
	\item $\theta^*$ --- plug-in оценка
    \item $\theta^*$ --- достаточно гладкая (обычно дифференцируема)
	\item у  $X$ достаточно много моментов (обычно конечная дисперсия)
	\item нужно генерировать большие выборки
	\item на очень больших данных трудозатратен
	\item на маленьких данных велика неустранимая ошибка
\end{itemize}
\end{document}
